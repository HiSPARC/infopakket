\documentclass[twoside, 11pt]{exam}

\input{../common_style}

\newcommand{\defaultstyle}{headandfoot}

% style definitions
\pagestyle{\defaultstyle}
\chead{\oddeven{\rightthumb}{\leftthumb}}
\cfoot{\theshorttitle\ -- \thepage}
\lfoot{\oddeven{}{\textcolor{gray}{\smaller Versie \theversion}}}
\rfoot{\oddeven{\textcolor{gray}{\smaller Versie \theversion}}{}}

\renewcommand{\thequestion}{\textbf{Opdracht \arabic{question}:}}
\renewcommand{\solutiontitle}{\noindent\textbf{Antwoord:}\enspace}
\newcommand{\makelines}[1]{\ifprintanswers\else\fillwithlines{#1\linefillheight}\fi}

\ifdefined\showanswers
  \printanswers
\else
  \noprintanswers
\fi



\usepackage{footnote}


\title{Data Retrieval}
\author{D.B.R.A. Fokkema}
\docwerkblad{5}{WDT}
\version{1.0}

\begin{document}

\maketitle

\begin{questions}

\uplevel{\section{Inleiding}}

\uplevel{Het \hisparc project verzamelt al jaren data van tientallen
stations in voornamelijk Nederland, Denemarken en Engeland. Het is
gebruikelijk in de wetenschap dat het analyseren van data niet gebeurt in
kant-en-klare programma's, zoals een spreadsheet-programma.  Een programma
als Excel wordt vaak gebruikt om de cijfers van een klas in te voeren of
een balans bij te houden.  In het bedrijfsleven wordt het ook vaak
(mis)bruikt om met veel ingewikkelde formules een analyse van
bedrijfsgegevens uit te voeren. Hierbij gaat het bijna altijd om relatief
weinig gegevens.  Een dag \hisparc data bestaat met gemak uit
\numrange{50000}{60000} regels.}

\uplevel{In de wetenschap weet je nooit wat je kunt verwachten als je
onderzoek doet. Een uitgebreide analyse van gegevens wordt daarom normaal
gesproken geprogrammeerd. Een programmeeromgeving als \emph{Mathematica},
of een programmeertaal als \emph{Python} behoort tot het
standaardgereedschap van de onderzoeker.  Het aanleren en zelf schrijven
van zo'n analyse kost alleen heel veel tijd.}

\uplevel{Onderzoek wordt vrijwel altijd gedaan in een onderzoeksgroep, met
meerdere wetenschappers.  Vaak schrijven één of twee wetenschappers de
grote lijnen van een analyse en worden deze programma's gebruikt door de
overige leden van de groep.}

\begin{savenotes}
\uplevel{Het \hisparc-team heeft een relatief gebruiksvriendelijk
programma geschreven dat beschikbaar is in de webbrowser: de \emph{data
retrieval
tool}\footnote{\url{http://data.hisparc.nl/media/jsparc/data_retrieval.html}}.
Met dit programma kun je alle \hisparc data downloaden en eenvoudige tot
enigszins complexe analyses uitvoeren. }
\end{savenotes}

\question Foo.


\end{questions}
\end{document}
