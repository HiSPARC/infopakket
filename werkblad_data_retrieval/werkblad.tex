\documentclass[twoside, 11pt]{exam}

\input{../common_style}

\newcommand{\defaultstyle}{headandfoot}

% style definitions
\pagestyle{\defaultstyle}
\chead{\oddeven{\rightthumb}{\leftthumb}}
\cfoot{\theshorttitle\ -- \thepage}
\lfoot{\oddeven{}{\textcolor{gray}{\smaller Versie \theversion}}}
\rfoot{\oddeven{\textcolor{gray}{\smaller Versie \theversion}}{}}

\renewcommand{\thequestion}{\textbf{Opdracht \arabic{question}:}}
\renewcommand{\solutiontitle}{\noindent\textbf{Antwoord:}\enspace}
\newcommand{\makelines}[1]{\ifprintanswers\else\fillwithlines{#1\linefillheight}\fi}

\ifdefined\showanswers
  \printanswers
\else
  \noprintanswers
\fi



\usepackage{footnote}


\newcommand{\screenscale}{.5}


\title{Data Retrieval}
\author{D.B.R.A. Fokkema}
\docwerkblad{5}{WDT}
\version{1.0}

\begin{document}

\maketitle

\begin{questions}

\uplevel{\section{Inleiding}}

\uplevel{Het \hisparc project verzamelt al jaren data van tientallen
stations in voornamelijk Nederland, Denemarken en Engeland. Het is
gebruikelijk in de wetenschap dat het analyseren van data niet gebeurt in
kant-en-klare programma's, zoals een spreadsheet-programma.  Een programma
als Excel wordt vaak gebruikt om de cijfers van een klas in te voeren of
een balans bij te houden.  In het bedrijfsleven wordt het ook vaak
(mis)bruikt om met veel ingewikkelde formules een analyse van
bedrijfsgegevens uit te voeren. Hierbij gaat het bijna altijd om relatief
weinig gegevens.  Een dag \hisparc data, daarentegen, bestaat met gemak
uit \numrange{50000}{60000} regels.}

\uplevel{In de wetenschap weet je nooit wat je kunt verwachten als je
onderzoek doet. Een uitgebreide analyse van gegevens wordt daarom normaal
gesproken geprogrammeerd. Een programmeeromgeving als \emph{Mathematica},
of een programmeertaal als \emph{Python}, behoort tot het
standaardgereedschap van de onderzoeker.  Het aanleren en zelf schrijven
van zo'n analyse kost alleen heel veel tijd.}

\uplevel{Onderzoek wordt vrijwel altijd gedaan in een onderzoeksgroep, met
meerdere wetenschappers.  Vaak schrijven één of twee wetenschappers de
grote lijnen van een analyse en worden deze programma's gebruikt door de
overige leden van de groep.}

\begin{savenotes}
\uplevel{Het \hisparc-team heeft een relatief gebruiksvriendelijk
programma geschreven dat beschikbaar is in de webbrowser: de \emph{data
retrieval
tool}\footnote{\url{http://data.hisparc.nl/media/jsparc/data_retrieval.html}}.
Met dit programma kun je alle \hisparc data downloaden en eenvoudige tot
enigszins complexe analyses uitvoeren. }
\end{savenotes}

\question Open een browser en ga naar
\url{http://data.hisparc.nl/media/jsparc/data_retrieval.html}.  Controleer
dat je pagina overeenkomt met \figref{fig:tool-landing}.  Bekijk de pagina
goed.

\begin{figure}
  \centering
  \includegraphics[scale=\screenscale]{tool-landing}
  \caption{Openingspagina van de data retrieval tool.}
  \label{fig:tool-landing}
\end{figure}

\uplevel{In dit werkblad richten we ons op de linkerkolom: \emph{download
data}. Hier is het mogelijk om data van de \hisparc servers te downloaden
in de browser om vervolgens de data te analyseren.}

\question We gaan in eerste instantie \emph{events} downloaden.  Kies
station 501 (Nikhef) en een startdatum van 1 november 2014, en een
einddatum van 2 november 2014. Klik op \emph{Get Data!}.

\uplevel{Tijdens het downloaden wordt het \hisparc logo rechtsbovenaan de
pagina geanimeerd.  De animatie bootst een shower na die op het
aardoppervlak wordt gedetecteerd. Als de animatie stopt is de dataset
gedownload.}

\question Bekijk \figref{fig:tool-dataset}. Dit kopje verschijnt op de
pagina zodra een dataset beschikbaar is. Download voor hetzelfde station
en dezelfde data ook eens meteorologische gegevens (\emph{data type:
weather}).  Download ook eens events voor station 202. Het kopje datasets
komt nu overeen met \figref{fig:tool-datasets}.

\begin{figure}
  \centering
  \includegraphics[scale=\screenscale]{tool-dataset}
  \caption{Na downloaden is er een dataset beschikbaar.}
  \label{fig:tool-dataset}
\end{figure}

\begin{figure}
  \centering
  \includegraphics[scale=\screenscale]{tool-datasets}
  \caption{Meerdere datasets kunnen na elkaar worden gedownload.}
  \label{fig:tool-datasets}
\end{figure}

\uplevel{Bekijk \figref{fig:tool-datasets}.  Het getal onder
\emph{entries} is het aantal metingen (events of meteorologische gegevens)
dat beschikbaar is.  Als je klikt op \emph{csv} onder emph{download}, dan
download je alle data als tekstbestand. Je kunt de gegevens dan inladen in
een ander programma. Om een dataset uit het geheugen te verwijderen klik
je op het kruisje onder \emph{remove}.}

\uplevel{Een overzicht van de ruwe data is beschikbaar door te klikken op
\emph{show} onder \emph{preview}.}

\question Bekijk een preview van de events van station 501. Klik voor het
bovenste event op \emph{show} onder het kopje \emph{trace} helemaal
rechts. Je scherm moet nu overeen komen met
\figref{fig:tool-preview-trace}. De grafiek is een weergave van de ruwe
data van een \hisparc event. Geladen deeltjes zijn door de detectoren
gegaan en hebben een lichtspoor nagelaten. Bekijk de grafiek en bestudeer
de getallen van het eerste event in de tabel. Verklaar de getallen onder
\emph{pulseheights}, \emph{arrival time} en \emph{trigger time}.

\begin{figure}
  \centering
  \includegraphics[width=\linewidth]{tool-preview-trace}
  \caption{Preview van een \hisparc event.}
  \label{fig:tool-preview-trace}
\end{figure}

\uplevel{De kolom \emph{pulseintegral} geeft aan hoe groot de oppervlakte
is onder het signaal. Dit is een maat voor hoeveel energie de deeltjes
hebben achtergelaten in de detector en wordt gebruikt om een schatting te
maken van het aantal geladen deeltjes dat door een detector ging. Deze
schatting is weergegeven onder het kopje \emph{number of MIPs}. MIP staat
voor \emph{minimum-ionizing particle}, ofwel een deeltje dat een minimale
hoeveelheid energie verliest door ionisatie. Deeltjes in showers vallen in
deze categorie.}

\uplevel{In de tabel komt soms het getal -999 voor. Dit betekent dat het
niet mogelijk was een voorlopige analyse uit te voeren op de data.  In de
praktijk betekent dit dat er in die detector geen deeltjes zijn
gedetecteerd. Ook kan het getal -1 voorkomen. Dat betekent dat er voor die
kolom geen data beschikbaar is, bijvoorbeeld als je data bekijkt van een
station met twee detectoren, in plaats van vier.}

\question Bekijk een preview van de data van station 202. Hoeveel
detectoren heeft dit station?



\end{questions}
\end{document}
