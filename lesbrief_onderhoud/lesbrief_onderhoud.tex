\documentclass[oneside, 11pt]{article}

\usepackage[T1]{fontenc}
\usepackage[utf8]{inputenc}
\usepackage[dutch]{babel}

\usepackage{fouriernc}
\usepackage[detect-all, load-configurations=binary,
            separate-uncertainty=true, per-mode=symbol,
            retain-explicit-plus, range-phrase={ tot }]{siunitx}

\usepackage{setspace}
\setstretch{1.2}

\setlength{\parskip}{\smallskipamount}
\setlength{\parindent}{0pt}

\usepackage{geometry}
\geometry{marginparwidth=0.5cm, verbose, a4paper, tmargin=3cm, bmargin=3cm, lmargin=2cm, rmargin=2cm}

\usepackage{float}

\usepackage[fleqn]{amsmath}
\numberwithin{equation}{section}
\numberwithin{figure}{section}

\usepackage{graphicx}
\graphicspath{{Figures/}}
\usepackage{subfig}

\usepackage{tikz}
\usetikzlibrary{plotmarks}

\usepackage{fancyhdr}
\pagestyle{fancy}
\fancyhf{}
\rhead{\thepage}
\renewcommand{\footrulewidth}{0pt}
\renewcommand{\headrulewidth}{0pt}

\usepackage{relsize}
\usepackage{xspace}
\usepackage{url}

\newcommand{\figref}[1]{Figuur~\ref{#1}}

\newcommand{\hisparc}{\textsmaller{HiSPARC}\xspace}
\newcommand{\kascade}{\textsmaller{KASCADE}\xspace}
\newcommand{\sapphire}{\textsmaller{SAPPHiRE}\xspace}
\newcommand{\jsparc}{\textsmaller{jSparc}\xspace}
\newcommand{\hdf}{\textsmaller{HDF5}\xspace}
\newcommand{\aires}{\textsmaller{AIRES}\xspace}
\newcommand{\csv}{\textsmaller{CSV}\xspace}
\newcommand{\python}{\textsmaller{PYTHON}\xspace}
\newcommand{\corsika}{\textsmaller{CORSIKA}\xspace}
\newcommand{\labview}{\textsmaller{LabVIEW}\xspace}
\newcommand{\daq}{\textsmaller{DAQ}\xspace}
\newcommand{\adc}{\textsmaller{ADC}\xspace}
\newcommand{\adcs}{\textsmaller{ADC}s\xspace}
\newcommand{\Adcs}{A\textsmaller{DC}s\xspace}
\newcommand{\hi}{\textsc{h i}\xspace}
\newcommand{\hii}{\textsc{h ii}\xspace}
\newcommand{\mip}{\textsmaller{MIP}\xspace}
\newcommand{\hisparcii}{\textsmaller{HiSPARC II}\xspace}
\newcommand{\hisparciii}{\textsmaller{HiSPARC III}\xspace}
\newcommand{\pmt}{\textsmaller{PMT}\xspace}
\newcommand{\pmts}{\textsmaller{PMT}s\xspace}

\DeclareSIUnit{\electronvolt}{\ensuremath{\mathrm{e\!\!\:V}}}

\DeclareSIUnit{\unitsigma}{\ensuremath{\sigma}}
\DeclareSIUnit{\mip}{\textsmaller{MIP}}
\DeclareSIUnit{\adc}{\textsmaller{ADC}}

\DeclareSIUnit{\gauss}{G}
\DeclareSIUnit{\parsec}{pc}
\DeclareSIUnit{\year}{yr}


\usepackage{xfrac}

\title{Lesbrief Stationsonderhoud}
\author{C.G.N. van Veen}
\docwerkblad{1}{LSO}
\version{1.0}

\begin{document}

\maketitle

\section{Inleiding}

Deze lesbrief gaat in op het onderhoud van een \hisparc station.
Leerlingen krijgen eerst een introductie over kosmische straling en gaan
dan aan de slag met het analyseren van een \hisparc station en
uiteindelijk stellen de leerlingen het station optimaal in. Er is
achtergrond materiaal beschikbaar op
\url{http://www.hisparc.nl/docent-student/lesmateriaal/informatie-pakket/} en 
op \url{http://www.hisparc.nl/docent-student/lesmateriaal/routenetpad/} respectievelijk 
het \textit{infopakket} en \textit{routenet}.

De lessenserie van 3 lessen is als volgt opgebouwd:
\begin{description}
    \item[Les 1] Introductie over kosmische straling, bijhorende terminologie en metingen van \hisparc.
    Bij les 1 hoort een werkblad.

    \item[Les 2] Les over hoe de stations meten, hoe deze meten en wat er af te lezen valt van 
    histogrammen op: \url{http://data.hisparc.nl}.  
    Bij les 2 hoort een werkblad. 
 
    \item[Les 3] Onderhoud en instellen station
    In deze les leren de leerlingen iets over de instelling van het \hisparc station.
    Zij gaan de fotomultipliers van de detectoren instellen en de resultaten van 
    hun instellingen bekijken. Bij les 3 hoort een opgaveblad.
\end{description}

Opmerking: leerlingen hebben bij deze lessenserie baat bij kennis van 
elektrische velden, versnelspanning, energie van fotonen, radioactief verval, 
deeltjesfysica en de lorentzfactor ($\gamma$):
\begin{equation}
    E = h \cdot f 
\end{equation}
\begin{equation}
    q \cdot {U_{AK}} = \sfrac{1}{2} \cdot m \cdot {v^2}
\end{equation}
\begin{equation}
    \gamma = \frac{1}{\sqrt{1-\frac{v^2}{c^2}}}
\end{equation}

\section{Les 1}

Als docent kunt u een aantal invalshoeken kiezen om de introductie van
kosmische straling aan te bieden.  In deze les starten we met
achtergrond materiaal en een werkblad. Tijdens deze les kunnen
leerlingen ook een station bezoeken als dat op school staat. Belangrijk
is dat er een introductie op kosmische straling gegeven wordt en dat de
terminologie duidelijk wordt gemaakt.
Behandel in ieder geval:
\begin{description}
    \item{Wat is kosmische straling?}
    \item{Hoe ontstaat een deeltjeslawine?}
    \item{Hoe worden deze deeltjes in de lawine op aarde gedetecteerd?}
\end{description}
Kies van het achtergrondmateriaal uit het infopakket:
Algemeen:  \textit{Terminologie, Cosmic air showers en$/$of Uitleg \hisparc}.
Het werkblad Cosmic air showers kan uitgedeeld worden aan leerlingen.

Of kies van Routenet: 
\textit{Kosmische straling}. De opgaven van dit stencil kunnen door de leerlingen zelfstandig gemaakt worden.

Van de \hisparc site (\url{www.hisparc.nl}) kunnen diverse bestaande powerpoint-presentaties gedownload worden.
Deze presentaties mogen naar believen aangepast worden, om in de klas te gebruiken.

\section{Les 2}
In deze les gaan we naar de meetstations van \hisparc kijken en met name naar de 
afstelling van de detectoren en het onderhoud van een station.
We beginnen met achtergrondinformatie over de detectoren.

In deze les behandelen we zaken als:
\begin{description}
    \item{Hoe meten de detectoren de deeltjes in de lawine?}
    \item{Wat is het pulshoogte-histogram?}
    \item{Hoe kunnen we foutmeldingen opsporen en oplossingen vinden?}
\end{description}

Kies van het achtergrondmateriaal uit het infopakket: \textit{Inregelen
PMT's}, \textit{Controle station} en \textit{Uitleg \hisparc}. In het
document `controle station' wordt uitgelegd hoe leerlingen zelf
problemen met stations kunnen constateren en oplossen. Vooral een
pulshoogte diagram, die er uitziet als in figuur 3.1 in dit document is
een probleem wat leerlingen zelf kunnen oplossen.

En$/$of kies van Routenet: 
Detecteren en Detector:
De opgaven van deze bladen kunnen in de les gemaakt worden.

\section{Les 3}
In deze les kijken we naar de DAQ software van het meetstation en gaan de leerlingen 
zelf aan de slag met het instellen van de PMT's van het station.
De leerlingen krijgen wat (meer) uitleg over de fotomultiplier en stellen via de DAQ
de juiste spanning in voor elke detector.


In deze les behandelen we zaken als:
\begin{description}
    \item{Hoe werkt een fotomultiplierbuis?}
    \item{Hoe kunnen we de spanning op de stations zo instellen dat er een 
    goed pulshoogte diagram ontstaat?}
\end{description}

Nog schrijven hoe de leerlingen de instellingen van de PMT kunnen wijzigen etc.

\end{document}
