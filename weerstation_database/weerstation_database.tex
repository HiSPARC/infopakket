\documentclass[oneside, 11pt]{article}

\usepackage[T1]{fontenc}
\usepackage[utf8]{inputenc}
\usepackage[dutch]{babel}

\usepackage[font={small,sf},labelfont={bf},labelsep=endash]{caption}
\usepackage{fouriernc}
\usepackage[detect-all, load-configurations=binary,
            separate-uncertainty=true, per-mode=symbol,
            retain-explicit-plus, range-phrase={ tot }]{siunitx}

\usepackage{setspace}
\setstretch{1.2}

\setlength{\parskip}{\smallskipamount}
\setlength{\parindent}{0pt}

\usepackage{geometry}
\geometry{marginparwidth=0.5cm, verbose, a4paper,
          tmargin=3cm, bmargin=3cm, lmargin=2cm, rmargin=2cm}

\usepackage{float}

\usepackage[fleqn]{amsmath}
\numberwithin{equation}{section}
\numberwithin{figure}{section}

\usepackage{graphicx}
\graphicspath{{Figures/}}
\usepackage{subfig}

\usepackage{tikz}
\usetikzlibrary{plotmarks,circuits.ee.IEC}

\usepackage{fancyhdr}
\pagestyle{fancy}
\fancyhf{}
\rhead{\thepage}
\renewcommand{\footrulewidth}{0pt}
\renewcommand{\headrulewidth}{0pt}

\usepackage{relsize}
\usepackage{xspace}
\usepackage{url}

\newcommand{\figref}[1]{Figuur~\ref{#1}}

\newcommand{\hisparc}{\textsmaller{HiSPARC}\xspace}
\newcommand{\kascade}{\textsmaller{KASCADE}\xspace}
\newcommand{\sapphire}{\textsmaller{SAPPHiRE}\xspace}
\newcommand{\jsparc}{\textsmaller{jSparc}\xspace}
\newcommand{\hdf}{\textsmaller{HDF5}\xspace}
\newcommand{\aires}{\textsmaller{AIRES}\xspace}
\newcommand{\csv}{\textsmaller{CSV}\xspace}
\newcommand{\python}{\textsmaller{PYTHON}\xspace}
\newcommand{\corsika}{\textsmaller{CORSIKA}\xspace}
\newcommand{\labview}{\textsmaller{LabVIEW}\xspace}
\newcommand{\dspmon}{\textsmaller{DSPMon}\xspace}
\newcommand{\daq}{\textsmaller{DAQ}\xspace}
\newcommand{\adc}{\textsmaller{ADC}\xspace}
\newcommand{\adcs}{\textsmaller{ADC}s\xspace}
\newcommand{\Adcs}{A\textsmaller{DC}s\xspace}
\newcommand{\hi}{\textsc{h i}\xspace}
\newcommand{\hii}{\textsc{h ii}\xspace}
\newcommand{\mip}{\textsmaller{MIP}\xspace}
\newcommand{\hisparcii}{\textsmaller{HiSPARC II}\xspace}
\newcommand{\hisparciii}{\textsmaller{HiSPARC III}\xspace}
\newcommand{\pmt}{\textsmaller{PMT}\xspace}
\newcommand{\pmts}{\textsmaller{PMT}s\xspace}
\newcommand{\gps}{\textsmaller{GPS}\xspace}

\DeclareSIUnit{\electronvolt}{\ensuremath{\mathrm{e\!\!\:V}}}

\DeclareSIUnit{\unitsigma}{\ensuremath{\sigma}}
\DeclareSIUnit{\mip}{\textsmaller{MIP}}
\DeclareSIUnit{\adc}{\textsmaller{ADC}}

\DeclareSIUnit{\gauss}{G}
\DeclareSIUnit{\parsec}{pc}
\DeclareSIUnit{\year}{yr}



\title{Verzenden data naar \hisparc database}
\author{C.G.N. van Veen}
\docweerstation{3}{WD}
\version{1.0}

\begin{document}

\maketitle

\section{Weerstation Data}

\paragraph{Inleiding} We hebben ons weerstation werkend gekregen en krijgen nu 
data binnen op de computer. De weerdata komt zelfs via een wireless
transmitter binnen. Je hebt zelf een behuizing gemaakt om je station
tegen diverse weersomstandigheden te beschermen. In deze tutorial gaan we kijken
hoe we onze metingen aan de database van \hisparc kunnen toevoegen. 
We hebben dan op een goedkope wijze een eigen weerstation verkregen, waarmee
we naast de metingen van kosmische straling met het \hisparc station, ook de
efficiëntie van detectoren, aantal showers ed. kunnen meten en 
combineren met onze eigen weerdata. 

\paragraph{Benodigdheden}

\begin{itemize}  
    \item Arduino software
    \item Python software
    \item PC van \hisparc station
    \item Arduino weerstation
    
\end{itemize}

\section{\hisparc database}

\hisparc heeft een uitgebreide database waarin gegevens van kosmische straling
wordt opgeslagen. Deze data is vrij voor iedereen om uit te lezen en te gebruiken
voor onderzoek of om eens te leren omgaan met data en hoe je deze data kunt manipuleren met Python.
Uitgebreide informatie over de \hisparc database en hoe deze te benaderen is te 
vinden op \url{http://docs.hisparc.nl/publicdb/}. Hier vind je ook voorbeeld
programma's in Python waarmee je zelf data kunt binnen halen.
Wat wij willen doen is nu onze eigen weerdata koppelen aan de data van het meetstation
wat je op school hebt, zodat de \daq ze meestuurt naar de database.


\subsection{Data uitlezen en manipuleren} De \hisparc database verwacht
de data in een bepaald format en een bepaalde volgorde. Dat betekent dat
we allereerst ons Arduino programma zo moeten maken dat de data er in de
juiste volgorde uitkomt.
In het onderstaande stukje code en de tabel zien we welke grootheden we in principe naar de 
\hisparc database zouden kunnen sturen mits we daar de sensoren van hadden aangesloten.
De database van \hisparc verwacht van het weerstation data in de volgende 
volgorde (zie tabel volgorde). 

\begin{center}
\begin{table}
    \begin{tabular}{ | l | l | l|}
    \hline
    \textbf{Grootheid (NL)}& \textbf{Grootheid (Engels in Python)} & \textbf{Eenheid}  \\ \hline
    1. datum & Date   & y-m-d  \\ \hline
    2. tijd & Time    & h-min-s   \\ \hline
    3. temperatuur (detector) & tempInside & \SI{}{\celsius}    \\ \hline
    4. temperatuur (buiten) & 4tempOutside  & \SI{}{\celsius}    \\ \hline
    5. luchtvochtigheid (binnen)  & humidityInside & \verb|%|     \\ \hline
    6. luchtvochtigheid (buiten) & humidityOutside  & \verb|%|    \\ \hline
    7. Luchtdruk & barometer &  \SI{}{\pascal}   \\ \hline
    8. Windrichting & windDir &  graden \\ \hline
    9. Windsnelheid & windSpeed  &  \SI{}{\meter\per\second} \\ \hline
    10. Zonne intensiteit & SolarRad & \SI{}{\watt\per\square\meter}    \\ \hline
    11. UV index & UV & (0-16)    \\ \hline
    12. Verdampingstranspiratie & ET &  \SI{}{\milli\meter}    \\ \hline
    13. Hoeveelheid regen & rainRate & \SI{}{\milli\meter} \\ \hline
    14. Gevoelstemperatuur & heatIndex & \SI{}{\celsius} \\ \hline
    15. Dauwpunt & dewPoint & \SI{}{\celsius}    \\ \hline
    16. Gevoelstemperatuur (wind) & windChill & \SI{}{\celsius}    \\ \hline
   \end{tabular}
   \caption{tabel met de 16 mogelijke grootheden die meegegeven kunnen worden aan
   de data van het weerstation. Een aantal van deze grootheden zoals gevoelstemperatuur
   kunnen berekend worden met behulp van de bekende grootheden als luchtvochtigheid
   en buitentemperatuur. Formules die daarvoor nodig zijn kunnen in de literatuur gevonden worden.
   Als de data geanalyseerd wordt kunnen deze grootheden alsnog berekend worden.
   Voor andere grootheden zoals bijvoorbeeld hoeveelheid neerslag moet 
   een extra sensor worden aangesloten op de Arduino.}
   \label{table:grootheden}
\end{table}
\end{center}

In de python code houden we uit de tabel de engelse grootheid aan. Als we een grootheid
niet meten of uitrekenen dan wordt er een \verb|"-999"| toegevoegd aan de datalijst.
Daar moeten we in het python programma ook rekening mee houden.

\subsection{Test-procedure} 
Voordat we in de database gaan sleutelen moeten we eerst kijken of we de
data in de juiste volgorde van de Arduino krijgen, alle sensoren hun data goed opsturen
en de data op de juiste manier in een array krijgen wat naar de \hisparc database 
gezonden kan worden.

Allereerst gaan we zorgen dat ons Arduino programma de data zonder tekst en 
in de juiste volgorde (zoals in tabel \ref{table:grootheden}) verstuurt. 
In code (1) hieronder zien we hoe dat moet.

\begin{minted}{c}
//code 1

// Programma geeft data van weerstation (Temperatuur (1 detector en buiten),
// luchtvochtigheid (binnen -> -999 en buiten ) en luchtdruk in Pa.
// Gescheiden door komma's

#include <OneWire.h>  // to use data from ultiple sensors send through one wire
#include <DallasTemperature.h> //library for ds18Bb20
#include <DHT.h>  // library for humidity sensors DHTxx (11, 21, 22)
#include <Wire.h>
#include <BMP085.h>

// code for digital Temperature sensor (ds18b20)
#define ONE_WIRE_BUS 3
// To database means we only use the temperature of one detector. 
// Place the right temperature sensor in one of the detector.

// Setup DTH library with right sensor
DHT dht = DHT();

// Setup a oneWire instance to communicate with any OneWire devices in this 
// case tempsensors
OneWire oneWire(ONE_WIRE_BUS); 
// Pass our oneWire reference to Dallas Temperature.
DallasTemperature tsensors(&oneWire);
BMP085 dps = BMP085();      // Digital Pressure Sensor -> dps is accessing library 
//needed for library of BMP085
long Temperature = 0, Pressure = 0, Altitude = 1000; 

void setup(void)
{
  // start serial port
  Serial.begin(9600);
  //Serial.println("HiSPARC Weather station measuring:");

  Wire.begin();
  dht.setup(5); // data pin 5 humidity
  dps.init(MODE_ULTRA_HIGHRES, 1000, true);  // 250 meters, true = using meter units
  tsensors.begin();
}
 
void loop(void){
  // DTH22 Reading temperature or humidity takes about 250 milliseconds!
  // DTH22 Sensor readings may also be up to 2 seconds 'old' (its a very slow sensor)
  delay(dht.getMinimumSamplingPeriod());
  
  tsensors.requestTemperatures(); //Get temperature of detectors (one is default)
   for  (int deviceA = 0; deviceA < 1; deviceA++) {
    printTemp(deviceA);
  }  
  float humidity = dht.getHumidity();
  float temperature = dht.getTemperature();
  
  // check if returns are valid, if they are nan (not a number) 
  // then something went wrong!
  if (isnan(temperature) || isnan(humidity)) {
    Serial.println("Failed to read from DHT");
  } 
  
  else {
    Serial.print(temperature, 1); //Temperature outside
  Serial.print(",");
  Serial.print("-999"); // Fill in when NOT measuring humidity inside with sensor.
  Serial.print(",");
  Serial.print(humidity, 1); //humidity outside
  Serial.print(",");
  }  
  dps.getTemperature(&Temperature);
  dps.getPressure(&Pressure);
  float pressure = Pressure/100;
  Serial.print(pressure); //luchtdruk in hPa
  Serial.println();  
 delay(2000);
}

void printTemp(int adress) { 
  float TempC = tsensors.getTempCByIndex(adress);
  String stringone = "Detector ";
  stringone += adress;
  Serial.print(" ");
  Serial.print(TempC,1); // print temperatuur van detector
  Serial.print(",");
}
    
\end{minted}

\subsection{Data gereed maken voor verzending naar database}

De data komt als volgt \verb|24.6,24.4,-999,37.8,1004.67| uit het Arduino programma.
Nu moeten we een tijdstempel toevoegen en \verb|-999| voor grootheden die 
we niet gemeten hebben. In ons geval hebben we al een \verb|-999| omdat we de 
luchtvochtigheid \textit{binnen} niet meten.
Nu schrijven we naast het programma wat we al hadden in de handleiding: `Wireless
weerstation' (code 4 in die handleiding) een toevoeging, die de data zo manipuleert dat het in het 
format komt wat de \hisparc database kan lezen.

Gebruik de het onderstaande stukje code (2) in de code die je al had.
Je moet wel helemaal bovenaan je python code ook : 'import datetime' toevoegen.

\begin{minted}{python}
# Code 2
# code om data het juiste format te geven. Dauwpunt wordt berekend.

def format_output(output):
    # This function creates the output needed to add to MYSQL database of HiSPARC station
    # weatherevent.py assumes date, time, temp_inside (detector), tempOutside, humidityInside ,
    # humidityOutside, barometer, windDir, windSpeed, solarRad, UV, ET, rainRate
    # heatIndex, dewPoint, windChill

    Tdew = dauwpunt_calc(output)
    # As we do not measure these weather variables we set them to '-999'
    # If the weatherstation measures more variables we can add them to the list
    # This is done similar to [dewPoint] in line 61
    Winddir = '-999'
    windSpeed = '-999'
    SolarRad  = '-999'
    UV = '-999'
    ET = '-999'
    rainRate = '-999'
    heatIndex = '-999'
    dewPoint = Tdew
    windChill = '-999'

    # Could have added all variables like 'dewPoint' but 8* ['-999'] works as well.
    # in this case without many sensors attached to weatherstation
    Temperor = output + 8 *['-999'] +[dewPoint] + ['-999'] 
    # volgorde tabel klopt nu met tabel 1.

    # Add timetstamp of measurement
    now = datetime.datetime.now()
    ctime = now.strftime('%Y-%m-%d %H:%M:%S')
    tempoutput = ctime + '\t' + '\t'.join(str(i) for i in Temperor)

    return tempoutput

def dauwpunt_calc(output):

    output = [float(i) for i in output]
    RH = output[3]/100  #calculate relative humidity
    Temp = output[1]  #outside temerature

    # calculate vaporpressure, Dewpoint: Formula from Vantage Pro Davis instruments

    dampdruk = RH*6.112 * numpy.exp((17.62*Temp)/(Temp+ 243.12))
    Numerator = 243.12*numpy.log(dampdruk)-440.1
    Denominator = 19.43-numpy.log(dampdruk)
    Tdew = Numerator/Denominator
    dampdruk2 = dampdruk
    Tdew = "%4.3f" %Tdew
    return Tdew

\end{minted}

\section{Data wegschrijven naar MySQL database: windows.}

We hebben nu de data in de volgende vorm: 
"2015-05-11 17:08:21	24.2	25.0	-999	45.6	1017.0	-999	-999	
-999	-999	-999	-999	-999	-999	12.4	-999". De data die we 
gemeten hebben staat als getal en alles wat we niet meten wordt weergegeven 
door de `-999'. We willen elke keer dat we zo elke regel met data kunnen
wegschrijven in de database van \hisparc.
Dat betekent dat we de SQL database moeten openen met een user en paswoord.
In het script `Datafromwirelesstodatabase.py' zie je een class Database, deze gaan we 
gebruiken om onze data weg te schrijven.
\begin{thebibliography}{9}
    
\end{thebibliography}


\end{document}
