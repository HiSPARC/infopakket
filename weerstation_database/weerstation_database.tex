\documentclass[oneside, 11pt]{article}

\usepackage[T1]{fontenc}
\usepackage[utf8]{inputenc}
\usepackage[dutch]{babel}

\usepackage[font={small,sf},labelfont={bf},labelsep=endash]{caption}
\usepackage{fouriernc}
\usepackage[detect-all, load-configurations=binary,
            separate-uncertainty=true, per-mode=symbol,
            retain-explicit-plus, range-phrase={ tot }]{siunitx}

\usepackage{setspace}
\setstretch{1.2}

\setlength{\parskip}{\smallskipamount}
\setlength{\parindent}{0pt}

\usepackage{geometry}
\geometry{marginparwidth=0.5cm, verbose, a4paper,
          tmargin=3cm, bmargin=3cm, lmargin=2cm, rmargin=2cm}

\usepackage{float}

\usepackage[fleqn]{amsmath}
\numberwithin{equation}{section}
\numberwithin{figure}{section}

\usepackage{graphicx}
\graphicspath{{Figures/}}
\usepackage{subfig}

\usepackage{tikz}
\usetikzlibrary{plotmarks,circuits.ee.IEC}

\usepackage{fancyhdr}
\pagestyle{fancy}
\fancyhf{}
\rhead{\thepage}
\renewcommand{\footrulewidth}{0pt}
\renewcommand{\headrulewidth}{0pt}

\usepackage{relsize}
\usepackage{xspace}
\usepackage{url}

\newcommand{\figref}[1]{Figuur~\ref{#1}}

\newcommand{\hisparc}{\textsmaller{HiSPARC}\xspace}
\newcommand{\kascade}{\textsmaller{KASCADE}\xspace}
\newcommand{\sapphire}{\textsmaller{SAPPHiRE}\xspace}
\newcommand{\jsparc}{\textsmaller{jSparc}\xspace}
\newcommand{\hdf}{\textsmaller{HDF5}\xspace}
\newcommand{\aires}{\textsmaller{AIRES}\xspace}
\newcommand{\csv}{\textsmaller{CSV}\xspace}
\newcommand{\python}{\textsmaller{PYTHON}\xspace}
\newcommand{\corsika}{\textsmaller{CORSIKA}\xspace}
\newcommand{\labview}{\textsmaller{LabVIEW}\xspace}
\newcommand{\dspmon}{\textsmaller{DSPMon}\xspace}
\newcommand{\daq}{\textsmaller{DAQ}\xspace}
\newcommand{\adc}{\textsmaller{ADC}\xspace}
\newcommand{\adcs}{\textsmaller{ADC}s\xspace}
\newcommand{\Adcs}{A\textsmaller{DC}s\xspace}
\newcommand{\hi}{\textsc{h i}\xspace}
\newcommand{\hii}{\textsc{h ii}\xspace}
\newcommand{\mip}{\textsmaller{MIP}\xspace}
\newcommand{\hisparcii}{\textsmaller{HiSPARC II}\xspace}
\newcommand{\hisparciii}{\textsmaller{HiSPARC III}\xspace}
\newcommand{\pmt}{\textsmaller{PMT}\xspace}
\newcommand{\pmts}{\textsmaller{PMT}s\xspace}
\newcommand{\gps}{\textsmaller{GPS}\xspace}

\DeclareSIUnit{\electronvolt}{\ensuremath{\mathrm{e\!\!\:V}}}

\DeclareSIUnit{\unitsigma}{\ensuremath{\sigma}}
\DeclareSIUnit{\mip}{\textsmaller{MIP}}
\DeclareSIUnit{\adc}{\textsmaller{ADC}}

\DeclareSIUnit{\gauss}{G}
\DeclareSIUnit{\parsec}{pc}
\DeclareSIUnit{\year}{yr}



\title{Weerdata versturen naar \hisparc}
\author{C.G.N. van Veen}
\docweerstation{3}{WD}
\version{1.1}

\begin{document}

\maketitle

\section{Weerstation Data}

\paragraph{Inleiding} We hebben ons weerstation werkend gekregen en krijgen nu 
data binnen op de computer. De weerdata komt zelfs via een wireless
transmitter binnen. Je hebt zelf een behuizing gemaakt om je station
tegen diverse weersomstandigheden te beschermen. In deze tutorial gaan we kijken
hoe we onze metingen aan de database van \hisparc kunnen toevoegen. 
We hebben dan op een goedkope wijze een eigen weerstation verkregen, waarmee
we de efficiëntie van detectoren bij veranderende temperatuur en bijvoorbeeld 
het aantal showers als functie van de luchtdruk kunnen meten. Ons station is uitgerust 
met luchtdruk, luchtvochtigheid en temperatuur sensoren, maar is gemakkelijk uit 
te breiden met andere sensoren. 

\paragraph{Benodigdheden}

\begin{itemize}  
    \item Arduino software
    \item Python software
    \item PC van \hisparc station
    \item Arduino weerstation
    
\end{itemize}

\section{\hisparc database}

\hisparc heeft een uitgebreide database waarin gegevens van kosmische straling
wordt opgeslagen. Deze data is vrij voor iedereen om uit te lezen, te gebruiken
voor onderzoek of om te leren omgaan met data en hoe je deze data kunt manipuleren met Python.
Uitgebreide informatie over de \hisparc database en hoe deze te benaderen is te 
vinden op \url{http://docs.hisparc.nl/publicdb/}. Hier vind je ook voorbeeld
programma's in Python waarmee je zelf data kunt binnen halen.
Wat wij willen doen is nu onze eigen weerdata koppelen aan de data van het \hisparc 
meetstation, zodat deze data wordt opgeslagen in de \hisparc database.


\subsection{Data uitlezen en manipuleren}

De \hisparc database verwacht
de data in een bepaald format en een bepaalde volgorde. Dat betekent dat
we allereerst ons Arduino programma zo moeten maken dat de data er in de
juiste volgorde uitkomt.
In het onderstaande stukje code en de tabel zien we welke grootheden we in principe naar de 
\hisparc database zouden kunnen sturen als we daar de sensoren van hadden aangesloten.
De database van \hisparc verwacht van het weerstation data in de volgende 
volgorde (zie tabel volgorde). 

\begin{center}
\begin{table}
    \begin{tabular}{ | l | l | l|}
    \hline
    \textbf{Grootheid (NL)}& \textbf{Grootheid (Engels in Python)} & \textbf{Eenheid}  \\ \hline
    1. datum & Date   & y-m-d  \\ \hline
    2. tijd & Time    & h-min-s   \\ \hline
    3. temperatuur (detector) & \verb|temp_inside| & \SI{}{\celsius}    \\ \hline
    4. temperatuur (buiten) & \verb|temp_outside| & \SI{}{\celsius}    \\ \hline
    5. luchtvochtigheid (binnen)  & \verb|humidity_inside|& \verb|%|     \\ \hline
    6. luchtvochtigheid (buiten) & \verb|humidity_outside|  & \verb|%|    \\ \hline
    7. Luchtdruk & barometer &  \SI{}{\pascal}   \\ \hline
    8. Windrichting & \verb|wind_dir| &  graden \\ \hline
    9. Windsnelheid & \verb|wind_speed|  &  \SI{}{\meter\per\second} \\ \hline
    10. Zonne intensiteit & \verb|solar_rad| & \SI{}{\watt\per\square\meter}    \\ \hline
    11. UV index & uv & (0-16)    \\ \hline
    12. Verdampingstranspiratie & evapotranspiration &  \SI{}{\milli\meter}    \\ \hline
    13. Hoeveelheid regen & \verb|rain_rate| & \SI{}{\milli\meter} \\ \hline
    14. Gevoelstemperatuur & \verb|heat_index| & \SI{}{\celsius} \\ \hline
    15. Dauwpunt & \verb|dew_point| & \SI{}{\celsius}    \\ \hline
    16. Gevoelstemperatuur (wind) & \verb|wind_chill| & \SI{}{\celsius}    \\ \hline
    
   \end{tabular}
   \caption{Tabel met de 16 mogelijke grootheden die meegegeven kunnen
   worden aan de data van het weerstation. Een aantal van deze
   grootheden zoals gevoelstemperatuur kunnen berekend worden met behulp
   van de bekende grootheden als luchtvochtigheid en buitentemperatuur.
   Formules die daarvoor nodig zijn kunnen in de literatuur gevonden
   worden.
   \protect\url{http://github.com/HiSPARC/weather/blob/master/doc/
   _static/Parameter_Manual.pdf}. Als de data geanalyseerd wordt kunnen
   deze grootheden alsnog berekend worden. Voor andere grootheden zoals
   bijvoorbeeld hoeveelheid neerslag moet een extra sensor worden
   aangesloten op de Arduino.}
   \label{table:grootheden}
\end{table}
\end{center}

In de Python code houden we uit de tabel de Engelse grootheid aan. Als we een grootheid
niet meten of kunnen uitrekenen dan wordt er een \verb|"-999"| toegevoegd aan de datalijst.
Daar moeten we in het Python programma ook rekening mee houden.


\subsection{Test-procedure} 

Voordat we in de database gaan sleutelen moeten we eerst kijken of we de
data in de juiste volgorde van de Arduino krijgen, alle sensoren hun data goed opsturen
zodat de juiste data naar de \hisparc database gezonden kan worden.

Allereerst gaan we zorgen dat ons Arduino programma de data zonder tekst en 
in de juiste volgorde (zoals in tabel \ref{table:grootheden}) verstuurt. 
In code (1) hieronder zien we hoe dat moet.

\begin{minted}{c}
//code 1

// Programma geeft data van weerstation (Temperatuur (van een detector en van de
// buitenlucht),
// luchtvochtigheid (binnen -> -999 en buiten ) en luchtdruk in Pa.
// Gescheiden door komma's

#include <OneWire.h>  // to use data from multiple sensors send through one wire
#include <DallasTemperature.h> //library for ds18Bb20
#include <DHT.h>  // library for humidity sensors DHTxx (11, 21, 22)
#include <Wire.h>
#include <BMP085.h>

// code for digital Temperature sensor (ds18b20)
#define ONE_WIRE_BUS 3
// To database means we only use the temperature of one detector. 
// Place the a temperature sensor in one of the detectors of the HiSPARC station.

// Setup DTH library with right sensor
DHT dht = DHT();

// Setup a oneWire instance to communicate with any OneWire devices in this 
// case tempsensors
OneWire oneWire(ONE_WIRE_BUS); 
// Pass our oneWire reference to Dallas Temperature.
DallasTemperature tsensors(&oneWire);
BMP085 dps = BMP085();      // Digital Pressure Sensor -> dps is accessing library 
//needed for library of BMP085
long Temperature = 0, Pressure = 0, Altitude = 1000; 

void setup(void)
{
  // start serial port
  Serial.begin(9600);
  //Serial.println("HiSPARC Weather station measuring:");

  Wire.begin();
  dht.setup(5); // data pin 5 humidity
  dps.init(MODE_ULTRA_HIGHRES, 1000, true);  // 250 meters, true = using meter units
  tsensors.begin();
}
 
void loop(void){
  // DTH22 Reading temperature or humidity takes about 250 milliseconds!
  // DTH22 Sensor readings may also be up to 2 seconds 'old' (its a very slow sensor)
  delay(dht.getMinimumSamplingPeriod());
  
  tsensors.requestTemperatures(); //Get temperature of detectors (one is default)
   for  (int deviceA = 0; deviceA < 1; deviceA++) {
     printTemp(deviceA);
  }  
  float humidity = dht.getHumidity();
  float temperature = dht.getTemperature();
  
  // check if returns are valid, if they are nan (not a number) 
  // then something went wrong!
  if (isnan(temperature) || isnan(humidity)) {
    continue;
  } 
  
  else {
    Serial.print(temperature, 1); //Temperature outside
    Serial.print(",");
    Serial.print(humidity, 1); //humidity outside
    Serial.print(",");
  }  
  dps.getTemperature(&Temperature);
  dps.getPressure(&Pressure);
  float pressure = Pressure/100;
  Serial.print(pressure); //luchtdruk in hPa
  Serial.println();  
 delay(2000);
}

void printTemp(int adress) { 
  float TempC = tsensors.getTempCByIndex(adress);
  String stringone = "Detector ";
  stringone += adress;
  Serial.print(" ");
  Serial.print(TempC,1); // print temperatuur van detector
  Serial.print(",");
}
    
\end{minted}


\subsection{Data gereed maken voor verzending naar database}

De data komt als volgt: \verb|24.6,24.4,37.8,1004.67| uit het Arduino programma.

Nu schrijven we naast het programma wat we al hadden in de handleiding: `Wireless
weerstation' (code 4 in die handleiding) een toevoeging, die de data zo manipuleert dat het in het 
format komt wat de \hisparc database kan lezen.

Gebruik het onderstaande stukje code (2) in de code die je al had.

\begin{minted}{python}
# Code 2
# code om data het juiste format te geven. Ook het Dauwpunt wordt berekend.
# we werken met een Class.

# in de onderstaande Class komt de output van de Arduino binnen.
class Measurement(object):

    def __init__(self, output):

        # Datastore assumes date, time, temp_inside (detector),
        # temp_outside, humidity_inside, humidity_outside, barometer,
        # wind_dir, wind_speed, solar_rad, uv, evapotranspiration, rain_rate, heat_index,
        # dewpoint, wind_chill

        # Extracts variables from output, order of variables is important.
        # Set the order of variables in the Arduino Program.
        self.temp_inside, self.temp_outside, self.humidity_outside, 
        self.barometer = output

        Tdew = self.dampdruk_calc(self.temp_outside, self.humidity_outside, 
                                                            self.barometer)

        # As we do not measure these weather variables we set them to
        # '-999' If the weatherstation does measure these variables we
        # can add them to the list and extract them from the Arduino
        # output.

        self.wind_dir = '-999'
        self.humidity_inside = '-999'
        self.wind_speed = '-999'
        self.solar_rad  = '-999'
        self.uv = '-999'
        self.evapotranspiration = '-999'
        self.rain_rate = '-999'
        self.heat_index = '-999'
        self.dew_point = Tdew       # dew_point is uitgerekend.
        self.wind_chill = '-999'

        # Add timetstamp of measurement
        self.datetime = datetime.datetime.now()
        self.nanoseconds = 0


    def dampdruk_calc(self, Tout, Hum_out, baro):
        RH = Hum_out/100  #calculate relative humidity

        # Calculate vaporpressure, Dewpoint: Formula from Vantage Pro Davis instruments
        # This document can be found at Github/HiSPARC/weather/doc/_static/
        #                                               Parameter_Manual.pdf

        dampdruk = RH * 6.112 * numpy.exp((17.62 * Tout)/(Tout + 243.12))
        Numerator = 243.12 * numpy.log(dampdruk)-440.1
        Denominator = 19.43 - numpy.log(dampdruk)
        Tdew = Numerator / Denominator
        Tdew = "%4.1f" %Tdew

        return Tdew

\end{minted}

In de `class Measurement' wordt ook het dauwpunt berekend. Als er andere afgeleide grootheden 
berekend moeten worden, kan dat gedaan worden door de \verb|`def dampdruk_calc'| te kopiëren en 
dan de berekening in te voeren met grootheden die je gemeten hebt. Je kunt
bijvoorbeeld de `\verb|wind_chill|' uit laten rekenen als je ook een sensor voor de 
windsnelheid hebt aangesloten. Voeg dan in de class Measurement in:

\begin{minted}{python}
# Code 3a
# voorbeeld als de windsnelheid bekend is
def wind_chill_calc(self, Tout, windspeed):
    
    # verdere berekeningen in het document Parameter_Manual.pdf en zie code 2.
    
    return wind_chill
    
\end{minted}        

Deze definitie berekent dan de gevoelstemperatuur. Deze kun je weer meegeven aan 
de lijst met data.

\begin{minted}{python}
# Code 3b
# voorbeeld als de windsnelheid bekend is

    def __init__(self, output):
        #aanroepen functie
        windchill = self.wind_chill_calc(self.temp_outside, self.windspeed)

        #toekennen waarde
        self.wind_chill = windchill
        
\end{minted}   
        
Op dezelfde wijze kunnen er andere definities voor afgeleide grootheden
binnen de class Measurement gemaakt worden. 

De andere classes in het Python programma hoeven niet verandert te worden.


\section{Data wegschrijven naar \hisparc database: Windows/MAC.}


\subsection{Libraries van python installeren}

Het gehele Python bestand waarmee je de data naar de \hisparc database kunt schrijven 
is te vinden op:
\url{https://github.com/HiSPARC/weather-arduino/tree/master/python}
Om het geheel te laten werken zijn, moeten wel een aantal libraries geïmporteerd worden.
Werk je in Canopy onder windows open dan een terminal. Ga naar `Tools' en open 
een Canopy Terminal.
Type in de terminal:
\begin{verbatim}
enpkg setuptools
enpkg pip
\end{verbatim}

Nu kun je eventueel ontbrekende libraries installeren. Met het commando: 
`pip freeze' kun je kijken welke libraries je al hebt. 

\begin{figure}
    \centering
    \includegraphics[width=0.80 \textwidth]{Canopy}
    \caption{Venster van Canopy.}
   \label{Canopy}
\end{figure}

Om te checken of libraries geïmporteerd zijn kun je in Canopy in het onderste venster
invoeren: \emph{import ......}. Zie \figref{fig:Canopy}

Als je geen foutmelding hebt dan is de library geïnstalleerd. 
In code 4 kun je zien welke libraries allemaal toegevoegd moeten zijn.
 
\begin{minted}{python}
# Code 4
import serial
import datetime
import time
import numpy
import sys
import time
import hashlib
import requests
import logging
import calendar
import cPickle 
from time import sleep
from pysparc import storage

\end{minted}

Waarschijnlijk moet `pysparc' nog geïnstalleerd worden. Open de terminal van
Canopy. Wat we gaan doen is een link naar de pysparc module van \hisparc
maken en deze gelijk installeren. type: 
\begin{verbatim} 
pip install http://github.com/HiSPARC/pysparc/archive/master.zip 
\end{verbatim}

Nu kun je \verb|import pysparc| gebruiken, Zie \figref{fig:Canopy} .


\subsection{Storage backup}

Als de internet verbinding wegvalt dan willen we niet dat de data verloren gaat.
We gebruiken daarom module \emph{pysparc}. Deze gebruikt een database (Redis) om data
op te slaan, totdat er weer een internetverbinding tot stand komt. 

- Voor \textbf{windows} ga je naar:
\begin{verbatim}
http://github.com/rgl/redis/downloads
\end{verbatim}

Download de versie van Redis, die je nodig hebt.
Installeer Redis op je computer. Start nu de `command prompt' als administrator.
Run het volgende commando: 

\begin{verbatim}
net start redis
\end{verbatim}

Elke keer dat je de (windows-)computer opnieuw opstart, moet wel Redis opnieuw opgestart worden.

- Voor de \textbf{Mac}:
Als je `Homebrew' geïnstalleerd hebt dan kun je eenvoudig het volgende 
in een terminal intypen: 
\begin{verbatim}
brew install redis
\end{verbatim}

Heb je geen Homebrew. Open dan een terminal en type de volgende commando's in. 

\begin{enumerate}
    \item curl -O http://download.redis.io/redis-stable.tar.gz
    \item tar -xvzf redis-stable.tar.gz 
    \item rm redis-stable.tar.gz 
    \item cd redis-stable 
    \item make 
    \item sudo make install
    \item redis-server
 \end{enumerate}
 
De Redis server is nu gestart. 


\subsection{Meten met het weerstation en data opsturen}

Zorg dat je in Arduino het weerprogramma upload naar de Arduino
(bijvoorbeeld met een USB kabel. Haal bij deze procedure het snoertje bij `Rx' op de
Arduino weg.) Bekijk in de Serial monitor of de de data correct wordt
weergegeven. Sluit nu de UART-APC220 module aan op de Arduino en de PC.  

Start Redis.

Open nu het Python programma (te vinden op
\url{http://github.com/HiSPARC/weather-arduino}. \verb|Datafromwirelesstodatabase.py|
Run het programma. 
\emph{!Pas op!} Pas alleen het programma alleen aan in de class die in deze
handleiding benoemd zijn. Dus alleen in de class `Measurement' kun je sensor waarden 
toevoegen.

Om te checken of het allemaal werkt draai je het programma in Canopy. Vul zelf 
je \hisparc station nr. in en het paswoord voor dat station. Het paswoord is te krijgen
bij \hisparc door te mailen naar \verb|info@hisparc.nl|. 


Door het Python script wordt alleen het dauwpunt geprint. Dit om te checken
of er nog steeds gemeten wordt. Als het goed is zou het dauwpunt tussen 
een waarde van 0-10 moeten liggen. Dit printen kun je als commentaar uitschakelen.

Sluit het internet even af en kijk of na weer inschakelen van het internet weer
data verzonden wordt. Als dat gebeurt is alles klaar voor gebruik.
Plaats nu je weerstation op het dak bij het station. Na een dag meten kun je 
bij de gegevens van het station ook de `weer grafiekjes' zien. Bekijk ze op
\url{data.hisparc.nl}.
Je eigen weerdata kun je nu altijd online uitlezen met dataretrieval tool. Zie de handleiding
'data retrieval' op \url{http://www.hisparc.nl/docent-student/lesmateriaal/informatie-pakket/}.





\end{document}
