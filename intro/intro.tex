\documentclass[oneside, 11pt]{article}

\usepackage[T1]{fontenc}
\usepackage[utf8]{inputenc}
\usepackage[dutch]{babel}

\usepackage{fouriernc}
\usepackage[detect-all, load-configurations=binary,
            separate-uncertainty=true, per-mode=symbol,
            retain-explicit-plus, range-phrase={ tot }]{siunitx}

\usepackage{setspace}
\setstretch{1.2}

\setlength{\parskip}{\smallskipamount}
\setlength{\parindent}{0pt}

\usepackage{geometry}
\geometry{marginparwidth=0.5cm, verbose, a4paper, tmargin=3cm, bmargin=3cm, lmargin=2cm, rmargin=2cm}

\usepackage{float}

\usepackage[fleqn]{amsmath}
\numberwithin{equation}{section}
\numberwithin{figure}{section}

\usepackage{graphicx}
\graphicspath{{Figures/}}
\usepackage{subfig}

\usepackage{tikz}
\usetikzlibrary{plotmarks}

\usepackage{fancyhdr}
\pagestyle{fancy}
\fancyhf{}
\rhead{\thepage}
\renewcommand{\footrulewidth}{0pt}
\renewcommand{\headrulewidth}{0pt}

\usepackage{relsize}
\usepackage{xspace}
\usepackage{url}

\newcommand{\figref}[1]{Figuur~\ref{#1}}

\newcommand{\hisparc}{\textsmaller{HiSPARC}\xspace}
\newcommand{\kascade}{\textsmaller{KASCADE}\xspace}
\newcommand{\sapphire}{\textsmaller{SAPPHiRE}\xspace}
\newcommand{\jsparc}{\textsmaller{jSparc}\xspace}
\newcommand{\hdf}{\textsmaller{HDF5}\xspace}
\newcommand{\aires}{\textsmaller{AIRES}\xspace}
\newcommand{\csv}{\textsmaller{CSV}\xspace}
\newcommand{\python}{\textsmaller{PYTHON}\xspace}
\newcommand{\corsika}{\textsmaller{CORSIKA}\xspace}
\newcommand{\labview}{\textsmaller{LabVIEW}\xspace}
\newcommand{\daq}{\textsmaller{DAQ}\xspace}
\newcommand{\adc}{\textsmaller{ADC}\xspace}
\newcommand{\adcs}{\textsmaller{ADC}s\xspace}
\newcommand{\Adcs}{A\textsmaller{DC}s\xspace}
\newcommand{\hi}{\textsc{h i}\xspace}
\newcommand{\hii}{\textsc{h ii}\xspace}
\newcommand{\mip}{\textsmaller{MIP}\xspace}
\newcommand{\hisparcii}{\textsmaller{HiSPARC II}\xspace}
\newcommand{\hisparciii}{\textsmaller{HiSPARC III}\xspace}
\newcommand{\pmt}{\textsmaller{PMT}\xspace}
\newcommand{\pmts}{\textsmaller{PMT}s\xspace}

\DeclareSIUnit{\electronvolt}{\ensuremath{\mathrm{e\!\!\:V}}}

\DeclareSIUnit{\unitsigma}{\ensuremath{\sigma}}
\DeclareSIUnit{\mip}{\textsmaller{MIP}}
\DeclareSIUnit{\adc}{\textsmaller{ADC}}

\DeclareSIUnit{\gauss}{G}
\DeclareSIUnit{\parsec}{pc}
\DeclareSIUnit{\year}{yr}



\title{Intro}
\author{J.M.C. Montanus}
\docalgemeen{0}{IN}
\version{1.0}

\begin{document}

\maketitle

\section{\hisparc}

\hisparc (High School Project on Astrophysics Research with Cosmics) is
een grootschalig experiment waarmee middelbare scholieren het heelal
kunnen onderzoeken. Het omvat momenteel een netwerk van meer dan 100
stations verspreid over Nederland met inmiddels uitbreidingen naar
Denemarken en Engeland. 


\section{Uniek}

Het unieke van \hisparc is dat de detectoren gebouwd en geplaatst zijn
door leerlingen uit het voortgezet onderwijs. Door de leerlingen de detectoren te laten onderhouden en de
meetgegevens te laten verwerken, worden ze direct betrokken bij
wetenschappelijk onderzoek naar kosmische straling. 


\section{Veelzijdig}

Kosmische deeltjes die onze atmosfeer binnendringen zullen door
botsingen met de luchtmoleculen een lawine van deeltjes teweegbrengen. De
processen die daarbij een rol spelen zijn te verklaren met de
\emph{quantummechanica}. In de ontwikkeling van een lawine komen vrijwel
alle \emph{elementaire deeltjes} voor. Met de
\emph{relativiteitstheorie} wordt verklaard waarom de kortlevende
\emph{muonen} uit de lawine toch het aardoppervlak kunnen bereiken. Allemaal onderdelen van het examenprogramma van Nieuwe Natuurkunde (NiNa). Het sluit tevens aan bij diverse NLT modules zoals \emph{Kosmische straling} en \emph{Kwantumstructuur van de materie}. Leerlingen die zich verdiepen in de werking van de detectoren komen in aanraking met diverse natuurkundige verschijnselen en technieken, zoals het
\emph{fluorescentielicht} dat onstaat als een elektron door de
scintillatorplaat van een \hisparc detector gaat, de versterking van dit
licht in de \pmt (Photo Multiplier Tube), de omzetting ervan tot een
elektrische signaal en tenslotte als digitale waarde wordt opgeslagen.      


\section{Uitdagend}

Door de meetgegevens op te vragen kunnen leerlingen bijvoorbeeld kijken
naar de frequentie van lawines.  Leerlingen kunnen daarbij verschillende
onderzoeksvragen stellen. Hangen de aantallen lawines af van het
seizoen, van dag of nacht, van de temperatuur, van de luchtdruk of van
andere weersomstandigheden zoals onweer? Zo ontstaat uitdagend
wetenschappelijk onderzoek. 


\section{Leerlingopdrachten}

Om docenten en leerlingen een ingang te bieden is een aantal
uitgewerkte opdrachten samengesteld. Door de beschreven handelingen uit
te voeren ervaren leerlingen hoe ze hun \hisparc station kunnen
onderhouden en calibreren, hoe ze de meetgegevens kunnen opvragen en hoe
ze op basis van de meetgegevens hun onderzoeksvragen kunnen beantwoorden.

\end{document}
