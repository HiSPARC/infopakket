\documentclass[oneside, 11pt]{article}

\usepackage[T1]{fontenc}
\usepackage[utf8]{inputenc}
\usepackage[dutch]{babel}

\usepackage[font={small,sf},labelfont={bf},labelsep=endash]{caption}
\usepackage{fouriernc}
\usepackage[detect-all, load-configurations=binary,
            separate-uncertainty=true, per-mode=symbol,
            retain-explicit-plus, range-phrase={ tot }]{siunitx}

\usepackage{setspace}
\setstretch{1.2}

\setlength{\parskip}{\smallskipamount}
\setlength{\parindent}{0pt}

\usepackage{geometry}
\geometry{marginparwidth=0.5cm, verbose, a4paper,
          tmargin=3cm, bmargin=3cm, lmargin=2cm, rmargin=2cm}

\usepackage{float}

\usepackage[fleqn]{amsmath}
\numberwithin{equation}{section}
\numberwithin{figure}{section}

\usepackage{graphicx}
\graphicspath{{Figures/}}
\usepackage{subfig}

\usepackage{tikz}
\usetikzlibrary{plotmarks,circuits.ee.IEC}

\usepackage{fancyhdr}
\pagestyle{fancy}
\fancyhf{}
\rhead{\thepage}
\renewcommand{\footrulewidth}{0pt}
\renewcommand{\headrulewidth}{0pt}

\usepackage{relsize}
\usepackage{xspace}
\usepackage{url}

\newcommand{\figref}[1]{Figuur~\ref{#1}}

\newcommand{\hisparc}{\textsmaller{HiSPARC}\xspace}
\newcommand{\kascade}{\textsmaller{KASCADE}\xspace}
\newcommand{\sapphire}{\textsmaller{SAPPHiRE}\xspace}
\newcommand{\jsparc}{\textsmaller{jSparc}\xspace}
\newcommand{\hdf}{\textsmaller{HDF5}\xspace}
\newcommand{\aires}{\textsmaller{AIRES}\xspace}
\newcommand{\csv}{\textsmaller{CSV}\xspace}
\newcommand{\python}{\textsmaller{PYTHON}\xspace}
\newcommand{\corsika}{\textsmaller{CORSIKA}\xspace}
\newcommand{\labview}{\textsmaller{LabVIEW}\xspace}
\newcommand{\dspmon}{\textsmaller{DSPMon}\xspace}
\newcommand{\daq}{\textsmaller{DAQ}\xspace}
\newcommand{\adc}{\textsmaller{ADC}\xspace}
\newcommand{\adcs}{\textsmaller{ADC}s\xspace}
\newcommand{\Adcs}{A\textsmaller{DC}s\xspace}
\newcommand{\hi}{\textsc{h i}\xspace}
\newcommand{\hii}{\textsc{h ii}\xspace}
\newcommand{\mip}{\textsmaller{MIP}\xspace}
\newcommand{\hisparcii}{\textsmaller{HiSPARC II}\xspace}
\newcommand{\hisparciii}{\textsmaller{HiSPARC III}\xspace}
\newcommand{\pmt}{\textsmaller{PMT}\xspace}
\newcommand{\pmts}{\textsmaller{PMT}s\xspace}
\newcommand{\gps}{\textsmaller{GPS}\xspace}

\DeclareSIUnit{\electronvolt}{\ensuremath{\mathrm{e\!\!\:V}}}

\DeclareSIUnit{\unitsigma}{\ensuremath{\sigma}}
\DeclareSIUnit{\mip}{\textsmaller{MIP}}
\DeclareSIUnit{\adc}{\textsmaller{ADC}}

\DeclareSIUnit{\gauss}{G}
\DeclareSIUnit{\parsec}{pc}
\DeclareSIUnit{\year}{yr}



\begin{document}

\title{Dagelijkse controle van een station} \author{C.G.N. van Veen}
\date{}

\maketitle

\section{Stations}

\hisparc heeft verschillende meetstations op scholen in heel Nederland
staan. De stations meten onophoudelijk en verzenden hun data naar de
\hisparc database. De status van de \hisparc stations wordt bijgehouden
door Nagios. Informatie over de status van een station is te vinden op
\url{http://data.hisparc.nl}. Dit document geeft aan hoe de status van
een station optimaal gemaakt kan worden.

\section{Status van een station controleren}

De school of instelling is verantwoordelijk voor het onderhoud van het
meetstation. Op de site: \url{http://data.hisparc.nl} kunnen de
prestaties van het station worden gevolgd. Van deze site is in figuur
\ref{fig:frontweb} een screenshot te zien is. Op de site staan alle
stations vermeld die in het netwerk van \hisparc zijn aangesloten. De
cirkels voor de stations geven met een kleurcode aan wat de status van
een station is. Als de kleur van de cirkel voor het betreffende station
groen is, dan is de status van het station `in orde'. Een gele cirkel
geeft aan dat er een probleem is met het station. Rood betekent dat de
pc niet `gepingt' kan worden, vaak staat de pc dan uit, heeft de pc geen
internet of VPN werkt niet.

\begin{figure}
    \centering \includegraphics[scale=0.60]{websitefront}
    \caption{Screenshot van de \hisparc data website.} 
    \label{fig:frontweb}
\end{figure}

\section{Probleem met een station}

In het geval van een gele of rode cirkel, kunt u klikken op de link van
het desbetreffende station (bijvoorbeeld klikken op:
\underline{501-Nikhef}) zoals te zien is in figuur \ref{fig:frontweb}.
Dan wordt u doorgelinkt naar de status pagina van het betreffende
station. Zo'n pagina met een overzicht van verzonden data van een
eerdere dagen van een station is weergeven in figuur
\ref{fig:websiteerror}. In figuur \ref{fig:websiteerror} is te zien dat
het station niet meer meet na 14.00 uur op de vorige dag,  omdat het
aantal gemeten events naar nul gaat. In het Pulseheight histogram is ook
te zien dat de \mip piek niet goed aanwezig is, wat duidt op een
verkeerde instelling van de \pmt. Zie ook \cite{inregelen}.

\begin{figure} 
    \centering 
    \includegraphics[scale=0.40]{websiteerror}
    \caption{Screenshot van het data-overzicht van een
             station. In het event histogram is duidelijk te zien dat
             het station niets meer heeft gemeten na 14.00 uur UTC tijd.
             Ook is er in het pulseheight histogram geen duidelijke MIP
             piek te zien.}
    \label{fig:websiteerror} 
\end{figure}

\section{Problemen oplossen}

Om precies te zien wat er mis is met het station en het probleem
daadwerkelijk op te lossen, klikt u op de site:
\url{http://data.hisparc.nl} op het gekleurde cirkeltje voor het
betreffende station. Als u op de status pagina zit kunt u op
\emph{status} klikken rechtsboven op de status pagina van het station.
In beide gevallen krijgt u een status en problemen overzicht, zie figuur
\ref{fig:statusstation}. In figuur \ref{fig:statusstation} is te zien
dat de \emph{TriggerRate} rood is en dus aandacht behoeft. Om de
problemen met de verschillende categorieën op te lossen is een speciale
website ontwikkeld. Ga naar \url{http://docs.hisparc.nl/maintenance/},
zie figuur \ref{fig:maintenance}. U kunt hier onder andere op
\emph{known issues} klikken en op \emph{frequently asked questions}. Op
de pagina \emph{known issues} bevindt zich een gecategoriseerde lijst
van problemen. U kunt daar in de tweede alinea op het betreffende
probleem klikken, voor het gegeven voorbeeld is dat \emph{TriggerRate}.
Op de site verschijnt dan een lijst met problemen die een foutmelding
\emph{TriggerRate} geven. In deze lijst is een oplossing te vinden voor
het probleem, zie figuur \ref{fig:solution}. Voor de foutmelding
\emph{TriggerRate} vinden we dan als mogelijke oplossing dat de button
\emph{DAQ Mode} moet worden gebruikt om van mode te wisselen. Bij andere
problemen met stations kunt u het bovenstaande stappenplan op dezelfde
manier volgen om diverse storingen en problemen met u station op te
lossen. Komt u er alsnog niet uit neem dan contact op met de
clustercoordinator of met \url{beheer@hisparc.nl}.
 

\begin{figure} 
\centering 
\includegraphics[scale=0.60]{statusstation}
\caption{Screenshot van de status van een station. In de tabel is in rood aangegeven 
welke foutmeldingen er voor het station zijn waargenomen. 
In dit geval is er een probleem met de \emph{TriggerRate}.}\label{fig:statusstation} 
\end{figure}

\begin{figure} 
\centering 
\includegraphics[scale=0.30]{maintenance}
\caption{Klik op \emph{frequently asked questions} en daarna op
         \emph{known issues} om naar verschillende foutmeldingen en
         oplossingen daarvan te gaan.}\label{fig:maintenance} 
\end{figure}
 
\begin{figure} 
\centering 
\includegraphics[scale=0.60]{solution}
\caption{Een mogelijke reden voor foutmelding en de bijhorende
         oplossing. Hier is de oplossing, om in het programma \hisparc
         DAQ op de button \emph{DAQ mode} te klikken.} 
\label{fig:solution} 
\end{figure}


\begin{thebibliography}{9} 	
	\bibitem{inregelen} D.B.R.A. Fokkema, \emph{Inregelen PMT's}, infopakket \hisparc 
\end{thebibliography}



\end{document}
