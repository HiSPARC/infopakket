\documentclass[oneside, 11pt]{article}

\usepackage[T1]{fontenc}
\usepackage[utf8]{inputenc}
\usepackage[dutch]{babel}

\usepackage[font={small,sf},labelfont={bf},labelsep=endash]{caption}
\usepackage{fouriernc}
\usepackage[detect-all, load-configurations=binary,
            separate-uncertainty=true, per-mode=symbol,
            retain-explicit-plus, range-phrase={ tot }]{siunitx}

\usepackage{setspace}
\setstretch{1.2}

\setlength{\parskip}{\smallskipamount}
\setlength{\parindent}{0pt}

\usepackage{geometry}
\geometry{marginparwidth=0.5cm, verbose, a4paper,
          tmargin=3cm, bmargin=3cm, lmargin=2cm, rmargin=2cm}

\usepackage{float}

\usepackage[fleqn]{amsmath}
\numberwithin{equation}{section}
\numberwithin{figure}{section}

\usepackage{graphicx}
\graphicspath{{Figures/}}
\usepackage{subfig}

\usepackage{tikz}
\usetikzlibrary{plotmarks,circuits.ee.IEC}

\usepackage{fancyhdr}
\pagestyle{fancy}
\fancyhf{}
\rhead{\thepage}
\renewcommand{\footrulewidth}{0pt}
\renewcommand{\headrulewidth}{0pt}

\usepackage{relsize}
\usepackage{xspace}
\usepackage{url}

\newcommand{\figref}[1]{Figuur~\ref{#1}}

\newcommand{\hisparc}{\textsmaller{HiSPARC}\xspace}
\newcommand{\kascade}{\textsmaller{KASCADE}\xspace}
\newcommand{\sapphire}{\textsmaller{SAPPHiRE}\xspace}
\newcommand{\jsparc}{\textsmaller{jSparc}\xspace}
\newcommand{\hdf}{\textsmaller{HDF5}\xspace}
\newcommand{\aires}{\textsmaller{AIRES}\xspace}
\newcommand{\csv}{\textsmaller{CSV}\xspace}
\newcommand{\python}{\textsmaller{PYTHON}\xspace}
\newcommand{\corsika}{\textsmaller{CORSIKA}\xspace}
\newcommand{\labview}{\textsmaller{LabVIEW}\xspace}
\newcommand{\dspmon}{\textsmaller{DSPMon}\xspace}
\newcommand{\daq}{\textsmaller{DAQ}\xspace}
\newcommand{\adc}{\textsmaller{ADC}\xspace}
\newcommand{\adcs}{\textsmaller{ADC}s\xspace}
\newcommand{\Adcs}{A\textsmaller{DC}s\xspace}
\newcommand{\hi}{\textsc{h i}\xspace}
\newcommand{\hii}{\textsc{h ii}\xspace}
\newcommand{\mip}{\textsmaller{MIP}\xspace}
\newcommand{\hisparcii}{\textsmaller{HiSPARC II}\xspace}
\newcommand{\hisparciii}{\textsmaller{HiSPARC III}\xspace}
\newcommand{\pmt}{\textsmaller{PMT}\xspace}
\newcommand{\pmts}{\textsmaller{PMT}s\xspace}
\newcommand{\gps}{\textsmaller{GPS}\xspace}

\DeclareSIUnit{\electronvolt}{\ensuremath{\mathrm{e\!\!\:V}}}

\DeclareSIUnit{\unitsigma}{\ensuremath{\sigma}}
\DeclareSIUnit{\mip}{\textsmaller{MIP}}
\DeclareSIUnit{\adc}{\textsmaller{ADC}}

\DeclareSIUnit{\gauss}{G}
\DeclareSIUnit{\parsec}{pc}
\DeclareSIUnit{\year}{yr}



\title{Wireless connectie PC en Arduino}
\author{C.G.N. van Veen}
\docweerstation{2}{WW}
\version{1.3}

\begin{document}

\maketitle

\section{Weerstation}

\paragraph{Inleiding} Ons weerstation werkt en geeft ons de data die we
willen van het weer. Het is echter nog wel afhankelijk van een fysieke
verbinding met de pc. Omdat we in sommige gevallen op afstand willen
meten, kan het handig zijn om de data van het weerstation wireless naar
de PC te versturen. Dit heeft een aantal voordelen. We hoeven geen kabel
tussen PC en Arduino te trekken en we kunnen over grote afstand een nieuw
programma zenden naar de Arduino. Het nadeel dat we wel constante
voeding nodig hebben bij het station (als er geen voeding beschikbaar), 
zouden we kunnen ondervangen met een zonnecel en backup batterij. 

\paragraph{Benodigdheden}
Om het geheel wireless te laten werken hebben we een aantal zaken nodig.
Het belangrijkste is de zend/ontvangmodule (APC220) module. Dit is een 
zend/ontvangmodule die makkelijk in te stellen is en makkelijk via een ander 
programma uit te lezen is.

\begin{itemize}  
    \item APC220 zendmodules (2x)    
    \item Arduino software
    \item USB to UART adapter
    \item APC programma
\end{itemize}

\begin{figure}
    \centering
    \subfloat[]{
        \includegraphics[width=.5\linewidth]{apc_receiver}
        \label{fig:apc_receiver}}
    \hfill
    \subfloat[]{
        \includegraphics[width=.4\linewidth]{USB_adapter}
        \label{fig:USB_adapter}}
    \caption{Een APC zend of ontvanger module (a) en zijn USB adapter
    (b), in het rode kader het nummer van de chip (CP210), check of dit
    nummer in de driver beschrijving staat bij het downloaden van de
    driver van SILABS.}
\end{figure}

\section{APC module installeren}

\subsection{Zenden en ontvangen} Telemetrie is een technologie, die
ervoor zorgt dat data verzonden en ontvangen wordt. De data wordt
verzonden via radiogolven en door ontvangers opgevangen, bewaard en/of
bewerkt. Ons weerstation moet de data van de sensoren doorsturen naar de
\hisparc computer op school, zodat deze de data kan versturen. De
APC-220 module (zie \figref{fig:apc_receiver}), die we gebruiken is een
radio zender of ontvanger met een laag vermogen. Een van de modules
gebruiken we als ontvanger, die wordt dan ook aangesloten op de pc. De
andere wordt aangesloten op de Arduino en zal als zender fungeren. Het
bereik van de zender is 1 km in open lucht. De zender kan door muren heen
zenden, maar het verdient aanbeveling om het aantal muren tussen het
meetstation en de computer beperkt te houden. De zender kan zenden
tussen \SIrange{430}{440}{\MHz}. De default waarde is echter
\SI{434}{\MHz}. In tabel zie je mogelijke waarden die we gebruiken, de APC220 
kan zenden tussen \SI{418}{\MHz} en \SI{454}{\MHz}.
In Nederland mag er echter alleen tussen deze frequentiegrenzen door
zendamateurs informatie verzonden worden.


\subsection{APC-220 installeren}
De ontvanger zal gekoppeld moeten worden aan de computer. De computer heeft 
een driver nodig om te werken. We hebben een VCP (Virtual Com Port) driver nodig, 
die we kunnen downloaden van Silabs.com:
\url{http://www.silabs.com/products/mcu/Pages/USBtoUARTBridgeVCPDrivers.aspx.}
Download de juiste driver voor je besturingssysteem op de gebruikte computer
en installeer deze. Zorg tijdens de installatie van de driver dat je de 
ontvangmodule nog niet in je pc hebt aangesloten.
Na installatie sluit je de APC module \figref{fig:APC_complete} aan op een 
USB ingang van de PC. Als Windows hem dan niet vind, verwijder de module en 
stop hem er opnieuw in.

Om de APC module te configureren gebruiken we de Arduino en een
programma `apc220cfg.ino' dat we downloaden van de site van NAROM of van
de universiteit van Aalborg \footnote{NAROM is een
raketbasis/organisatie in Noorwegen die participeert in CANSAT (een
Europees project van de ESA). Zij hebben samen met de Aalborg
universiteit op basis van Arduino een pakket ontwikkeld waarmee een
CANSAT satelliet gemaakt kan worden. De CANSAT zendt draadloos data via
de APC-220 module. Andere tools om de APC-module te configureren bestaan
wel, maar geven veel problemen. Een voorbeeld van zo'n problematisch
programma is "RF-Magic". Gebruik Google met zoekterm `apc220cfg.ino'} of van
Github/\hisparc. 

 
\begin{figure}
    \centering
    \includegraphics[width=0.60\textwidth]{APC_complete}
    \caption{De APC module op de aangesloten op de USB adapter.}
   \label{fig:APC_complete}
\end{figure}

\begin{figure}
    \centering
    \includegraphics[width=0.5\textwidth]{APC220_configurat1}
    \caption{De APC module aangesloten op de Arduino om te configureren.
    Sluit de module aan op pin 8-13 en ground van de Arduino.}
   \label{fig:APC220_configurat1}
\end{figure}

\begin{figure}
    \centering
    \includegraphics[width=0.60\textwidth]{scrshotConfigAPC}
    
    \caption{Screenshot van de "serial monitor", type 'm' in de invoerbalk
    van de serial monitor. Na het typen van 'm' wordt dit menu getoond.
    Type `r' om de instellingen van de APC module uit te lezen. Type `w'
    om de APC module configureren. Bepaal welke frequentie je wilt
    gebruiken voor transmissie en ontvangst uit tabel \ref{table:frequenties}.
    Voorbeeld: w 434150 3 9 3 0 (denk aan de spaties) betekent: write
    frequentie \SI{434150}{\MHz} met 9600 bps in lucht, met een zendvermogen van
    \SI{20}{\milli\watt} ontvangen van data met 9600 bps over UART zonder parity
    check.}
   
   \label{fig:scrshotConfigAPC}
\end{figure}

\begin{figure}
    \centering
    \includegraphics[width=0.80 \textwidth]{APCTXRX}
    \caption{De APC module aangesloten op de Arduino om te zenden.}
   \label{fig:APCTXRX}
\end{figure}

\subsection{Configureren van de APC-220}

Start nu Arduino software op en open het programma `apc220cfg.ino'. We
willen nu de frequentie gaan instellen waarmee de module zendt en
ontvangt. Upload het programma naar de Arduino voordat je de APC module
aansluit op de Arduino. Maak nu de USB kabel los en verwijder eventueel
de voeding van de Arduino. Plaats nu de APC module op de Arduino zoals
is aangegeven in \figref{fig:APC220_configurat1}. De APC moet zo
geplaatst worden dat de pinnen van de APC in pin 8-13 en ground van de
Arduino zitten. Sluit de Arduino nu opnieuw aan met behulp van de USB
kabel en open de `Serial Monitor' van Arduino. Type in de invoerbalk van
de "Serial Monitor", `m'. Dan opent het menu wat in
\figref{fig:scrshotConfigAPC} te zien is. Met dit menu kun je de APC-module
configureren. Het invoeren van de letter `r' geeft de huidige
configuratie van de APC module weer. Bepaal met behulp van tabel
\ref{table:frequenties} op welke frequentie je beide modules wilt laten
werken. Beide modules moeten op dezelfde frequentie werken. Voer dan je
keuze in. Bijvoorbeeld de invoer: w 434790 3 9 3 0 staat voor: write
frequentie \SI{434790}{\MHz} met 9600 bps in lucht en
\SI{20}{\milli\watt} 9600 bps wegschrijven over UART zonder parity check.
Herhaal deze stap voor de andere APC module. Belangrijk is dat de Baudrate op 
9600 staat en paritycheck `none' is.

\begin{center}
\begin{table}
    \begin{tabular}{ | l | l | }
    \hline
    Frequentie (MHz) & Frequentie (MHz)  \\ \hline
    433050 & 433950     \\ \hline
    433100 & 434000     \\ \hline
    433150 & 434050     \\ \hline
    433200 & 434100     \\ \hline
    433250 & 434150     \\ \hline
    433300 & 434200     \\ \hline
    433350 & 434250     \\ \hline
    433400 & 434300     \\ \hline
    433450 & 434350     \\ \hline
    433500 & 434400     \\ \hline
    433550 & 434450     \\ \hline
    433600 & 434500     \\ \hline
    433650 & 434550     \\ \hline
    433700 & 434600     \\ \hline
    433750 & 434650     \\ \hline
    433800 & 434700     \\ \hline
    433850 & 434750     \\ \hline
    433900 & 434790     \\ \hline
   \end{tabular}
   \caption{tabel met frequenties: Volgens \cite{Radio} mag in Nederland
   dit bereik van radiofrequenties gebruikt worden voor telemetrie op
   een vermogen van \SI{10}{\milli\watt}}
   \label{table:frequenties}
\end{table}
\end{center}

\subsection{Zenden en ontvangen met de APC module}

We hebben nu de modules zo geconfigureerd dat ze in principe klaar zijn
om te zenden en ontvangen. Nu moeten we ons weerprogramma weer uploaden
naar de Arduino. Sluit de USB kabel aan op de Arduino en upload het
weerprogramma (Zorg dat er geen kabeltje in de Rx<-0 ingang van de Arduino zit, 
anders krijg je een foutmelding). Run het programma, kijk in de serial monitor van de
arduino software of er data verschijnt. Sluit de serial monitor. We gaan
nu wireless meten. Haal de usb kabel nu los.

Sluit een voeding aan op de Arduino.
We sluiten de APC-modules aan zoals in \figref{fig:APCTXRX} (let op de
verbindingen RX $\longleftrightarrow$ TX) en plaatsen de APC module met UART
in de USB ingang van de pc. De Arduino software zal de UART vinden en
geeft ook de poort aan, waar deze zit. Ga naar \emph {"Tools"},
\emph{"Serial Port"}. Daar vind je de poort waar de UART zit, zorg dat je deze checkt.
Het adres van deze poort is nodig om met een andere programmeertaal (in ons
geval: Python) de gegevens van het weerstation uit te lezen. 
Op de MAC en Windows computers gaat dat net even anders, vandaar dat de onderstaande
paragraven uitleg voor beide `operating systems' bevatten.

\subsubsection{Python voor MAC}
Een pakket om Python in te draaien is bijvoorbeeld: Canopy.
Dat is gratis te downloaden van: \url{https://www.enthought.com/products/canopy/}
Download de free version en installeer deze. Je hebt dan een venster waar je kunt
programmeren en een uitvoer venster. 

Uitleg over hoe je moet werken met Canopy vind je hier:
\url{http://docs.enthought.com/canopy/}. Waarschijnlijk moet je de package 'serial'
installeren. Ga daarvoor naar de terminal in Canopy, die vind je onder:
Tools - canopy terminal. Typ in de terminal: pip install pyserial. Je kunt nu de
package `pyserial' gebruiken. Typ in het runvenster van Canopy: `import serial'. Als het goed is 
krijg je nu geen foutmelding meer als je dit commando gerund hebt.

Een voorbeeld programma in Python om data van het weerstation (MAC) uit te lezen.
Als een regel voorafgegaan wordt door een \verb|#| dan is dat een commentaar 
regel en die wordt niet gecompiled en uitgevoerd.

\begin{minted}{python}
# code 1

# code om de data van de APC220 op de MAC uit te lezen
import sys, serial
import time
from time import sleep
      
def main():
strPort = '/dev/tty.SLAB_USBtoUART' # Adres van een APC-UART module voor MAC. 
ser = serial.Serial(strPort, baudrate=9600, parity=serial.PARITY_NONE, 
    bytesize=serial.EIGHTBITS, stopbits=serial.STOPBITS_ONE, timeout=1.0)
    
while True:
try:
    line = ser.readline()
    print line
    time.sleep(10)
 
except KeyboardInterrupt:
    print 'exiting'
    break
    
if __name__ == '__main__':
main()
    
\end{minted}

\subsubsection{Python voor Windows}

Als je de wireless connectie maakt op een windowscomputer dan kun je 
waarschijnlijk de COM poort waar de UART zit niet vinden met de behulp van de 
Arduino software. Download Canopy voor windows.
Die is gratis te downloaden van: \url{https://www.enthought.com/products/canopy/}
Download de free version en installeer deze. Je hebt dan een venster waar je kunt
programmeren en een uitvoer venster.

Uitleg over hoe je moet werken met Canopy vind je hier:
\url{http://docs.enthought.com/canopy/}.

Eerst willen we vinden waar de COM poort van de UART zit. Gebruik het onderstaande 
stukje code \footnote{bron: \url{http://asvresearch.com/main/index.php/tutorials/tutorial-5}}.

\begin{minted}{python}
# Code 2

# code om COM poort van de UART op windows te vinden.

import serial
 
def get_first_valid_port():
    # Windows
    found = False
    port_number = 0
    while not found and port_number < 256:
        try:
            s = serial.Serial(port_number)
            s.close()
            found = True
        except serial.SerialException:
            port_number += 1
    if found:
        return 'COM' + str(port_number + 1)
    return None
if __name__ == '__main__': 

    poort = get_first_valid_port()
    print poort
    
\end{minted}

Het stukje code om de informatie over de UART in te lezen (Windows):
De \verb|port_id| variabele heb je gevonden in het vorige stukje code (2).
Dat is adres van de UART aansluiting.

\begin{minted}{python}
# Code 3

# Code om de COM poort op de juiste manier te benaderen vul bij 
# port_id het adres in wat in code (2) gevonden is.

import serial 

port = serial.Serial(port_id, baudrate=9600, parity=serial.PARITY_NONE, 
       bytesize=serial.EIGHTBITS, stopbits=serial.STOPBITS_ONE, timeout=1.0) 

port.setRTS(0)

\end{minted}

Het stukje code om daadwerkelijk data uit te lezen. De output leest de
data in blokjes van 50 karakters en print deze als er data is. De
library \emph{msvcrt} moet je importeren met Python. \verb|kbhit()|
functie detecteert een keyboard input en de code stopt als op het keyboard de
letter `s' wordt getypt. \verb|getch()| functie haalt de waarde van de keyboard 
 op en vergelijkt met de opgegeven `s'.

\begin{minted}{python}
# Code 4

# Download data via APC-220  
 
while True:
    output = port.read(50)
    if output:
        print output
    if msvcrt.kbhit() and msvcrt.getch() == 's':
        break
\end{minted}

\subsection{Uitlezen en wegschrijven van de data}

We hebben nu een basisprogramma waarmee we de data van de Arduino kunnen
uitlezen. Een aantal problemen steken nu de kop op. Het tempo waarmee de
Arduino verzendt en de frequentie waarmee Python data uitleest zullen
verschillen of uiteindelijk uit de pas gaan lopen. Ook is het niet
interessant om meerdere keren dezelfde waarde van de temperatuur of
luchtdruk te hebben. Het is interessanter als we alleen verschillen met
de vorige waarden opslaan samen met een tijdstempel. Dit zullen we moeten opvangen met een stukje
programmeercode. De oplossing van dit probleem kun je zelf bedenken.
In deze paragraaf zullen we wel wat oplossingen aandragen.
In Python kun je de data die je uitleest wegschrijven als een tekstfile.
Hoe dat moet kun je hier lezen:
\url{http://www.pythonforbeginners.com/files/reading-and-writing-files-in-python}

Voordat je wegschrijft moet je zorgen dat de nieuwe data vergelijkt met de oude data, als deze exact hetzelfde is
als de vorige kun je in het programma aangeven dat je pas data wegschrijft als de data verschilt.
Ook komt het wel eens voor dat de string met data leeg is. Dat wil je ook voorkomen.
Daarnaast weet je dat de lengte van je data altijd hetzelfde moet zijn.

Gebruik hiervoor een `if' statement met je voorwaarden. Bijvoorbeeld de volgende code:
(code 4)

\begin{minted}{python}
# code 4
if nieuwe_data != ' ' and nieuwe_data !=vorige_data and len(nieuwe_data)>32:
    file = open(".\Documents\Arduino\Metingen\Metingen.txt", "a") #"a
            file.write(output + "  {:%d, %b %Y} | ".format(datetime.date.today())
                         + (time.strftime("%H:%M:%S")) + "\n")
            file.close()
            vorige_data = nieuwe_data
\end{minted}

In de bovenstaande code staat in het if statement een controle op 3
punten. Namelijk of de \verb|nieuwe_data| niet gelijk is aan de oude data, of
\verb|nieuwe_data| niet leeg is en het statement kijkt of de lengte van de data
gelijk is aan het minimale aantal characters wat je nodig had voor
bijvoorbeeld 3x een temperatuurmeting, luchtdruk en luchtvochtigheid. In ons geval is 
het minimale karakters in de string gelijk aan 32. Als we data naar de \hisparc database gaan
sturen zal het aantal karakters minder zijn. 
De file wordt elke keer geopend en gesloten als er data wordt weggeschreven, \verb|"a"|
zorgt ervoor dat er data toegevoegd wordt (\textbf{a}ppend) aan de textfile.

\section{Wireless weerstation plaatsen}

\subsection{Behuizing en voeding Arduino weerstation} 
Het weerstation moet kunnen
meten aan het weer, dus de behuizing moet voldoen aan de eigenschappen
van een weerhuisje, zoals waterdicht zijn, wind doorlatend en niet in de
volle zon zijn opgesteld. Daarbij komt dat de Arduino spanning nodig heeft. Als
er buiten in de buurt van het weerstation een voeding is, dan is dit
probleem zo opgelost. Anders kan een oplossing met een zonnecel
en een oplaadbare batterij uitkomst bieden. Bijvoorbeeld de `Solar
MintyBoost', `LiPo USB Charger Hookup' \footnote{Deze producten kunnen
gevonden door de naam te 'googlen.} of zelf een combinatie maken van een
zonnecel en een oplaadbare batterij. Er zijn voor Arduino, wat
printplaatjes te koop die een zonnecel combineren met een oplaadbare
batterij zoals de `USB LiPoly Charge'. Onze voorkeur gaat echter uit
naar een kant en klare zonnecel met batterij met een USB uitgang, zodat we de Arduino
via USB kunnen voeden. Let er op dat de zonnecel waterbestendig is, als dit niet zo is maak dan een
constructie waardoor deze afgeschermd is van de weersinvloeden.
De Arduino heeft bij (5V) een stroomsterkte van ongeveer \SI{40}{\milli\ampere} nodig per 
aangesloten pin, daarnaast \SI{50}{\milli\ampere} voor de \SI{3,3}{\volt} uitgang.
De zender (APC220) heeft \SI{40}{\milli\ampere} nodig.
Als we uitgaan van metingen in de winter dan kunnen we op zonnecel ongeveer
8 uur licht verwachten. Dat betekent dat onze Arduino 24 uur moeten kunnen laten werken
op een batterij die in korte tijd is opgeladen.
Een snelle rekensom leert ons dat we dan een batterij met een capaciteit van rond
de \SI{2000}{\milli\ampere\hour} nodig hebben. Een hogere capaciteit is wel aan te bevelen.

In het volgende document gaan we daadwerkelijk data naar de database van \hisparc schrijven. 

\begin{thebibliography}{9}
    \bibitem{CANSAT}
        NAROM, 2013 \emph{The CANSAT BOOK 2013-edition}, 
        CANSAT project
    \bibitem{Radio}
    Agentschap Telecom, 2012 \emph{Vergunningsvrije radiotoepassingen}, 
    Ministerie van Economische zaken, Landbouw en Innovatie
\end{thebibliography}


\end{document}
