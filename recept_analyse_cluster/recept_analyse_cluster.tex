\documentclass[oneside, 11pt]{article}

\usepackage[T1]{fontenc}
\usepackage[utf8]{inputenc}
\usepackage[dutch]{babel}

\usepackage{fouriernc}
\usepackage[detect-all, load-configurations=binary,
            separate-uncertainty=true, per-mode=symbol,
            retain-explicit-plus, range-phrase={ tot }]{siunitx}

\usepackage{setspace}
\setstretch{1.2}

\setlength{\parskip}{\smallskipamount}
\setlength{\parindent}{0pt}

\usepackage{geometry}
\geometry{marginparwidth=0.5cm, verbose, a4paper, tmargin=3cm, bmargin=3cm, lmargin=2cm, rmargin=2cm}

\usepackage{float}

\usepackage[fleqn]{amsmath}
\numberwithin{equation}{section}
\numberwithin{figure}{section}

\usepackage{graphicx}
\graphicspath{{Figures/}}
\usepackage{subfig}

\usepackage{tikz}
\usetikzlibrary{plotmarks}

\usepackage{fancyhdr}
\pagestyle{fancy}
\fancyhf{}
\rhead{\thepage}
\renewcommand{\footrulewidth}{0pt}
\renewcommand{\headrulewidth}{0pt}

\usepackage{relsize}
\usepackage{xspace}
\usepackage{url}

\newcommand{\figref}[1]{Figuur~\ref{#1}}

\newcommand{\hisparc}{\textsmaller{HiSPARC}\xspace}
\newcommand{\kascade}{\textsmaller{KASCADE}\xspace}
\newcommand{\sapphire}{\textsmaller{SAPPHiRE}\xspace}
\newcommand{\jsparc}{\textsmaller{jSparc}\xspace}
\newcommand{\hdf}{\textsmaller{HDF5}\xspace}
\newcommand{\aires}{\textsmaller{AIRES}\xspace}
\newcommand{\csv}{\textsmaller{CSV}\xspace}
\newcommand{\python}{\textsmaller{PYTHON}\xspace}
\newcommand{\corsika}{\textsmaller{CORSIKA}\xspace}
\newcommand{\labview}{\textsmaller{LabVIEW}\xspace}
\newcommand{\daq}{\textsmaller{DAQ}\xspace}
\newcommand{\adc}{\textsmaller{ADC}\xspace}
\newcommand{\adcs}{\textsmaller{ADC}s\xspace}
\newcommand{\Adcs}{A\textsmaller{DC}s\xspace}
\newcommand{\hi}{\textsc{h i}\xspace}
\newcommand{\hii}{\textsc{h ii}\xspace}
\newcommand{\mip}{\textsmaller{MIP}\xspace}
\newcommand{\hisparcii}{\textsmaller{HiSPARC II}\xspace}
\newcommand{\hisparciii}{\textsmaller{HiSPARC III}\xspace}
\newcommand{\pmt}{\textsmaller{PMT}\xspace}
\newcommand{\pmts}{\textsmaller{PMT}s\xspace}

\DeclareSIUnit{\electronvolt}{\ensuremath{\mathrm{e\!\!\:V}}}

\DeclareSIUnit{\unitsigma}{\ensuremath{\sigma}}
\DeclareSIUnit{\mip}{\textsmaller{MIP}}
\DeclareSIUnit{\adc}{\textsmaller{ADC}}

\DeclareSIUnit{\gauss}{G}
\DeclareSIUnit{\parsec}{pc}
\DeclareSIUnit{\year}{yr}



\title{Analyse van een cluster}
\author{N.G. Schultheiss}
\docrecept{2}{AC}
\version{1.0}

\begin{document}

\maketitle

\begin{tabular}{|>{\raggedright}p{3cm}|>{\raggedright}p{12cm}|}
\hline
leerjaar & niveau \tabularnewline
\hline
3 HAVO/VWO & beginnend \tabularnewline
\hline
\end{tabular}

\section{lesmateriaal}

\begin{tabular}{ |>{\raggedright}p{2.5cm}|>{\raggedright}p{8cm}|>{\raggedright}p{4cm}|}
\hline
lesnummer & lesbeschrijving & bron \tabularnewline
\hline
1 & 
Werkblad, hoofdstuk 1.6 uit Kosmische straling, Woudstra en de Gier 2013 & 
\url{http://www.hisparc.nl/docent-student/lesmateriaal/onderbouw/} \tabularnewline
\hline
2 & 
Interactief practicum met leerling computers, leerlingen werken individueel of met tweetallen. Voor dit practicum moet een sessie aangevraagd worden. &
\url{https://data.hisparc.nl/media/jsparc/jsparc.html} \tabularnewline
\hline
3 & 
Werkblad, Infopakket > opdracht 'Traces'. Een uitwerkbladen is ook in het infopakket beschikbaar. & 
\url{http://www.hisparc.nl/docent-student/lesmateriaal/informatie-pakket/} \tabularnewline
\hline
\end{tabular}

\section{Les 1}

De eerste les wordt gebruikt om enkele begrippen van HiSPARC uit te leggen:

\begin{itemize}
\item Wat is kosmische straling?
\item Hoe ontstaat een deeltjeslawine?
\item Hoe worden deeltjes in de lawine gedetecteerd?
\end{itemize}

\section{Les 2}

In de tweede les wordt een het practicum jSparc uitgevoerd.

\begin{itemize}
\item Leerlingen voeren het practicum uit aan de hand van een practicum-instructie.
\item De resultaten van de energiereconstructie is op een beamer aan de leerlingen te tonen.
\item Leerlingen bewaren een screenshot van een reconstructie, deze wordt in de derde les uitgewerkt.
\end{itemize}

Voordat dit practicum wordt een sessie aangevraagd op \url{http://www.hisparc.nl/hisparc-data/jsparc/sessie-aanvragen/}; Een sessie is een verzameling metingen die door op te geven stations in een op te geven periode zijn gedaan. Er worden co\"{i}incidenties tussen de stations gezocht. Kies stations die niet te ver uit elkaar liggen, deze hebben weinig of geen co\"{i}incidenties.De energie van een deeltjes-lawine is te bepalen met de verdeling van deeltjes op het aardoppervlak. De analyse gebeurd interactief door de leerling. Dit kan met een computer, een laptop of een tablet op \url{https://data.hisparc.nl/media/jsparc/jsparc.html}. Op \url{http://www.hisparc.nl/docent-student/lesmateriaal/informatie-pakket/} is een instructie voor de jSparc analyse te vinden. 

\section{Les 3}

In de derde les wordt het screenshot via een begeleidde instructie ge\"interpreteerd.
\begin{itemize}
\item Leerlingen voeren het practicum uit aan de hand van de practicum-instructie 'jSparc analyse'. Deze is te vinden op: \url{}
\item De resultaten van de energiereconstructie is op een beamer aan de leerlingen te tonen.
\item Leerlingen bewaren een screenshot van een reconstructie, deze wordt in de derde les uitgewerkt.
\end{itemize}

Het informatiepakket bevat een werkblad "Traces". In dit werkblad wordt de betekenis van de grafieken (traces) aan de hand van praktische opdrachten toegelicht. De opdrachten kunnen eventueel als huiswerk worden opgegeven, het is ook mogelijk om van de gehele lessenreeks een verslag te laten maken..

\section{Verdieping}

\begin{tabular}{ |>{\raggedright}p{2.5cm}|>{\raggedright}p{8cm}|>{\raggedright}p{4cm}|}
\hline
leerdoelen & beschrijving & bron \tabularnewline
\hline
Kosmische straling & 
Deze module is in lengte vergelijkbaar met een hoofdstuk in een leerboek. Onderwerpen uit de tweede klas worden herhaald binnen het kader van kosmische straling. Het practicum uit paragraaf 1.2.4 kan afzonderlijk bij het onderwerp 'Elektriciteit' worden behandeld. & 
\url{http://www.hisparc.nl/docent-student/lesmateriaal/onderbouw/} \tabularnewline
\hline
Bovenbouw & 
Op de Woudschoten conferentie in 2013 is een presentatie gegeven over de opbouw van deeltjes-lawines. Deze presentatie kan in de bovenbouw als eerste les van de reeks worden gebruikt. worden gebruikt. Hierna kunnen de leerlingen in een veel zelfstandiger vorm de laatste twee lessen volgen. &
\url{http://www.hisparc.nl/fileadmin/HiSPARC/presentaties/diversen/131214_Schultheiss_Woudschoten.pdf} \tabularnewline
\hline
\end{tabular}

\end{document}
