\documentclass[oneside, 11pt]{article}

\usepackage[T1]{fontenc}
\usepackage[utf8]{inputenc}
\usepackage[dutch]{babel}

\usepackage{fouriernc}
\usepackage[detect-all, load-configurations=binary,
            separate-uncertainty=true, per-mode=symbol,
            retain-explicit-plus, range-phrase={ tot }]{siunitx}

\usepackage{setspace}
\setstretch{1.2}

\setlength{\parskip}{\smallskipamount}
\setlength{\parindent}{0pt}

\usepackage{geometry}
\geometry{marginparwidth=0.5cm, verbose, a4paper, tmargin=3cm, bmargin=3cm, lmargin=2cm, rmargin=2cm}

\usepackage{float}

\usepackage[fleqn]{amsmath}
\numberwithin{equation}{section}
\numberwithin{figure}{section}

\usepackage{graphicx}
\graphicspath{{Figures/}}
\usepackage{subfig}

\usepackage{tikz}
\usetikzlibrary{plotmarks}

\usepackage{fancyhdr}
\pagestyle{fancy}
\fancyhf{}
\rhead{\thepage}
\renewcommand{\footrulewidth}{0pt}
\renewcommand{\headrulewidth}{0pt}

\usepackage{relsize}
\usepackage{xspace}
\usepackage{url}

\newcommand{\figref}[1]{Figuur~\ref{#1}}

\newcommand{\hisparc}{\textsmaller{HiSPARC}\xspace}
\newcommand{\kascade}{\textsmaller{KASCADE}\xspace}
\newcommand{\sapphire}{\textsmaller{SAPPHiRE}\xspace}
\newcommand{\jsparc}{\textsmaller{jSparc}\xspace}
\newcommand{\hdf}{\textsmaller{HDF5}\xspace}
\newcommand{\aires}{\textsmaller{AIRES}\xspace}
\newcommand{\csv}{\textsmaller{CSV}\xspace}
\newcommand{\python}{\textsmaller{PYTHON}\xspace}
\newcommand{\corsika}{\textsmaller{CORSIKA}\xspace}
\newcommand{\labview}{\textsmaller{LabVIEW}\xspace}
\newcommand{\daq}{\textsmaller{DAQ}\xspace}
\newcommand{\adc}{\textsmaller{ADC}\xspace}
\newcommand{\adcs}{\textsmaller{ADC}s\xspace}
\newcommand{\Adcs}{A\textsmaller{DC}s\xspace}
\newcommand{\hi}{\textsc{h i}\xspace}
\newcommand{\hii}{\textsc{h ii}\xspace}
\newcommand{\mip}{\textsmaller{MIP}\xspace}
\newcommand{\hisparcii}{\textsmaller{HiSPARC II}\xspace}
\newcommand{\hisparciii}{\textsmaller{HiSPARC III}\xspace}
\newcommand{\pmt}{\textsmaller{PMT}\xspace}
\newcommand{\pmts}{\textsmaller{PMT}s\xspace}

\DeclareSIUnit{\electronvolt}{\ensuremath{\mathrm{e\!\!\:V}}}

\DeclareSIUnit{\unitsigma}{\ensuremath{\sigma}}
\DeclareSIUnit{\mip}{\textsmaller{MIP}}
\DeclareSIUnit{\adc}{\textsmaller{ADC}}

\DeclareSIUnit{\gauss}{G}
\DeclareSIUnit{\parsec}{pc}
\DeclareSIUnit{\year}{yr}



\title{Analyse van een cluster}
\author{N.G. Schultheiss}
\docdetector{2}{RC}
\version{1.0}

\begin{document}

\maketitle

\section{Inleiding}

Dit recept is te gebruiken voor een reeks van drie lessen over de reconstructie van energie van kosmische deeltjes met HiSPARC. In deze lessen komen de volgende onderwerpen aan bod:

Les 1: De eerste les wordt gebruikt om enkele begrippen van HiSPARC uit te leggen:

\begin{itemize}
\item Wat is kosmische straling?
\item Hoe ontstaat een deeltjeslawine?
\item Hoe worden deeltjes in de lawine gedetecteerd?
\end{itemize}

Les 2: In de tweede les wordt een het practicum jSparc uitgevoerd.

\begin{itemize}
\item Leerlingen voeren het practicum uit aan de hand van een practicum-instructie.
\item De resultaten van de energiereconstructie is op een beamer aan de leerlingen te tonen.
\item Leerlingen bewaren een screenshot van een reconstructie, deze wordt in de derde les uitgewerkt.
\end{itemize}

Les 3: In de derde les wordt het screenshot via een begeleidde instructie ge\"interpreteerd.
\begin{itemize}
\item Leerlingen voeren het practicum uit aan de hand van een practicum-instructie.
\item De resultaten van de energiereconstructie is op een beamer aan de leerlingen te tonen.
\item Leerlingen bewaren een screenshot van een reconstructie, deze wordt in de derde les uitgewerkt.
\end{itemize}

Bronnen:

Het informatiepakket, routenet en de module Kosmische straling zijn te vinden op:

\url{http://www.hisparc.nl/docent-student/lesmateriaal/}

Voor de bovenbouw is een presentatie over de opbouw van deeltjes-lawines beschikbaar op \url{http://www.hisparc.nl/fileadmin/HiSPARC/presentaties/diversen/131214_Schultheiss_Woudschoten.pdf}. Dit is een ingekorte versie van de presentatie zoals deze in 2014 op de Woudschoten conferentie gegeven is.

\section{Les 1}

Hoofdstuk 1.6 van Kosmische Straling / Wouda en de Gier ($2^{\mathrm{de}}$ klas) kan worden gebruikt als inleiding voor kosmische straling. Indien er ruimte is kan de lessenreeks Kosmische Straling in de tweede klas worden gegeven. Het praktijkum 1.2.4 is eventueel als losse les aan het onderwerp Elektriciteit te koppelen.
Achtergronden voor deze les kunnen worden gevonden in het infopakket en het RouteNetHiSPARC:

\begin{itemize}
\item Infopakket: Het infopakket bestaat uit een aantal instructies waarin de werking van HiSPARC is uitgewerkt. In het infopakket zijn werkbladen te vinden die leerlingen kunnen gebruiken. Cosmic air showers
\item De resultaten van de energiereconstructie is op een beamer aan de leerlingen te tonen.
\end{itemize}

\section{Les 2}

Voordat dit practicum wordt een sessie aangevraagd op \url{http://www.hisparc.nl/hisparc-data/jsparc/sessie-aanvragen/}; Een sessie is een verzameling metingen die door op te geven stations in een op te geven periode zijn gedaan. Er worden co\"{i}incidenties tussen de stations gezocht. Kies stations die niet te ver uit elkaar liggen, deze hebben weinig of geen co\"{i}incidenties.De energie van een deeltjes-lawine is te bepalen met de verdeling van deeltjes op het aardoppervlak. De analyse gebeurd interactief door de leerling. Dit kan met een computer, een laptop of een tablet op \url{http://data.hisparc.nl/media/jsparc/jsparc.html}.

\begin{itemize}
\item Leerlingen gebruiken een instructie voor de bepaaling van de kern van 
\item De resultaten van de energiereconstructie is op een beamer aan de leerlingen te tonen.
\end{itemize}

Een jSparc lesinstructie is in het informatiepakket opgenomen onder het kopje Data-analyse.


\section{Les 3}
Het informatiepakket bevat een werkblad "Traces". In dit werkblad wordt de betekenis van de grafieken (traces) aan de hand van praktische opdrachten toegelicht. De opdrachten kunnen eventueel als huiswerk worden opgegeven, het is ook mogelijk om van de gehele lessenreeks een verslag te laten maken.. 

\end{document}
