\documentclass[oneside, 11pt]{article}

\usepackage[T1]{fontenc}
\usepackage[utf8]{inputenc}
\usepackage[dutch]{babel}

\usepackage{fouriernc}
\usepackage[detect-all, load-configurations=binary,
            separate-uncertainty=true, per-mode=symbol,
            retain-explicit-plus, range-phrase={ tot }]{siunitx}

\usepackage{setspace}
\setstretch{1.2}

\setlength{\parskip}{\smallskipamount}
\setlength{\parindent}{0pt}

\usepackage{geometry}
\geometry{marginparwidth=0.5cm, verbose, a4paper, tmargin=3cm, bmargin=3cm, lmargin=2cm, rmargin=2cm}

\usepackage{float}

\usepackage[fleqn]{amsmath}
\numberwithin{equation}{section}
\numberwithin{figure}{section}

\usepackage{graphicx}
\graphicspath{{Figures/}}
\usepackage{subfig}

\usepackage{tikz}
\usetikzlibrary{plotmarks}

\usepackage{fancyhdr}
\pagestyle{fancy}
\fancyhf{}
\rhead{\thepage}
\renewcommand{\footrulewidth}{0pt}
\renewcommand{\headrulewidth}{0pt}

\usepackage{relsize}
\usepackage{xspace}
\usepackage{url}

\newcommand{\figref}[1]{Figuur~\ref{#1}}

\newcommand{\hisparc}{\textsmaller{HiSPARC}\xspace}
\newcommand{\kascade}{\textsmaller{KASCADE}\xspace}
\newcommand{\sapphire}{\textsmaller{SAPPHiRE}\xspace}
\newcommand{\jsparc}{\textsmaller{jSparc}\xspace}
\newcommand{\hdf}{\textsmaller{HDF5}\xspace}
\newcommand{\aires}{\textsmaller{AIRES}\xspace}
\newcommand{\csv}{\textsmaller{CSV}\xspace}
\newcommand{\python}{\textsmaller{PYTHON}\xspace}
\newcommand{\corsika}{\textsmaller{CORSIKA}\xspace}
\newcommand{\labview}{\textsmaller{LabVIEW}\xspace}
\newcommand{\daq}{\textsmaller{DAQ}\xspace}
\newcommand{\adc}{\textsmaller{ADC}\xspace}
\newcommand{\adcs}{\textsmaller{ADC}s\xspace}
\newcommand{\Adcs}{A\textsmaller{DC}s\xspace}
\newcommand{\hi}{\textsc{h i}\xspace}
\newcommand{\hii}{\textsc{h ii}\xspace}
\newcommand{\mip}{\textsmaller{MIP}\xspace}
\newcommand{\hisparcii}{\textsmaller{HiSPARC II}\xspace}
\newcommand{\hisparciii}{\textsmaller{HiSPARC III}\xspace}
\newcommand{\pmt}{\textsmaller{PMT}\xspace}
\newcommand{\pmts}{\textsmaller{PMT}s\xspace}

\DeclareSIUnit{\electronvolt}{\ensuremath{\mathrm{e\!\!\:V}}}

\DeclareSIUnit{\unitsigma}{\ensuremath{\sigma}}
\DeclareSIUnit{\mip}{\textsmaller{MIP}}
\DeclareSIUnit{\adc}{\textsmaller{ADC}}

\DeclareSIUnit{\gauss}{G}
\DeclareSIUnit{\parsec}{pc}
\DeclareSIUnit{\year}{yr}


\newcolumntype{x}[1]{>{\centering\arraybackslash\hspace{0pt}}p{#1}}

\title{Notebooks voor python}
\author{N.G. Schultheiss}
\docwerkblad{9}{NB}
\version{1.1}

\begin{document}

\maketitle

\section{Inleiding}

De programmeertaal python wordt gebruikt voor het verwerken van HiSPARC-meetgegevens. Een aantal
standaarden voor gegevensverwerking is voor de gebruiker draait onzichtbaar op de achtergrond. Nieuwe
software voor interpretatie van meetgegevens kan direct in python geschreven worden. Documentatie is te
vinden op \url{http://docs.hisparc.nl/}.

Een aantal interactieve werkbladen is in de vorm van notebooks geschreven. Deze notebooks geven 
voorbeelden voor:
\begin{itemize}
\item Het ophalen van meetgegegevens.
\item Het ophalen van stationsgegevens.
\item Het verwerken van de opgehaalde gegevens
\end{itemize}
De notebooks maken gebruik van ipython. 

\section{Software-installatie}

Voordat notebooks kunnen worden gebruikt, installeren we ipython 2.7. Onder windows hebben we de keuze uit:
\begin{itemize}
\item pythonxy (\url{https://python-xy.github.io/downloads.html})
\item winpython (\url{https://github.com/winpython/winpython/wiki/Installation})
\end{itemize} 
De notebook module is te installeren door 'pip install notebook' in een command-scherm te draaien. Hierna
voeren we he commando 'ipython notebook' ingevoerd. De browser start met een scherm waarin de notebooks
te zien zijn.

\begin{thebibliography}{9}
    \bibitem{nikhefhisparc} Nikhef, \emph{\hisparc Elektronica},\\
    \url{https://www.nikhef.nl/wetenschap-techniek/technische-afdelingen/
	  elektronicatechnologie/projecten/}.

    \bibitem{wiki} Wikipedia, \\
    \url{http://www.wikipedia.org}.

    \bibitem{veen2013richting} C.G.N. van Veen, \emph{Richting
    Reconstructie} (2013).
\end{thebibliography}

\end{document}
