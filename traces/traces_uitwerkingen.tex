\documentclass[oneside, 11pt]{article}

\usepackage[T1]{fontenc}
\usepackage[utf8]{inputenc}
\usepackage[dutch]{babel}

\usepackage{fouriernc}
\usepackage[detect-all, load-configurations=binary,
            separate-uncertainty=true, per-mode=symbol,
            retain-explicit-plus, range-phrase={ tot }]{siunitx}

\usepackage{setspace}
\setstretch{1.2}

\setlength{\parskip}{\smallskipamount}
\setlength{\parindent}{0pt}

\usepackage{geometry}
\geometry{marginparwidth=0.5cm, verbose, a4paper, tmargin=3cm, bmargin=3cm, lmargin=2cm, rmargin=2cm}

\usepackage{float}

\usepackage[fleqn]{amsmath}
\numberwithin{equation}{section}
\numberwithin{figure}{section}

\usepackage{graphicx}
\graphicspath{{Figures/}}
\usepackage{subfig}

\usepackage{tikz}
\usetikzlibrary{plotmarks}

\usepackage{fancyhdr}
\pagestyle{fancy}
\fancyhf{}
\rhead{\thepage}
\renewcommand{\footrulewidth}{0pt}
\renewcommand{\headrulewidth}{0pt}

\usepackage{relsize}
\usepackage{xspace}
\usepackage{url}

\newcommand{\figref}[1]{Figuur~\ref{#1}}

\newcommand{\hisparc}{\textsmaller{HiSPARC}\xspace}
\newcommand{\kascade}{\textsmaller{KASCADE}\xspace}
\newcommand{\sapphire}{\textsmaller{SAPPHiRE}\xspace}
\newcommand{\jsparc}{\textsmaller{jSparc}\xspace}
\newcommand{\hdf}{\textsmaller{HDF5}\xspace}
\newcommand{\aires}{\textsmaller{AIRES}\xspace}
\newcommand{\csv}{\textsmaller{CSV}\xspace}
\newcommand{\python}{\textsmaller{PYTHON}\xspace}
\newcommand{\corsika}{\textsmaller{CORSIKA}\xspace}
\newcommand{\labview}{\textsmaller{LabVIEW}\xspace}
\newcommand{\daq}{\textsmaller{DAQ}\xspace}
\newcommand{\adc}{\textsmaller{ADC}\xspace}
\newcommand{\adcs}{\textsmaller{ADC}s\xspace}
\newcommand{\Adcs}{A\textsmaller{DC}s\xspace}
\newcommand{\hi}{\textsc{h i}\xspace}
\newcommand{\hii}{\textsc{h ii}\xspace}
\newcommand{\mip}{\textsmaller{MIP}\xspace}
\newcommand{\hisparcii}{\textsmaller{HiSPARC II}\xspace}
\newcommand{\hisparciii}{\textsmaller{HiSPARC III}\xspace}
\newcommand{\pmt}{\textsmaller{PMT}\xspace}
\newcommand{\pmts}{\textsmaller{PMT}s\xspace}

\DeclareSIUnit{\electronvolt}{\ensuremath{\mathrm{e\!\!\:V}}}

\DeclareSIUnit{\unitsigma}{\ensuremath{\sigma}}
\DeclareSIUnit{\mip}{\textsmaller{MIP}}
\DeclareSIUnit{\adc}{\textsmaller{ADC}}

\DeclareSIUnit{\gauss}{G}
\DeclareSIUnit{\parsec}{pc}
\DeclareSIUnit{\year}{yr}



\title{Pulshoogte en pulsintegraal}
\author{N.G. Schultheiss}
\docdocent{2}{UP}
\version{1.0}

\begin{document}

\maketitle

\section{Inleiding}

De pulshoogte en de pulsintegraal (het pulsoppervlak) histogrammen zijn
voor iedere dag op te vragen op: \url{https://data.hisparc.nl} door op de
stationsnaam te klikken. Rechtsboven beide histogrammen is een link
waarmee de gegevens in een spreadsheet, zoals Excel, te laden zijn. Deze
gegevens kunnen ook worden gebruikt voor eigen practica. De top in de
grafiek is een maat voor de spanning die wordt afgegeven als een enkel
deeltje door de detector gaat. Het aantal deeltjes dat de detector treft
is te bepalen door de gemeten waarde te delen door de topwaarde. Binnen
een foutenmarge is dus uit te rekenen hoeveel kosmische deeltjes er door
je schieten als je ligt te zonnebaden, hardlopen, etc. Globaal komen de
deeltjes op het oppervlak met een zenithhoek tussen \SI{0}{\degree} en
\SI{45}{\degree}.


\section{De pulsvorm}


\subsection{Pulsen ophalen uit de HiSPARC data opslag}

De afbeeldingen in deze module zijn afkomstig van \jsparc. Dit is een
interactieve practicumomgeving waarmee de energie van een kosmisch
deeltje dat de atmosfeer binnendringt kan worden bepaald. Op
\url{https://docs.hisparc.nl/jsparc/} wordt beschreven hoe
dit practicum werkt.


\subsection{Eenvoudige pulsvormen}

\begin{minipage}[t]{1\columnwidth}%

\paragraph{Opdracht 1:}

De groene grafiek van detector 3 in figuur \ref{fig:Eenvoudige-pulsen}
heeft een minder vloeiend verloop. Stel een hypothese op waarmee dit
minder vloeiende verloop kan worden verklaard.

In de groene grafiek zijn een aantal treden te zien. Deze zijn
bijvoorbeeld te verklaren als elk trede door een deeltje wordt
veroorzaakt. De treedjes zijn ook te verklaren met lichtstralen die via
verschillende wegen op de detector vallen. Het licht heeft ongeveer
\SI{3}{\nano\second} nodig om in vacuum \SI{1}{\meter} af te leggen.
Kunnen de treedjes nu verklaard worden? (Brekingsindex?)

\end{minipage}

\begin{minipage}[t]{1\columnwidth}%

\paragraph{Opdracht 2:}

Geef een verklaring voor het verloop van de grafiek van detector
1 in figuur 2.3.

In de zwarte grafiek zijn drie bobbels te herkennen: rond de
\SI{35}{\nano\second}, de \SI{95}{\nano\second} en de
\SI{210}{\nano\second}. Elke bobbel duidt op fluorescentie veroorzaakt
door 1 of meer deeltjes. (Dit kan ook op verval binnen de detector
wijzen! De levensduur van muonen is \SI{2.2}{\micro\second}, dus?)

\end{minipage}

\begin{minipage}[t]{1\columnwidth}

\paragraph{Opdracht 3:}

Bereken de afstand tussen de waargenomen deeltjes in de grafiek
van detector 1 in figuur 2.3.

$s=c*t\Rightarrow s=3.0*10^{8}*\left(95-35\right)*10^{-9}=18\mathrm{m}$

$s=c*t\Rightarrow s=3.0*10^{8}*\left(210-95\right)*10^{-9}=35\mathrm{m}$
(Eigenlijk 1 significant getal meer.)
\end{minipage}


\subsection{Ingewikkelde pulsvormen}

\begin{minipage}[t]{1\columnwidth}

\paragraph{Opdracht 4:}

Leg met de pulsintegraal (het pulsoppervlak) uit waarom er
waarschijnlijk evenveel deeltjes door detector 1 als door detector
2 zijn gegaan.

De puls van detector 2 kan ontstaan als twee deeltjes vlak na elkaar
komen. Als deze deeltjes tegelijk komen ontstaat puls 1.

\end{minipage}

\begin{minipage}[t]{1\columnwidth}

\paragraph{Opdracht 5:}

Verklaar waarom er binnen een station soms detectoren zijn
die een aantal deeltjes meten, terwijl ander detectoren (bijna) niets
meten.

Er worden slechts enkele deeltjes gemeten. Soms zullen er meer deeltjes
dan het gemiddelde worden gemeten. Soms minder. Het aantal deeltjes
dat gemeten wordt, is verdeeld volgens de Poisson distributie.

\end{minipage}
\end{document}
