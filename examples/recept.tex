\documentclass[oneside, 11pt]{article}

\usepackage[T1]{fontenc}
\usepackage[utf8]{inputenc}
\usepackage[dutch]{babel}

\usepackage{fouriernc}
\usepackage[detect-all, load-configurations=binary,
            separate-uncertainty=true, per-mode=symbol,
            retain-explicit-plus, range-phrase={ tot }]{siunitx}

\usepackage{setspace}
\setstretch{1.2}

\setlength{\parskip}{\smallskipamount}
\setlength{\parindent}{0pt}

\usepackage{geometry}
\geometry{marginparwidth=0.5cm, verbose, a4paper, tmargin=3cm, bmargin=3cm, lmargin=2cm, rmargin=2cm}

\usepackage{float}

\usepackage[fleqn]{amsmath}
\numberwithin{equation}{section}
\numberwithin{figure}{section}

\usepackage{graphicx}
\graphicspath{{Figures/}}
\usepackage{subfig}

\usepackage{tikz}
\usetikzlibrary{plotmarks}

\usepackage{fancyhdr}
\pagestyle{fancy}
\fancyhf{}
\rhead{\thepage}
\renewcommand{\footrulewidth}{0pt}
\renewcommand{\headrulewidth}{0pt}

\usepackage{relsize}
\usepackage{xspace}
\usepackage{url}

\newcommand{\figref}[1]{Figuur~\ref{#1}}

\newcommand{\hisparc}{\textsmaller{HiSPARC}\xspace}
\newcommand{\kascade}{\textsmaller{KASCADE}\xspace}
\newcommand{\sapphire}{\textsmaller{SAPPHiRE}\xspace}
\newcommand{\jsparc}{\textsmaller{jSparc}\xspace}
\newcommand{\hdf}{\textsmaller{HDF5}\xspace}
\newcommand{\aires}{\textsmaller{AIRES}\xspace}
\newcommand{\csv}{\textsmaller{CSV}\xspace}
\newcommand{\python}{\textsmaller{PYTHON}\xspace}
\newcommand{\corsika}{\textsmaller{CORSIKA}\xspace}
\newcommand{\labview}{\textsmaller{LabVIEW}\xspace}
\newcommand{\daq}{\textsmaller{DAQ}\xspace}
\newcommand{\adc}{\textsmaller{ADC}\xspace}
\newcommand{\adcs}{\textsmaller{ADC}s\xspace}
\newcommand{\Adcs}{A\textsmaller{DC}s\xspace}
\newcommand{\hi}{\textsc{h i}\xspace}
\newcommand{\hii}{\textsc{h ii}\xspace}
\newcommand{\mip}{\textsmaller{MIP}\xspace}
\newcommand{\hisparcii}{\textsmaller{HiSPARC II}\xspace}
\newcommand{\hisparciii}{\textsmaller{HiSPARC III}\xspace}
\newcommand{\pmt}{\textsmaller{PMT}\xspace}
\newcommand{\pmts}{\textsmaller{PMT}s\xspace}

\DeclareSIUnit{\electronvolt}{\ensuremath{\mathrm{e\!\!\:V}}}

\DeclareSIUnit{\unitsigma}{\ensuremath{\sigma}}
\DeclareSIUnit{\mip}{\textsmaller{MIP}}
\DeclareSIUnit{\adc}{\textsmaller{ADC}}

\DeclareSIUnit{\gauss}{G}
\DeclareSIUnit{\parsec}{pc}
\DeclareSIUnit{\year}{yr}



\title{Hier komt de titel}
\author{hier komt de naam van de auteur}
\docrecept{0}{CODE}
\version{1.1}

\begin{document}

\maketitle

\begin{tabular}{|>{\raggedright}p{2.5cm}|>{\raggedright}p{12.5cm}|}
\hline
leerjaar & niveau \tabularnewline
\hline
1 & beginnend \tabularnewline
\hline
2, 3 & gevorderd \tabularnewline
\hline
4, 5 & expert \tabularnewline
\hline
\end{tabular}

\section{lesmateriaal}

\begin{tabular}{ |>{\raggedright}p{2.5cm}|>{\raggedright}p{8cm}|>{\raggedright}p{4cm}|}
\hline
lesnummer & lesbeschrijving & bron \tabularnewline
\hline
1 & werkblad & \url{www.hisparc.nl/werkblad.doc} \tabularnewline
\hline
2 & interactief prc. & \url{www.hisparc.nl/practicum.htm} \tabularnewline
\hline
3 & opdracht & \url{op/een/adres} \tabularnewline
\hline
\end{tabular}

\section{Les 1}

Korte uitleg met leerdoelen en gebruikte lesmethoden voor de eerste les.

\section{Les 2}

Voor les 2

\section{Les 3}

Voor les 3

\section{Verdieping}

\begin{tabular}{ |>{\raggedright}p{2.5cm}|>{\raggedright}p{8cm}|>{\raggedright}p{4cm}|}
\hline
leerdoelen & beschrijving bron & bron \tabularnewline
\hline
HiSPARC en Excell & werkblad met databron & \url{www.hisparc.nl/excell.doc} \tabularnewline
\hline
HiSPARC elektronica & interactief lesmateriaal & \url{www.hisparc.nl/elektro.htm} \tabularnewline
\hline
\end{tabular}

\end{document}
