\documentclass[oneside, 11pt]{article}

\usepackage[T1]{fontenc}
\usepackage[utf8]{inputenc}
\usepackage[dutch]{babel}

\usepackage{fouriernc}
\usepackage[detect-all, load-configurations=binary,
            separate-uncertainty=true, per-mode=symbol,
            retain-explicit-plus, range-phrase={ tot }]{siunitx}

\usepackage{setspace}
\setstretch{1.2}

\setlength{\parskip}{\smallskipamount}
\setlength{\parindent}{0pt}

\usepackage{geometry}
\geometry{marginparwidth=0.5cm, verbose, a4paper, tmargin=3cm, bmargin=3cm, lmargin=2cm, rmargin=2cm}

\usepackage{float}

\usepackage[fleqn]{amsmath}
\numberwithin{equation}{section}
\numberwithin{figure}{section}

\usepackage{graphicx}
\graphicspath{{Figures/}}
\usepackage{subfig}

\usepackage{tikz}
\usetikzlibrary{plotmarks}

\usepackage{fancyhdr}
\pagestyle{fancy}
\fancyhf{}
\rhead{\thepage}
\renewcommand{\footrulewidth}{0pt}
\renewcommand{\headrulewidth}{0pt}

\usepackage{relsize}
\usepackage{xspace}
\usepackage{url}

\newcommand{\figref}[1]{Figuur~\ref{#1}}

\newcommand{\hisparc}{\textsmaller{HiSPARC}\xspace}
\newcommand{\kascade}{\textsmaller{KASCADE}\xspace}
\newcommand{\sapphire}{\textsmaller{SAPPHiRE}\xspace}
\newcommand{\jsparc}{\textsmaller{jSparc}\xspace}
\newcommand{\hdf}{\textsmaller{HDF5}\xspace}
\newcommand{\aires}{\textsmaller{AIRES}\xspace}
\newcommand{\csv}{\textsmaller{CSV}\xspace}
\newcommand{\python}{\textsmaller{PYTHON}\xspace}
\newcommand{\corsika}{\textsmaller{CORSIKA}\xspace}
\newcommand{\labview}{\textsmaller{LabVIEW}\xspace}
\newcommand{\daq}{\textsmaller{DAQ}\xspace}
\newcommand{\adc}{\textsmaller{ADC}\xspace}
\newcommand{\adcs}{\textsmaller{ADC}s\xspace}
\newcommand{\Adcs}{A\textsmaller{DC}s\xspace}
\newcommand{\hi}{\textsc{h i}\xspace}
\newcommand{\hii}{\textsc{h ii}\xspace}
\newcommand{\mip}{\textsmaller{MIP}\xspace}
\newcommand{\hisparcii}{\textsmaller{HiSPARC II}\xspace}
\newcommand{\hisparciii}{\textsmaller{HiSPARC III}\xspace}
\newcommand{\pmt}{\textsmaller{PMT}\xspace}
\newcommand{\pmts}{\textsmaller{PMT}s\xspace}

\DeclareSIUnit{\electronvolt}{\ensuremath{\mathrm{e\!\!\:V}}}

\DeclareSIUnit{\unitsigma}{\ensuremath{\sigma}}
\DeclareSIUnit{\mip}{\textsmaller{MIP}}
\DeclareSIUnit{\adc}{\textsmaller{ADC}}

\DeclareSIUnit{\gauss}{G}
\DeclareSIUnit{\parsec}{pc}
\DeclareSIUnit{\year}{yr}



\begin{document}

\title{De stationsplattegrond}
\author{A.P.L.S. de Laat, N.G. Schultheiss}

\maketitle

\section{Inleiding}

Een \hisparc station bestaat uit twee of vier detectoren die op het dak
van een (school)gebouw staan. Elk paar detectoren is aangesloten op een \hisparc II of \hisparc
III unit. De \hisparc units zijn in twee varianten beschikbaar, een
master en een slave. Een \hisparc station heeft altijd een \hisparc
master, op deze unit wordt ook een GPS antenne aangesloten. Sommige
stations hebben naast een master ook een slave, master en slave werken
samen als een eenheid. De master en eventuele slave van een station zijn
aangesloten op een meetcomputer, die de metingen naar de \hisparc server
stuurt. Met de computer zijn de aangesloten units ook af te regelen.

Met de GPS antenne is de locatie van een station te bepalen.
GPS-satellieten zenden radiosignalen uit, in dit signaal wordt de tijd
van de satelliet op het moment van uitzenden op 1 ns nauwkeurig meegestuurd.
Met het verschil tussen de aankomsttijd van het signaal en de lokale
tijd van de antenne zijn de afstanden uit rekenen. Deze tijd kunnen we
ook gebruiken om  vast te leggen wanneer de detectoren iets meten. 

Omdat de GPS antenne zowel de plaats als de tijd van eent station definieert,
wordt de plaats van de detectoren ten opzichte van de GPS antenne
vastgelegd.


\section{Kaarten}

De plaats van de GPS antenne wordt via het internet naar de \hisparc
server gestuurd. De configuratie van het station is in een Firefox,
Chrome of Safari browser op te halen op: 

\texttt{\small{http://data.hisparc.nl/api/station/}}{\small{\{Stationsnummer}}\texttt{\small{\}/config/}}{\small \par}

Op de plaats van \{Stationsnummer\} vul je het stationsnummer in. Let
op: Internet Explorer werkt niet! 

Met enig zoeken zijn de Oosterlengte (longitude) en Noorderbreedte
(latitude), beide in graden te vinden. De hoogte (altitude) wordt in
meter vanaf de WGS 84 ellipsoïde gemeten. Deze ellipsoïde ligt in
Amsterdam tientallen meters onder NAP (het Nieuw Amsterdams Peil).

\begin{minipage}[t]{1\columnwidth}%

\paragraph{Opdracht 1:}

Bepaal hoever de GPS antenne van het te meten station boven
        de WGS 84 ellipsoïde ligt en hoever dit boven NAP is.

\begin{center}
    \rule{\textwidth}{0.3mm}\\
    \rule{\textwidth}{0.3mm}\\
    \rule{\textwidth}{0.3mm}\\
\end{center}
\end{minipage}


\section{Het meten van de opstelling}


\subsection{Benodigdheden}

Om de opstelling te meten hebben we het volgende nodig:

\begin{itemize}
    \item Kompas
    \item Meetlint (liefst langer dan 10m)
    \item Veiligheidsmateriaal (harnas + zekeringslijn)
    \item Pen en papier om de metingen te noteren
    \item Sleutels van de `skiboxen'
\end{itemize}

\subsection{Veiligheid}

De verrichtingen op het dak moeten volgens de richtlijnen van de ARBO worden uitgevoerd.
Meestal staat de opstelling op een plat dak. Op dit dak zijn twee zones
te onderscheiden:

\begin{itemize}
    \item Een deel in het midden van het dak dat wordt begrensd door een
          markering. Daar mag je je zonder zonder
          veiligheidsmateriaal bevinden.
    \item Een deel tussen de markering en de rand. Daar moet je
          een veiligheidsharnas aan en moet je gezekerd zijn met een lijn aan
          bevestigingspunten op het dak.
\end{itemize}
Als er geen markering op het dak aanwezig is, dien je altijd op minstens 4 m afstand van de rand van het dak te blijven.

\textbf{Neem contact op met de gebouwenbeheerder op je school. Deze
dient op de hoogte te zijn van het practicum en heeft eventueel
aanvullende eisen!} Zonodig kan de gebouwbeheerder voorafgaand aan het practicum de markering aanbrengen. 
Verder is het aan te raden om met niet meer dan drie
leerlingen \'en onder begeleiding van een TOA of docent het dak op te gaan.
De leerlingen doen de meting, de TOA of docent draagt zorg voor een
veilige meetsituatie. Bij metingen nabij de rand van het dak zorgt een leerling voor
de zekeringslijn en een gezekerde leerling voor de metingen bij de rand. De derde leerling noteert de meetgegevens of verricht de metingen binnen de markeringen. De TOA of docent blijft dicht bij de leerling die zorgt
voor de zekeringslijn en let goed op. \textbf{Bespreek van te voren -en
dus niet op het dak- wie wat gaat doen!}


\subsection{De meting}

De detectoren liggen in skiboxen. Voordat de plaats van een detector
gemeten kan worden moet de skibox geopend worden. Met behulp van
diagonalen is het midden van de scintillator te bepalen. De scintillator
is de rechthoekige plaat van 100 cm lang en 50 cm breed. Deze plaat is
lichtdicht ingepakt. Let op dat de verpakking niet beschadigd wordt! We
meten nu de afstand van het midden van de scintillator tot de GPS
antenne (een plastic paddestoeltje). Deze is in tabel
\ref{tab:Meetgegevens} op te schrijven.

Naast de afstand moeten er nog twee hoeken worden gemeten. Hoek $\alpha$
is de hoek tussen het noorden en het meetlint van de GPS antenne naar de
detector. Staat de detector ten noorden van de GPS antenne dan meten we
een waarde van $0^{\mathrm{o}}$, staat de detector ten oosten van de
detector dan meten we $90^{\mathrm{o}}$. Hoek $\beta$ is de hoek tussen
het noorden en de lange zijde van de scintillator in de richting van de
PMT buis naar het einde zonder PMT buis. De PMT buis is te herkennen aan
de elektrische aansluitingen. Zit de PMT buis in het zuiden dan is deze
hoek $0\mathrm{^{o}}$. Deze metingen kunnen natuurlijk een aantal malen
worden uitgevoerd zodat toevallige fouten kunnen worden verkleind.

\begin{table}[h]
    \centering
    \begin{tabular}{|>{\centering}p{3.5cm}|>{\centering}p{3.5cm}|>{\centering}p{3.5cm}|>{\centering}p{3.5cm}|}
        \hline 
        Detector & Afstand {[}m{]} & $\alpha$ {[}$^{\mathrm{o}}${]} & $\beta$ {[}$^{\mathrm{o}}${]}\tabularnewline
        \hline 
        \hline 
        1 &  &  & \tabularnewline
        \hline 
        2 &  &  & \tabularnewline
        \hline 
        3 &  &  & \tabularnewline
        \hline 
        4 &  &  & \tabularnewline
        \hline 
    \end{tabular}

\caption{\label{tab:Meetgegevens}Meetgegevens}
\end{table}

\paragraph{Opdracht 2:}

Meet de locatie van de detectoren en vul de waarden in tabel
\ref{tab:Meetgegevens} in.

Detector 1 en 2 zijn aangesloten op de \hisparc master unit, deze unit is
te herkennen aan de GPS-aansluiting. Detector 3 en 4 zijn als PMT 1
respectievelijk PMT 2 aangesloten op de \hisparc slave (zonder
GPS-aansluiting).


\section{Doorgeven van meetgegevens}

Hier moet nog een formulier voor komen. Mail-adres opgeven. Code wordt
verzonden, hiermee wordt het formulier geopend. Naast de meetwaarden
wordt het mail-adres opgeslagen.

\end{document}
