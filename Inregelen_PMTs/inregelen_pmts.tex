\documentclass[oneside, 11pt]{article}

\usepackage[T1]{fontenc}
\usepackage[utf8]{inputenc}
\usepackage[dutch]{babel}

\usepackage[font={small,sf},labelfont={bf},labelsep=endash]{caption}
\usepackage{fouriernc}
\usepackage[detect-all, load-configurations=binary,
            separate-uncertainty=true, per-mode=symbol,
            retain-explicit-plus, range-phrase={ tot }]{siunitx}

\usepackage{setspace}
\setstretch{1.2}

\setlength{\parskip}{\smallskipamount}
\setlength{\parindent}{0pt}

\usepackage{geometry}
\geometry{marginparwidth=0.5cm, verbose, a4paper,
          tmargin=3cm, bmargin=3cm, lmargin=2cm, rmargin=2cm}

\usepackage{float}

\usepackage[fleqn]{amsmath}
\numberwithin{equation}{section}
\numberwithin{figure}{section}

\usepackage{graphicx}
\graphicspath{{Figures/}}
\usepackage{subfig}

\usepackage{tikz}
\usetikzlibrary{plotmarks,circuits.ee.IEC}

\usepackage{fancyhdr}
\pagestyle{fancy}
\fancyhf{}
\rhead{\thepage}
\renewcommand{\footrulewidth}{0pt}
\renewcommand{\headrulewidth}{0pt}

\usepackage{relsize}
\usepackage{xspace}
\usepackage{url}

\newcommand{\figref}[1]{Figuur~\ref{#1}}

\newcommand{\hisparc}{\textsmaller{HiSPARC}\xspace}
\newcommand{\kascade}{\textsmaller{KASCADE}\xspace}
\newcommand{\sapphire}{\textsmaller{SAPPHiRE}\xspace}
\newcommand{\jsparc}{\textsmaller{jSparc}\xspace}
\newcommand{\hdf}{\textsmaller{HDF5}\xspace}
\newcommand{\aires}{\textsmaller{AIRES}\xspace}
\newcommand{\csv}{\textsmaller{CSV}\xspace}
\newcommand{\python}{\textsmaller{PYTHON}\xspace}
\newcommand{\corsika}{\textsmaller{CORSIKA}\xspace}
\newcommand{\labview}{\textsmaller{LabVIEW}\xspace}
\newcommand{\dspmon}{\textsmaller{DSPMon}\xspace}
\newcommand{\daq}{\textsmaller{DAQ}\xspace}
\newcommand{\adc}{\textsmaller{ADC}\xspace}
\newcommand{\adcs}{\textsmaller{ADC}s\xspace}
\newcommand{\Adcs}{A\textsmaller{DC}s\xspace}
\newcommand{\hi}{\textsc{h i}\xspace}
\newcommand{\hii}{\textsc{h ii}\xspace}
\newcommand{\mip}{\textsmaller{MIP}\xspace}
\newcommand{\hisparcii}{\textsmaller{HiSPARC II}\xspace}
\newcommand{\hisparciii}{\textsmaller{HiSPARC III}\xspace}
\newcommand{\pmt}{\textsmaller{PMT}\xspace}
\newcommand{\pmts}{\textsmaller{PMT}s\xspace}
\newcommand{\gps}{\textsmaller{GPS}\xspace}

\DeclareSIUnit{\electronvolt}{\ensuremath{\mathrm{e\!\!\:V}}}

\DeclareSIUnit{\unitsigma}{\ensuremath{\sigma}}
\DeclareSIUnit{\mip}{\textsmaller{MIP}}
\DeclareSIUnit{\adc}{\textsmaller{ADC}}

\DeclareSIUnit{\gauss}{G}
\DeclareSIUnit{\parsec}{pc}
\DeclareSIUnit{\year}{yr}



\makeatletter
\pgfdeclareshape{openthumb}{
  \inheritsavedanchors[from=rectangle]
  \inheritanchorborder[from=rectangle]
  \inheritanchor[from=rectangle]{center}
  \inheritanchor[from=rectangle]{base}
  \inheritanchor[from=rectangle]{north}
  \inheritanchor[from=rectangle]{north east}
  \inheritanchor[from=rectangle]{east}
  \inheritanchor[from=rectangle]{south east}
  \inheritanchor[from=rectangle]{south}
  \inheritanchor[from=rectangle]{south west}
  \inheritanchor[from=rectangle]{west}
  \inheritanchor[from=rectangle]{north west}
  \backgroundpath{
    %  store lower right in xa/ya and upper right in xb/yb
    \southwest \pgf@xa=\pgf@x \pgf@ya=\pgf@y
    \northeast \pgf@xb=\pgf@x \pgf@yb=\pgf@y
    \pgfpathmoveto{\pgfpoint{\pgf@xa}{\pgf@ya}}
    \pgfpathlineto{\pgfpoint{\pgf@xb}{\pgf@ya}}
    \pgfpathmoveto{\pgfpoint{\pgf@xa}{\pgf@yb}}
    \pgfpathlineto{\pgfpoint{\pgf@xb}{\pgf@yb}}
 }
}
\makeatother

\newcommand{\thumb}{
\begin{tikzpicture}[remember picture,overlay,font=\sffamily]
  \node[fill,text=white,anchor=north east,yshift=-1cm,text width=3cm,align=right] at (current page.north east) {Theorie\rule{.8cm}{0pt}};

  \node[openthumb,draw,text=black,anchor=north east,yshift=-1.5cm-\thepage*.5cm,text width=2cm,align=right] at (current page.north east) {GC\rule{.8cm}{0pt}};
\end{tikzpicture}
}
\fancyhead{\thumb}

%\tikzexternalize

\begin{document}

\title{Inregelen \pmts}
\author{D. B. R. A. Fokkema}
\date{}

\maketitle
\thispagestyle{fancy}

\section{Werking van een \pmt}

Een fotoversterkbuis (photomultiplier tube, of \pmt) is een
elektronenbuis die in staat is om hele kleine lichtflitsjes om te zetten
in een elektrisch signaal.  Het is zelfs mogelijk om afzonderlijke fotonen
te tellen.  Geladen deeltjes uit kosmische straling die door de \hisparc
detectoren gaan verliezen energie in het materiaal van de detectoren.  In
de scintillator wordt dat energieverlies omgezet in een zwak
lichtschijnsel.  Dit paarsblauwe licht verspreidt zich door de detector
weerkaatst zo veel mogelijk aan de randen en een deel komt terecht bij de
\pmt, die het licht detecteert en een signaal afgeeft aan de \hisparc
elektronica.

Een \pmt maakt gebruik van het fotoelektrisch effect.  Dit effect werd
verklaard door Einstein en hiervoor ontving hij in 1921 de
Nobelprijs.\footnote{Veel mensen gaan er van uit dat Einstein de Nobelprijs
ontving voor de relativiteitstheorie, maar dit klopt niet.}  Wanneer licht
op een metaal schijnt, kunnen er elektronen worden losgeslagen uit het
oppervlak.  Dit gebeurt wanneer de energie per foton hoger is dan de
energie die een elektron nodig heeft om los te komen uit het metaal.  Deze
drempelenergie is afhankelijk van het soort metaal.  De energie per foton
wordt gegeven door
\begin{equation}
E_f = \frac{hc}{\lambda},
\end{equation}
met $E_f$ de energie van het foton, $h$ de constante van Planck, $c$ de
lichtsnelheid en $\lambda$ de golflengte van het licht.  Zie ook Tabel 7
en Tabel 35 in je Binas.

De energie van een foton is dus omgekeerd evenredig met de golflengte:
$E_f \propto 1/\lambda$.  Hoe kleiner de golflengte van het licht, hoe
groter de energie per foton.  Dit betekent dat hoe `blauwer' het licht,
hoe makkelijker een elektron wordt losgemaakt.  Is het licht \emph{te}
`rood', dan lukt dat nooit.  Daarom hebben we een scintillator gekozen die
een violet licht uitstraalt.

\begin{figure}
\centering
\includegraphics{pmt-schematic}
\caption{Schematische weergave van een \pmt.  Figuur overgenomen uit
\cite{hamamatsu}.}
\label{fig:pmt}
\end{figure}

De \pmt is een verzegelde glazen buis die vacuüm gemaakt
is (\figref{fig:pmt}).  De voorkant van een \pmt bestaat uit een dunne
glaslaag.  Aan de binnenkant van het glas is een zeer dun metaallaagje
opgedampt.  Het laagje is zó dun, dat het doorzichtig is.  Licht dat op de
\pmt valt, gaat door het glas en raakt het metaal.  Het violette licht uit
de scintillator heeft een $E_f$ die hoog genoeg is om elektronen vrij te
maken uit het metaallaagje.  Om er voor te zorgen dat er veel elektronen
beschikbaar zijn wordt het metaalplaatje op een grote negatieve spanning
gezet.  De elektronen staan dan feitelijk te dringen om het metaal te
verlaten.  Het metaallaagje heet de \emph{kathode}\footnote{In een
vacuümbuis is de kathode de pool waar elektronen uit worden vrijgemaakt,
zoals hier gebeurt.} van de \pmt.  De \emph{anode} van de fotobuis wordt
geaard, waardoor er een sterk elektrisch veld in de buis ontstaat.  De
elektronen versnellen richting de anode.  Om een grote versterkingsfactor
te krijgen is de buis opgedeeld in meerdere trappen.  Iedere trap heeft
een \emph{dynode}, een metalen plaatje met een iets minder negatieve
spanning dan de voorgaande dynode.  Dus bij een \pmt met drie dynodes,
zijn de spanningen bijvoorbeeld als volgt: kathode (\SI{-1000}{\volt}),
eerste dynode (\SI{-750}{\volt}), tweede dynode (\SI{-500}{\volt}), derde
dynode (\SI{-250}{\volt}) en kathode (\SI{0}{\volt}).  Zo blijven de
elektronen versnellen van kathode, langs alle dynodes en uiteindelijk naar
de anode.  De versterking treedt op zodra de elektronen een dynode raken.
De grote snelheid waarmee een elektron het metaal intreedt maakt een
aantal elektronen los.  Deze losgeslagen elektronen worden vervolgens
versneld naar de volgende dynode.  Als per dynode per elektron
bijvoorbeeld drie elektronen worden losgeslagen, dan is de totale
versterking in een \pmt met tien dynodes gelijk aan $3^{10} \approx
\num{60000}$.  De hoogspanning die over de \pmt staat bepaalt in grote
mate de versterkingsfactor.  Hoe hoger de spanning, hoe groter de
versterkingsfactor.  De hoogspanning bepaalt namelijk enerzijds de
versnelling van de elektronen en anderzijds het aantal elektronen dat
staat te dringen om de dynode te verlaten.

De \pmt die gebruikt wordt door \hisparc is een 9107B van Electron
Tubes \cite{9107B}.  Deze heeft 11 dynodes en een typische versterking van
\num{3e6} bij \SI{850}{\volt}.  Dat komt overeen met $\num{3.9}^{11}$.  Voor
ieder elektron dat een dynode raakt worden er gemiddeld bijna vier
elektronen vrijgemaakt.


\section{Signaal uit \hisparc detectoren}

Zodra een air shower een detector bereikt gaan één of meerdere deeltjes
door de detector.  Dit veroorzaakt een pulsvormig signaal.  De relatief
lange staart van het signaal wordt veroorzaakt door licht dat via een
aantal reflecties alsnog bij de \pmt terecht komt, maar ook door deeltjes
die een beetje achterliepen in de shower en vrij laat door de detector
gaan.  Het signaal uit de \pmt wordt via kabels met een lengte van
\SI{30}{\meter} naar de \hisparc elektronica geleid.  Alle kabels zijn
precies even lang, om te zorgen dat het signaal uit alle detectoren op
hetzelfde moment bij de elektronica aankomt.

In \figref{fig:traces} staat het signaal van een \hisparc event.  Het
bestaat uit een flinke puls met wat kleinere pieken in de staart.  Dit
signaal is veroorzaakt door meerdere deeltjes die (vrijwel) gelijktijdig
door de detector gingen.  De eerste grote puls is een optelsom van het
licht van meerdere deeltjes.

\begin{figure}
\centering
% \usepackage{tikz}
% \usetikzlibrary{arrows}
% \usepackage{pgfplots}
% \pgfplotsset{compat=1.3}
% \usepackage[detect-family]{siunitx}
% \usepackage[eulergreek]{sansmath}
% \sisetup{text-sf=\sansmath}
% \usepackage{relsize}
%
\pgfkeysifdefined{/artist/width}
    {\pgfkeysgetvalue{/artist/width}{\defaultwidth}}
    {\def\defaultwidth{ .5\linewidth }}
%
%
%\begin{sansmath}
\begin{tikzpicture}[
        %font=\sffamily,
        every pin/.style={inner sep=2pt},%, font={\sffamily\smaller}},
        every label/.style={inner sep=2pt},%, font={\sffamily\smaller}},
        every pin edge/.style={<-, >=stealth', shorten <=2pt},
        pin distance=2.5ex,
    ]
    \begin{axis}[
            width=\defaultwidth,
            %
            title={  },
            %
            xlabel={ Tijd [\si{\nano\second}] },
            ylabel={ Signaal [\si{\volt}] },
            %
            xmin={ 0 },
            xmax={ 200 },
            ymin={  },
            ymax={  },
            %
            xtick={  },
            ytick={  },
            %
            tick align=outside,
            max space between ticks=40,
            every tick/.style={},
        ]

        

        
            
            % Draw series plot
%            \addplot[no markers,black] coordinates {
%                (0, 0.00114)
%                (2, 0.00057)
%                (5, 0.00057)
%                (7, 0.00057)
%                (10, 0.00057)
%                (12, 0.00057)
%                (15, 0.00057)
%                (17, 0.00057)
%                (20, 0.00057)
%                (22, -0)
%                (25, -0)
%                (27, -0.00057)
%                (30, -0.04047)
%                (32, -0.14706)
%                (35, -0.34599)
%                (37, -0.63042)
%                (40, -0.8778)
%                (42, -1.07502)
%                (45, -1.18503)
%                (47, -1.21809)
%                (50, -1.18674)
%                (52, -1.11435)
%                (55, -1.03968)
%                (57, -0.94563)
%                (60, -0.88806)
%                (62, -0.85386)
%                (65, -0.80769)
%                (67, -0.74328)
%                (70, -0.68058)
%                (72, -0.62472)
%                (75, -0.58482)
%                (77, -0.51927)
%                (80, -0.44631)
%                (82, -0.37791)
%                (85, -0.33117)
%                (87, -0.29184)
%                (90, -0.27645)
%                (92, -0.25479)
%                (95, -0.23313)
%                (97, -0.20805)
%                (100, -0.19152)
%                (102, -0.17385)
%                (105, -0.15561)
%                (107, -0.13908)
%                (110, -0.12597)
%                (112, -0.11343)
%                (115, -0.10089)
%                (117, -0.09291)
%                (120, -0.0912)
%                (122, -0.09006)
%                (125, -0.08949)
%                (127, -0.07638)
%                (130, -0.07581)
%                (132, -0.07467)
%                (135, -0.07296)
%                (137, -0.05871)
%                (140, -0.05985)
%                (142, -0.06042)
%                (145, -0.05871)
%                (147, -0.04731)
%                (150, -0.04617)
%                (152, -0.04845)
%                (155, -0.04902)
%                (157, -0.04674)
%                (160, -0.0456)
%                (162, -0.04617)
%                (165, -0.04845)
%                (167, -0.04902)
%                (170, -0.05244)
%                (172, -0.05073)
%                (175, -0.04788)
%                (177, -0.0342)
%                (180, -0.03306)
%                (182, -0.03249)
%                (185, -0.03192)
%                (187, -0.02679)
%                (190, -0.03021)
%                (192, -0.04503)
%                (195, -0.03477)
%                (197, -0.03762)
%                (200, -0.03591)
%                (202, -0.03591)
%                (205, -0.03477)
%                (207, -0.02166)
%                (210, -0.02166)
%                (212, -0.02394)
%                (215, -0.02223)
%                (217, -0.02052)
%                (220, -0.01824)
%                (222, -0.01824)
%                (225, -0.01881)
%                (227, -0.01881)
%                (230, -0.01881)
%                (232, -0.0171)
%                (235, -0.01767)
%                (237, -0.01767)
%                (240, -0.01767)
%                (242, -0.01824)
%                (245, -0.01824)
%                (247, -0.01824)
%                (250, -0.01767)
%                (252, -0.0171)
%                (255, -0.01311)
%                (257, -0.01425)
%                (260, -0.01368)
%                (262, -0.01311)
%                (265, -0.01311)
%                (267, -0.01425)
%                (270, -0.01596)
%                (272, -0.02565)
%                (275, -0.02223)
%                (277, -0.02052)
%                (280, -0.01995)
%                (282, -0.01995)
%                (285, -0.01995)
%                (287, -0.01824)
%                (290, -0.0114)
%                (292, -0.01368)
%                (295, -0.01254)
%                (297, -0.01254)
%                (300, -0.01368)
%                (302, -0.01596)
%                (305, -0.01653)
%                (307, -0.01767)
%                (310, -0.01938)
%                (312, -0.01881)
%                (315, -0.01824)
%                (317, -0.01824)
%                (320, -0.01596)
%                (322, -0.01197)
%                (325, -0.01254)
%                (327, -0.01083)
%                (330, -0.0114)
%                (332, -0.01197)
%                (335, -0.0114)
%                (337, -0.0114)
%                (340, -0.0114)
%                (342, -0.01026)
%                (345, -0.00969)
%                (347, -0.01026)
%                (350, -0.00969)
%                (352, -0.00855)
%                (355, -0.00627)
%                (357, -0.00741)
%                (360, -0.00684)
%                (362, -0.00684)
%                (365, -0.00684)
%                (367, -0.00684)
%                (370, -0.00741)
%                (372, -0.00798)
%                (375, -0.00855)
%                (377, -0.00912)
%                (380, -0.00912)
%                (382, -0.00912)
%                (385, -0.00855)
%                (387, -0.00855)
%                (390, -0.00855)
%                (392, -0.00798)
%                (395, -0.00855)
%                (397, -0.00855)
%                (400, -0.00798)
%                (402, -0.00798)
%                (405, -0.00285)
%                (407, -0.00342)
%                (410, -0.00456)
%                (412, -0.0057)
%                (415, -0.00627)
%                (417, -0.0057)
%                (420, -0.0057)
%                (422, -0.0057)
%                (425, -0.0057)
%                (427, -0.00456)
%                (430, -0.00399)
%                (432, -0.00342)
%                (435, -0.00228)
%                (437, -0.00228)
%                (440, -0.00342)
%                (442, -0.00399)
%                (445, -0.00456)
%                (447, -0.00456)
%                (450, -0.00456)
%                (452, -0.00456)
%                (455, -0.00399)
%                (457, -0.00399)
%                (460, -0.00285)
%                (462, -0.00114)
%                (465, -0.00114)
%                (467, -0.00114)
%                (470, -0)
%                (472, -0)
%                (475, 0.00114)
%                (477, 0.00171)
%                (480, 0.00114)
%                (482, 0.00114)
%                (485, -0.00114)
%                (487, -0.00228)
%                (490, -0.00627)
%                (492, -0.0057)
%                (495, -0.00456)
%                (497, -0.00456)
%                (500, -0.00342)
%                (502, 0.00057)
%                (505, 0.00171)
%                (507, 0.00285)
%                (510, 0.00228)
%                (512, 0.00228)
%                (515, 0.00228)
%                (517, 0.00228)
%                (520, 0.00228)
%                (522, 0.00228)
%                (525, 0.00228)
%                (527, 0.00228)
%                (530, 0.00228)
%                (532, 0.00171)
%                (535, 0.00171)
%                (537, 0.00171)
%                (540, 0.00228)
%                (542, 0.00228)
%                (545, 0.00228)
%                (547, 0.00228)
%                (550, 0.00228)
%                (552, 0.00228)
%                (555, 0.00228)
%                (557, 0.00228)
%                (560, 0.00228)
%                (562, 0.00228)
%                (565, 0.00228)
%                (567, 0.00228)
%                (570, 0.00171)
%                (572, -0)
%                (575, -0)
%                (577, -0.00171)
%                (580, -0.00171)
%                (582, -0.00228)
%                (585, -0.00171)
%                (587, 0.00171)
%                (590, 0.00171)
%                (592, 0.00057)
%                (595, 0.00057)
%                (597, 0.00057)
%                (600, 0.00057)
%                (602, 0.00057)
%                (605, 0.00057)
%                (607, 0.00114)
%                (610, 0.00171)
%                (612, 0.00114)
%                (615, 0.00114)
%                (617, 0.00114)
%                (620, 0.00114)
%                (622, 0.00114)
%                (625, 0.00057)
%                (627, 0.00057)
%                (630, 0.00057)
%                (632, -0.00057)
%                (635, -0.00057)
%                (637, -0.00057)
%                (640, -0.00114)
%                (642, -0.00114)
%                (645, -0.00114)
%                (647, -0.00114)
%                (650, -0.00057)
%                (652, -0)
%                (655, 0.00057)
%                (657, 0.00114)
%                (660, 0.00114)
%                (662, 0.00171)
%                (665, 0.00171)
%                (667, 0.00228)
%                (670, 0.00228)
%                (672, 0.00228)
%                (675, 0.00228)
%                (677, 0.00228)
%                (680, 0.00228)
%                (682, 0.00228)
%                (685, 0.00228)
%                (687, 0.00228)
%                (690, 0.00228)
%                (692, 0.00228)
%                (695, 0.00228)
%                (697, 0.00228)
%                (700, 0.00228)
%                (702, 0.00228)
%                (705, 0.00228)
%                (707, 0.00114)
%                (710, 0.00057)
%                (712, 0.00057)
%                (715, 0.00057)
%                (717, 0.00057)
%                (720, -0)
%                (722, -0)
%                (725, -0.00171)
%                (727, -0.00171)
%                (730, -0.00228)
%                (732, -0.00228)
%                (735, -0.00228)
%                (737, -0.00342)
%                (740, -0.00342)
%                (742, -0.00342)
%                (745, -0)
%                (747, -0.00228)
%                (750, -0.00114)
%                (752, -0.00057)
%                (755, 0.00057)
%                (757, 0.00057)
%                (760, 0.00057)
%                (762, 0.00057)
%                (765, 0.00171)
%                (767, 0.00114)
%                (770, 0.00114)
%                (772, 0.00114)
%                (775, 0.00114)
%                (777, 0.00171)
%                (780, 0.00171)
%                (782, 0.00171)
%                (785, 0.00171)
%                (787, 0.00171)
%                (790, 0.00171)
%                (792, 0.00171)
%                (795, 0.00171)
%                (797, 0.00114)
%                (800, 0.00114)
%                (802, 0.00114)
%                (805, 0.00114)
%                (807, 0.00114)
%                (810, 0.00114)
%                (812, 0.00114)
%                (815, 0.00114)
%                (817, 0.00114)
%                (820, 0.00114)
%                (822, 0.00114)
%                (825, 0.00171)
%                (827, 0.00171)
%                (830, 0.00171)
%                (832, 0.00171)
%                (835, 0.00171)
%                (837, 0.00171)
%                (840, 0.00114)
%                (842, 0.00114)
%                (845, 0.00114)
%                (847, 0.00114)
%                (850, 0.00114)
%                (852, 0.00114)
%                (855, -0)
%                (857, -0)
%                (860, -0.00057)
%                (862, -0.00114)
%                (865, -0.00057)
%                (867, -0.00114)
%                (870, -0)
%                (872, -0)
%                (875, 0.00057)
%                (877, 0.00114)
%                (880, 0.00114)
%                (882, 0.00114)
%                (885, -0)
%                (887, -0)
%                (890, -0)
%                (892, -0.00057)
%                (895, -0)
%                (897, -0)
%                (900, -0)
%                (902, -0.00057)
%                (905, -0.00114)
%                (907, -0.00114)
%                (910, -0.00114)
%                (912, -0.00114)
%                (915, -0.00114)
%                (917, -0.00114)
%                (920, -0.00057)
%                (922, -0.00057)
%                (925, -0)
%                (927, -0)
%                (930, 0.00057)
%                (932, 0.00114)
%                (935, 0.00114)
%                (937, 0.00114)
%                (940, 0.00114)
%                (942, 0.00114)
%                (945, 0.00114)
%                (947, 0.00114)
%                (950, 0.00114)
%                (952, -0)
%                (955, -0.00057)
%                (957, -0.00114)
%                (960, -0.00171)
%                (962, -0.00114)
%                (965, -0.00114)
%                (967, -0.00171)
%                (970, -0.00114)
%                (972, -0.00114)
%                (975, -0.00057)
%                (977, -0)
%                (980, 0.00057)
%                (982, 0.00057)
%                (985, -0)
%                (987, 0.00057)
%                (990, 0.00057)
%                (992, 0.00114)
%                (995, 0.00057)
%                (997, 0.00114)
%                (1000, 0.00114)
%                (1002, 0.00114)
%                (1005, 0.00114)
%                (1007, 0.00057)
%                (1010, 0.00057)
%                (1012, 0.00057)
%                (1015, 0.00057)
%                (1017, 0.00057)
%                (1020, 0.00114)
%                (1022, 0.00114)
%                (1025, 0.00114)
%                (1027, 0.00114)
%                (1030, 0.00114)
%                (1032, 0.00114)
%                (1035, 0.00114)
%                (1037, 0.00114)
%                (1040, 0.00114)
%                (1042, 0.00114)
%                (1045, 0.00114)
%                (1047, 0.00114)
%                (1050, 0.00114)
%                (1052, 0.00114)
%                (1055, 0.00114)
%                (1057, 0.00114)
%                (1060, 0.00114)
%                (1062, 0.00114)
%                (1065, 0.00114)
%                (1067, 0.00114)
%                (1070, 0.00114)
%                (1072, 0.00114)
%                (1075, 0.00114)
%                (1077, 0.00114)
%                (1080, 0.00171)
%                (1082, 0.00171)
%                (1085, 0.00171)
%                (1087, 0.00171)
%                (1090, 0.00171)
%                (1092, 0.00114)
%                (1095, 0.00114)
%                (1097, 0.00114)
%                (1100, 0.00114)
%                (1102, 0.00114)
%                (1105, 0.00114)
%                (1107, 0.00114)
%                (1110, 0.00057)
%                (1112, -0)
%                (1115, -0.00057)
%                (1117, -0.00114)
%                (1120, -0.00114)
%                (1122, -0.00114)
%                (1125, -0.00057)
%                (1127, -0.00114)
%                (1130, -0)
%                (1132, -0)
%                (1135, -0)
%                (1137, -0)
%                (1140, -0)
%                (1142, -0)
%                (1145, -0)
%                (1147, -0)
%                (1150, -0)
%                (1152, -0)
%                (1155, -0)
%                (1157, -0)
%                (1160, -0)
%                (1162, -0)
%                (1165, 0.00114)
%                (1167, 0.00114)
%                (1170, 0.00114)
%                (1172, 0.00114)
%                (1175, 0.00114)
%                (1177, 0.00114)
%                (1180, 0.00057)
%                (1182, -0)
%                (1185, -0.00114)
%                (1187, -0.00171)
%                (1190, -0.00171)
%                (1192, -0.00228)
%                (1195, -0.00114)
%                (1197, -0.00114)
%                (1200, -0.00114)
%                (1202, -0.00057)
%                (1205, -0.00057)
%                (1207, -0.00057)
%                (1210, -0.00114)
%                (1212, -0.00171)
%                (1215, -0.00228)
%                (1217, -0.00228)
%                (1220, -0.00228)
%                (1222, -0.00285)
%                (1225, -0.00171)
%                (1227, -0.00171)
%                (1230, -0.00114)
%                (1232, -0)
%                (1235, -0)
%                (1237, -0)
%                (1240, 0.00057)
%                (1242, 0.00114)
%                (1245, 0.00114)
%                (1247, 0.00114)
%                (1250, 0.00114)
%                (1252, 0.00114)
%                (1255, 0.00114)
%                (1257, -0)
%                (1260, -0.00228)
%                (1262, -0.00285)
%                (1265, -0.00228)
%                (1267, -0.00171)
%                (1270, -0.00114)
%                (1272, -0.00171)
%                (1275, -0.00114)
%                (1277, -0.00057)
%                (1280, -0.00057)
%                (1282, -0.00057)
%                (1285, -0)
%                (1287, -0)
%                (1290, 0.00057)
%                (1292, 0.00057)
%                (1295, 0.00114)
%                (1297, 0.00114)
%                (1300, 0.00114)
%                (1302, 0.00114)
%                (1305, 0.00114)
%                (1307, 0.00057)
%                (1310, 0.00114)
%                (1312, 0.00057)
%                (1315, 0.00057)
%                (1317, -0.00114)
%                (1320, -0.00228)
%                (1322, -0.00114)
%                (1325, -0.00171)
%                (1327, -0.00171)
%                (1330, -0.00171)
%                (1332, -0.00228)
%                (1335, -0.00171)
%                (1337, -0.00114)
%                (1340, -0.00114)
%                (1342, -0.00057)
%                (1345, -0)
%                (1347, 0.00057)
%                (1350, 0.00057)
%                (1352, 0.00057)
%            };
        
            
            % Draw series plot
%            \addplot[no markers,black!80] coordinates {
%                (0, 0.00114)
%                (2, 0.00114)
%                (5, 0.00057)
%                (7, -0)
%                (10, -0)
%                (12, -0)
%                (15, -0)
%                (17, -0.00057)
%                (20, -0.00342)
%                (22, -0.03192)
%                (25, -0.0741)
%                (27, -0.1596)
%                (30, -0.25878)
%                (32, -0.37449)
%                (35, -0.46341)
%                (37, -0.52953)
%                (40, -0.55461)
%                (42, -0.54036)
%                (45, -0.49818)
%                (47, -0.45201)
%                (50, -0.41895)
%                (52, -0.36993)
%                (55, -0.31521)
%                (57, -0.25764)
%                (60, -0.22572)
%                (62, -0.19608)
%                (65, -0.1767)
%                (67, -0.15618)
%                (70, -0.1368)
%                (72, -0.12084)
%                (75, -0.10488)
%                (77, -0.09177)
%                (80, -0.08835)
%                (82, -0.08322)
%                (85, -0.07923)
%                (87, -0.07125)
%                (90, -0.06783)
%                (92, -0.06498)
%                (95, -0.06042)
%                (97, -0.05415)
%                (100, -0.05073)
%                (102, -0.04674)
%                (105, -0.04332)
%                (107, -0.04332)
%                (110, -0.04617)
%                (112, -0.08151)
%                (115, -0.11856)
%                (117, -0.16245)
%                (120, -0.18753)
%                (122, -0.20805)
%                (125, -0.21318)
%                (127, -0.21318)
%                (130, -0.21432)
%                (132, -0.21147)
%                (135, -0.1881)
%                (137, -0.16758)
%                (140, -0.15675)
%                (142, -0.14478)
%                (145, -0.13395)
%                (147, -0.1197)
%                (150, -0.10716)
%                (152, -0.09177)
%                (155, -0.07581)
%                (157, -0.057)
%                (160, -0.05016)
%                (162, -0.04389)
%                (165, -0.03933)
%                (167, -0.03591)
%                (170, -0.03591)
%                (172, -0.03363)
%                (175, -0.03021)
%                (177, -0.02793)
%                (180, -0.02964)
%                (182, -0.02907)
%                (185, -0.0285)
%                (187, -0.02793)
%                (190, -0.02736)
%                (192, -0.02679)
%                (195, -0.02622)
%                (197, -0.02622)
%                (200, -0.02565)
%                (202, -0.02394)
%                (205, -0.02337)
%                (207, -0.02223)
%                (210, -0.01653)
%                (212, -0.01539)
%                (215, -0.01539)
%                (217, -0.01482)
%                (220, -0.01482)
%                (222, -0.01482)
%                (225, -0.01539)
%                (227, -0.01482)
%                (230, -0.01482)
%                (232, -0.01482)
%                (235, -0.01482)
%                (237, -0.01482)
%                (240, -0.01254)
%                (242, -0.01368)
%                (245, -0.01425)
%                (247, -0.01368)
%                (250, -0.01368)
%                (252, -0.01368)
%                (255, -0.01368)
%                (257, -0.01368)
%                (260, -0.01254)
%                (262, -0.01083)
%                (265, -0.01026)
%                (267, -0.01083)
%                (270, -0.01026)
%                (272, -0.00969)
%                (275, -0.00969)
%                (277, -0.00912)
%                (280, -0.00912)
%                (282, -0.00855)
%                (285, -0.00855)
%                (287, -0.00741)
%                (290, -0.00684)
%                (292, -0.00684)
%                (295, -0.00627)
%                (297, -0.0057)
%                (300, -0.00627)
%                (302, -0.00627)
%                (305, -0.00684)
%                (307, -0.00855)
%                (310, -0.00855)
%                (312, -0.00912)
%                (315, -0.00969)
%                (317, -0.01026)
%                (320, -0.00969)
%                (322, -0.01026)
%                (325, -0.01026)
%                (327, -0.00969)
%                (330, -0.00969)
%                (332, -0.00969)
%                (335, -0.00912)
%                (337, -0.00855)
%                (340, -0.00855)
%                (342, -0.00912)
%                (345, -0.00912)
%                (347, -0.00912)
%                (350, -0.00855)
%                (352, -0.00798)
%                (355, -0.00798)
%                (357, -0.00741)
%                (360, -0.00684)
%                (362, -0.00627)
%                (365, -0.0057)
%                (367, -0.00513)
%                (370, -0.00513)
%                (372, -0.00456)
%                (375, -0.00513)
%                (377, -0.00513)
%                (380, -0.00513)
%                (382, -0.00513)
%                (385, -0.00513)
%                (387, -0.00456)
%                (390, -0.00513)
%                (392, -0.00513)
%                (395, -0.00513)
%                (397, -0.00513)
%                (400, -0.00399)
%                (402, -0.00342)
%                (405, -0.00171)
%                (407, -0.00285)
%                (410, -0.00228)
%                (412, -0.00228)
%                (415, -0.00228)
%                (417, -0.00228)
%                (420, -0.00228)
%                (422, -0.00228)
%                (425, -0.00228)
%                (427, -0.00228)
%                (430, -0.00228)
%                (432, -0.00171)
%                (435, -0.00114)
%                (437, -0.00114)
%                (440, -0.00057)
%                (442, -0)
%                (445, -0)
%                (447, -0)
%                (450, -0)
%                (452, 0.00057)
%                (455, -0)
%                (457, -0)
%                (460, 0.00057)
%                (462, 0.00057)
%                (465, 0.00057)
%                (467, 0.00057)
%                (470, -0)
%                (472, -0)
%                (475, -0)
%                (477, -0)
%                (480, 0.00057)
%                (482, 0.00057)
%                (485, 0.00057)
%                (487, 0.00114)
%                (490, 0.00057)
%                (492, 0.00057)
%                (495, -0)
%                (497, -0)
%                (500, -0.00057)
%                (502, -0.00057)
%                (505, -0)
%                (507, -0)
%                (510, 0.00057)
%                (512, 0.00057)
%                (515, 0.00057)
%                (517, 0.00057)
%                (520, -0)
%                (522, -0)
%                (525, -0)
%                (527, 0.00057)
%                (530, 0.00114)
%                (532, 0.00114)
%                (535, 0.00057)
%                (537, 0.00114)
%                (540, 0.00114)
%                (542, 0.00114)
%                (545, 0.00114)
%                (547, 0.00057)
%                (550, 0.00057)
%                (552, 0.00057)
%                (555, 0.00057)
%                (557, 0.00114)
%                (560, 0.00171)
%                (562, 0.00114)
%                (565, 0.00114)
%                (567, 0.00114)
%                (570, 0.00114)
%                (572, 0.00114)
%                (575, 0.00057)
%                (577, 0.00057)
%                (580, 0.00057)
%                (582, 0.00057)
%                (585, 0.00228)
%                (587, 0.00171)
%                (590, 0.00171)
%                (592, 0.00228)
%                (595, 0.00228)
%                (597, 0.00228)
%                (600, 0.00228)
%                (602, 0.00228)
%                (605, 0.00228)
%                (607, 0.00228)
%                (610, 0.00228)
%                (612, 0.00228)
%                (615, 0.00228)
%                (617, 0.00171)
%                (620, 0.00171)
%                (622, 0.00171)
%                (625, 0.00114)
%                (627, 0.00114)
%                (630, 0.00114)
%                (632, 0.00114)
%                (635, 0.00114)
%                (637, 0.00114)
%                (640, 0.00114)
%                (642, 0.00114)
%                (645, 0.00114)
%                (647, 0.00114)
%                (650, 0.00114)
%                (652, 0.00114)
%                (655, 0.00114)
%                (657, 0.00114)
%                (660, 0.00114)
%                (662, 0.00114)
%                (665, 0.00114)
%                (667, 0.00114)
%                (670, 0.00114)
%                (672, 0.00114)
%                (675, 0.00114)
%                (677, 0.00114)
%                (680, 0.00171)
%                (682, 0.00114)
%                (685, 0.00114)
%                (687, 0.00114)
%                (690, 0.00171)
%                (692, 0.00114)
%                (695, 0.00114)
%                (697, 0.00114)
%                (700, 0.00114)
%                (702, 0.00114)
%                (705, 0.00114)
%                (707, 0.00114)
%                (710, 0.00114)
%                (712, 0.00114)
%                (715, 0.00114)
%                (717, 0.00114)
%                (720, 0.00114)
%                (722, 0.00114)
%                (725, 0.00114)
%                (727, 0.00114)
%                (730, 0.00114)
%                (732, 0.00114)
%                (735, 0.00114)
%                (737, 0.00114)
%                (740, 0.00114)
%                (742, 0.00114)
%                (745, 0.00114)
%                (747, 0.00114)
%                (750, 0.00114)
%                (752, 0.00114)
%                (755, 0.00114)
%                (757, 0.00114)
%                (760, 0.00114)
%                (762, 0.00114)
%                (765, 0.00114)
%                (767, 0.00057)
%                (770, 0.00057)
%                (772, -0)
%                (775, -0)
%                (777, -0)
%                (780, -0)
%                (782, -0)
%                (785, -0)
%                (787, -0)
%                (790, -0)
%                (792, -0)
%                (795, 0.00057)
%                (797, 0.00057)
%                (800, 0.00057)
%                (802, 0.00114)
%                (805, 0.00114)
%                (807, 0.00114)
%                (810, 0.00114)
%                (812, 0.00114)
%                (815, 0.00114)
%                (817, 0.00114)
%                (820, 0.00114)
%                (822, 0.00114)
%                (825, 0.00114)
%                (827, 0.00114)
%                (830, 0.00114)
%                (832, 0.00114)
%                (835, 0.00114)
%                (837, 0.00114)
%                (840, 0.00114)
%                (842, 0.00114)
%                (845, 0.00114)
%                (847, 0.00114)
%                (850, 0.00114)
%                (852, 0.00171)
%                (855, 0.00114)
%                (857, 0.00114)
%                (860, 0.00114)
%                (862, 0.00114)
%                (865, 0.00114)
%                (867, 0.00114)
%                (870, 0.00114)
%                (872, 0.00114)
%                (875, 0.00114)
%                (877, 0.00114)
%                (880, 0.00114)
%                (882, 0.00114)
%                (885, 0.00114)
%                (887, 0.00114)
%                (890, 0.00114)
%                (892, 0.00114)
%                (895, 0.00114)
%                (897, 0.00057)
%                (900, -0)
%                (902, -0.00114)
%                (905, -0.00228)
%                (907, -0.00171)
%                (910, -0.00171)
%                (912, -0.00171)
%                (915, -0.00114)
%                (917, -0.00057)
%                (920, -0)
%                (922, -0)
%                (925, -0)
%                (927, 0.00114)
%                (930, 0.00114)
%                (932, 0.00114)
%                (935, 0.00114)
%                (937, 0.00114)
%                (940, 0.00114)
%                (942, 0.00114)
%                (945, 0.00114)
%                (947, 0.00057)
%                (950, -0.00114)
%                (952, -0.00114)
%                (955, -0.00171)
%                (957, -0.00228)
%                (960, -0.00285)
%                (962, -0.00285)
%                (965, -0.00228)
%                (967, -0.00228)
%                (970, -0.00171)
%                (972, -0.00114)
%                (975, -0.00114)
%                (977, -0)
%                (980, -0)
%                (982, -0)
%                (985, -0)
%                (987, -0)
%                (990, -0)
%                (992, -0)
%                (995, 0.00057)
%                (997, 0.00114)
%                (1000, 0.00057)
%                (1002, 0.00114)
%                (1005, 0.00057)
%                (1007, 0.00057)
%                (1010, 0.00057)
%                (1012, 0.00057)
%                (1015, 0.00057)
%                (1017, 0.00057)
%                (1020, 0.00057)
%                (1022, 0.00057)
%                (1025, 0.00114)
%                (1027, 0.00114)
%                (1030, 0.00114)
%                (1032, 0.00114)
%                (1035, 0.00114)
%                (1037, 0.00114)
%                (1040, 0.00114)
%                (1042, 0.00114)
%                (1045, 0.00114)
%                (1047, 0.00114)
%                (1050, 0.00114)
%                (1052, 0.00114)
%                (1055, 0.00114)
%                (1057, 0.00114)
%                (1060, 0.00114)
%                (1062, 0.00114)
%                (1065, 0.00114)
%                (1067, 0.00114)
%                (1070, 0.00114)
%                (1072, 0.00114)
%                (1075, 0.00114)
%                (1077, 0.00114)
%                (1080, 0.00114)
%                (1082, 0.00114)
%                (1085, 0.00114)
%                (1087, 0.00057)
%                (1090, 0.00114)
%                (1092, 0.00114)
%                (1095, 0.00114)
%                (1097, 0.00057)
%                (1100, 0.00114)
%                (1102, 0.00114)
%                (1105, 0.00114)
%                (1107, 0.00114)
%                (1110, 0.00114)
%                (1112, 0.00114)
%                (1115, 0.00114)
%                (1117, 0.00114)
%                (1120, 0.00114)
%                (1122, 0.00114)
%                (1125, 0.00114)
%                (1127, 0.00114)
%                (1130, 0.00114)
%                (1132, 0.00114)
%                (1135, 0.00057)
%                (1137, 0.00057)
%                (1140, 0.00057)
%                (1142, 0.00057)
%                (1145, 0.00057)
%                (1147, 0.00057)
%                (1150, 0.00057)
%                (1152, 0.00057)
%                (1155, 0.00057)
%                (1157, 0.00057)
%                (1160, 0.00114)
%                (1162, 0.00114)
%                (1165, 0.00114)
%                (1167, 0.00114)
%                (1170, 0.00114)
%                (1172, 0.00114)
%                (1175, 0.00114)
%                (1177, 0.00114)
%                (1180, 0.00114)
%                (1182, 0.00114)
%                (1185, 0.00114)
%                (1187, 0.00114)
%                (1190, 0.00114)
%                (1192, 0.00114)
%                (1195, 0.00114)
%                (1197, 0.00114)
%                (1200, 0.00114)
%                (1202, 0.00114)
%                (1205, 0.00114)
%                (1207, 0.00114)
%                (1210, 0.00114)
%                (1212, 0.00114)
%                (1215, 0.00057)
%                (1217, 0.00057)
%                (1220, 0.00057)
%                (1222, -0)
%                (1225, -0)
%                (1227, -0)
%                (1230, -0)
%                (1232, -0.00057)
%                (1235, -0.00057)
%                (1237, -0.00057)
%                (1240, -0.00057)
%                (1242, -0)
%                (1245, 0.00057)
%                (1247, 0.00057)
%                (1250, 0.00057)
%                (1252, 0.00114)
%                (1255, 0.00114)
%                (1257, 0.00114)
%                (1260, 0.00114)
%                (1262, 0.00114)
%                (1265, 0.00057)
%                (1267, 0.00057)
%                (1270, 0.00057)
%                (1272, 0.00057)
%                (1275, 0.00057)
%                (1277, 0.00057)
%                (1280, 0.00057)
%                (1282, 0.00057)
%                (1285, 0.00057)
%                (1287, 0.00114)
%                (1290, 0.00114)
%                (1292, 0.00114)
%                (1295, 0.00114)
%                (1297, 0.00114)
%                (1300, 0.00114)
%                (1302, 0.00114)
%                (1305, 0.00057)
%                (1307, 0.00057)
%                (1310, 0.00057)
%                (1312, 0.00057)
%                (1315, 0.00057)
%                (1317, 0.00057)
%                (1320, -0)
%                (1322, -0.00057)
%                (1325, -0.00114)
%                (1327, -0.00114)
%                (1330, -0.00114)
%                (1332, -0.00114)
%                (1335, -0.00057)
%                (1337, -0)
%                (1340, -0)
%                (1342, -0)
%                (1345, 0.00057)
%                (1347, 0.00114)
%                (1350, 0.00057)
%                (1352, 0.00057)
%            };
        
            
            % Draw series plot
%            \addplot[no markers,black!60] coordinates {
%                (0, -0.00057)
%                (2, -0)
%                (5, -0.00057)
%                (7, -0.00114)
%                (10, -0.00057)
%                (12, -0.00057)
%                (15, -0.00057)
%                (17, -0.00114)
%                (20, -0.03591)
%                (22, -0.10488)
%                (25, -0.23028)
%                (27, -0.39729)
%                (30, -0.55974)
%                (32, -0.67944)
%                (35, -0.75924)
%                (37, -0.81225)
%                (40, -0.84246)
%                (42, -0.84018)
%                (45, -0.81852)
%                (47, -0.78261)
%                (50, -0.73188)
%                (52, -0.66462)
%                (55, -0.58995)
%                (57, -0.52497)
%                (60, -0.46968)
%                (62, -0.40812)
%                (65, -0.35112)
%                (67, -0.30495)
%                (70, -0.2679)
%                (72, -0.22914)
%                (75, -0.19551)
%                (77, -0.17727)
%                (80, -0.15675)
%                (82, -0.12882)
%                (85, -0.11001)
%                (87, -0.10374)
%                (90, -0.1026)
%                (92, -0.0969)
%                (95, -0.09177)
%                (97, -0.08265)
%                (100, -0.0798)
%                (102, -0.07296)
%                (105, -0.06555)
%                (107, -0.05643)
%                (110, -0.05187)
%                (112, -0.04617)
%                (115, -0.04047)
%                (117, -0.03933)
%                (120, -0.03762)
%                (122, -0.0342)
%                (125, -0.03648)
%                (127, -0.03648)
%                (130, -0.03534)
%                (132, -0.03534)
%                (135, -0.0342)
%                (137, -0.02793)
%                (140, -0.02907)
%                (142, -0.02907)
%                (145, -0.03078)
%                (147, -0.02964)
%                (150, -0.02964)
%                (152, -0.02964)
%                (155, -0.02907)
%                (157, -0.0285)
%                (160, -0.0285)
%                (162, -0.0285)
%                (165, -0.0285)
%                (167, -0.02964)
%                (170, -0.03078)
%                (172, -0.03135)
%                (175, -0.03192)
%                (177, -0.03078)
%                (180, -0.03135)
%                (182, -0.02907)
%                (185, -0.02565)
%                (187, -0.02394)
%                (190, -0.02394)
%                (192, -0.02508)
%                (195, -0.02451)
%                (197, -0.02508)
%                (200, -0.02508)
%                (202, -0.02394)
%                (205, -0.0228)
%                (207, -0.02109)
%                (210, -0.01653)
%                (212, -0.01197)
%                (215, -0.01197)
%                (217, -0.01197)
%                (220, -0.0114)
%                (222, -0.00969)
%                (225, -0.00798)
%                (227, -0.0057)
%                (230, -0.00798)
%                (232, -0.00969)
%                (235, -0.00855)
%                (237, -0.00912)
%                (240, -0.00912)
%                (242, -0.00855)
%                (245, -0.00855)
%                (247, -0.00912)
%                (250, -0.00969)
%                (252, -0.00969)
%                (255, -0.01026)
%                (257, -0.01026)
%                (260, -0.01197)
%                (262, -0.01311)
%                (265, -0.01254)
%                (267, -0.01197)
%                (270, -0.01254)
%                (272, -0.01311)
%                (275, -0.01368)
%                (277, -0.01425)
%                (280, -0.01482)
%                (282, -0.01539)
%                (285, -0.01596)
%                (287, -0.01596)
%                (290, -0.01653)
%                (292, -0.0171)
%                (295, -0.01653)
%                (297, -0.01539)
%                (300, -0.01539)
%                (302, -0.01083)
%                (305, -0.01254)
%                (307, -0.0114)
%                (310, -0.01197)
%                (312, -0.01254)
%                (315, -0.01311)
%                (317, -0.01311)
%                (320, -0.01482)
%                (322, -0.01482)
%                (325, -0.0171)
%                (327, -0.0171)
%                (330, -0.01653)
%                (332, -0.01596)
%                (335, -0.01311)
%                (337, -0.01254)
%                (340, -0.01083)
%                (342, -0.00855)
%                (345, -0.00855)
%                (347, -0.00855)
%                (350, -0.00798)
%                (352, -0.00855)
%                (355, -0.00798)
%                (357, -0.00741)
%                (360, -0.00684)
%                (362, -0.00627)
%                (365, -0.00627)
%                (367, -0.00741)
%                (370, -0.00798)
%                (372, -0.00684)
%                (375, -0.00741)
%                (377, -0.00741)
%                (380, -0.00684)
%                (382, -0.00684)
%                (385, -0.00684)
%                (387, -0.00684)
%                (390, -0.00684)
%                (392, -0.00798)
%                (395, -0.00741)
%                (397, -0.00741)
%                (400, -0.00627)
%                (402, -0.0057)
%                (405, -0.00456)
%                (407, -0.00228)
%                (410, -0.00114)
%                (412, -0.00057)
%                (415, -0.00057)
%                (417, -0.00057)
%                (420, -0.00057)
%                (422, -0.00057)
%                (425, -0.00171)
%                (427, -0.00114)
%                (430, -0.00171)
%                (432, -0.00228)
%                (435, -0.00114)
%                (437, -0.00171)
%                (440, -0.00057)
%                (442, -0)
%                (445, -0)
%                (447, 0.00171)
%                (450, 0.00171)
%                (452, 0.00228)
%                (455, 0.00171)
%                (457, 0.00228)
%                (460, 0.00171)
%                (462, 0.00171)
%                (465, 0.00114)
%                (467, 0.00171)
%                (470, 0.00171)
%                (472, 0.00114)
%                (475, 0.00171)
%                (477, 0.00114)
%                (480, 0.00285)
%                (482, 0.00171)
%                (485, 0.00171)
%                (487, 0.00228)
%                (490, 0.00228)
%                (492, 0.00228)
%                (495, 0.00228)
%                (497, 0.00228)
%                (500, 0.00228)
%                (502, 0.00228)
%                (505, 0.00228)
%                (507, 0.00342)
%                (510, 0.00285)
%                (512, 0.00228)
%                (515, 0.00285)
%                (517, 0.00171)
%                (520, 0.00171)
%                (522, 0.00228)
%                (525, 0.00171)
%                (527, 0.00114)
%                (530, 0.00171)
%                (532, 0.00171)
%                (535, 0.00171)
%                (537, 0.00114)
%                (540, -0)
%                (542, -0)
%                (545, -0.00228)
%                (547, -0.00399)
%                (550, -0.00513)
%                (552, -0.0057)
%                (555, -0.0057)
%                (557, -0.00456)
%                (560, -0.00399)
%                (562, -0)
%                (565, 0.00057)
%                (567, 0.00228)
%                (570, 0.00114)
%                (572, 0.00114)
%                (575, 0.00114)
%                (577, 0.00114)
%                (580, 0.00114)
%                (582, 0.00114)
%                (585, 0.00171)
%                (587, 0.00228)
%                (590, 0.00228)
%                (592, 0.00114)
%                (595, 0.00114)
%                (597, -0)
%                (600, -0)
%                (602, -0.00114)
%                (605, -0.0057)
%                (607, -0.0057)
%                (610, -0.00456)
%                (612, -0.00342)
%                (615, -0.00171)
%                (617, -0.00171)
%                (620, -0.00057)
%                (622, -0)
%                (625, -0)
%                (627, 0.00057)
%                (630, 0.00057)
%                (632, -0)
%                (635, -0)
%                (637, -0)
%                (640, -0)
%                (642, 0.00057)
%                (645, 0.00057)
%                (647, 0.00057)
%                (650, 0.00114)
%                (652, 0.00171)
%                (655, 0.00171)
%                (657, 0.00114)
%                (660, 0.00114)
%                (662, 0.00114)
%                (665, 0.00114)
%                (667, 0.00171)
%                (670, 0.00171)
%                (672, 0.00114)
%                (675, 0.00171)
%                (677, 0.00057)
%                (680, 0.00114)
%                (682, 0.00114)
%                (685, 0.00057)
%                (687, -0.00114)
%                (690, -0.00114)
%                (692, -0.00342)
%                (695, -0.00342)
%                (697, -0.00399)
%                (700, -0.00342)
%                (702, -0.00285)
%                (705, -0.00228)
%                (707, 0.00399)
%                (710, -0)
%                (712, -0)
%                (715, -0)
%                (717, -0.00057)
%                (720, -0)
%                (722, 0.00114)
%                (725, 0.00057)
%                (727, 0.00057)
%                (730, 0.00114)
%                (732, 0.00114)
%                (735, 0.00114)
%                (737, 0.00114)
%                (740, 0.00114)
%                (742, 0.00114)
%                (745, 0.00057)
%                (747, 0.00114)
%                (750, 0.00057)
%                (752, 0.00114)
%                (755, 0.00114)
%                (757, 0.00057)
%                (760, 0.00114)
%                (762, 0.00114)
%                (765, 0.00114)
%                (767, 0.00114)
%                (770, 0.00114)
%                (772, 0.00057)
%                (775, 0.00057)
%                (777, 0.00057)
%                (780, 0.00057)
%                (782, -0)
%                (785, -0.00114)
%                (787, -0.00057)
%                (790, -0.00114)
%                (792, -0.00114)
%                (795, -0.00057)
%                (797, -0.00114)
%                (800, -0.00114)
%                (802, -0.00114)
%                (805, -0.00114)
%                (807, -0.00342)
%                (810, -0.00399)
%                (812, -0.00342)
%                (815, -0.00285)
%                (817, -0.00228)
%                (820, -0.00057)
%                (822, 0.00114)
%                (825, -0.00057)
%                (827, -0)
%                (830, -0)
%                (832, -0)
%                (835, -0)
%                (837, -0.00114)
%                (840, -0.00114)
%                (842, -0.00171)
%                (845, -0.00171)
%                (847, -0.00228)
%                (850, -0.00228)
%                (852, -0.00228)
%                (855, -0.00285)
%                (857, -0.00228)
%                (860, -0.00171)
%                (862, -0.00114)
%                (865, -0.00057)
%                (867, -0)
%                (870, -0)
%                (872, -0)
%                (875, -0)
%                (877, -0.00057)
%                (880, -0.00057)
%                (882, -0)
%                (885, -0.00114)
%                (887, -0)
%                (890, -0)
%                (892, -0)
%                (895, -0.00114)
%                (897, -0.00171)
%                (900, -0.00114)
%                (902, -0.00114)
%                (905, -0.00114)
%                (907, 0.00057)
%                (910, -0)
%                (912, -0)
%                (915, -0.00057)
%                (917, -0.00114)
%                (920, -0.00057)
%                (922, 0.00057)
%                (925, -0)
%                (927, -0.00114)
%                (930, -0.00057)
%                (932, -0.00057)
%                (935, -0.00057)
%                (937, -0)
%                (940, -0)
%                (942, -0)
%                (945, -0)
%                (947, -0)
%                (950, -0)
%                (952, -0.00057)
%                (955, -0.00114)
%                (957, -0.00342)
%                (960, -0.00399)
%                (962, -0.00513)
%                (965, -0.00627)
%                (967, -0.0057)
%                (970, -0.00513)
%                (972, -0.00456)
%                (975, -0.00342)
%                (977, -0.00285)
%                (980, -0.00285)
%                (982, -0.00228)
%                (985, -0.00228)
%                (987, -0.00228)
%                (990, -0.00114)
%                (992, -0.00114)
%                (995, -0.00057)
%                (997, -0.00114)
%                (1000, -0.00114)
%                (1002, -0.00057)
%                (1005, -0.00114)
%                (1007, -0)
%                (1010, -0)
%                (1012, 0.00057)
%                (1015, 0.00057)
%                (1017, -0)
%                (1020, -0)
%                (1022, -0)
%                (1025, -0)
%                (1027, -0)
%                (1030, -0)
%                (1032, -0.00057)
%                (1035, -0)
%                (1037, -0.00057)
%                (1040, -0)
%                (1042, -0)
%                (1045, -0)
%                (1047, -0)
%                (1050, -0.00057)
%                (1052, -0.00114)
%                (1055, -0.00171)
%                (1057, -0.00228)
%                (1060, -0.00342)
%                (1062, -0.00513)
%                (1065, -0.00456)
%                (1067, -0.00342)
%                (1070, -0.00342)
%                (1072, -0.00342)
%                (1075, -0.00399)
%                (1077, -0.00399)
%                (1080, -0.00342)
%                (1082, -0.00342)
%                (1085, -0.00285)
%                (1087, -0.00285)
%                (1090, -0.00171)
%                (1092, -0.00114)
%                (1095, -0.00114)
%                (1097, -0.00057)
%                (1100, -0.00114)
%                (1102, -0.00057)
%                (1105, -0.00057)
%                (1107, 0.00171)
%                (1110, -0)
%                (1112, -0)
%                (1115, -0)
%                (1117, -0.00057)
%                (1120, -0)
%                (1122, -0)
%                (1125, -0.00057)
%                (1127, -0.00057)
%                (1130, -0.00057)
%                (1132, -0.00057)
%                (1135, -0.00057)
%                (1137, -0)
%                (1140, -0)
%                (1142, -0)
%                (1145, -0)
%                (1147, 0.00114)
%                (1150, 0.00114)
%                (1152, 0.00057)
%                (1155, -0)
%                (1157, -0)
%                (1160, -0)
%                (1162, -0)
%                (1165, -0)
%                (1167, -0.00057)
%                (1170, -0.00057)
%                (1172, -0.00057)
%                (1175, -0.00114)
%                (1177, -0)
%                (1180, -0)
%                (1182, 0.00057)
%                (1185, 0.00057)
%                (1187, 0.00057)
%                (1190, 0.00057)
%                (1192, 0.00057)
%                (1195, 0.00057)
%                (1197, -0)
%                (1200, 0.00057)
%                (1202, -0)
%                (1205, -0)
%                (1207, 0.00057)
%                (1210, 0.00057)
%                (1212, 0.00114)
%                (1215, 0.00057)
%                (1217, 0.00057)
%                (1220, 0.00057)
%                (1222, 0.00057)
%                (1225, 0.00057)
%                (1227, 0.00057)
%                (1230, 0.00057)
%                (1232, 0.00057)
%                (1235, 0.00057)
%                (1237, -0)
%                (1240, 0.00057)
%                (1242, 0.00114)
%                (1245, 0.00057)
%                (1247, 0.00114)
%                (1250, 0.00114)
%                (1252, 0.00114)
%                (1255, 0.00057)
%                (1257, 0.00057)
%                (1260, -0)
%                (1262, 0.00114)
%                (1265, 0.00057)
%                (1267, 0.00171)
%                (1270, 0.00114)
%                (1272, 0.00114)
%                (1275, 0.00114)
%                (1277, 0.00057)
%                (1280, 0.00114)
%                (1282, 0.00228)
%                (1285, 0.00114)
%                (1287, 0.00114)
%                (1290, 0.00114)
%                (1292, 0.00114)
%                (1295, 0.00114)
%                (1297, 0.00057)
%                (1300, 0.00057)
%                (1302, 0.00114)
%                (1305, 0.00057)
%                (1307, 0.00171)
%                (1310, 0.00171)
%                (1312, 0.00114)
%                (1315, 0.00114)
%                (1317, 0.00057)
%                (1320, 0.00057)
%                (1322, 0.00057)
%                (1325, -0)
%                (1327, -0)
%                (1330, -0.00285)
%                (1332, -0.00912)
%                (1335, -0.01083)
%                (1337, -0.00969)
%                (1340, -0.00741)
%                (1342, -0.00741)
%                (1345, -0.00627)
%                (1347, 0.00057)
%                (1350, -0.00114)
%                (1352, -0.00114)
%            };
        
            
            % Draw series plot
            \addplot[no markers,black] coordinates {
                (0, -0.00057)
                (2, -0.00057)
                (5, -0)
                (7, -0.00057)
                (10, -0.00114)
                (12, -0.02451)
                (15, -0.10089)
                (17, -0.24339)
                (20, -0.43719)
                (22, -0.60819)
                (25, -0.7239)
                (27, -0.76779)
                (30, -0.77349)
                (32, -0.72789)
                (35, -0.66234)
                (37, -0.59337)
                (40, -0.5415)
                (42, -0.49305)
                (45, -0.43776)
                (47, -0.37221)
                (50, -0.31635)
                (52, -0.27189)
                (55, -0.24909)
                (57, -0.23997)
                (60, -0.23313)
                (62, -0.20406)
                (65, -0.17328)
                (67, -0.14763)
                (70, -0.13908)
                (72, -0.14877)
                (75, -0.17385)
                (77, -0.2052)
                (80, -0.21546)
                (82, -0.20577)
                (85, -0.17898)
                (87, -0.15333)
                (90, -0.13338)
                (92, -0.12255)
                (95, -0.12882)
                (97, -0.12939)
                (100, -0.1254)
                (102, -0.09861)
                (105, -0.08265)
                (107, -0.06954)
                (110, -0.06327)
                (112, -0.06156)
                (115, -0.05358)
                (117, -0.04788)
                (120, -0.0399)
                (122, -0.0342)
                (125, -0.03477)
                (127, -0.03477)
                (130, -0.0342)
                (132, -0.03021)
                (135, -0.03306)
                (137, -0.0342)
                (140, -0.0342)
                (142, -0.03306)
                (145, -0.03306)
                (147, -0.03192)
                (150, -0.03078)
                (152, -0.03249)
                (155, -0.03249)
                (157, -0.03363)
                (160, -0.03363)
                (162, -0.03192)
                (165, -0.03078)
                (167, -0.02964)
                (170, -0.02223)
                (172, -0.0228)
                (175, -0.02223)
                (177, -0.02166)
                (180, -0.02166)
                (182, -0.02109)
                (185, -0.02166)
                (187, -0.02166)
                (190, -0.02166)
                (192, -0.02223)
                (195, -0.02166)
                (197, -0.02166)
                (200, -0.02052)
                (202, -0.02052)
                (205, -0.01881)
                (207, -0.01653)
                (210, -0.01425)
                (212, -0.01254)
                (215, -0.01425)
                (217, -0.01539)
                (220, -0.01425)
                (222, -0.01482)
                (225, -0.01482)
                (227, -0.01311)
                (230, -0.01197)
                (232, -0.01254)
                (235, -0.01197)
                (237, -0.01197)
                (240, -0.0114)
                (242, -0.01083)
                (245, -0.01026)
                (247, -0.00912)
                (250, -0.00855)
                (252, -0.00627)
                (255, -0.00741)
                (257, -0.00798)
                (260, -0.00741)
                (262, -0.00798)
                (265, -0.00855)
                (267, -0.00912)
                (270, -0.01026)
                (272, -0.01083)
                (275, -0.01026)
                (277, -0.01026)
                (280, -0.00912)
                (282, -0.00684)
                (285, -0.00627)
                (287, -0.00684)
                (290, -0.00684)
                (292, -0.00741)
                (295, -0.00741)
                (297, -0.00684)
                (300, -0.00684)
                (302, -0.00684)
                (305, -0.00684)
                (307, -0.00513)
                (310, -0.00627)
                (312, -0.00684)
                (315, -0.00684)
                (317, -0.00798)
                (320, -0.00798)
                (322, -0.00912)
                (325, -0.00969)
                (327, -0.00912)
                (330, -0.00912)
                (332, -0.00855)
                (335, -0.00684)
                (337, -0.00741)
                (340, -0.00684)
                (342, -0.00741)
                (345, -0.00684)
                (347, -0.00627)
                (350, -0.00627)
                (352, -0.0057)
                (355, -0.0057)
                (357, -0.00627)
                (360, -0.0057)
                (362, -0.0057)
                (365, -0.0057)
                (367, -0.00513)
                (370, -0.00513)
                (372, -0.00513)
                (375, -0.0057)
                (377, -0.00627)
                (380, -0.00684)
                (382, -0.00684)
                (385, -0.00684)
                (387, -0.00627)
                (390, -0.00399)
                (392, -0.00456)
                (395, -0.00228)
                (397, -0.00285)
                (400, -0.00171)
                (402, 0.00171)
                (405, -0.00057)
                (407, -0.00171)
                (410, -0.00057)
                (412, -0.00114)
                (415, -0.00114)
                (417, -0.00171)
                (420, -0.00228)
                (422, -0.00228)
                (425, -0.00171)
                (427, -0.00114)
                (430, -0.00114)
                (432, -0.00114)
                (435, -0.00114)
                (437, -0.00171)
                (440, -0.00114)
                (442, -0.00057)
                (445, -0)
                (447, 0.00057)
                (450, 0.00114)
                (452, 0.00114)
                (455, 0.00114)
                (457, 0.00171)
                (460, 0.00171)
                (462, 0.00228)
                (465, 0.00228)
                (467, 0.00456)
                (470, 0.00285)
                (472, 0.00228)
                (475, 0.00171)
                (477, 0.00057)
                (480, 0.00057)
                (482, -0.00057)
                (485, -0.00114)
                (487, -0.00114)
                (490, -0.00114)
                (492, -0.00114)
                (495, -0)
                (497, -0)
                (500, 0.00057)
                (502, 0.00114)
                (505, 0.00171)
                (507, 0.00228)
                (510, 0.00228)
                (512, 0.00171)
                (515, 0.00228)
                (517, 0.00171)
                (520, 0.00171)
                (522, 0.00228)
                (525, 0.00171)
                (527, 0.00114)
                (530, 0.00171)
                (532, 0.00171)
                (535, 0.00171)
                (537, 0.00171)
                (540, 0.00171)
                (542, 0.00171)
                (545, 0.00171)
                (547, 0.00285)
                (550, 0.00228)
                (552, 0.00228)
                (555, 0.00228)
                (557, 0.00114)
                (560, 0.00171)
                (562, 0.00171)
                (565, 0.00228)
                (567, 0.00228)
                (570, 0.00171)
                (572, 0.00171)
                (575, 0.00171)
                (577, 0.00114)
                (580, 0.00114)
                (582, 0.00114)
                (585, 0.00114)
                (587, 0.00228)
                (590, 0.00171)
                (592, 0.00114)
                (595, 0.00114)
                (597, 0.00114)
                (600, 0.00114)
                (602, 0.00057)
                (605, 0.00114)
                (607, 0.00114)
                (610, 0.00057)
                (612, 0.00057)
                (615, -0.00057)
                (617, -0.00114)
                (620, -0.00114)
                (622, -0.00057)
                (625, -0.00114)
                (627, -0)
                (630, -0)
                (632, -0)
                (635, 0.00114)
                (637, -0)
                (640, -0)
                (642, 0.00057)
                (645, 0.00057)
                (647, -0)
                (650, 0.00057)
                (652, 0.00057)
                (655, 0.00057)
                (657, 0.00114)
                (660, 0.00057)
                (662, 0.00114)
                (665, 0.00114)
                (667, 0.00114)
                (670, 0.00171)
                (672, 0.00171)
                (675, 0.00171)
                (677, 0.00114)
                (680, 0.00057)
                (682, 0.00057)
                (685, 0.00114)
                (687, 0.00057)
                (690, 0.00057)
                (692, 0.00057)
                (695, 0.00057)
                (697, 0.00057)
                (700, 0.00057)
                (702, 0.00057)
                (705, 0.00114)
                (707, 0.00114)
                (710, 0.00114)
                (712, 0.00114)
                (715, 0.00114)
                (717, 0.00057)
                (720, 0.00057)
                (722, 0.00114)
                (725, 0.00057)
                (727, 0.00057)
                (730, 0.00114)
                (732, 0.00057)
                (735, 0.00114)
                (737, 0.00114)
                (740, 0.00114)
                (742, 0.00114)
                (745, 0.00114)
                (747, 0.00114)
                (750, 0.00114)
                (752, 0.00114)
                (755, 0.00114)
                (757, 0.00057)
                (760, 0.00057)
                (762, 0.00057)
                (765, 0.00114)
                (767, 0.00057)
                (770, -0)
                (772, -0)
                (775, -0)
                (777, -0)
                (780, -0)
                (782, -0)
                (785, -0)
                (787, 0.00171)
                (790, 0.00057)
                (792, 0.00114)
                (795, 0.00114)
                (797, 0.00114)
                (800, 0.00114)
                (802, 0.00171)
                (805, 0.00114)
                (807, 0.00114)
                (810, 0.00114)
                (812, 0.00114)
                (815, 0.00114)
                (817, 0.00057)
                (820, 0.00114)
                (822, 0.00114)
                (825, 0.00114)
                (827, 0.00171)
                (830, 0.00114)
                (832, 0.00057)
                (835, -0)
                (837, -0.00114)
                (840, -0.00171)
                (842, -0.00285)
                (845, -0.00342)
                (847, -0.00285)
                (850, -0.00342)
                (852, -0.00114)
                (855, -0.00171)
                (857, -0.00114)
                (860, -0.00114)
                (862, -0.00114)
                (865, 0.00057)
                (867, -0)
                (870, -0)
                (872, -0)
                (875, -0)
                (877, -0)
                (880, -0)
                (882, -0)
                (885, -0)
                (887, 0.00057)
                (890, 0.00114)
                (892, 0.00057)
                (895, 0.00057)
                (897, 0.00057)
                (900, -0)
                (902, 0.00057)
                (905, 0.00057)
                (907, 0.00057)
                (910, -0)
                (912, -0)
                (915, -0)
                (917, -0.00057)
                (920, -0)
                (922, 0.00057)
                (925, 0.00057)
                (927, -0)
                (930, -0)
                (932, -0)
                (935, -0)
                (937, -0)
                (940, -0)
                (942, -0)
                (945, -0)
                (947, 0.00057)
                (950, -0)
                (952, -0)
                (955, -0)
                (957, -0.00057)
                (960, -0)
                (962, -0)
                (965, 0.00057)
                (967, -0)
                (970, -0)
                (972, 0.00057)
                (975, -0)
                (977, 0.00057)
                (980, -0)
                (982, -0)
                (985, -0)
                (987, -0)
                (990, -0)
                (992, -0)
                (995, -0)
                (997, -0.00057)
                (1000, -0)
                (1002, -0)
                (1005, 0.00057)
                (1007, 0.00057)
                (1010, 0.00057)
                (1012, 0.00114)
                (1015, 0.00057)
                (1017, 0.00057)
                (1020, 0.00057)
                (1022, -0)
                (1025, -0)
                (1027, 0.00057)
                (1030, 0.00057)
                (1032, -0)
                (1035, -0)
                (1037, -0.00057)
                (1040, -0)
                (1042, -0.00057)
                (1045, -0.00057)
                (1047, -0)
                (1050, -0)
                (1052, -0)
                (1055, -0)
                (1057, -0)
                (1060, 0.00057)
                (1062, -0)
                (1065, -0)
                (1067, 0.00057)
                (1070, -0)
                (1072, -0)
                (1075, -0)
                (1077, -0.00057)
                (1080, -0)
                (1082, -0)
                (1085, -0)
                (1087, -0)
                (1090, -0)
                (1092, -0)
                (1095, -0)
                (1097, -0)
                (1100, -0)
                (1102, -0)
                (1105, -0)
                (1107, 0.00114)
                (1110, 0.00057)
                (1112, -0)
                (1115, 0.00057)
                (1117, -0)
                (1120, 0.00057)
                (1122, 0.00057)
                (1125, 0.00057)
                (1127, 0.00057)
                (1130, -0)
                (1132, -0)
                (1135, -0)
                (1137, -0)
                (1140, -0)
                (1142, -0)
                (1145, -0)
                (1147, -0)
                (1150, -0)
                (1152, -0)
                (1155, -0)
                (1157, -0.00057)
                (1160, -0.00057)
                (1162, -0.00057)
                (1165, -0.00057)
                (1167, -0.00057)
                (1170, -0.00057)
                (1172, -0.00057)
                (1175, -0.00057)
                (1177, 0.00057)
                (1180, -0)
                (1182, -0)
                (1185, -0)
                (1187, -0)
                (1190, -0)
                (1192, -0)
                (1195, -0)
                (1197, -0)
                (1200, -0)
                (1202, -0)
                (1205, -0)
                (1207, 0.00057)
                (1210, 0.00057)
                (1212, 0.00114)
                (1215, 0.00057)
                (1217, 0.00057)
                (1220, 0.00057)
                (1222, 0.00057)
                (1225, 0.00057)
                (1227, 0.00057)
                (1230, 0.00057)
                (1232, 0.00057)
                (1235, 0.00114)
                (1237, -0)
                (1240, 0.00057)
                (1242, 0.00057)
                (1245, 0.00057)
                (1247, 0.00057)
                (1250, -0)
                (1252, -0)
                (1255, -0)
                (1257, -0)
                (1260, -0)
                (1262, -0)
                (1265, -0)
                (1267, 0.00057)
                (1270, 0.00057)
                (1272, 0.00057)
                (1275, 0.00114)
                (1277, -0)
                (1280, 0.00057)
                (1282, 0.00057)
                (1285, 0.00057)
                (1287, 0.00057)
                (1290, 0.00057)
                (1292, 0.00057)
                (1295, 0.00057)
                (1297, 0.00057)
                (1300, 0.00057)
                (1302, 0.00114)
                (1305, 0.00057)
                (1307, 0.00114)
                (1310, 0.00114)
                (1312, 0.00114)
                (1315, 0.00114)
                (1317, 0.00057)
                (1320, 0.00114)
                (1322, 0.00114)
                (1325, 0.00114)
                (1327, 0.00114)
                (1330, 0.00114)
                (1332, 0.00057)
                (1335, 0.00057)
                (1337, 0.00114)
                (1340, 0.00114)
                (1342, 0.00114)
                (1345, 0.00057)
                (1347, 0.00114)
                (1350, 0.00114)
                (1352, 0.00114)
            };
        

        

        

        
    \end{axis}
\end{tikzpicture}
%\end{sansmath}

\caption{Een signaal van een event in een \hisparc
detector. % Station 501, detector 4, 1 maart 2012 om 01:54:04.
Het signaal bestaat uit een spanning uit een \pmt.  De (negatieve)
spanning is recht evenredig met de lichtintensiteit die op het venster van
de \pmt valt.  In de staart van het signaal zijn meerdere piekjes te zien.
Dit zijn deeltjes die op een relatief laat tijdstip de detector
bereikten.}
\label{fig:traces}
\end{figure}


\subsection{Pulshoogtehistogram}

Als je alle signalen uit een \hisparc detector bekijkt, dan zie je grote
verschillen.  Dit komt doordate een event veroorzaakt kan worden door één,
twee, drie, of zelfs méér geladen deeltjes die door de detector gaan.  Ook
ontstaan er hoogenergetische fotonen in air showers.  Deze fotonen geven,
met een kleine kans, een relatief zwak signaal in de detectoren.  Maar
omdat het aantal fotonen in een air shower enorm groot is zie je dit terug
in signalen uit de \hisparc detectoren.

In \figref{fig:spectrum_componenten} is een histogram gemaakt van de
componenten van het signaal uit een \hisparc detector.  Er kunnen 1, 2, 3,
of meerdere deeltjes door een detector gaan.  Je zou dus kunnen verwachten
dat in het signaal van een \hisparc detector de stappen duidelijk te zien
zijn, zoals in de bovenste plot.  Doordat het energieverlies van geladen
deeltjes een kansproces is, net als het precieze aantal fotonen dat
uiteindelijk bij de \pmt uitkomt, is het signaal van bv. 2 deeltjes soms
iets kleiner, en soms iets groter.  De componenten zijn dus verbreedt
(middelste plot).  Verder is een afvallende distributie toegevoegd die de
bijdrage van fotonen laat zien.  Fotonen geven over het algemeen een klein
signaal, maar er zijn ontzettend veel fotonen die de plaat raken.  De
energie van de fotonen is zó hoog (gammastraling) dat deze fotonen door
het plastic dringen en door de detector gaan.  In de onderste plot is
tenslotte het totale signaal (zwart) als een optelsom van de verschillende
deeltjescomponenten (grijs) weergegeven.  Dit is het histogram dat op
\url{http://data.hisparc.nl/} en in de \daq software wordt weergegeven.

\begin{figure}
\centering
% \usepackage{tikz}
% \usetikzlibrary{arrows,pgfplots.groupplots,external}
% \usepackage{pgfplots}
% \pgfplotsset{compat=1.10}
% \usepackage[detect-family]{siunitx}
% \usepackage[eulergreek]{sansmath}
% \sisetup{text-sf=\sansmath}
% \usepackage{relsize}
%
    \tikzsetnextfilename{externalized-spectrum_componenten}
\pgfkeysifdefined{/artist/width}
    {\pgfkeysgetvalue{/artist/width}{\defaultwidth}}
    {\def\defaultwidth{ 0.5\linewidth }}
%
%
\begin{sansmath}
\begin{tikzpicture}[font=\sffamily]
    \node[inner sep=0pt] (plot) {
        \begin{tikzpicture}[
                inner sep=.3333em,
                font=\sffamily,
                every pin/.style={inner sep=2pt, font={\sffamily\smaller}},
                every label/.style={inner sep=2pt, font={\sffamily\smaller}},
                every pin edge/.style={<-, >=stealth', shorten <=2pt},
                pin distance=2.5ex,
            ]
            \begin{groupplot}[
                    xmode=normal,
                    ymode=log,
                    width=\defaultwidth,
                    scale only axis,
                    %
                    xmin={ 0 },
                    xmax={ 10.5 },
                    ymin={ 1 },
                    ymax={ 100000.0 },
                    %
                    group style={rows=3,columns=1,
                                 horizontal sep=4pt, vertical sep=4pt},
                    %
                    tick align=outside,
                    max space between ticks=40,
                    every tick/.style={},
                    axis on top,
                    %
                    xtick=\empty, ytick=\empty,
                    scaled ticks=false,
                    point meta min={  },
                    point meta max={  },
                        colormap={coolwarm}{
                            rgb255(0cm)=( 59, 76,192);
                            rgb255(1cm)=( 98,130,234);
                            rgb255(2cm)=(141,176,254);
                            rgb255(3cm)=(184,208,249);
                            rgb255(4cm)=(221,221,221);
                            rgb255(5cm)=(245,196,173);
                            rgb255(6cm)=(244,154,123);
                            rgb255(7cm)=(222, 96, 77);
                            rgb255(8cm)=(180,  4, 38)},
                ]
                
                    
                    \nextgroupplot[
                        % Default: empty ticks all round the border of the
                        % multiplot
                                xtick={  },
                                % 'right' means 'top'
                                xtick pos=right,
                                xticklabel=\empty,
                                ytick={  },
                                ytick pos=both,
                                yticklabel=\empty,
                            xtick={ 0, 1, 2, 3, 4, 5, 6, 7, 8, 9, 10, 11, 12, 13, 14, 15, 16, 17, 18, 19 },
                                xticklabel={},
                        axis equal=false,
                        %
                        title={  },
                        xlabel={  },
                        ylabel={  },
                    ]

                    

                    


    
    % Draw series plot
    \addplot[no markers,solid,const plot] coordinates {
            (0.0, 1e-10)
            (0.111111111111, 1e-10)
            (0.222222222222, 1e-10)
            (0.333333333333, 1e-10)
            (0.444444444444, 1e-10)
            (0.555555555556, 1e-10)
            (0.666666666667, 1e-10)
            (0.777777777778, 1e-10)
            (0.888888888889, 1e-10)
            (1.0, 10000.0)
            (1.11111111111, 1e-10)
            (1.22222222222, 1e-10)
            (1.33333333333, 1e-10)
            (1.44444444444, 1e-10)
            (1.55555555556, 1e-10)
            (1.66666666667, 1e-10)
            (1.77777777778, 1e-10)
            (1.88888888889, 1e-10)
            (2.0, 1250.0)
            (2.11111111111, 1e-10)
            (2.22222222222, 1e-10)
            (2.33333333333, 1e-10)
            (2.44444444444, 1e-10)
            (2.55555555556, 1e-10)
            (2.66666666667, 1e-10)
            (2.77777777778, 1e-10)
            (2.88888888889, 1e-10)
            (3.0, 370.0)
            (3.11111111111, 1e-10)
            (3.22222222222, 1e-10)
            (3.33333333333, 1e-10)
            (3.44444444444, 1e-10)
            (3.55555555556, 1e-10)
            (3.66666666667, 1e-10)
            (3.77777777778, 1e-10)
            (3.88888888889, 1e-10)
            (4.0, 156.0)
            (4.11111111111, 1e-10)
            (4.22222222222, 1e-10)
            (4.33333333333, 1e-10)
            (4.44444444444, 1e-10)
            (4.55555555556, 1e-10)
            (4.66666666667, 1e-10)
            (4.77777777778, 1e-10)
            (4.88888888889, 1e-10)
            (5.0, 80.0)
            (5.11111111111, 1e-10)
            (5.22222222222, 1e-10)
            (5.33333333333, 1e-10)
            (5.44444444444, 1e-10)
            (5.55555555556, 1e-10)
            (5.66666666667, 1e-10)
            (5.77777777778, 1e-10)
            (5.88888888889, 1e-10)
            (6.0, 46.0)
            (6.11111111111, 1e-10)
            (6.22222222222, 1e-10)
            (6.33333333333, 1e-10)
            (6.44444444444, 1e-10)
            (6.55555555556, 1e-10)
            (6.66666666667, 1e-10)
            (6.77777777778, 1e-10)
            (6.88888888889, 1e-10)
            (7.0, 29.0)
            (7.11111111111, 1e-10)
            (7.22222222222, 1e-10)
            (7.33333333333, 1e-10)
            (7.44444444444, 1e-10)
            (7.55555555556, 1e-10)
            (7.66666666667, 1e-10)
            (7.77777777778, 1e-10)
            (7.88888888889, 1e-10)
            (8.0, 19.0)
            (8.11111111111, 1e-10)
            (8.22222222222, 1e-10)
            (8.33333333333, 1e-10)
            (8.44444444444, 1e-10)
            (8.55555555556, 1e-10)
            (8.66666666667, 1e-10)
            (8.77777777778, 1e-10)
            (8.88888888889, 1e-10)
            (9.0, 13.0)
            (9.11111111111, 1e-10)
            (9.22222222222, 1e-10)
            (9.33333333333, 1e-10)
            (9.44444444444, 1e-10)
            (9.55555555556, 1e-10)
            (9.66666666667, 1e-10)
            (9.77777777778, 1e-10)
            (9.88888888889, 1e-10)
            (10.0, 10.0)
            (10.1111111111, 1e-10)
            (10.2222222222, 1e-10)
            (10.3333333333, 1e-10)
            (10.4444444444, 1e-10)
            (10.5555555556, 1e-10)
            (10.6666666667, 1e-10)
            (10.7777777778, 1e-10)
            (10.8888888889, 7.0)
            (11.0, 7.0)
    };
    


                
                    
                    \nextgroupplot[
                        % Default: empty ticks all round the border of the
                        % multiplot
                                ytick={  },
                                ytick pos=both,
                                yticklabel=\empty,
                        axis equal=false,
                        %
                        title={  },
                        xlabel={  },
                        ylabel={  },
                    ]

                    

                    


    
    % Draw series plot
    \addplot[no markers,solid] coordinates {
            (1e-10, 153.922273845)
            (0.055276382009, 238.701111536)
            (0.110552763918, 361.056313561)
            (0.165829145827, 532.675812853)
            (0.221105527736, 766.511265817)
            (0.276381909645, 1075.82501607)
            (0.331658291554, 1472.76097042)
            (0.386934673463, 1966.48403786)
            (0.442211055372, 2561.03855182)
            (0.497487437281, 3253.18932019)
            (0.55276381919, 4030.60362478)
            (0.608040201099, 4870.77844795)
            (0.663316583009, 5741.0878394)
            (0.718592964918, 6600.20673603)
            (0.773869346827, 7400.96614734)
            (0.829145728736, 8094.44036483)
            (0.884422110645, 8634.8096701)
            (0.939698492554, 8984.34130705)
            (0.994974874463, 9117.7408907)
            (1.05025125637, 9025.17811761)
            (1.10552763828, 8713.48421234)
            (1.16080402019, 8205.31856637)
            (1.2160804021, 7536.4457376)
            (1.27135678401, 6751.57719513)
            (1.32663316592, 5899.44871358)
            (1.38190954783, 5027.88289267)
            (1.43718592974, 4179.52001033)
            (1.49246231164, 3388.7161727)
            (1.54773869355, 2679.8563395)
            (1.60301507546, 2067.0708493)
            (1.65829145737, 1555.1301561)
            (1.71356783928, 1141.15767361)
            (1.76884422119, 816.755557602)
            (1.8241206031, 570.172245296)
            (1.87939698501, 388.228647617)
            (1.93467336692, 257.831880711)
            (1.98994974883, 167.014118359)
            (2.04522613073, 105.520608141)
            (2.10050251264, 65.0262810893)
            (2.15577889455, 39.0848167358)
            (2.21105527646, 22.9136763353)
            (2.26633165837, 13.102344069)
            (2.32160804028, 7.3075325225)
            (2.37688442219, 3.97520971642)
            (2.4321608041, 2.1091956405)
            (2.48743718601, 1.09154392305)
            (2.54271356792, 0.550976528811)
            (2.59798994983, 0.27126421924)
            (2.65326633173, 0.130262514638)
            (2.70854271364, 0.0610118135359)
            (2.76381909555, 0.0278725011636)
            (2.81909547746, 0.0124195391716)
            (2.87437185937, 0.0053976230566)
            (2.92964824128, 0.00228805865943)
            (2.98492462319, 0.000946017755903)
            (3.0402010051, 0.000381503874194)
            (3.09547738701, 0.000150060416932)
            (3.15075376892, 5.75706159899e-05)
            (3.20603015082, 2.15428488938e-05)
            (3.26130653273, 7.86272208667e-06)
            (3.31658291464, 2.79904736217e-06)
            (3.37185929655, 9.71885494232e-07)
            (3.42713567846, 3.29145168536e-07)
            (3.48241206037, 1.08724495288e-07)
            (3.53768844228, 3.50295733213e-08)
            (3.59296482419, 1.10080356127e-08)
            (3.6482412061, 3.37405638346e-09)
            (3.70351758801, 1.00870082721e-09)
            (3.75879396992, 2.94130424394e-10)
            (3.81407035182, 8.36536805427e-11)
            (3.86934673373, 2.32058611239e-11)
            (3.92462311564, 6.27881743523e-12)
            (3.97989949755, 1.6570116187e-12)
            (4.03517587946, 4.26521357786e-13)
            (4.09045226137, 1.07083743193e-13)
            (4.14572864328, 2.62224844197e-14)
            (4.20100502519, 6.26313336597e-15)
            (4.2562814071, 1.45907297828e-15)
            (4.31155778901, 3.31535394929e-16)
            (4.36683417091, 7.34768142792e-17)
            (4.42211055282, 1.58832096147e-17)
            (4.47738693473, 3.34883498311e-18)
            (4.53266331664, 6.88678872168e-19)
            (4.58793969855, 1.38136135715e-19)
            (4.64321608046, 2.70249797477e-20)
            (4.69849246237, 5.15692699197e-21)
            (4.75376884428, 9.59807458082e-22)
            (4.80904522619, 1.7423876688e-22)
            (4.8643216081, 3.0851265896e-23)
            (4.91959799001, 5.32805525577e-24)
            (4.97487437191, 8.97494886451e-25)
            (5.03015075382, 1.4745611142e-25)
            (5.08542713573, 2.3629857484e-26)
            (5.14070351764, 3.69340522407e-27)
            (5.19597989955, 5.63067352872e-28)
            (5.25125628146, 8.37261780401e-29)
            (5.30653266337, 1.21431040892e-29)
            (5.36180904528, 1.71777271451e-30)
            (5.41708542719, 2.37011388022e-31)
            (5.4723618091, 3.18962964208e-32)
            (5.527638191, 4.18676741125e-33)
            (5.58291457291, 5.36024873276e-34)
            (5.63819095482, 6.69358190504e-35)
            (5.69346733673, 8.15266791471e-36)
            (5.74874371864, 9.68519689846e-37)
            (5.80402010055, 1.12223726944e-37)
            (5.85929648246, 1.26831886468e-38)
            (5.91457286437, 1.39810486861e-39)
            (5.96984924628, 1.50320625896e-40)
            (6.02512562819, 1.57639458395e-41)
            (6.08040201009, 1.61242240499e-42)
            (6.135678392, 1.6086451191e-43)
            (6.19095477391, 1.56534185707e-44)
            (6.24623115582, 1.48568138741e-45)
            (6.30150753773, 1.37533880934e-46)
            (6.35678391964, 1.24182742002e-47)
            (6.41206030155, 1.0936549508e-48)
            (6.46733668346, 9.39435425075e-50)
            (6.52261306537, 7.87083981648e-51)
            (6.57788944728, 6.43195128307e-52)
            (6.63316582919, 5.12662971042e-53)
            (6.68844221109, 3.98555388485e-54)
            (6.743718593, 3.02212867538e-55)
            (6.79899497491, 2.23514009666e-56)
            (6.85427135682, 1.61236766009e-57)
            (6.90954773873, 1.1344645757e-58)
            (6.96482412064, 7.78547888697e-60)
            (7.02010050255, 5.21131415483e-61)
            (7.07537688446, 3.40233215252e-62)
            (7.13065326637, 2.16657483963e-63)
            (7.18592964828, 1.34566889181e-64)
            (7.24120603018, 8.15211432597e-66)
            (7.29648241209, 4.81692452791e-67)
            (7.351758794, 2.77611180783e-68)
            (7.40703517591, 1.56052811059e-69)
            (7.46231155782, 8.55605899691e-71)
            (7.51758793973, 4.57555179769e-72)
            (7.57286432164, 2.38660591787e-73)
            (7.62814070355, 1.21418672846e-74)
            (7.68341708546, 6.02501012812e-76)
            (7.73869346737, 2.91606776708e-77)
            (7.79396984928, 1.37659115547e-78)
            (7.84924623118, 6.33840329609e-80)
            (7.90452261309, 2.846572792e-81)
            (7.959798995, 1.24690179626e-82)
            (8.01507537691, 5.32733170091e-84)
            (8.07035175882, 2.22000911634e-85)
            (8.12562814073, 9.02333972238e-87)
            (8.18090452264, 3.57723433563e-88)
            (8.23618090455, 1.38323192549e-89)
            (8.29145728646, 5.21687161265e-91)
            (8.34673366837, 1.919078876e-92)
            (8.40201005027, 6.88561970114e-94)
            (8.45728643218, 2.40968775892e-95)
            (8.51256281409, 8.22519194892e-97)
            (8.567839196, 2.73841234132e-98)
            (8.62311557791, 8.8924038166e-100)
            (8.67839195982, 2.816482394e-101)
            (8.73366834173, 8.70086512426e-103)
            (8.78894472364, 2.62171430444e-104)
            (8.84422110555, 7.70505640152e-106)
            (8.89949748746, 2.20868515386e-107)
            (8.95477386937, 6.17531809333e-109)
            (9.01005025127, 1.68403993397e-110)
            (9.06532663318, 4.47932908107e-112)
            (9.12060301509, 1.16209353492e-113)
            (9.175879397, 2.9406052738e-115)
            (9.23115577891, 7.257714996e-117)
            (9.28643216082, 1.74715171671e-118)
            (9.34170854273, 4.10231363095e-120)
            (9.39698492464, 9.3949534422e-122)
            (9.45226130655, 2.09859162704e-123)
            (9.50753768846, 4.57223738856e-125)
            (9.56281407036, 9.7162154625e-127)
            (9.61809045227, 2.0138774296e-128)
            (9.67336683418, 4.0713314835e-130)
            (9.72864321609, 8.02800111244e-132)
            (9.783919598, 1.54399509293e-133)
            (9.83919597991, 2.89635598391e-135)
            (9.89447236182, 5.29938525888e-137)
            (9.94974874373, 9.45728722734e-139)
            (10.0050251256, 1.64617194689e-140)
            (10.0603015075, 2.79480392826e-142)
            (10.1155778895, 4.62801784611e-144)
            (10.1708542714, 7.47491461775e-146)
            (10.2261306533, 1.17756524418e-147)
            (10.2814070352, 1.80938611408e-149)
            (10.3366834171, 2.71172133788e-151)
            (10.391959799, 3.96393382173e-153)
            (10.4472361809, 5.6516500623e-155)
            (10.5025125628, 7.8594410886e-157)
            (10.5577889447, 1.06604513828e-158)
            (10.6130653266, 1.41035049782e-160)
            (10.6683417085, 1.81989381829e-162)
            (10.7236180905, 2.29051201811e-164)
            (10.7788944724, 2.81181427356e-166)
            (10.8341708543, 3.36672945786e-168)
            (10.8894472362, 3.93185359343e-170)
            (10.9447236181, 4.47872078806e-172)
            (11.0, 4.97597477851e-174)
    };
    

    
    % Draw series plot
    \addplot[no markers,solid] coordinates {
            (1e-10, 0.229649734272)
            (0.055276382009, 0.358366194811)
            (0.110552763918, 0.552295988826)
            (0.165829145827, 0.840621529144)
            (0.221105527736, 1.26360953293)
            (0.276381909645, 1.87589724018)
            (0.331658291554, 2.75035631272)
            (0.386934673463, 3.98247119403)
            (0.442211055372, 5.69508343484)
            (0.497487437281, 8.04324512793)
            (0.55276381919, 11.2187968479)
            (0.608040201099, 15.4541466365)
            (0.663316583009, 21.024590191)
            (0.718592964918, 28.2483974875)
            (0.773869346827, 37.4838211993)
            (0.829145728736, 49.1221837412)
            (0.884422110645, 63.5762984274)
            (0.939698492554, 81.26369727)
            (0.994974874463, 102.584484953)
            (1.05025125637, 127.894112551)
            (1.10552763828, 157.47194396)
            (1.16080402019, 191.487130243)
            (1.2160804021, 229.963949083)
            (1.27135678401, 272.74932919)
            (1.32663316592, 319.485674844)
            (1.38190954783, 369.592248772)
            (1.43718592974, 422.258192783)
            (1.49246231164, 476.449724882)
            (1.54773869355, 530.933149662)
            (1.60301507546, 584.314103984)
            (1.65829145737, 635.092029359)
            (1.71356783928, 681.727354948)
            (1.76884422119, 722.717457043)
            (1.8241206031, 756.676306133)
            (1.87939698501, 782.411978476)
            (1.93467336692, 798.996012809)
            (1.98994974883, 805.818992434)
            (2.04522613073, 802.62771468)
            (2.10050251264, 789.540787009)
            (2.15577889455, 767.041310386)
            (2.21105527646, 735.947279202)
            (2.26633165837, 697.362226379)
            (2.32160804028, 652.610265062)
            (2.37688442219, 603.160853996)
            (2.4321608041, 550.549230472)
            (2.48743718601, 496.298472031)
            (2.54271356792, 441.848598702)
            (2.59798994983, 388.497109175)
            (2.65326633173, 337.354002203)
            (2.70854271364, 289.31283946)
            (2.76381909555, 245.037929586)
            (2.81909547746, 204.966406275)
            (2.87437185937, 169.322949328)
            (2.92964824128, 138.144223058)
            (2.98492462319, 111.309798756)
            (3.0402010051, 88.5763606895)
            (3.09547738701, 69.6123080384)
            (3.15075376892, 54.0303772272)
            (3.20603015082, 41.4165321142)
            (3.26130653273, 31.3540198915)
            (3.31658291464, 23.4420988523)
            (3.37185929655, 17.3094601046)
            (3.42713567846, 12.6227597671)
            (3.48241206037, 9.09094179978)
            (3.53768844228, 6.46617126974)
            (3.59296482419, 4.54223182119)
            (3.6482412061, 3.15119408804)
            (3.70351758801, 2.15906032199)
            (3.75879396992, 1.46095946089)
            (3.81407035182, 0.976327045037)
            (3.86934673373, 0.644371375917)
            (3.92462311564, 0.420011252816)
            (3.97989949755, 0.270376771349)
            (4.03517587946, 0.171894347211)
            (4.09045226137, 0.107928857072)
            (4.14572864328, 0.0669263787382)
            (4.20100502519, 0.0409865006981)
            (4.2562814071, 0.0247895202082)
            (4.31155778901, 0.0148074122279)
            (4.36683417091, 0.00873522263719)
            (4.42211055282, 0.00508923558952)
            (4.47738693473, 0.00292829548851)
            (4.53266331664, 0.00166402944512)
            (4.58793969855, 0.000933879583776)
            (4.64321608046, 0.000517612220192)
            (4.69849246237, 0.000283336109631)
            (4.75376884428, 0.000153173310134)
            (4.80904522619, 8.17801748288e-05)
            (4.8643216081, 4.31217857295e-05)
            (4.91959799001, 2.24558348445e-05)
            (4.97487437191, 1.15490281394e-05)
            (5.03015075382, 5.86604566081e-06)
            (5.08542713573, 2.94258627184e-06)
            (5.14070351764, 1.45779593719e-06)
            (5.19597989955, 7.13260267711e-07)
            (5.25125628146, 3.44653818603e-07)
            (5.30653266337, 1.64475767907e-07)
            (5.36180904528, 7.75183493601e-08)
            (5.41708542719, 3.60820227432e-08)
            (5.4723618091, 1.65867386243e-08)
            (5.527638191, 7.53034531498e-09)
            (5.58291457291, 3.37638955711e-09)
            (5.63819095482, 1.49511283769e-09)
            (5.69346733673, 6.53851551223e-10)
            (5.74874371864, 2.82402223593e-10)
            (5.80402010055, 1.20459437597e-10)
            (5.85929648246, 5.07454743067e-11)
            (5.91457286437, 2.11123981136e-11)
            (5.96984924628, 8.67484219044e-12)
            (6.02512562819, 3.52021642498e-12)
            (6.08040201009, 1.41078525727e-12)
            (6.135678392, 5.58388192137e-13)
            (6.19095477391, 2.18270631527e-13)
            (6.24623115582, 8.42632391405e-14)
            (6.30150753773, 3.21265995613e-14)
            (6.35678391964, 1.20969291256e-14)
            (6.41206030155, 4.49851608757e-15)
            (6.46733668346, 1.65214129093e-15)
            (6.52261306537, 5.99251125674e-16)
            (6.57788944728, 2.1466156799e-16)
            (6.63316582919, 7.59422572835e-17)
            (6.68844221109, 2.65336179771e-17)
            (6.743718593, 9.15573486802e-18)
            (6.79899497491, 3.12013732936e-18)
            (6.85427135682, 1.0501176951e-18)
            (6.90954773873, 3.49048687722e-19)
            (6.96482412064, 1.14582371929e-19)
            (7.02010050255, 3.71478234185e-20)
            (7.07537688446, 1.18941305893e-20)
            (7.13065326637, 3.76110794591e-21)
            (7.18592964828, 1.17458015621e-21)
            (7.24120603018, 3.62270763562e-22)
            (7.29648241209, 1.1034881637e-22)
            (7.351758794, 3.31960062408e-23)
            (7.40703517591, 9.86251692796e-24)
            (7.46231155782, 2.89383291046e-24)
            (7.51758793973, 8.38576923128e-25)
            (7.57286432164, 2.39991647985e-25)
            (7.62814070355, 6.78317645503e-26)
            (7.68341708546, 1.89345011342e-26)
            (7.73869346737, 5.21985435538e-27)
            (7.79396984928, 1.4211720881e-27)
            (7.84924623118, 3.82136673615e-28)
            (7.90452261309, 1.01478615669e-28)
            (7.959798995, 2.66142428266e-29)
            (8.01507537691, 6.89346320224e-30)
            (8.07035175882, 1.76337457194e-30)
            (8.12562814073, 4.45487427464e-31)
            (8.18090452264, 1.11150160858e-31)
            (8.23618090455, 2.73885162211e-32)
            (8.29145728646, 6.66516154418e-33)
            (8.34673366837, 1.60190449719e-33)
            (8.40201005027, 3.80229954008e-34)
            (8.45728643218, 8.91332616823e-35)
            (8.51256281409, 2.06355969555e-35)
            (8.567839196, 4.71821840407e-36)
            (8.62311557791, 1.065424819e-36)
            (8.67839195982, 2.37602685116e-37)
            (8.73366834173, 5.2331555604e-38)
            (8.78894472364, 1.13830779778e-38)
            (8.84422110555, 2.44534163758e-39)
            (8.89949748746, 5.18803790328e-40)
            (8.95477386937, 1.08705246097e-40)
            (9.01005025127, 2.2494773908e-41)
            (9.06532663318, 4.5972331226e-42)
            (9.12060301509, 9.27887207712e-43)
            (9.175879397, 1.84959913378e-43)
            (9.23115577891, 3.64119388747e-44)
            (9.28643216082, 7.07935616418e-45)
            (9.34170854273, 1.35933808667e-45)
            (9.39698492464, 2.57777480182e-46)
            (9.45226130655, 4.82776618476e-47)
            (9.50753768846, 8.9295844428e-48)
            (9.56281407036, 1.63117303806e-48)
            (9.61809045227, 2.94274469697e-49)
            (9.67336683418, 5.24310919843e-50)
            (9.72864321609, 9.22590515643e-51)
            (9.783919598, 1.60329277018e-51)
            (9.83919597991, 2.75169597367e-52)
            (9.89447236182, 4.66414268912e-53)
            (9.94974874373, 7.80776932241e-54)
            (10.0050251256, 1.29082059225e-54)
            (10.0603015075, 2.10760180545e-55)
            (10.1155778895, 3.39856049256e-56)
            (10.1708542714, 5.41234176863e-57)
            (10.2261306533, 8.51254041436e-58)
            (10.2814070352, 1.32226020173e-58)
            (10.3366834171, 2.02842255853e-59)
            (10.391959799, 3.07314960506e-60)
            (10.4472361809, 4.59825169405e-61)
            (10.5025125628, 6.79493868086e-62)
            (10.5577889447, 9.91658434235e-63)
            (10.6130653266, 1.42929687359e-63)
            (10.6683417085, 2.03454145453e-64)
            (10.7236180905, 2.86018686789e-65)
            (10.7788944724, 3.97105625223e-66)
            (10.8341708543, 5.44504452784e-67)
            (10.8894472362, 7.37361737442e-68)
            (10.9447236181, 9.86151241203e-69)
            (11.0, 1.30253745127e-69)
    };
    

    
    % Draw series plot
    \addplot[no markers,solid] coordinates {
            (1e-10, 0.00093781246772)
            (0.055276382009, 0.00146649233023)
            (0.110552763918, 0.0022742215311)
            (0.165829145827, 0.00349763845303)
            (0.221105527736, 0.00533465379491)
            (0.276381909645, 0.0080691305582)
            (0.331658291554, 0.0121042092197)
            (0.386934673463, 0.0180067476507)
            (0.442211055372, 0.026565825279)
            (0.497487437281, 0.0388687454025)
            (0.55276381919, 0.056398416531)
            (0.608040201099, 0.0811563442595)
            (0.663316583009, 0.115815641717)
            (0.718592964918, 0.163908372157)
            (0.773869346827, 0.230051057275)
            (0.829145728736, 0.320211191844)
            (0.884422110645, 0.442015967003)
            (0.939698492554, 0.60510199596)
            (0.994974874463, 0.821501556)
            (1.05025125637, 1.10605665292)
            (1.10552763828, 1.47684708971)
            (1.16080402019, 1.95561278467)
            (1.2160804021, 2.56814405602)
            (1.27135678401, 3.34460682501)
            (1.32663316592, 4.31976318892)
            (1.38190954783, 5.53304221879)
            (1.43718592974, 7.02841190816)
            (1.49246231164, 8.85400177873)
            (1.54773869355, 11.0614275974)
            (1.60301507546, 13.70477577)
            (1.65829145737, 16.8392158779)
            (1.71356783928, 20.5192258699)
            (1.76884422119, 24.7964355821)
            (1.8241206031, 29.7171200306)
            (1.87939698501, 35.3194032562)
            (1.93467336692, 41.6302647928)
            (1.98994974883, 48.6624719468)
            (2.04522613073, 56.4115894535)
            (2.10050251264, 64.8532409062)
            (2.15577889455, 73.9408107769)
            (2.21105527646, 83.6037792744)
            (2.26633165837, 93.7468726306)
            (2.32160804028, 104.250187448)
            (2.37688442219, 114.970409306)
            (2.4321608041, 125.743193976)
            (2.48743718601, 136.386716776)
            (2.54271356792, 146.706325329)
            (2.59798994983, 156.500158078)
            (2.65326633173, 165.565520633)
            (2.70854271364, 173.705750057)
            (2.76381909555, 180.737249017)
            (2.81909547746, 186.496341892)
            (2.87437185937, 190.845596938)
            (2.92964824128, 193.67927415)
            (2.98492462319, 194.927597517)
            (3.0402010051, 194.559610689)
            (3.09547738701, 192.584452888)
            (3.15075376892, 189.05098166)
            (3.20603015082, 184.045764187)
            (3.26130653273, 177.689552465)
            (3.31658291464, 170.132442734)
            (3.37185929655, 161.547989965)
            (3.42713567846, 152.126599321)
            (3.48241206037, 142.068544939)
            (3.53768844228, 131.576971181)
            (3.59296482419, 120.851213227)
            (3.6482412061, 110.080735199)
            (3.70351758801, 99.4399289527)
            (3.75879396992, 89.08395018)
            (3.81407035182, 79.1456963513)
            (3.86934673373, 69.7339586352)
            (3.92462311564, 60.9327125626)
            (3.97989949755, 52.801453975)
            (4.03517587946, 45.3764408318)
            (4.09045226137, 38.6726695445)
            (4.14572864328, 32.6863971343)
            (4.20100502519, 27.3980170133)
            (4.2562814071, 22.7751049096)
            (4.31155778901, 18.7754700454)
            (4.36683417091, 15.3500723852)
            (4.42211055282, 12.4456967348)
            (4.47738693473, 10.007305959)
            (4.53266331664, 7.9800262264)
            (4.58793969855, 6.31074506774)
            (4.64321608046, 4.94932680753)
            (4.69849246237, 3.8494687967)
            (4.75376884428, 2.96923555861)
            (4.80904522619, 2.27131662234)
            (4.8643216081, 1.7230579453)
            (4.91959799001, 1.29631713998)
            (4.97487437191, 0.96719005225)
            (5.03015075382, 0.715651450608)
            (5.08542713573, 0.525146490997)
            (5.14070351764, 0.382162929137)
            (5.19597989955, 0.275807342184)
            (5.25125628146, 0.197402326403)
            (5.30653266337, 0.140116050442)
            (5.36180904528, 0.098630829272)
            (5.41708542719, 0.0688536011291)
            (5.4723618091, 0.0476683161627)
            (5.527638191, 0.0327282014829)
            (5.58291457291, 0.0222845385861)
            (5.63819095482, 0.0150478457028)
            (5.69346733673, 0.0100770680127)
            (5.74874371864, 0.00669242059747)
            (5.80402010055, 0.00440779554637)
            (5.85929648246, 0.00287904752188)
            (5.91457286437, 0.00186494231151)
            (5.96984924628, 0.00119803953034)
            (6.02512562819, 0.000763248696503)
            (6.08040201009, 0.000482225499326)
            (6.135678392, 0.000302150591494)
            (6.19095477391, 0.000187752575516)
            (6.24623115582, 0.000115701112477)
            (6.30150753773, 7.07095939689e-05)
            (6.35678391964, 4.28556739506e-05)
            (6.41206030155, 2.57589106195e-05)
            (6.46733668346, 1.5354505425e-05)
            (6.52261306537, 9.07681218783e-06)
            (6.57788944728, 5.32132817816e-06)
            (6.63316582919, 3.09382621512e-06)
            (6.68844221109, 1.78386080635e-06)
            (6.743718593, 1.02003527128e-06)
            (6.79899497491, 5.78440365751e-07)
            (6.85427135682, 3.25305324661e-07)
            (6.90954773873, 1.81431599358e-07)
            (6.96482412064, 1.00351498767e-07)
            (7.02010050255, 5.50457729457e-08)
            (7.07537688446, 2.9944237784e-08)
            (7.13065326637, 1.61544333305e-08)
            (7.18592964828, 8.64289766776e-09)
            (7.24120603018, 4.585811412e-09)
            (7.29648241209, 2.41302703471e-09)
            (7.351758794, 1.25920767366e-09)
            (7.40703517591, 6.51660973353e-10)
            (7.46231155782, 3.34453111779e-10)
            (7.51758793973, 1.70230724067e-10)
            (7.57286432164, 8.59270355189e-11)
            (7.62814070355, 4.30141047637e-11)
            (7.68341708546, 2.13540927772e-11)
            (7.73869346737, 1.0513335872e-11)
            (7.79396984928, 5.13321116413e-12)
            (7.84924623118, 2.48557514955e-12)
            (7.90452261309, 1.19358634993e-12)
            (7.959798995, 5.68420805674e-13)
            (8.01507537691, 2.68457329276e-13)
            (8.07035175882, 1.2573892468e-13)
            (8.12562814073, 5.84054528584e-14)
            (8.18090452264, 2.69045806033e-14)
            (8.23618090455, 1.2291028749e-14)
            (8.29145728646, 5.56851532992e-15)
            (8.34673366837, 2.50195665674e-15)
            (8.40201005027, 1.11483173048e-15)
            (8.45728643218, 4.92638146559e-16)
            (8.51256281409, 2.15891692593e-16)
            (8.567839196, 9.38281172595e-17)
            (8.62311557791, 4.04407545158e-17)
            (8.67839195982, 1.72860066094e-17)
            (8.73366834173, 7.32755820406e-18)
            (8.78894472364, 3.08044214112e-18)
            (8.84422110555, 1.28426903772e-18)
            (8.89949748746, 5.30992205148e-19)
            (8.95477386937, 2.17725580103e-19)
            (9.01005025127, 8.8536009603e-20)
            (9.06532663318, 3.57042248515e-20)
            (9.12060301509, 1.42793503984e-20)
            (9.175879397, 5.66352040894e-21)
            (9.23115577891, 2.22768441261e-21)
            (9.28643216082, 8.68980516821e-22)
            (9.34170854273, 3.36167395794e-22)
            (9.39698492464, 1.28970480343e-22)
            (9.45226130655, 4.90697925591e-23)
            (9.50753768846, 1.85151529414e-23)
            (9.56281407036, 6.92834591424e-24)
            (9.61809045227, 2.57111202051e-24)
            (9.67336683418, 9.46240673485e-25)
            (9.72864321609, 3.45359496305e-25)
            (9.783919598, 1.25005859252e-25)
            (9.83919597991, 4.48723189749e-26)
            (9.89447236182, 1.59740792922e-26)
            (9.94974874373, 5.63952251215e-27)
            (10.0050251256, 1.97450396664e-27)
            (10.0603015075, 6.85587457174e-28)
            (10.1155778895, 2.36078748229e-28)
            (10.1708542714, 8.06195001205e-29)
            (10.2261306533, 2.73031330468e-29)
            (10.2814070352, 9.1700994769e-30)
            (10.3366834171, 3.05439208434e-30)
            (10.391959799, 1.00893851569e-30)
            (10.4472361809, 3.30516993952e-31)
            (10.5025125628, 1.07377197336e-31)
            (10.5577889447, 3.45954920029e-32)
            (10.6130653266, 1.10539153788e-32)
            (10.6683417085, 3.50269189129e-33)
            (10.7236180905, 1.10072014657e-33)
            (10.7788944724, 3.43037083325e-34)
            (10.8341708543, 1.06021612171e-34)
            (10.8894472362, 3.24965243507e-35)
            (10.9447236181, 9.87799042173e-36)
            (11.0, 2.97775844387e-36)
    };
    

    
    % Draw series plot
    \addplot[no markers,solid] coordinates {
            (1e-10, 5.78362179496e-06)
            (0.055276382009, 9.05346928034e-06)
            (0.110552763918, 1.4083871755e-05)
            (0.165829145827, 2.17731354423e-05)
            (0.221105527736, 3.34512051956e-05)
            (0.276381909645, 5.10733620392e-05)
            (0.331658291554, 7.74941659188e-05)
            (0.386934673463, 0.000116851812178)
            (0.442211055372, 0.000175103077295)
            (0.497487437281, 0.000260761810613)
            (0.55276381919, 0.000385910042361)
            (0.608040201099, 0.00056757081461)
            (0.663316583009, 0.000829556390325)
            (0.718592964918, 0.00120493509527)
            (0.773869346827, 0.00173929511643)
            (0.829145728736, 0.00249502432449)
            (0.884422110645, 0.00355687148113)
            (0.939698492554, 0.00503910543599)
            (0.994974874463, 0.00709464387591)
            (1.05025125637, 0.00992657979986)
            (1.10552763828, 0.0138025890863)
            (1.16080402019, 0.0190727520285)
            (1.2160804021, 0.0261913599257)
            (1.27135678401, 0.035743297669)
            (1.32663316592, 0.0484755862442)
            (1.38190954783, 0.065334625317)
            (1.43718592974, 0.087509584653)
            (1.49246231164, 0.116482242522)
            (1.54773869355, 0.154083348021)
            (1.60301507546, 0.202555282028)
            (1.65829145737, 0.264620400153)
            (1.71356783928, 0.343553955986)
            (1.76884422119, 0.443259924887)
            (1.8241206031, 0.56834738502)
            (1.87939698501, 0.724204379131)
            (1.93467336692, 0.917065403468)
            (1.98994974883, 1.15406788558)
            (2.04522613073, 1.44329226789)
            (2.10050251264, 1.79377966683)
            (2.15577889455, 2.21552059492)
            (2.21105527646, 2.71940798843)
            (2.26633165837, 3.31714785195)
            (2.32160804028, 4.02112128531)
            (2.37688442219, 4.84419256254)
            (2.4321608041, 5.79945933303)
            (2.48743718601, 6.89994293745)
            (2.54271356792, 8.15821926886)
            (2.59798994983, 9.58599352027)
            (2.65326633173, 11.1936254616)
            (2.70854271364, 12.9896154556)
            (2.76381909555, 14.9800650871)
            (2.81909547746, 17.1681298388)
            (2.87437185937, 19.5534844675)
            (2.92964824128, 22.1318243707)
            (2.98492462319, 24.8944280306)
            (3.0402010051, 27.8278063564)
            (3.09547738701, 30.9134642159)
            (3.15075376892, 34.127797516)
            (3.20603015082, 37.4421457975)
            (3.26130653273, 40.8230154706)
            (3.31658291464, 44.2324826578)
            (3.37185929655, 47.6287773454)
            (3.42713567846, 50.9670424901)
            (3.48241206037, 54.2002532777)
            (3.53768844228, 57.2802733358)
            (3.59296482419, 60.1590168634)
            (3.6482412061, 62.7896788341)
            (3.70351758801, 65.127990129)
            (3.75879396992, 67.1334510457)
            (3.81407035182, 68.7704954082)
            (3.86934673373, 70.0095386413)
            (3.92462311564, 70.8278667055)
            (3.97989949755, 71.2103285872)
            (4.03517587946, 71.1498028519)
            (4.09045226137, 70.6474181891)
            (4.14572864328, 69.7125184043)
            (4.20100502519, 68.3623733778)
            (4.2562814071, 66.6216484808)
            (4.31155778901, 64.5216552222)
            (4.36683417091, 62.0994149298)
            (4.42211055282, 59.3965745632)
            (4.47738693473, 56.4582189601)
            (4.53266331664, 53.3316267019)
            (4.58793969855, 50.0650172677)
            (4.64321608046, 46.70633531)
            (4.69849246237, 43.3021139214)
            (4.75376884428, 39.8964530091)
            (4.80904522619, 36.5301417616)
            (4.8643216081, 33.2399461467)
            (4.91959799001, 30.0580739278)
            (4.97487437191, 27.0118212964)
            (5.03015075382, 24.1233973456)
            (5.08542713573, 21.4099156173)
            (5.14070351764, 18.8835361392)
            (5.19597989955, 16.551736915)
            (5.25125628146, 14.4176908361)
            (5.30653266337, 12.4807224467)
            (5.36180904528, 10.7368188227)
            (5.41708542719, 9.17916987132)
            (5.4723618091, 7.79871540736)
            (5.527638191, 6.58467918372)
            (5.58291457291, 5.52507338984)
            (5.63819095482, 4.60716074187)
            (5.69346733673, 3.81786494532)
            (5.74874371864, 3.14412381994)
            (5.80402010055, 2.5731825825)
            (5.85929648246, 2.09282757123)
            (5.91457286437, 1.69156299582)
            (5.96984924628, 1.35873507567)
            (6.02512562819, 1.08460919128)
            (6.08040201009, 0.860406449008)
            (6.135678392, 0.678306402222)
            (6.19095477391, 0.531422647062)
            (6.24623115582, 0.413757694527)
            (6.30150753773, 0.320142985851)
            (6.35678391964, 0.246169238259)
            (6.41206030155, 0.18811154712)
            (6.46733668346, 0.14285288368)
            (6.52261306537, 0.107808859281)
            (6.57788944728, 0.0808559106643)
            (6.63316582919, 0.0602644195235)
            (6.68844221109, 0.0446377263501)
            (6.743718593, 0.0328575392982)
            (6.79899497491, 0.0240358723778)
            (6.85427135682, 0.017473368085)
            (6.90954773873, 0.0126236587148)
            (6.96482412064, 0.00906328729148)
            (7.02010050255, 0.00646663176741)
            (7.07537688446, 0.00458524344857)
            (7.13065326637, 0.00323101183985)
            (7.18592964828, 0.00226259374806)
            (7.24120603018, 0.00157458641036)
            (7.29648241209, 0.00108897597355)
            (7.351758794, 0.000748448598849)
            (7.40703517591, 0.000511207885569)
            (7.46231155782, 0.000346996418168)
            (7.51758793973, 0.0002340692277)
            (7.57286432164, 0.000156911775065)
            (7.62814070355, 0.000104534255719)
            (7.68341708546, 6.92075715347e-05)
            (7.73869346737, 4.55344895069e-05)
            (7.79396984928, 2.97727682266e-05)
            (7.84924623118, 1.93459403321e-05)
            (7.90452261309, 1.24925861829e-05)
            (7.959798995, 8.01690518512e-06)
            (8.01507537691, 5.11273193882e-06)
            (8.07035175882, 3.24034447705e-06)
            (8.12562814073, 2.04089764048e-06)
            (8.18090452264, 1.27744775683e-06)
            (8.23618090455, 7.94615359211e-07)
            (8.29145728646, 4.91204849122e-07)
            (8.34673366837, 3.01758990683e-07)
            (8.40201005027, 1.84225470459e-07)
            (8.45728643218, 1.11771482915e-07)
            (8.51256281409, 6.73913621149e-08)
            (8.567839196, 4.03802811827e-08)
            (8.62311557791, 2.40450855151e-08)
            (8.67839195982, 1.42290267552e-08)
            (8.73366834173, 8.36788973626e-09)
            (8.78894472364, 4.89044731477e-09)
            (8.84422110555, 2.84035813709e-09)
            (8.89949748746, 1.63941726545e-09)
            (8.95477386937, 9.40367884744e-10)
            (9.01005025127, 5.36040934757e-10)
            (9.06532663318, 3.03661690735e-10)
            (9.12060301509, 1.7095190257e-10)
            (9.175879397, 9.5642243335e-11)
            (9.23115577891, 5.31762187845e-11)
            (9.28643216082, 2.93817076994e-11)
            (9.34170854273, 1.61334964565e-11)
            (9.39698492464, 8.80383399701e-12)
            (9.45226130655, 4.77427108545e-12)
            (9.50753768846, 2.57296683028e-12)
            (9.56281407036, 1.37801258218e-12)
            (9.61809045227, 7.33439104458e-13)
            (9.67336683418, 3.87942019732e-13)
            (9.72864321609, 2.03920771973e-13)
            (9.783919598, 1.06524129384e-13)
            (9.83919597991, 5.53001619464e-14)
            (9.89447236182, 2.85296665479e-14)
            (9.94974874373, 1.46271215749e-14)
            (10.0050251256, 7.45268781233e-15)
            (10.0603015075, 3.77362612955e-15)
            (10.1155778895, 1.89887638862e-15)
            (10.1708542714, 9.4956872248e-16)
            (10.2261306533, 4.71897841353e-16)
            (10.2814070352, 2.33056630483e-16)
            (10.3366834171, 1.14384405271e-16)
            (10.391959799, 5.57909885411e-17)
            (10.4472361809, 2.70428947293e-17)
            (10.5025125628, 1.30266930941e-17)
            (10.5577889447, 6.23601380656e-18)
            (10.6130653266, 2.96668774466e-18)
            (10.6683417085, 1.40258273894e-18)
            (10.7236180905, 6.5898726985e-19)
            (10.7788944724, 3.07692877382e-19)
            (10.8341708543, 1.42774202828e-19)
            (10.8894472362, 6.5837590402e-20)
            (10.9447236181, 3.01710201054e-20)
            (11.0, 1.37403546905e-20)
    };
    

    
    % Draw series plot
    \addplot[no markers,solid] coordinates {
            (1e-10, 4.47079031791e-08)
            (0.055276382009, 7.00277641055e-08)
            (0.110552763918, 1.09141451372e-07)
            (0.165829145827, 1.69255463211e-07)
            (0.221105527736, 2.61173538367e-07)
            (0.276381909645, 4.01004417037e-07)
            (0.331658291554, 6.12636187172e-07)
            (0.386934673463, 9.31300084028e-07)
            (0.442211055372, 1.40867285036e-06)
            (0.497487437281, 2.12013811171e-06)
            (0.55276381919, 3.17505803452e-06)
            (0.608040201099, 4.73121505312e-06)
            (0.663316583009, 7.01499258574e-06)
            (0.718592964918, 1.03494018455e-05)
            (0.773869346827, 1.51927639393e-05)
            (0.829145728736, 2.21917645594e-05)
            (0.884422110645, 3.2253762867e-05)
            (0.939698492554, 4.66447153794e-05)
            (0.994974874463, 6.71209372901e-05)
            (1.05025125637, 9.61052435731e-05)
            (1.10552763828, 0.00013692087357)
            (1.16080402019, 0.00019410009376)
            (1.2160804021, 0.000273788583776)
            (1.27135678401, 0.000384271726569)
            (1.32663316592, 0.000536654820021)
            (1.38190954783, 0.000745736059062)
            (1.43718592974, 0.00103111892652)
            (1.49246231164, 0.00141861935228)
            (1.54773869355, 0.00194203256366)
            (1.60301507546, 0.0026453347795)
            (1.65829145737, 0.00358540551496)
            (1.71356783928, 0.00483536684997)
            (1.76884422119, 0.00648864600816)
            (1.8241206031, 0.00866387625629)
            (1.87939698501, 0.0115107575409)
            (1.93467336692, 0.0152170013046)
            (1.98994974883, 0.0200164822514)
            (2.04522613073, 0.0261987119574)
            (2.10050251264, 0.0341197335152)
            (2.15577889455, 0.0442145111141)
            (2.21105527646, 0.0570108518688)
            (2.26633165837, 0.0731448476547)
            (2.32160804028, 0.0933777607939)
            (2.37688442219, 0.118614198116)
            (2.4321608041, 0.14992132275)
            (2.48743718601, 0.188548742266)
            (2.54271356792, 0.235948586764)
            (2.59798994983, 0.293795153567)
            (2.65326633173, 0.364003350067)
            (2.70854271364, 0.448745018127)
            (2.76381909555, 0.550462078876)
            (2.81909547746, 0.671875303877)
            (2.87437185937, 0.815987406899)
            (2.92964824128, 0.986079070399)
            (2.98492462319, 1.1856964835)
            (3.0402010051, 1.4186289851)
            (3.09547738701, 1.68887548785)
            (3.15075376892, 2.0005985152)
            (3.20603015082, 2.35806492336)
            (3.26130653273, 2.76557270576)
            (3.31658291464, 3.22736369233)
            (3.37185929655, 3.74752245414)
            (3.42713567846, 4.3298622981)
            (3.48241206037, 4.97779987112)
            (3.53768844228, 5.69422056841)
            (3.59296482419, 6.48133763076)
            (3.6482412061, 7.3405484891)
            (3.70351758801, 8.27229253829)
            (3.75879396992, 9.27591505653)
            (3.81407035182, 10.3495423966)
            (3.86934673373, 11.4899738219)
            (3.92462311564, 12.6925954124)
            (3.97989949755, 13.9513212968)
            (4.03517587946, 15.2585670578)
            (4.09045226137, 16.6052595019)
            (4.14572864328, 17.9808860849)
            (4.20100502519, 19.3735861612)
            (4.2562814071, 20.7702849026)
            (4.31155778901, 22.1568692647)
            (4.36683417091, 23.5184038094)
            (4.42211055282, 24.8393825972)
            (4.47738693473, 26.1040118047)
            (4.53266331664, 27.2965162807)
            (4.58793969855, 28.4014620053)
            (4.64321608046, 29.4040854143)
            (4.69849246237, 30.2906198798)
            (4.75376884428, 31.0486093149)
            (4.80904522619, 31.6671989516)
            (4.8643216081, 32.1373938187)
            (4.91959799001, 32.4522763219)
            (4.97487437191, 32.6071755743)
            (5.03015075382, 32.5997826817)
            (5.08542713573, 32.4302080073)
            (5.14070351764, 32.1009784306)
            (5.19597989955, 31.6169746947)
            (5.25125628146, 30.9853110152)
            (5.30653266337, 30.2151610982)
            (5.36180904528, 29.3175365172)
            (5.41708542719, 28.3050249278)
            (5.4723618091, 27.191496813)
            (5.527638191, 25.9917902927)
            (5.58291457291, 24.7213839723)
            (5.63819095482, 23.3960678481)
            (5.69346733673, 22.0316219348)
            (5.74874371864, 20.6435115692)
            (5.80402010055, 19.246607327)
            (5.85929648246, 17.8549362109)
            (5.91457286437, 16.4814693195)
            (5.96984924628, 15.1379496362)
            (6.02512562819, 13.834761987)
            (6.08040201009, 12.5808456494)
            (6.135678392, 11.3836486399)
            (6.19095477391, 10.2491214013)
            (6.24623115582, 9.18174650688)
            (6.30150753773, 8.18460012111)
            (6.35678391964, 7.25944032101)
            (6.41206030155, 6.40681699839)
            (6.46733668346, 5.62619791298)
            (6.52261306537, 4.9161055383)
            (6.57788944728, 4.27425960562)
            (6.63316582919, 3.69772067294)
            (6.68844221109, 3.18303059128)
            (6.743718593, 2.72634636918)
            (6.79899497491, 2.32356461342)
            (6.85427135682, 1.97043441357)
            (6.90954773873, 1.66265721071)
            (6.96482412064, 1.39597282068)
            (7.02010050255, 1.16623134971)
            (7.07537688446, 0.969451231665)
            (7.13065326637, 0.801864022263)
            (7.18592964828, 0.659946904046)
            (7.24120603018, 0.540444087253)
            (7.29648241209, 0.440378441363)
            (7.351758794, 0.357054767705)
            (7.40703517591, 0.288056135046)
            (7.46231155782, 0.231234658341)
            (7.51758793973, 0.184698017424)
            (7.57286432164, 0.146792898282)
            (7.62814070355, 0.116086405299)
            (7.68341708546, 0.0913463476282)
            (7.73869346737, 0.0715211547411)
            (7.79396984928, 0.0557200316472)
            (7.84924623118, 0.0431938282802)
            (7.90452261309, 0.0333169737478)
            (7.959798995, 0.0255707167867)
            (8.01507537691, 0.0195278200299)
            (8.07035175882, 0.0148387778124)
            (8.12562814073, 0.0112195646734)
            (8.18090452264, 0.00844087340709)
            (8.23618090455, 0.00631876601735)
            (8.29145728646, 0.00470663659254)
            (8.34673366837, 0.00348837018886)
            (8.40201005027, 0.00257257456888)
            (8.45728643218, 0.00188776045453)
            (8.51256281409, 0.00137834933911)
            (8.567839196, 0.00100139455787)
            (8.62311557791, 0.000723910136206)
            (8.67839195982, 0.000520712010165)
            (8.73366834173, 0.00037268682327)
            (8.78894472364, 0.000265414098093)
            (8.84422110555, 0.000188077759953)
            (8.89949748746, 0.000132612482704)
            (8.95477386937, 9.30389657973e-05)
            (9.01005025127, 6.49499541653e-05)
            (9.06532663318, 4.51155576598e-05)
            (9.12060301509, 3.11822405265e-05)
            (9.175879397, 2.14447906414e-05)
            (9.23115577891, 1.46747195626e-05)
            (9.28643216082, 9.99197449045e-06)
            (9.34170854273, 6.76965176627e-06)
            (9.39698492464, 4.56367645889e-06)
            (9.45226130655, 3.06123642668e-06)
            (9.50753768846, 2.04320716877e-06)
            (9.56281407036, 1.35694248804e-06)
            (9.61809045227, 8.96693424075e-07)
            (9.67336683418, 5.89603475489e-07)
            (9.72864321609, 3.85753247707e-07)
            (9.783919598, 2.51126563512e-07)
            (9.83919597991, 1.62670657668e-07)
            (9.89447236182, 1.04847794063e-07)
            (9.94974874373, 6.72423493981e-08)
            (10.0050251256, 4.29101434705e-08)
            (10.0603015075, 2.72464898635e-08)
            (10.1155778895, 1.72145098178e-08)
            (10.1708542714, 1.08221180542e-08)
            (10.2261306533, 6.7696057973e-09)
            (10.2814070352, 4.21354851637e-09)
            (10.3366834171, 2.60955301442e-09)
            (10.391959799, 1.60811741734e-09)
            (10.4472361809, 9.86058982352e-10)
            (10.5025125628, 6.01618992022e-10)
            (10.5577889447, 3.65236092719e-10)
            (10.6130653266, 2.20627348019e-10)
            (10.6683417085, 1.32610681713e-10)
            (10.7236180905, 7.93105893046e-11)
            (10.7788944724, 4.71973214647e-11)
            (10.8341708543, 2.79471183493e-11)
            (10.8894472362, 1.64660800378e-11)
            (10.9447236181, 9.65332474399e-12)
            (11.0, 5.63115027116e-12)
    };
    

    
    % Draw series plot
    \addplot[no markers,solid] coordinates {
            (1e-10, 3.98675366542e-10)
            (0.055276382009, 6.24720707575e-10)
            (0.110552763918, 9.7487063726e-10)
            (0.165829145827, 1.51496523347e-09)
            (0.221105527736, 2.34451454649e-09)
            (0.276381909645, 3.61324812033e-09)
            (0.331658291554, 5.54545554017e-09)
            (0.386934673463, 8.47561573652e-09)
            (0.442211055372, 1.29003020213e-08)
            (0.497487437281, 1.95534360851e-08)
            (0.55276381919, 2.95148705441e-08)
            (0.608040201099, 4.43663057514e-08)
            (0.663316583009, 6.641409232e-08)
            (0.718592964918, 9.90060659086e-08)
            (0.773869346827, 1.46979902384e-07)
            (0.829145728736, 2.17294480026e-07)
            (0.884422110645, 3.19914571586e-07)
            (0.939698492554, 4.69044381888e-07)
            (0.994974874463, 6.84838935459e-07)
            (1.05025125637, 9.9576655887e-07)
            (1.10552763828, 1.44185377669e-06)
            (1.16080402019, 2.0791196839e-06)
            (1.2160804021, 2.98560500011e-06)
            (1.27135678401, 4.26952732384e-06)
            (1.32663316592, 6.08025556229e-06)
            (1.38190954783, 8.62300144156e-06)
            (1.43718592974, 1.21783842375e-05)
            (1.49246231164, 1.71283478565e-05)
            (1.54773869355, 2.39903102849e-05)
            (1.60301507546, 3.34619190548e-05)
            (1.65829145737, 4.64793891882e-05)
            (1.71356783928, 6.42931299e-05)
            (1.76884422119, 8.85652419489e-05)
            (1.8241206031, 0.000121494508064)
            (1.87939698501, 0.000165975722927)
            (1.93467336692, 0.000225801633588)
            (1.98994974883, 0.000305917399433)
            (2.04522613073, 0.000412739341041)
            (2.10050251264, 0.000554551829962)
            (2.15577889455, 0.000741998466324)
            (2.21105527646, 0.00098868617423)
            (2.26633165837, 0.00131192347439)
            (2.32160804028, 0.00173361690615)
            (2.37688442219, 0.00228135227826)
            (2.4321608041, 0.00298969001044)
            (2.48743718601, 0.00390170613531)
            (2.54271356792, 0.00507081237273)
            (2.59798994983, 0.00656288984181)
            (2.65326633173, 0.00845877117341)
            (2.70854271364, 0.0108571047278)
            (2.76381909555, 0.013877631974)
            (2.81909547746, 0.0176649044915)
            (2.87437185937, 0.0223924601413)
            (2.92964824128, 0.0282674683486)
            (2.98492462319, 0.0355358418031)
            (3.0402010051, 0.0444877959048)
            (3.09547738701, 0.0554638177425)
            (3.15075376892, 0.0688609831603)
            (3.20603015082, 0.0851395335622)
            (3.26130653273, 0.104829593722)
            (3.31658291464, 0.12853787839)
            (3.37185929655, 0.156954199584)
            (3.42713567846, 0.190857548986)
            (3.48241206037, 0.231121492093)
            (3.53768844228, 0.278718574081)
            (3.59296482419, 0.334723403594)
            (3.6482412061, 0.400314051807)
            (3.70351758801, 0.476771382398)
            (3.75879396992, 0.565475915885)
            (3.81407035182, 0.667901831494)
            (3.86934673373, 0.785607723821)
            (3.92462311564, 0.920223762077)
            (3.97989949755, 1.07343494864)
            (4.03517587946, 1.24696024228)
            (4.09045226137, 1.44252740066)
            (4.14572864328, 1.66184350608)
            (4.20100502519, 1.90656126777)
            (4.2562814071, 2.17824134041)
            (4.31155778901, 2.47831105946)
            (4.36683417091, 2.80802016504)
            (4.42211055282, 3.16839426204)
            (4.47738693473, 3.56018693863)
            (4.53266331664, 3.98383163125)
            (4.58793969855, 4.43939447391)
            (4.64321608046, 4.92652949477)
            (4.69849246237, 5.44443761611)
            (4.75376884428, 5.99183096671)
            (4.80904522619, 6.56690402213)
            (4.8643216081, 7.16731304264)
            (4.91959799001, 7.79016517714)
            (4.97487437191, 8.43201844197)
            (5.03015075382, 9.0888935673)
            (5.08542713573, 9.75629843345)
            (5.14070351764, 10.4292655009)
            (5.19597989955, 11.1024022791)
            (5.25125628146, 11.76995449)
            (5.30653266337, 12.4258811802)
            (5.36180904528, 13.0639406245)
            (5.41708542719, 13.6777854736)
            (5.4723618091, 14.2610652314)
            (5.527638191, 14.8075338288)
            (5.58291457291, 15.3111597993)
            (5.63819095482, 15.7662363732)
            (5.69346733673, 16.1674887033)
            (5.74874371864, 16.5101754148)
            (5.80402010055, 16.7901817545)
            (5.85929648246, 17.0041017828)
            (5.91457286437, 17.1493073136)
            (5.96984924628, 17.2240016539)
            (6.02512562819, 17.2272566036)
            (6.08040201009, 17.1590316535)
            (6.135678392, 17.0201748243)
            (6.19095477391, 16.8124051228)
            (6.24623115582, 16.5382771235)
            (6.30150753773, 16.2011286944)
            (6.35678391964, 15.8050133639)
            (6.41206030155, 15.3546192431)
            (6.46733668346, 14.8551767714)
            (6.52261306537, 14.3123578179)
            (6.57788944728, 13.7321688579)
            (6.63316582919, 13.1208410224)
            (6.68844221109, 12.4847198168)
            (6.743718593, 11.8301572046)
            (6.79899497491, 11.1634085699)
            (6.85427135682, 10.4905368189)
            (6.90954773873, 9.81732556652)
            (6.96482412064, 9.14920298927)
            (7.02010050255, 8.49117753851)
            (7.07537688446, 7.8477862973)
            (7.13065326637, 7.22305636134)
            (7.18592964828, 6.6204792329)
            (7.24120603018, 6.04299785528)
            (7.29648241209, 5.49300559247)
            (7.351758794, 4.97235618373)
            (7.40703517591, 4.4823834815)
            (7.46231155782, 4.02392961633)
            (7.51758793973, 3.59738012622)
            (7.57286432164, 3.20270453642)
            (7.62814070355, 2.83950087842)
            (7.68341708546, 2.50704268519)
            (7.73869346737, 2.20432708998)
            (7.79396984928, 1.93012277816)
            (7.84924623118, 1.68301668982)
            (7.90452261309, 1.46145853517)
            (7.959798995, 1.26380235928)
            (8.01507537691, 1.08834456832)
            (8.07035175882, 0.933358001743)
            (8.12562814073, 0.797121796337)
            (8.18090452264, 0.677946936521)
            (8.23618090455, 0.574197515765)
            (8.29145728646, 0.484307845464)
            (8.34673366837, 0.406795638663)
            (8.40201005027, 0.340271566476)
            (8.45728643218, 0.28344553575)
            (8.51256281409, 0.235130068667)
            (8.567839196, 0.194241180429)
            (8.62311557791, 0.159797152153)
            (8.67839195982, 0.130915584874)
            (8.73366834173, 0.106809099598)
            (8.78894472364, 0.0867800200625)
            (8.84422110555, 0.0702143414644)
            (8.89949748746, 0.0565752519018)
            (8.95477386937, 0.0453964355604)
            (9.01005025127, 0.0362753491081)
            (9.06532663318, 0.0288666266825)
            (9.12060301509, 0.0228757351286)
            (9.175879397, 0.0180529704588)
            (9.23115577891, 0.0141878592756)
            (9.28643216082, 0.0111040053276)
            (9.34170854273, 0.00865440149902)
            (9.39698492464, 0.00671721123536)
            (9.45226130655, 0.00519201048326)
            (9.50753768846, 0.00399647135248)
            (9.56281407036, 0.00306346156755)
            (9.61809045227, 0.0023385289838)
            (9.67336683418, 0.00177773763313)
            (9.72864321609, 0.00134582057615)
            (9.783919598, 0.00101461493373)
            (9.83919597991, 0.000761745548794)
            (9.89447236182, 0.000569525519126)
            (9.94974874373, 0.000424044115264)
            (10.0050251256, 0.000314415162274)
            (10.0603015075, 0.000232161665303)
            (10.1155778895, 0.000170715175164)
            (10.1708542714, 0.000125011030233)
            (10.2261306533, 9.11631092684e-05)
            (10.2814070352, 6.62040430596e-05)
            (10.3366834171, 4.78789353142e-05)
            (10.391959799, 3.44825238231e-05)
            (10.4472361809, 2.47313710887e-05)
            (10.5025125628, 1.7664116958e-05)
            (10.5577889447, 1.25640674087e-05)
            (10.6130653266, 8.89945010586e-06)
            (10.6683417085, 6.27755716029e-06)
            (10.7236180905, 4.40973780368e-06)
            (10.7788944724, 3.08481730269e-06)
            (10.8341708543, 2.14902128594e-06)
            (10.8894472362, 1.49089334011e-06)
            (10.9447236181, 1.03002323342e-06)
            (11.0, 7.08666748363e-07)
    };
    

    
    % Draw series plot
    \addplot[no markers,solid] coordinates {
            (1e-10, 3.92351303779e-12)
            (0.055276382009, 6.1499354523e-12)
            (0.110552763918, 9.60546777961e-12)
            (0.165829145827, 1.49492355853e-11)
            (0.221105527736, 2.31831239353e-11)
            (0.276381909645, 3.58242768385e-11)
            (0.331658291554, 5.5161416786e-11)
            (0.386934673463, 8.46342006198e-11)
            (0.442211055372, 1.29392434893e-10)
            (0.497487437281, 1.97117135231e-10)
            (0.55276381919, 2.99221211226e-10)
            (0.608040201099, 4.52598258858e-10)
            (0.663316583009, 6.82159427777e-10)
            (0.718592964918, 1.02449866526e-09)
            (0.773869346827, 1.53316682269e-09)
            (0.829145728736, 2.28623007164e-09)
            (0.884422110645, 3.39705796629e-09)
            (0.939698492554, 5.02965836244e-09)
            (0.994974874463, 7.4203863942e-09)
            (1.05025125637, 1.09085508037e-08)
            (1.10552763828, 1.59793864954e-08)
            (1.16080402019, 2.3324140471e-08)
            (1.2160804021, 3.39237379614e-08)
            (1.27135678401, 4.91647976417e-08)
            (1.32663316592, 7.09998312547e-08)
            (1.38190954783, 1.02167527072e-07)
            (1.43718592974, 1.4649437562e-07)
            (1.49246231164, 2.09305926263e-07)
            (1.54773869355, 2.97985137561e-07)
            (1.60301507546, 4.22727193668e-07)
            (1.65829145737, 5.97555535605e-07)
            (1.71356783928, 8.41683601419e-07)
            (1.76884422119, 1.18133198354e-06)
            (1.8241206031, 1.65214272803e-06)
            (1.87939698501, 2.30237291983e-06)
            (1.93467336692, 3.19710042597e-06)
            (1.98994974883, 4.42373795033e-06)
            (2.04522613073, 6.09923001015e-06)
            (2.10050251264, 8.37940409687e-06)
            (2.15577889455, 1.1471065578e-05)
            (2.21105527646, 1.56475697024e-05)
            (2.26633165837, 2.12687776745e-05)
            (2.32160804028, 2.88065118197e-05)
            (2.37688442219, 3.88768723354e-05)
            (2.4321608041, 5.22810701467e-05)
            (2.48743718601, 7.00567721263e-05)
            (2.54271356792, 9.35423513404e-05)
            (2.59798994983, 0.000124456890472)
            (2.65326633173, 0.000164999304705)
            (2.70854271364, 0.000217970533273)
            (2.76381909555, 0.000286923396867)
            (2.81909547746, 0.000376345428787)
            (2.87437185937, 0.000491880755423)
            (2.92964824128, 0.000640597916553)
            (2.98492462319, 0.000831311363056)
            (3.0402010051, 0.00107496522799)
            (3.09547738701, 0.00138508880872)
            (3.15075376892, 0.00177833398707)
            (3.20603015082, 0.00227510550664)
            (3.26130653273, 0.0029002955684)
            (3.31658291464, 0.00368413453292)
            (3.37185929655, 0.00466316955864)
            (3.42713567846, 0.00588138267723)
            (3.48241206037, 0.00739145901949)
            (3.53768844228, 0.00925621456284)
            (3.59296482419, 0.0115501907728)
            (3.6482412061, 0.0143614207569)
            (3.70351758801, 0.0177933679429)
            (3.75879396992, 0.0219670337459)
            (3.81407035182, 0.0270232251332)
            (3.86934673373, 0.0331249663744)
            (3.92462311564, 0.0404600315654)
            (3.97989949755, 0.0492435657541)
            (4.03517587946, 0.0597207527341)
            (4.09045226137, 0.0721694769329)
            (4.14572864328, 0.086902915486)
            (4.20100502519, 0.104271984793)
            (4.2562814071, 0.124667553926)
            (4.31155778901, 0.148522325602)
            (4.36683417091, 0.17631227445)
            (4.42211055282, 0.208557522648)
            (4.47738693473, 0.245822525071)
            (4.53266331664, 0.288715430724)
            (4.58793969855, 0.3378864849)
            (4.64321608046, 0.394025337979)
            (4.69849246237, 0.457857132624)
            (4.75376884428, 0.530137251948)
            (4.80904522619, 0.611644627515)
            (4.8643216081, 0.703173528025)
            (4.91959799001, 0.805523777666)
            (4.97487437191, 0.919489387041)
            (5.03015075382, 1.04584561939)
            (5.08542713573, 1.18533455979)
            (5.14070351764, 1.33864930451)
            (5.19597989955, 1.50641694054)
            (5.25125628146, 1.6891805404)
            (5.30653266337, 1.88738045283)
            (5.36180904528, 2.10133522404)
            (5.41708542719, 2.33122253547)
            (5.4723618091, 2.57706058903)
            (5.527638191, 2.83869040914)
            (5.58291457291, 3.11575955874)
            (5.63819095482, 3.40770778278)
            (5.69346733673, 3.71375509563)
            (5.74874371864, 4.03289281629)
            (5.80402010055, 4.36387802686)
            (5.85929648246, 4.70523188433)
            (5.91457286437, 5.05524215369)
            (5.96984924628, 5.41197025215)
            (6.02512562819, 5.77326300066)
            (6.08040201009, 6.13676917305)
            (6.135678392, 6.49996081579)
            (6.19095477391, 6.86015918731)
            (6.24623115582, 7.21456503771)
            (6.30150753773, 7.56029282133)
            (6.35678391964, 7.89440831125)
            (6.41206030155, 8.21396896955)
            (6.46733668346, 8.51606632501)
            (6.52261306537, 8.79786952444)
            (6.57788944728, 9.05666915896)
            (6.63316582919, 9.289920425)
            (6.68844221109, 9.49528466359)
            (6.743718593, 9.67066833284)
            (6.79899497491, 9.81425850697)
            (6.85427135682, 9.92455406087)
            (6.90954773873, 10.0003917905)
            (6.96482412064, 10.0409668333)
            (7.02010050255, 10.0458468882)
            (7.07537688446, 10.0149798827)
            (7.13065326637, 9.94869489815)
            (7.18592964828, 9.84769632888)
            (7.24120603018, 9.71305142167)
            (7.29648241209, 9.54617150252)
            (7.351758794, 9.34878735301)
            (7.40703517591, 9.12291933704)
            (7.46231155782, 8.87084299838)
            (7.51758793973, 8.59505094792)
            (7.57286432164, 8.29821193177)
            (7.62814070355, 7.98312801766)
            (7.68341708546, 7.65269085575)
            (7.73869346737, 7.30983796092)
            (7.79396984928, 6.95750992921)
            (7.84924623118, 6.59860944195)
            (7.90452261309, 6.23596283101)
            (7.959798995, 5.87228488071)
            (8.01507537691, 5.51014742985)
            (8.07035175882, 5.15195221548)
            (8.12562814073, 4.79990827295)
            (8.18090452264, 4.45601407789)
            (8.23618090455, 4.12204449044)
            (8.29145728646, 3.79954244243)
            (8.34673366837, 3.48981519908)
            (8.40201005027, 3.19393492935)
            (8.45728643218, 2.91274323698)
            (8.51256281409, 2.64685923714)
            (8.567839196, 2.39669071426)
            (8.62311557791, 2.16244786287)
            (8.67839195982, 1.9441590973)
            (8.73366834173, 1.74168841421)
            (8.78894472364, 1.55475380524)
            (8.84422110555, 1.3829462417)
            (8.89949748746, 1.22574878911)
            (8.95477386937, 1.0825554527)
            (9.01005025127, 0.952689404765)
            (9.06532663318, 0.835420298386)
            (9.12060301509, 0.729980427145)
            (9.175879397, 0.635579546059)
            (9.23115577891, 0.551418222353)
            (9.28643216082, 0.476699635327)
            (9.34170854273, 0.410639790816)
            (9.39698492464, 0.352476157039)
            (9.45226130655, 0.301474764177)
            (9.50753768846, 0.256935839776)
            (9.56281407036, 0.218198075665)
            (9.61809045227, 0.184641639974)
            (9.67336683418, 0.155690060024)
            (9.72864321609, 0.130811109061)
            (9.783919598, 0.109516832379)
            (9.83919597991, 0.0913628470502)
            (9.89447236182, 0.0759470448325)
            (9.94974874373, 0.0629078205652)
            (10.0050251256, 0.0519219391009)
            (10.0603015075, 0.0427021431447)
            (10.1155778895, 0.0349945928573)
            (10.1708542714, 0.0285762161783)
            (10.2261306533, 0.0232520369631)
            (10.2814070352, 0.0188525365385)
            (10.3366834171, 0.0152310934299)
            (10.391959799, 0.0122615360018)
            (10.4472361809, 0.00983583370714)
            (10.5025125628, 0.00786194465102)
            (10.5577889447, 0.00626183025919)
            (10.6130653266, 0.0049696419988)
            (10.6683417085, 0.0039300802833)
            (10.7236180905, 0.00309692183829)
            (10.7788944724, 0.0024317088277)
            (10.8341708543, 0.00190259084522)
            (10.8894472362, 0.00148330936436)
            (10.9447236181, 0.00115231331125)
            (11.0, 0.000891993980364)
    };
    

    
    % Draw series plot
    \addplot[no markers,solid] coordinates {
            (1e-10, 4.15023556101e-14)
            (0.055276382009, 6.50676202344e-14)
            (0.110552763918, 1.01695805513e-13)
            (0.165829145827, 1.58448123214e-13)
            (0.221105527736, 2.46103108717e-13)
            (0.276381909645, 3.8105971761e-13)
            (0.331658291554, 5.88186326565e-13)
            (0.386934673463, 9.05071206801e-13)
            (0.442211055372, 1.38834216824e-12)
            (0.497487437281, 2.12303044452e-12)
            (0.55276381919, 3.2363976026e-12)
            (0.608040201099, 4.91828239247e-12)
            (0.663316583009, 7.45093873666e-12)
            (0.718592964918, 1.12526410525e-11)
            (0.773869346827, 1.69411887035e-11)
            (0.829145728736, 2.5426069855e-11)
            (0.884422110645, 3.80417540222e-11)
            (0.939698492554, 5.67397962313e-11)
            (0.994974874463, 8.43647360141e-11)
            (1.05025125637, 1.2504895691e-10)
            (1.10552763828, 1.84775819199e-10)
            (1.16080402019, 2.72179956771e-10)
            (1.2160804021, 3.99680616476e-10)
            (1.27135678401, 5.85080971406e-10)
            (1.32663316592, 8.53817009441e-10)
            (1.38190954783, 1.24210861073e-09)
            (1.43718592974, 1.80135904246e-09)
            (1.49246231164, 2.60427556938e-09)
            (1.54773869355, 3.75335403519e-09)
            (1.60301507546, 5.39259825602e-09)
            (1.65829145737, 7.72364926635e-09)
            (1.71356783928, 1.10279036405e-08)
            (1.76884422119, 1.5696734867e-08)
            (1.8241206031, 2.22726362373e-08)
            (1.87939698501, 3.15050277845e-08)
            (1.93467336692, 4.44256766436e-08)
            (1.98994974883, 6.24502493737e-08)
            (2.04522613073, 8.75145459789e-08)
            (2.10050251264, 1.2225658278e-07)
            (2.15577889455, 1.70259048708e-07)
            (2.21105527646, 2.36370946308e-07)
            (2.26633165837, 3.27132676322e-07)
            (2.32160804028, 4.51335714513e-07)
            (2.37688442219, 6.20756701192e-07)
            (2.4321608041, 8.51116624732e-07)
            (2.48743718601, 1.16332931492e-06)
            (2.54271356792, 1.58512024213e-06)
            (2.59798994983, 2.15311731413e-06)
            (2.65326633173, 2.91554075133e-06)
            (2.70854271364, 3.93565009826e-06)
            (2.76381909555, 5.29614401126e-06)
            (2.81909547746, 7.10475379725e-06)
            (2.87437185937, 9.50132603971e-06)
            (2.92964824128, 1.26667544333e-05)
            (2.98492462319, 1.68341976673e-05)
            (3.0402010051, 2.23031104467e-05)
            (3.09547738701, 2.94567201742e-05)
            (3.15075376892, 3.87837040982e-05)
            (3.20603015082, 5.09049624763e-05)
            (3.26130653273, 6.66065439895e-05)
            (3.31658291464, 8.68799615059e-05)
            (3.37185929655, 0.000112971340235)
            (3.42713567846, 0.000146441066695)
            (3.48241206037, 0.000189235855432)
            (3.53768844228, 0.000243775419922)
            (3.59296482419, 0.000313056222215)
            (3.6482412061, 0.000400775079043)
            (3.70351758801, 0.000511475715073)
            (3.75879396992, 0.000650721669584)
            (3.81407035182, 0.000825299271845)
            (3.86934673373, 0.00104345469127)
            (3.92462311564, 0.00131516932663)
            (3.97989949755, 0.00165247800735)
            (4.03517587946, 0.00206983461914)
            (4.09045226137, 0.00258452981291)
            (4.14572864328, 0.00321716538511)
            (4.20100502519, 0.00399218970008)
            (4.2562814071, 0.0049384981314)
            (4.31155778901, 0.00609010189741)
            (4.36683417091, 0.0074868678232)
            (4.42211055282, 0.00917533044627)
            (4.47738693473, 0.0112095764647)
            (4.53266331664, 0.0136521997783)
            (4.58793969855, 0.0165753232739)
            (4.64321608046, 0.0200616810377)
            (4.69849246237, 0.0242057518416)
            (4.75376884428, 0.0291149315437)
            (4.80904522619, 0.0349107284939)
            (4.8643216081, 0.0417299621756)
            (4.91959799001, 0.0497259411977)
            (4.97487437191, 0.0590695924554)
            (5.03015075382, 0.0699505088924)
            (5.08542713573, 0.0825778789469)
            (5.14070351764, 0.0971812565797)
            (5.19597989955, 0.114011126928)
            (5.25125628146, 0.133339219283)
            (5.30653266337, 0.155458516456)
            (5.36180904528, 0.180682907828)
            (5.41708542719, 0.209346432816)
            (5.4723618091, 0.241802062139)
            (5.527638191, 0.278419966535)
            (5.58291457291, 0.319585226547)
            (5.63819095482, 0.365694942764)
            (5.69346733673, 0.417154713751)
            (5.74874371864, 0.474374458724)
            (5.80402010055, 0.537763573957)
            (5.85929648246, 0.607725425827)
            (5.91457286437, 0.684651199214)
            (5.96984924628, 0.768913137427)
            (6.02512562819, 0.860857228689)
            (6.08040201009, 0.960795414037)
            (6.135678392, 1.06899741186)
            (6.19095477391, 1.18568227465)
            (6.24623115582, 1.31100981336)
            (6.30150753773, 1.44507204305)
            (6.35678391964, 1.5878848204)
            (6.41206030155, 1.73937985706)
            (6.46733668346, 1.89939730388)
            (6.52261306537, 2.06767910703)
            (6.57788944728, 2.24386333944)
            (6.63316582919, 2.42747970753)
            (6.68844221109, 2.61794642451)
            (6.743718593, 2.81456862745)
            (6.79899497491, 3.01653849476)
            (6.85427135682, 3.22293719512)
            (6.90954773873, 3.43273876745)
            (6.96482412064, 3.64481599537)
            (7.02010050255, 3.85794829924)
            (7.07537688446, 4.07083162496)
            (7.13065326637, 4.28209026268)
            (7.18592964828, 4.49029048151)
            (7.24120603018, 4.69395581894)
            (7.29648241209, 4.89158381852)
            (7.351758794, 5.08166396634)
            (7.40703517591, 5.26269653884)
            (7.46231155782, 5.43321204144)
            (7.51758793973, 5.5917908915)
            (7.57286432164, 5.73708298094)
            (7.62814070355, 5.86782674388)
            (7.68341708546, 5.98286735436)
            (7.73869346737, 6.08117368789)
            (7.79396984928, 6.16185369876)
            (7.84924623118, 6.22416789283)
            (7.90452261309, 6.2675406111)
            (7.959798995, 6.29156888347)
            (8.01507537691, 6.29602866208)
            (8.07035175882, 6.28087829955)
            (8.12562814073, 6.24625919653)
            (8.18090452264, 6.19249360445)
            (8.23618090455, 6.12007963124)
            (8.29145728646, 6.02968355831)
            (8.34673366837, 5.92212963448)
            (8.40201005027, 5.7983875655)
            (8.45728643218, 5.65955796465)
            (8.51256281409, 5.50685606961)
            (8.567839196, 5.34159406214)
            (8.62311557791, 5.16516234966)
            (8.67839195982, 4.97901018092)
            (8.73366834173, 4.78462597176)
            (8.78894472364, 4.58351771117)
            (8.84422110555, 4.37719380349)
            (8.89949748746, 4.16714467987)
            (8.95477386937, 3.95482548241)
            (9.01005025127, 3.74164008812)
            (9.06532663318, 3.52892669936)
            (9.12060301509, 3.31794518271)
            (9.175879397, 3.10986629173)
            (9.23115577891, 2.9057628621)
            (9.28643216082, 2.70660302033)
            (9.34170854273, 2.51324540271)
            (9.39698492464, 2.32643633897)
            (9.45226130655, 2.14680891672)
            (9.50753768846, 1.97488380956)
            (9.56281407036, 1.81107172287)
            (9.61809045227, 1.65567728895)
            (9.67336683418, 1.50890422572)
            (9.72864321609, 1.37086156245)
            (9.783919598, 1.24157072982)
            (9.83919597991, 1.1209733117)
            (9.89447236182, 1.0089392604)
            (9.94974874373, 0.905275385829)
            (10.0050251256, 0.809733941535)
            (10.0603015075, 0.722021146101)
            (10.1155778895, 0.641805495848)
            (10.1708542714, 0.568725744228)
            (10.2261306533, 0.502398443375)
            (10.2814070352, 0.44242496374)
            (10.3366834171, 0.388397927857)
            (10.391959799, 0.339907013656)
            (10.4472361809, 0.296544100851)
            (10.5025125628, 0.257907750537)
            (10.5577889447, 0.223607022934)
            (10.6130653266, 0.193264651036)
            (10.6683417085, 0.166519598737)
            (10.7236180905, 0.143029040729)
            (10.7788944724, 0.122469808159)
            (10.8341708543, 0.104539348839)
            (10.8894472362, 0.0889562537668)
            (10.9447236181, 0.07546040313)
            (11.0, 0.0638127848877)
    };
    

    
    % Draw series plot
    \addplot[no markers,solid] coordinates {
            (1e-10, 4.63882917976e-16)
            (0.055276382009, 7.27404162251e-16)
            (0.110552763918, 1.13746901608e-15)
            (0.165829145827, 1.7737800327e-15)
            (0.221105527736, 2.75839425944e-15)
            (0.276381909645, 4.27769002731e-15)
            (0.331658291554, 6.61543908117e-15)
            (0.386934673463, 1.02024489467e-14)
            (0.442211055372, 1.56908538872e-14)
            (0.497487437281, 2.40649585906e-14)
            (0.55276381919, 3.68061211678e-14)
            (0.608040201099, 5.61372813693e-14)
            (0.663316583009, 8.53845107417e-14)
            (0.718592964918, 1.29509970411e-13)
            (0.773869346827, 1.95895160215e-13)
            (0.829145728736, 2.9548851205e-13)
            (0.884422110645, 4.4448170314e-13)
            (0.939698492554, 6.66750833984e-13)
            (0.994974874463, 9.97400632707e-13)
            (1.05025125637, 1.48789438058e-12)
            (1.10552763828, 2.21345634033e-12)
            (1.16080402019, 3.2837206873e-12)
            (1.2160804021, 4.85800380131e-12)
            (1.27135678401, 7.16713993528e-12)
            (1.32663316592, 1.05446048605e-11)
            (1.38190954783, 1.54707414586e-11)
            (1.43718592974, 2.26354086362e-11)
            (1.49246231164, 3.30264536736e-11)
            (1.54773869355, 4.80542648771e-11)
            (1.60301507546, 6.97265738472e-11)
            (1.65829145737, 1.00893016344e-10)
            (1.71356783928, 1.45586222882e-10)
            (1.76884422119, 2.0949605238e-10)
            (1.8241206031, 3.00626874164e-10)
            (1.87939698501, 4.30205690804e-10)
            (1.93467336692, 6.13932879872e-10)
            (1.98994974883, 8.73699380287e-10)
            (2.04522613073, 1.23993681423e-09)
            (2.10050251264, 1.75482362993e-09)
            (2.15577889455, 2.47664514728e-09)
            (2.21105527646, 3.48570387748e-09)
            (2.26633165837, 4.89230569382e-09)
            (2.32160804028, 6.84751629965e-09)
            (2.37688442219, 9.55760231463e-09)
            (2.4321608041, 1.3303356501e-08)
            (2.48743718601, 1.84658751656e-08)
            (2.54271356792, 2.55608300956e-08)
            (2.59798994983, 3.52838855105e-08)
            (2.65326633173, 4.85706871012e-08)
            (2.70854271364, 6.66758379841e-08)
            (2.76381909555, 9.12765276793e-08)
            (2.81909547746, 1.24608058829e-07)
            (2.87437185937, 1.69640499697e-07)
            (2.92964824128, 2.30308171835e-07)
            (2.98492462319, 3.11806773438e-07)
            (3.0402010051, 4.2097677315e-07)
            (3.09547738701, 5.6679644399e-07)
            (3.15075376892, 7.61013728252e-07)
            (3.20603015082, 1.01895324807e-06)
            (3.26130653273, 1.36054345408e-06)
            (3.31658291464, 1.81161942477e-06)
            (3.37185929655, 2.40556952117e-06)
            (3.42713567846, 3.1854093378e-06)
            (3.48241206037, 4.20638458738e-06)
            (3.53768844228, 5.53922617583e-06)
            (3.59296482419, 7.27420626928e-06)
            (3.6482412061, 9.52617417065e-06)
            (3.70351758801, 1.24407858887e-05)
            (3.75879396992, 1.62021819978e-05)
            (3.81407035182, 2.10424153684e-05)
            (3.86934673373, 2.72529841976e-05)
            (3.92462311564, 3.51988870697e-05)
            (3.97989949755, 4.53356860373e-05)
            (4.03517587946, 5.82301413602e-05)
            (4.09045226137, 7.45850678532e-05)
            (4.14572864328, 9.52691578681e-05)
            (4.20100502519, 0.000121352619612)
            (4.2562814071, 0.000154149591319)
            (4.31155778901, 0.000195268410873)
            (4.36683417091, 0.000246670945528)
            (4.42211055282, 0.00031074231548)
            (4.47738693473, 0.000390372475681)
            (4.53266331664, 0.000489051249289)
            (4.58793969855, 0.000610978529334)
            (4.64321608046, 0.000761191477684)
            (4.69849246237, 0.000945710646355)
            (4.75376884428, 0.00117170701861)
            (4.80904522619, 0.00144769200841)
            (4.8643216081, 0.00178373245756)
            (4.91959799001, 0.00219169262056)
            (4.97487437191, 0.00268550501732)
            (5.03015075382, 0.00328147185144)
            (5.08542713573, 0.0039985984261)
            (5.14070351764, 0.00485895962743)
            (5.19597989955, 0.00588810007617)
            (5.25125628146, 0.00711546795999)
            (5.30653266337, 0.00857488184269)
            (5.36180904528, 0.0103050288927)
            (5.41708542719, 0.0123499919782)
            (5.4723618091, 0.0147598019343)
            (5.527638191, 0.0175910100246)
            (5.58291457291, 0.0209072741944)
            (5.63819095482, 0.0247799511668)
            (5.69346733673, 0.0292886847692)
            (5.74874371864, 0.0345219791383)
            (5.80402010055, 0.0405777436471)
            (5.85929648246, 0.0475637945866)
            (5.91457286437, 0.0555982968471)
            (5.96984924628, 0.0648101271426)
            (6.02512562819, 0.0753391387665)
            (6.08040201009, 0.0873363065191)
            (6.135678392, 0.100963729387)
            (6.19095477391, 0.116394467853)
            (6.24623115582, 0.133812192442)
            (6.30150753773, 0.153410620372)
            (6.35678391964, 0.175392718004)
            (6.41206030155, 0.199969648278)
            (6.46733668346, 0.227359444549)
            (6.52261306537, 0.257785395193)
            (6.57788944728, 0.291474127125)
            (6.63316582919, 0.328653380962)
            (6.68844221109, 0.369549475914)
            (6.743718593, 0.414384468623)
            (6.79899497491, 0.463373017037)
            (6.85427135682, 0.516718967796)
            (6.90954773873, 0.574611693593)
            (6.96482412064, 0.637222215239)
            (7.02010050255, 0.704699151575)
            (7.07537688446, 0.77716454884)
            (7.13065326637, 0.854709649192)
            (7.18592964828, 0.937390665761)
            (7.24120603018, 1.02522463849)
            (7.29648241209, 1.11818545085)
            (7.351758794, 1.21620009225)
            (7.40703517591, 1.31914525377)
            (7.46231155782, 1.42684434653)
            (7.51758793973, 1.53906503119)
            (7.57286432164, 1.65551734441)
            (7.62814070355, 1.77585250327)
            (7.68341708546, 1.8996624615)
            (7.73869346737, 2.026480282)
            (7.79396984928, 2.15578137836)
            (7.84924623118, 2.28698566472)
            (7.90452261309, 2.4194606377)
            (7.959798995, 2.55252539697)
            (8.01507537691, 2.68545559298)
            (8.07035175882, 2.81748927098)
            (8.12562814073, 2.94783356115)
            (8.18090452264, 3.07567214498)
            (8.23618090455, 3.20017340925)
            (8.29145728646, 3.32049918108)
            (8.34673366837, 3.43581392131)
            (8.40201005027, 3.54529423923)
            (8.45728643218, 3.64813858006)
            (8.51256281409, 3.74357692783)
            (8.567839196, 3.83088036061)
            (8.62311557791, 3.90937029333)
            (8.67839195982, 3.97842724465)
            (8.73366834173, 4.0374989697)
            (8.78894472364, 4.08610780982)
            (8.84422110555, 4.12385712213)
            (8.89949748746, 4.15043666835)
            (8.95477386937, 4.1656268601)
            (9.01005025127, 4.16930177968)
            (9.06532663318, 4.16143091814)
            (9.12060301509, 4.14207959698)
            (9.175879397, 4.11140806548)
            (9.23115577891, 4.06966929105)
            (9.28643216082, 4.01720548512)
            (9.34170854273, 3.95444343149)
            (9.39698492464, 3.88188870618)
            (9.45226130655, 3.8001188983)
            (9.50753768846, 3.70977595914)
            (9.56281407036, 3.61155782092)
            (9.61809045227, 3.50620943824)
            (9.67336683418, 3.39451341261)
            (9.72864321609, 3.27728036445)
            (9.783919598, 3.15533921719)
            (9.83919597991, 3.02952755475)
            (9.89447236182, 2.90068220711)
            (9.94974874373, 2.76963020835)
            (10.0050251256, 2.63718025959)
            (10.0603015075, 2.50411481356)
            (10.1155778895, 2.37118288127)
            (10.1708542714, 2.23909364248)
            (10.2261306533, 2.10851092278)
            (10.2814070352, 1.98004858053)
            (10.3366834171, 1.8542668272)
            (10.391959799, 1.73166948612)
            (10.4472361809, 1.6127021765)
            (10.5025125628, 1.4977513932)
            (10.5577889447, 1.38714443782)
            (10.6130653266, 1.28115014388)
            (10.6683417085, 1.17998032785)
            (10.7236180905, 1.08379188946)
            (10.7788944724, 0.992689478197)
            (10.8341708543, 0.906728638954)
            (10.8894472362, 0.825919347753)
            (10.9447236181, 0.750229848739)
            (11.0, 0.679590705546)
    };
    

    
    % Draw series plot
    \addplot[no markers,solid] coordinates {
            (1e-10, 5.41535117479e-18)
            (0.055276382009, 8.49286472812e-18)
            (0.110552763918, 1.32861313532e-17)
            (0.165829145827, 2.07328798005e-17)
            (0.221105527736, 3.2272861269e-17)
            (0.276381909645, 5.01108841497e-17)
            (0.331658291554, 7.7614598092e-17)
            (0.386934673463, 1.19914448266e-16)
            (0.442211055372, 1.84806127753e-16)
            (0.497487437281, 2.84104409449e-16)
            (0.55276381919, 4.35668674431e-16)
            (0.608040201099, 6.66425255992e-16)
            (0.663316583009, 1.01686503374e-15)
            (0.718592964918, 1.54771835556e-15)
            (0.773869346827, 2.34983466269e-15)
            (0.829145728736, 3.55876599057e-15)
            (0.884422110645, 5.37623582419e-15)
            (0.939698492554, 8.10165867772e-15)
            (0.994974874463, 1.2178290447e-14)
            (1.05025125637, 1.82606177661e-14)
            (1.10552763828, 2.73124938967e-14)
            (1.16080402019, 4.07496611556e-14)
            (1.2160804021, 6.06461728657e-14)
            (1.27135678401, 9.00325498849e-14)
            (1.32663316592, 1.33325264606e-13)
            (1.38190954783, 1.9694370843e-13)
            (1.43718592974, 2.9019406202e-13)
            (1.49246231164, 4.26532064713e-13)
            (1.54773869355, 6.25362163586e-13)
            (1.60301507546, 9.14593838225e-13)
            (1.65829145737, 1.33426377114e-12)
            (1.71356783928, 1.94165412534e-12)
            (1.76884422119, 2.81850493057e-12)
            (1.8241206031, 4.08114928401e-12)
            (1.87939698501, 5.89471641044e-12)
            (1.93467336692, 8.49298006856e-12)
            (1.98994974883, 1.22060191358e-11)
            (2.04522613073, 1.74986579191e-11)
            (2.10050251264, 2.50237382754e-11)
            (2.15577889455, 3.5695740038e-11)
            (2.21105527646, 5.07922374372e-11)
            (2.26633165837, 7.20933242629e-11)
            (2.32160804028, 1.02072678659e-10)
            (2.37688442219, 1.44158656274e-10)
            (2.4321608041, 2.03090071374e-10)
            (2.48743718601, 2.85399634146e-10)
            (2.54271356792, 4.00068985595e-10)
            (2.59798994983, 5.59413716912e-10)
            (2.65326633173, 7.80275715121e-10)
            (2.70854271364, 1.08562496743e-09)
            (2.76381909555, 1.50670527338e-09)
            (2.81909547746, 2.08590030176e-09)
            (2.87437185937, 2.88055080058e-09)
            (2.92964824128, 3.96802393968e-09)
            (2.98492462319, 5.45242602067e-09)
            (3.0402010051, 7.47346548118e-09)
            (3.09547738701, 1.02181209081e-08)
            (3.15075376892, 1.39359569071e-08)
            (3.20603015082, 1.89591693276e-08)
            (3.26130653273, 2.57287430119e-08)
            (3.31658291464, 3.48284851954e-08)
            (3.37185929655, 4.70291745364e-08)
            (3.42713567846, 6.33456620037e-08)
            (3.48241206037, 8.51105026621e-08)
            (3.53768844228, 1.14068619318e-07)
            (3.59296482419, 1.52498638895e-07)
            (3.6482412061, 2.03367946404e-07)
            (3.70351758801, 2.7053022394e-07)
            (3.75879396992, 3.58976347166e-07)
            (3.81407035182, 4.75152073839e-07)
            (3.86934673373, 6.27359064534e-07)
            (3.92462311564, 8.26259524539e-07)
            (3.97989949755, 1.08550926178e-06)
            (4.03517587946, 1.42254934809e-06)
            (4.09045226137, 1.85959299591e-06)
            (4.14572864328, 2.42485188255e-06)
            (4.20100502519, 3.15405514921e-06)
            (4.2562814071, 4.09232487026e-06)
            (4.31155778901, 5.29648414225e-06)
            (4.36683417091, 6.83788831104e-06)
            (4.42211055282, 8.80588647878e-06)
            (4.47738693473, 1.13120395588e-05)
            (4.53266331664, 1.44952430262e-05)
            (4.58793969855, 1.85279273927e-05)
            (4.64321608046, 2.36235375464e-05)
            (4.69849246237, 3.00455236551e-05)
            (4.75376884428, 3.81181114977e-05)
            (4.80904522619, 4.82391589908e-05)
            (4.8643216081, 6.08954483489e-05)
            (4.91959799001, 7.66808097196e-05)
            (4.97487437191, 9.63175220922e-05)
            (5.03015075382, 0.000120681490488)
            (5.08542713573, 0.000150831754408)
            (5.14070351764, 0.000188044940554)
            (5.19597989955, 0.000233855332045)
            (5.25125628146, 0.00029010128547)
            (5.30653266337, 0.000358978784761)
            (5.36180904528, 0.000443102975075)
            (5.41708542719, 0.000545578568624)
            (5.4723618091, 0.00067008005496)
            (5.527638191, 0.000820942677825)
            (5.58291457291, 0.00100326515586)
            (5.63819095482, 0.0012230251216)
            (5.69346733673, 0.00148720822812)
            (5.74874371864, 0.001803951821)
            (5.80402010055, 0.00218270399079)
            (5.85929648246, 0.00263439870173)
            (5.91457286437, 0.00317164753344)
            (5.96984924628, 0.00380894836628)
            (6.02512562819, 0.00456291108551)
            (6.08040201009, 0.00545250006927)
            (6.135678392, 0.00649929285688)
            (6.19095477391, 0.00772775396512)
            (6.24623115582, 0.00916552232894)
            (6.30150753773, 0.0108437102894)
            (6.35678391964, 0.0127972114367)
            (6.41206030155, 0.0150650139448)
            (6.46733668346, 0.0176905153086)
            (6.52261306537, 0.0207218336287)
            (6.57788944728, 0.0242121097858)
            (6.63316582919, 0.0282197940278)
            (6.68844221109, 0.0328089096686)
            (6.743718593, 0.0380492857891)
            (6.79899497491, 0.0440167500659)
            (6.85427135682, 0.0507932721437)
            (6.90954773873, 0.0584670473594)
            (6.96482412064, 0.0671325101265)
            (7.02010050255, 0.0768902659553)
            (7.07537688446, 0.0878469309174)
            (7.13065326637, 0.100114867428)
            (7.18592964828, 0.113811805517)
            (7.24120603018, 0.129060339341)
            (7.29648241209, 0.145987289559)
            (7.351758794, 0.164722923407)
            (7.40703517591, 0.185400025823)
            (7.46231155782, 0.208152816874)
            (7.51758793973, 0.233115712954)
            (7.57286432164, 0.260421931799)
            (7.62814070355, 0.290201944241)
            (7.68341708546, 0.322581778813)
            (7.73869346737, 0.357681188759)
            (7.79396984928, 0.395611694607)
            (7.84924623118, 0.436474519275)
            (7.90452261309, 0.480358436479)
            (7.959798995, 0.527337557057)
            (8.01507537691, 0.577469081516)
            (8.07035175882, 0.630791050613)
            (8.12562814073, 0.687320128966)
            (8.18090452264, 0.747049459453)
            (8.23618090455, 0.809946628415)
            (8.29145728646, 0.875951783276)
            (8.34673366837, 0.944975945119)
            (8.40201005027, 1.01689955883)
            (8.45728643218, 1.09157132267)
            (8.51256281409, 1.16880733736)
            (8.567839196, 1.24839061227)
            (8.62311557791, 1.33007096234)
            (8.67839195982, 1.41356532513)
            (8.73366834173, 1.49855852159)
            (8.78894472364, 1.58470447778)
            (8.84422110555, 1.67162791773)
            (8.89949748746, 1.75892652965)
            (8.95477386937, 1.84617359945)
            (9.01005025127, 1.9329210968)
            (9.06532663318, 2.01870318998)
            (9.12060301509, 2.10304015724)
            (9.175879397, 2.18544265327)
            (9.23115577891, 2.26541628162)
            (9.28643216082, 2.34246641614)
            (9.34170854273, 2.41610320778)
            (9.39698492464, 2.4858467075)
            (9.45226130655, 2.55123203142)
            (9.50753768846, 2.61181449127)
            (9.56281407036, 2.66717461155)
            (9.61809045227, 2.71692295433)
            (9.67336683418, 2.76070467459)
            (9.72864321609, 2.79820373139)
            (9.783919598, 2.82914668521)
            (9.83919597991, 2.85330601757)
            (9.89447236182, 2.87050291668)
            (9.94974874373, 2.88060948124)
            (10.0050251256, 2.88355030457)
            (10.0603015075, 2.87930341154)
            (10.1155778895, 2.86790053203)
            (10.1708542714, 2.84942670611)
            (10.2261306533, 2.82401922796)
            (10.2814070352, 2.79186594666)
            (10.3366834171, 2.75320295351)
            (10.391959799, 2.70831169543)
            (10.4472361809, 2.65751556413)
            (10.5025125628, 2.60117601853)
            (10.5577889447, 2.53968830564)
            (10.6130653266, 2.47347685069)
            (10.6683417085, 2.40299039139)
            (10.7236180905, 2.32869693443)
            (10.7788944724, 2.25107861279)
            (10.8341708543, 2.1706265227)
            (10.8894472362, 2.08783561641)
            (10.9447236181, 2.00319972411)
            (11.0, 1.91720677323)
    };
    

    
    % Draw series plot
    \addplot[no markers,solid] coordinates {
            (1e-10, 6.54819627712e-20)
            (0.055276382009, 1.02706636967e-19)
            (0.110552763918, 1.60727636819e-19)
            (0.165829145827, 2.50956148141e-19)
            (0.221105527736, 3.90949219607e-19)
            (0.276381909645, 6.07656420967e-19)
            (0.331658291554, 9.42347455628e-19)
            (0.386934673463, 1.45807298451e-18)
            (0.442211055372, 2.25093365453e-18)
            (0.497487437281, 3.46705998652e-18)
            (0.55276381919, 5.32813533292e-18)
            (0.608040201099, 8.16966726289e-18)
            (0.663316583009, 1.24982359834e-17)
            (0.718592964918, 1.90769219238e-17)
            (0.773869346827, 2.90524735315e-17)
            (0.829145728736, 4.41441497344e-17)
            (0.884422110645, 6.69234664946e-17)
            (0.939698492554, 1.01227597763e-16)
            (0.994974874463, 1.5276879855e-16)
            (1.05025125637, 2.30030607126e-16)
            (1.10552763828, 3.455825607e-16)
            (1.16080402019, 5.18004152415e-16)
            (1.2160804021, 7.74693467874e-16)
            (1.27135678401, 1.15595726051e-15)
            (1.32663316592, 1.7209525267e-15)
            (1.38190954783, 2.55629654804e-15)
            (1.43718592974, 3.78851321935e-15)
            (1.49246231164, 5.60198072926e-15)
            (1.54773869355, 8.26474853531e-15)
            (1.60301507546, 1.21655824935e-14)
            (1.65829145737, 1.78669903109e-14)
            (1.71356783928, 2.61809334011e-14)
            (1.76884422119, 3.82766640217e-14)
            (1.8241206031, 5.58339385669e-14)
            (1.87939698501, 8.12601610634e-14)
            (1.93467336692, 1.17997368844e-13)
            (1.98994974883, 1.70955153779e-13)
            (2.04522613073, 2.47119662849e-13)
            (2.10050251264, 3.56408159173e-13)
            (2.15577889455, 5.12865166119e-13)
            (2.21105527646, 7.36332547867e-13)
            (2.26633165837, 1.05477547469e-12)
            (2.32160804028, 1.50751373801e-12)
            (2.37688442219, 2.14969954284e-12)
            (2.4321608041, 3.0585070067e-12)
            (2.48743718601, 4.34166614965e-12)
            (2.54271356792, 6.14919975831e-12)
            (2.59798994983, 8.68952440846e-12)
            (2.65326633173, 1.22514824889e-11)
            (2.70854271364, 1.72344135505e-11)
            (2.76381909555, 2.41890942279e-11)
            (2.81909547746, 3.38733330974e-11)
            (2.87437185937, 4.73272739114e-11)
            (2.92964824128, 6.59751332832e-11)
            (2.98492462319, 9.17623005577e-11)
            (3.0402010051, 1.27339617258e-10)
            (3.09547738701, 1.76310435156e-10)
            (3.15075376892, 2.43560986603e-10)
            (3.20603015082, 3.35701020913e-10)
            (3.26130653273, 4.61649988972e-10)
            (3.31658291464, 6.33414829961e-10)
            (3.37185929655, 8.67119304497e-10)
            (3.42713567846, 1.1843626767e-09)
            (3.48241206037, 1.61400843792e-09)
            (3.53768844228, 2.19453301577e-09)
            (3.59296482419, 2.97710167631e-09)
            (3.6482412061, 4.02958615473e-09)
            (3.70351758801, 5.44179846963e-09)
            (3.75879396992, 7.33229100236e-09)
            (3.81407035182, 9.85716807026e-09)
            (3.86934673373, 1.32214735486e-08)
            (3.92462311564, 1.76938682636e-08)
            (3.97989949755, 2.36254967428e-08)
            (4.03517587946, 3.14741737443e-08)
            (4.09045226137, 4.18353067073e-08)
            (4.14572864328, 5.54813227623e-08)
            (4.20100502519, 7.34118023096e-08)
            (4.2562814071, 9.6917052176e-08)
            (4.31155778901, 1.27658499692e-07)
            (4.36683417091, 1.67770077861e-07)
            (4.42211055282, 2.1998572809e-07)
            (4.47738693473, 2.87799302112e-07)
            (4.53266331664, 3.75664534898e-07)
            (4.58793969855, 4.89244427081e-07)
            (4.64321608046, 6.35721366135e-07)
            (4.69849246237, 8.24181683804e-07)
            (4.75376884428, 1.06609115369e-06)
            (4.80904522619, 1.37588124478e-06)
            (4.8643216081, 1.77166983895e-06)
            (4.91959799001, 2.27614467525e-06)
            (4.97487437191, 2.91764309036e-06)
            (5.03015075382, 3.73146777934e-06)
            (5.08542713573, 4.76148540579e-06)
            (5.14070351764, 6.06206305336e-06)
            (5.19597989955, 7.70040684093e-06)
            (5.25125628146, 9.75937763481e-06)
            (5.30653266337, 1.23408707914e-05)
            (5.36180904528, 1.55698603588e-05)
            (5.41708542719, 1.95992232512e-05)
            (5.4723618091, 2.46154756643e-05)
            (5.527638191, 3.08455724844e-05)
            (5.58291457291, 3.85649406889e-05)
            (5.63819095482, 4.81069397332e-05)
            (5.69346733673, 5.98739656208e-05)
            (5.74874371864, 7.43504406462e-05)
            (5.80402010055, 9.21179575146e-05)
            (5.85929648246, 0.000113872874412)
            (5.91457286437, 0.000140446686285)
            (5.96984924628, 0.000172829526647)
            (6.02512562819, 0.000212197183025)
            (6.08040201009, 0.000259942037113)
            (6.135678392, 0.00031770836682)
            (6.19095477391, 0.000387432470777)
            (6.24623115582, 0.00047138809533)
            (6.30150753773, 0.000572237658264)
            (6.35678391964, 0.000693089770923)
            (6.41206030155, 0.000837563559486)
            (6.46733668346, 0.001009860275)
            (6.52261306537, 0.00121484265846)
            (6.57788944728, 0.00145812248979)
            (6.63316582919, 0.00174615669565)
            (6.68844221109, 0.00208635231866)
            (6.743718593, 0.00248718055784)
            (6.79899497491, 0.00295829997379)
            (6.85427135682, 0.00351068881233)
            (6.90954773873, 0.00415678623274)
            (6.96482412064, 0.00491064203289)
            (7.02010050255, 0.00578807423928)
            (7.07537688446, 0.00680683367829)
            (7.13065326637, 0.0079867743626)
            (7.18592964828, 0.00935002821765)
            (7.24120603018, 0.0109211823369)
            (7.29648241209, 0.0127274565951)
            (7.351758794, 0.0147988790707)
            (7.40703517591, 0.0171684563328)
            (7.46231155782, 0.0198723352496)
            (7.51758793973, 0.0229499525673)
            (7.57286432164, 0.0264441681175)
            (7.62814070355, 0.0304013771305)
            (7.68341708546, 0.0348715967826)
            (7.73869346737, 0.0399085217986)
            (7.79396984928, 0.0455695436738)
            (7.84924623118, 0.0519157278937)
            (7.90452261309, 0.0590117434215)
            (7.959798995, 0.0669257387163)
            (8.01507537691, 0.0757291586391)
            (8.07035175882, 0.0854964968305)
            (8.12562814073, 0.0963049784944)
            (8.18090452264, 0.10823416903)
            (8.23618090455, 0.121365504604)
            (8.29145728646, 0.135781741573)
            (8.34673366837, 0.151566322639)
            (8.40201005027, 0.168802658769)
            (8.45728643218, 0.187573327187)
            (8.51256281409, 0.20795918721)
            (8.567839196, 0.23003841725)
            (8.62311557791, 0.253885478029)
            (8.67839195982, 0.279570008804)
            (8.73366834173, 0.307155665257)
            (8.78894472364, 0.336698909601)
            (8.84422110555, 0.368247765295)
            (8.89949748746, 0.401840550605)
            (8.95477386937, 0.437504606981)
            (9.01005025127, 0.475255039822)
            (9.06532663318, 0.515093490606)
            (9.12060301509, 0.557006960564)
            (9.175879397, 0.600966706983)
            (9.23115577891, 0.646927233817)
            (9.28643216082, 0.694825398515)
            (9.34170854273, 0.744579656831)
            (9.39698492464, 0.796089466779)
            (9.45226130655, 0.849234871895)
            (9.50753768846, 0.903876282488)
            (9.56281407036, 0.959854471627)
            (9.61809045227, 1.01699080023)
            (9.67336683418, 1.07508768287)
            (9.72864321609, 1.13392930257)
            (9.783919598, 1.19328257957)
            (9.83919597991, 1.25289839484)
            (9.89447236182, 1.31251306543)
            (9.94974874373, 1.37185006401)
            (10.0050251256, 1.43062197119)
            (10.0603015075, 1.48853264427)
            (10.1155778895, 1.54527958232)
            (10.1708542714, 1.600556463)
            (10.2261306533, 1.65405582323)
            (10.2814070352, 1.70547185177)
            (10.3366834171, 1.75450325944)
            (10.391959799, 1.8008561898)
            (10.4472361809, 1.84424713146)
            (10.5025125628, 1.88440579205)
            (10.5577889447, 1.9210778932)
            (10.6130653266, 1.9540278464)
            (10.6683417085, 1.9830412702)
            (10.7236180905, 2.00792731149)
            (10.7788944724, 2.02852073546)
            (10.8341708543, 2.04468375234)
            (10.8894472362, 2.05630755269)
            (10.9447236181, 2.06331352732)
            (11.0, 2.06565415259)
    };
    

    
    % Draw series plot
    \addplot[no markers,solid] coordinates {
            (1e-10, 8.1513808932e-22)
            (0.055276382009, 1.27864221703e-21)
            (0.110552763918, 2.00153959627e-21)
            (0.165829145827, 3.12663098239e-21)
            (0.221105527736, 4.87400939864e-21)
            (0.276381909645, 7.58216780527e-21)
            (0.331658291554, 1.17705759942e-20)
            (0.386934673463, 1.82347298545e-20)
            (0.442211055372, 2.81902057749e-20)
            (0.497487437281, 4.34905052052e-20)
            (0.55276381919, 6.69557614768e-20)
            (0.608040201099, 1.02867633471e-19)
            (0.663316583009, 1.57712762251e-19)
            (0.718592964918, 2.41297166428e-19)
            (0.773869346827, 3.68412960551e-19)
            (0.829145728736, 5.61325629836e-19)
            (0.884422110645, 8.53477622856e-19)
            (0.939698492554, 1.29499084989e-18)
            (0.994974874463, 1.96082373374e-18)
            (1.05025125637, 2.96283662411e-18)
            (1.10552763828, 4.46759862e-18)
            (1.16080402019, 6.72260951914e-18)
            (1.2160804021, 1.00948278779e-17)
            (1.27135678401, 1.51271545837e-17)
            (1.32663316592, 2.26210555273e-17)
            (1.38190954783, 3.37571505757e-17)
            (1.43718592974, 5.02708224281e-17)
            (1.49246231164, 7.47073779541e-17)
            (1.54773869355, 1.10791971864e-16)
            (1.60301507546, 1.63964709615e-16)
            (1.65829145737, 2.4215295367e-16)
            (1.71356783928, 3.568834853e-16)
            (1.76884422119, 5.24880485707e-16)
            (1.8241206031, 7.70356401827e-16)
            (1.87939698501, 1.1282887546e-15)
            (1.93467336692, 1.64909668084e-15)
            (1.98994974883, 2.40530010249e-15)
            (2.04522613073, 3.50098069723e-15)
            (2.10050251264, 5.08519319383e-15)
            (2.15577889455, 7.37093348391e-15)
            (2.21105527646, 1.06619053164e-14)
            (2.26633165837, 1.53902064282e-14)
            (2.32160804028, 2.21692684807e-14)
            (2.37688442219, 3.18680562535e-14)
            (2.4321608041, 4.57148271024e-14)
            (2.48743718601, 6.54418973504e-14)
            (2.54271356792, 9.34871622719e-14)
            (2.59798994983, 1.33273979766e-13)
            (2.65326633173, 1.89599003397e-13)
            (2.70854271364, 2.69168372161e-13)
            (2.76381909555, 3.81337311331e-13)
            (2.81909547746, 5.39127973685e-13)
            (2.87437185937, 7.60626982134e-13)
            (2.92964824128, 1.07089991056e-12)
            (2.98492462319, 1.50460787848e-12)
            (3.0402010051, 2.10957550998e-12)
            (3.09547738701, 2.95164493244e-12)
            (3.15075376892, 4.12126409079e-12)
            (3.20603015082, 5.74240833568e-12)
            (3.26130653273, 7.98463360206e-12)
            (3.31658291464, 1.10793225741e-11)
            (3.37185929655, 1.53415315047e-11)
            (3.42713567846, 2.11992999501e-11)
            (3.48241206037, 2.92328809246e-11)
            (3.53768844228, 4.02271263532e-11)
            (3.59296482419, 5.52412752773e-11)
            (3.6482412061, 7.57017077088e-11)
            (3.70351758801, 1.03524931427e-10)
            (3.75879396992, 1.4128027146e-10)
            (3.81407035182, 1.9240457043e-10)
            (3.86934673373, 2.614848562e-10)
            (3.92462311564, 3.54629609605e-10)
            (3.97989949755, 4.79955243036e-10)
            (4.03517587946, 6.48222021503e-10)
            (4.09045226137, 8.73663346412e-10)
            (4.14572864328, 1.1750646054e-09)
            (4.20100502519, 1.57716332067e-09)
            (4.2562814071, 2.11246190596e-09)
            (4.31155778901, 2.82356895404e-09)
            (4.36683417091, 3.76621582904e-09)
            (4.42211055282, 5.01313390222e-09)
            (4.47738693473, 6.65902583363e-09)
            (4.53266331664, 8.8269240313e-09)
            (4.58793969855, 1.16763034264e-08)
            (4.64321608046, 1.54134071272e-08)
            (4.69849246237, 2.03043561181e-08)
            (4.75376884428, 2.66917524363e-08)
            (4.80904522619, 3.50156545195e-08)
            (4.8643216081, 4.58400099771e-08)
            (4.91959799001, 5.98858823223e-08)
            (4.97487437191, 7.8073112938e-08)
            (5.03015075382, 1.01572427902e-07)
            (5.08542713573, 1.31870443113e-07)
            (5.14070351764, 1.70850554135e-07)
            (5.19597989955, 2.20893335042e-07)
            (5.25125628146, 2.85000831438e-07)
            (5.30653266337, 3.66950037178e-07)
            (5.36180904528, 4.71481915595e-07)
            (5.41708542719, 6.0453359027e-07)
            (5.4723618091, 7.73522817077e-07)
            (5.527638191, 9.87695590851e-07)
            (5.58291457291, 1.25854977255e-06)
            (5.63819095482, 1.60034998546e-06)
            (5.69346733673, 2.03075176448e-06)
            (5.74874371864, 2.57155609647e-06)
            (5.80402010055, 3.24961911084e-06)
            (5.85929648246, 4.09794581817e-06)
            (5.91457286437, 5.15700150475e-06)
            (5.96984924628, 6.47627972369e-06)
            (6.02512562819, 8.11617183386e-06)
            (6.08040201009, 1.01501897758e-05)
            (6.135678392, 1.26676012871e-05)
            (6.19095477391, 1.57765450907e-05)
            (6.24623115582, 1.96077027684e-05)
            (6.30150753773, 2.43186140877e-05)
            (6.35678391964, 3.00987334786e-05)
            (6.41206030155, 3.71753371589e-05)
            (6.46733668346, 4.58204030399e-05)
            (6.52261306537, 5.63585989479e-05)
            (6.57788944728, 6.917652878e-05)
            (6.63316582919, 8.47334008366e-05)
            (6.68844221109, 0.00010357329757)
            (6.743718593, 0.000126339241121)
            (6.79899497491, 0.000153789264009)
            (6.85427135682, 0.000186814708867)
            (6.90954773873, 0.000226460994736)
            (6.96482412064, 0.000273951099686)
            (7.02010050255, 0.0003307120199)
            (7.07537688446, 0.000398404473153)
            (7.13065326637, 0.000478956119177)
            (7.18592964828, 0.000574598569962)
            (7.24120603018, 0.000687908458698)
            (7.29648241209, 0.00082185282598)
            (7.351758794, 0.000979839065024)
            (7.40703517591, 0.00116576964306)
            (7.46231155782, 0.00138410178263)
            (7.51758793973, 0.00163991224338)
            (7.57286432164, 0.00193896729068)
            (7.62814070355, 0.00228779787185)
            (7.68341708546, 0.00269377994184)
            (7.73869346737, 0.00316521978838)
            (7.79396984928, 0.00371144410064)
            (7.84924623118, 0.00434289440472)
            (7.90452261309, 0.00507122535429)
            (7.959798995, 0.00590940621532)
            (8.01507537691, 0.00687182472037)
            (8.07035175882, 0.00797439229165)
            (8.12562814073, 0.00923464944417)
            (8.18090452264, 0.0106718699827)
            (8.23618090455, 0.0123071624014)
            (8.29145728646, 0.0141635666854)
            (8.34673366837, 0.0162661445043)
            (8.40201005027, 0.0186420605788)
            (8.45728643218, 0.0213206528046)
            (8.51256281409, 0.02433348853)
            (8.567839196, 0.0277144042178)
            (8.62311557791, 0.0314995255766)
            (8.67839195982, 0.0357272651393)
            (8.73366834173, 0.0404382941889)
            (8.78894472364, 0.0456754859059)
            (8.84422110555, 0.0514838266322)
            (8.89949748746, 0.0579102922278)
            (8.95477386937, 0.0650036866403)
            (9.01005025127, 0.0728144400203)
            (9.06532663318, 0.0813943640063)
            (9.12060301509, 0.0907963621635)
            (9.175879397, 0.101074094011)
            (9.23115577891, 0.112281591599)
            (9.28643216082, 0.124472828192)
            (9.34170854273, 0.137701239323)
            (9.39698492464, 0.152019197219)
            (9.45226130655, 0.16747744042)
            (9.50753768846, 0.184124461323)
            (9.56281407036, 0.202005855273)
            (9.61809045227, 0.221163635808)
            (9.67336683418, 0.241635521636)
            (9.72864321609, 0.263454201888)
            (9.783919598, 0.286646587171)
            (9.83919597991, 0.31123305484)
            (9.89447236182, 0.337226697786)
            (9.94974874373, 0.364632586799)
            (10.0050251256, 0.393447057227)
            (10.0603015075, 0.423657031213)
            (10.1155778895, 0.455239387166)
            (10.1708542714, 0.488160388337)
            (10.2261306533, 0.522375182427)
            (10.2814070352, 0.557827383956)
            (10.3366834171, 0.594448750776)
            (10.391959799, 0.632158965456)
            (10.4472361809, 0.670865531488)
            (10.5025125628, 0.710463793157)
            (10.5577889447, 0.750837086665)
            (10.6130653266, 0.791857028601)
            (10.6683417085, 0.833383946154)
            (10.7236180905, 0.87526745162)
            (10.7788944724, 0.917347161733)
            (10.8341708543, 0.959453560213)
            (10.8894472362, 1.00140899974)
            (10.9447236181, 1.04302883726)
            (11.0, 1.08412269434)
    };
    

    
    % Draw series plot
    \addplot[no markers,solid] coordinates {
            (1e-10, 1.03976040908e-23)
            (0.055276382009, 1.63111975876e-23)
            (0.110552763918, 2.55390744209e-23)
            (0.165829145827, 3.99108697209e-23)
            (0.221105527736, 6.22506604186e-23)
            (0.276381909645, 9.69088562236e-23)
            (0.331658291554, 1.50573902211e-22)
            (0.386934673463, 2.33508490549e-22)
            (0.442211055372, 3.61428487726e-22)
            (0.497487437281, 5.5835295366e-22)
            (0.55276381919, 8.60918396959e-22)
            (0.608040201099, 1.32489631451e-21)
            (0.663316583009, 2.03501928335e-21)
            (0.718592964918, 3.11976512477e-21)
            (0.773869346827, 4.77355578356e-21)
            (0.829145728736, 7.29002199193e-21)
            (0.884422110645, 1.11117486249e-20)
            (0.939698492554, 1.69045148308e-20)
            (0.994974874463, 2.56678650154e-20)
            (1.05025125637, 3.88994549226e-20)
            (1.10552763828, 5.88388286259e-20)
            (1.16080402019, 8.88282801533e-20)
            (1.2160804021, 1.33845946215e-19)
            (1.27135678401, 2.01291726151e-19)
            (1.32663316592, 3.02143571946e-19)
            (1.38190954783, 4.52655214109e-19)
            (1.43718592974, 6.7684375898e-19)
            (1.49246231164, 1.01012719457e-18)
            (1.54773869355, 1.50463247484e-18)
            (1.60301507546, 2.2369255602e-18)
            (1.65829145737, 3.31924543107e-18)
            (1.71356783928, 4.91579692327e-18)
            (1.76884422119, 7.26633194873e-18)
            (1.8241206031, 1.07202093746e-17)
            (1.87939698501, 1.57854887997e-17)
            (1.93467336692, 2.31995485821e-17)
            (1.98994974883, 3.40304559284e-17)
            (2.04522613073, 4.98221808743e-17)
            (2.10050251264, 7.28021883053e-17)
            (2.15577889455, 1.06177590688e-16)
            (2.21105527646, 1.54556769736e-16)
            (2.26633165837, 2.24548382718e-16)
            (2.32160804028, 3.25610621891e-16)
            (2.37688442219, 4.71252784398e-16)
            (2.4321608041, 6.80731782793e-16)
            (2.48743718601, 9.81442498201e-16)
            (2.54271356792, 1.41227875098e-15)
            (2.59798994983, 2.02834919027e-15)
            (2.65326633173, 2.90758054599e-15)
            (2.70854271364, 4.15994433226e-15)
            (2.76381909555, 5.94032240843e-15)
            (2.81909547746, 8.46640905706e-15)
            (2.87437185937, 1.20435691858e-14)
            (2.92964824128, 1.70992833555e-14)
            (2.98492462319, 2.4230776939e-14)
            (3.0402010051, 3.42707416701e-14)
            (3.09547738701, 4.83778317932e-14)
            (3.15075376892, 6.81610120542e-14)
            (3.20603015082, 9.58500609612e-14)
            (3.26130653273, 1.34528868624e-13)
            (3.31658291464, 1.88453982419e-13)
            (3.37185929655, 2.63488631701e-13)
            (3.42713567846, 3.67692844317e-13)
            (3.48241206037, 5.12124092364e-13)
            (3.53768844228, 7.11921271759e-13)
            (3.59296482419, 9.87769165733e-13)
            (3.6482412061, 1.36787273704e-12)
            (3.70351758801, 1.89061306878e-12)
            (3.75879396992, 2.60811268047e-12)
            (3.81407035182, 3.59101135899e-12)
            (3.86934673373, 4.93484981727e-12)
            (3.92462311564, 6.76858418507e-12)
            (3.97989949755, 9.26591821564e-12)
            (4.03517587946, 1.2660353251e-11)
            (4.09045226137, 1.72651325824e-11)
            (4.14572864328, 2.34996148849e-11)
            (4.20100502519, 3.19240737586e-11)
            (4.2562814071, 4.32855160188e-11)
            (4.31155778901, 5.8577876773e-11)
            (4.36683417091, 7.9120930502e-11)
            (4.42211055282, 1.06663512054e-10)
            (4.47738693473, 1.43518241953e-10)
            (4.53266331664, 1.92736986942e-10)
            (4.58793969855, 2.58338871517e-10)
            (4.64321608046, 3.456059288e-10)
            (4.69849246237, 4.61465611903e-10)
            (4.75376884428, 6.14984592839e-10)
            (4.80904522619, 8.18004816912e-10)
            (4.8643216081, 1.08596097748e-09)
            (4.91959799001, 1.43892882184e-09)
            (4.97487437191, 1.90296647231e-09)
            (5.03015075382, 2.51182682929e-09)
            (5.08542713573, 3.30913881933e-09)
            (5.14070351764, 4.35117961248e-09)
            (5.19597989955, 5.71038998184e-09)
            (5.25125628146, 7.47982194372e-09)
            (5.30653266337, 9.77875316463e-09)
            (5.36180904528, 1.27597581011e-08)
            (5.41708542719, 1.66175935228e-08)
            (5.4723618091, 2.16003384139e-08)
            (5.527638191, 2.8023328139e-08)
            (5.58291457291, 3.62865435923e-08)
            (5.63819095482, 4.68962617917e-08)
            (5.69346733673, 6.04919496364e-08)
            (5.74874371864, 7.78795926922e-08)
            (5.80402010055, 1.00072902064e-07)
            (5.85929648246, 1.28344141781e-07)
            (5.91457286437, 1.6428667477e-07)
            (5.96984924628, 2.09891746705e-07)
            (6.02512562819, 2.67642524252e-07)
            (6.08040201009, 3.40628989314e-07)
            (6.135678392, 4.32687977088e-07)
            (6.19095477391, 5.48573447717e-07)
            (6.24623115582, 6.94163015589e-07)
            (6.30150753773, 8.76707844575e-07)
            (6.35678391964, 1.10513427161e-06)
            (6.41206030155, 1.39040696595e-06)
            (6.46733668346, 1.74596509007e-06)
            (6.52261306537, 2.18824482487e-06)
            (6.57788944728, 2.73730378154e-06)
            (6.63316582919, 3.41756527199e-06)
            (6.68844221109, 4.25870317617e-06)
            (6.743718593, 5.29669125377e-06)
            (6.79899497491, 6.57504422833e-06)
            (6.85427135682, 8.14628184605e-06)
            (6.90954773873, 1.00736514057e-05)
            (6.96482412064, 1.24331489861e-05)
            (7.02010050255, 1.53158847828e-05)
            (7.07537688446, 1.88308436104e-05)
            (7.13065326637, 2.31080977366e-05)
            (7.18592964828, 2.83025357784e-05)
            (7.24120603018, 3.45981783905e-05)
            (7.29648241209, 4.2213158876e-05)
            (7.351758794, 5.14054546038e-05)
            (7.40703517591, 6.24794631508e-05)
            (7.46231155782, 7.57935253184e-05)
            (7.51758793973, 9.17685054763e-05)
            (7.57286432164, 0.000110897547937)
            (7.62814070355, 0.000133757136071)
            (7.68341708546, 0.000161019588457)
            (7.73869346737, 0.000193467133268)
            (7.79396984928, 0.000232007708055)
            (7.84924623118, 0.000277692636822)
            (7.90452261309, 0.000331736339393)
            (7.959798995, 0.000395538229233)
            (8.01507537691, 0.000470706954639)
            (8.07035175882, 0.000559087134115)
            (8.12562814073, 0.000662788729366)
            (8.18090452264, 0.000784219188087)
            (8.23618090455, 0.000926118473171)
            (8.29145728646, 0.00109159707451)
            (8.34673366837, 0.00128417707377)
            (8.40201005027, 0.00150783630066)
            (8.45728643218, 0.00176705558139)
            (8.51256281409, 0.00206686903478)
            (8.567839196, 0.00241291731952)
            (8.62311557791, 0.00281150367679)
            (8.67839195982, 0.00326965254472)
            (8.73366834173, 0.0037951704469)
            (8.78894472364, 0.00439670877428)
            (8.84422110555, 0.00508382799037)
            (8.89949748746, 0.00586706269266)
            (8.95477386937, 0.00675798686097)
            (9.01005025127, 0.00776927851503)
            (9.06532663318, 0.00891478289191)
            (9.12060301509, 0.0102095731391)
            (9.175879397, 0.0116700074036)
            (9.23115577891, 0.0133137810838)
            (9.28643216082, 0.0151599728984)
            (9.34170854273, 0.0172290833236)
            (9.39698492464, 0.0195430638519)
            (9.45226130655, 0.0221253354433)
            (9.50753768846, 0.0250007944677)
            (9.56281407036, 0.0281958043886)
            (9.61809045227, 0.0317381714076)
            (9.67336683418, 0.0356571022844)
            (9.72864321609, 0.0399831425718)
            (9.783919598, 0.044748093561)
            (9.83919597991, 0.0499849063231)
            (9.89447236182, 0.0557275513597)
            (9.94974874373, 0.0620108625451)
            (10.0050251256, 0.0688703542486)
            (10.0603015075, 0.0763420107788)
            (10.1155778895, 0.0844620475861)
            (10.1708542714, 0.0932666439921)
            (10.2261306533, 0.102791647597)
            (10.2814070352, 0.113072250928)
            (10.3366834171, 0.124142641338)
            (10.391959799, 0.136035625659)
            (10.4472361809, 0.148782231605)
            (10.5025125628, 0.162411288449)
            (10.5577889447, 0.176948990038)
            (10.6130653266, 0.192418443748)
            (10.6683417085, 0.208839209493)
            (10.7236180905, 0.22622683345)
            (10.7788944724, 0.244592381606)
            (10.8341708543, 0.263941978707)
            (10.8894472362, 0.284276358564)
            (10.9447236181, 0.305590432023)
            (11.0, 0.327872879135)
    };
    

    
    % Draw series plot
    \addplot[no markers,solid] coordinates {
            (1e-10, 1.3541174871e-25)
            (0.055276382009, 2.12441166766e-25)
            (0.110552763918, 3.32695756548e-25)
            (0.165829145827, 5.20094303956e-25)
            (0.221105527736, 8.11602143786e-25)
            (0.276381909645, 1.26424290739e-24)
            (0.331658291554, 1.96582167119e-24)
            (0.386934673463, 3.05129335487e-24)
            (0.442211055372, 4.72770163323e-24)
            (0.497487437281, 7.31210483891e-24)
            (0.55276381919, 1.12891439565e-23)
            (0.608040201099, 1.7398261931e-23)
            (0.663316583009, 2.67655992322e-23)
            (0.718592964918, 4.11030764796e-23)
            (0.773869346827, 6.30083257322e-23)
            (0.829145728736, 9.64157096971e-23)
            (0.884422110645, 1.47273265759e-22)
            (0.939698492554, 2.24556838106e-22)
            (0.994974874463, 3.41786499362e-22)
            (1.05025125637, 5.19289780297e-22)
            (1.10552763828, 7.87573146215e-22)
            (1.16080402019, 1.1923349542e-21)
            (1.2160804021, 1.80190505745e-21)
            (1.27135678401, 2.71826494768e-21)
            (1.32663316592, 4.09334092726e-21)
            (1.38190954783, 6.15304799841e-21)
            (1.43718592974, 9.23270459482e-21)
            (1.49246231164, 1.38290972249e-20)
            (1.54773869355, 2.06768783095e-20)
            (1.60301507546, 3.08604579112e-20)
            (1.65829145737, 4.59775697843e-20)
            (1.71356783928, 6.83779239415e-20)
            (1.76884422119, 1.01510755871e-19)
            (1.8241206031, 1.50430003186e-19)
            (1.87939698501, 2.22527208534e-19)
            (1.93467336692, 3.28592788232e-19)
            (1.98994974883, 4.84349869813e-19)
            (2.04522613073, 7.12666918087e-19)
            (2.10050251264, 1.04674347811e-18)
            (2.15577889455, 1.53468829273e-18)
            (2.21105527646, 2.24608583461e-18)
            (2.26633165837, 3.28139694842e-18)
            (2.32160804028, 4.78539116135e-18)
            (2.37688442219, 6.96630313424e-18)
            (2.4321608041, 1.01231004496e-17)
            (2.48743718601, 1.46842231693e-17)
            (2.54271356792, 2.12625162452e-17)
            (2.59798994983, 3.0732974306e-17)
            (2.65326633173, 4.43425617562e-17)
            (2.70854271364, 6.38650444023e-17)
            (2.76381909555, 9.18188627989e-17)
            (2.81909547746, 1.31773127322e-16)
            (2.87437185937, 1.8877655165e-16)
            (2.92964824128, 2.69957561481e-16)
            (2.98492462319, 3.8536227451e-16)
            (3.0402010051, 5.49122386121e-16)
            (3.09547738701, 7.81079705047e-16)
            (3.15075376892, 1.1090415327e-15)
            (3.20603015082, 1.57190581123e-15)
            (3.26130653273, 2.22398309027e-15)
            (3.31658291464, 3.14096208782e-15)
            (3.37185929655, 4.42812793917e-15)
            (3.42713567846, 6.2316618377e-15)
            (3.48241206037, 8.75414727183e-15)
            (3.53768844228, 1.22758075425e-14)
            (3.59296482419, 1.71835363442e-14)
            (3.6482412061, 2.40105038042e-14)
            (3.70351758801, 3.34900851728e-14)
            (3.75879396992, 4.66291483613e-14)
            (3.81407035182, 6.48074550437e-14)
            (3.86934673373, 8.9912216351e-14)
            (3.92462311564, 1.24519879414e-13)
            (3.97989949755, 1.72141241923e-13)
            (4.03517587946, 2.37551307183e-13)
            (4.09045226137, 3.27232303494e-13)
            (4.14572864328, 4.49967520452e-13)
            (4.20100502519, 6.17635736044e-13)
            (4.2562814071, 8.46271894407e-13)
            (4.31155778901, 1.15748046909e-12)
            (4.36683417091, 1.58031505605e-12)
            (4.42211055282, 2.1537730389e-12)
            (4.47738693473, 2.93010002902e-12)
            (4.53266331664, 3.97915823475e-12)
            (4.58793969855, 5.39418981508e-12)
            (4.64321608046, 7.29940552642e-12)
            (4.69849246237, 9.85995678758e-12)
            (4.75376884428, 1.32950135232e-11)
            (4.80904522619, 1.78948807021e-11)
            (4.8643216081, 2.40433558268e-11)
            (4.91959799001, 3.22468733794e-11)
            (4.97487437191, 4.31724199674e-11)
            (5.03015075382, 5.76967600479e-11)
            (5.08542713573, 7.69702170725e-11)
            (5.14070351764, 1.02499146449e-10)
            (5.19597989955, 1.36252361651e-10)
            (5.25125628146, 1.80798190924e-10)
            (5.30653266337, 2.39480620537e-10)
            (5.36180904528, 3.16645208965e-10)
            (5.41708542719, 4.17928241877e-10)
            (5.4723618091, 5.50626071551e-10)
            (5.527638191, 7.24165905813e-10)
            (5.58291457291, 9.50704675027e-10)
            (5.63819095482, 1.245889246e-09)
            (5.69346733673, 1.62981945301e-09)
            (5.74874371864, 2.12826552077e-09)
            (5.80402010055, 2.77420387257e-09)
            (5.85929648246, 3.60975054189e-09)
            (5.91457286437, 4.68859002368e-09)
            (5.96984924628, 6.07902011233e-09)
            (6.02512562819, 7.86776090516e-09)
            (6.08040201009, 1.01647096838e-08)
            (6.135678392, 1.31088639754e-08)
            (6.19095477391, 1.68756840986e-08)
            (6.24623115582, 2.16862255005e-08)
            (6.30150753773, 2.78184420459e-08)
            (6.35678391964, 3.56211462718e-08)
            (6.41206030155, 4.55312139697e-08)
            (6.46733668346, 5.80947411598e-08)
            (6.52261306537, 7.39930048884e-08)
            (6.57788944728, 9.40742490686e-08)
            (6.63316582919, 1.19392517097e-07)
            (6.68844221109, 1.51254989074e-07)
            (6.743718593, 1.9127955865e-07)
            (6.79899497491, 2.41464708914e-07)
            (6.85427135682, 3.04274125361e-07)
            (6.90954773873, 3.82738924279e-07)
            (6.96482412064, 4.80580885565e-07)
            (7.02010050255, 6.02360669102e-07)
            (7.07537688446, 7.53655673677e-07)
            (7.13065326637, 9.41272977885e-07)
            (7.18592964828, 1.17350369534e-06)
            (7.24120603018, 1.46042609436e-06)
            (7.29648241209, 1.81426598846e-06)
            (7.351758794, 2.24982421172e-06)
            (7.40703517591, 2.78498246791e-06)
            (7.46231155782, 3.44130049651e-06)
            (7.51758793973, 4.24471934903e-06)
            (7.57286432164, 5.22638762754e-06)
            (7.62814070355, 6.42362981807e-06)
            (7.68341708546, 7.88107836615e-06)
            (7.73869346737, 9.6519939003e-06)
            (7.79396984928, 1.17998010191e-05)
            (7.84924623118, 1.43998703234e-05)
            (7.90452261309, 1.75415808972e-05)
            (7.959798995, 2.13307012126e-05)
            (8.01507537691, 2.58921304507e-05)
            (8.07035175882, 3.13730464632e-05)
            (8.12562814073, 3.79465110355e-05)
            (8.18090452264, 4.58155877057e-05)
            (8.23618090455, 5.52180321004e-05)
            (8.29145728646, 6.64316195153e-05)
            (8.34673366837, 7.97801792142e-05)
            (8.40201005027, 9.56404095656e-05)
            (8.45728643218, 0.000114449552571)
            (8.51256281409, 0.000136714010456)
            (8.567839196, 0.000163018990632)
            (8.62311557791, 0.000194039268369)
            (8.67839195982, 0.000230551158756)
            (8.73366834173, 0.000273445790728)
            (8.78894472364, 0.000323743776024)
            (8.84422110555, 0.00038261136448)
            (8.89949748746, 0.000451378174021)
            (8.95477386937, 0.000531556578655)
            (9.01005025127, 0.000624862830542)
            (9.06532663318, 0.000733239982479)
            (9.12060301509, 0.000858882664624)
            (9.175879397, 0.00100426375373)
            (9.23115577891, 0.00117216295425)
            (9.28643216082, 0.00136569728836)
            (9.34170854273, 0.0015883534654)
            (9.39698492464, 0.0018440220715)
            (9.45226130655, 0.00213703348519)
            (9.50753768846, 0.00247219538684)
            (9.56281407036, 0.00285483168638)
            (9.61809045227, 0.00329082264719)
            (9.67336683418, 0.00378664593265)
            (9.72864321609, 0.00434941824702)
            (9.783919598, 0.00498693718404)
            (9.83919597991, 0.00570772283526)
            (9.89447236182, 0.00652105864605)
            (9.94974874373, 0.00743703094243)
            (10.0050251256, 0.00846656648493)
            (10.0603015075, 0.00962146734027)
            (10.1155778895, 0.0109144422969)
            (10.1708542714, 0.0123591339882)
            (10.2261306533, 0.0139701408306)
            (10.2814070352, 0.0157630328296)
            (10.3366834171, 0.0177543602649)
            (10.391959799, 0.0199616542272)
            (10.4472361809, 0.0224034179571)
            (10.5025125628, 0.0250991079223)
            (10.5577889447, 0.0280691035736)
            (10.6130653266, 0.0313346647378)
            (10.6683417085, 0.0349178756422)
            (10.7236180905, 0.0388415746231)
            (10.7788944724, 0.0431292686466)
            (10.8341708543, 0.0478050318695)
            (10.8894472362, 0.0528933875908)
            (10.9447236181, 0.0584191730897)
            (11.0, 0.0644073870157)
    };
    

    
    % Draw series plot
    \addplot[no markers,solid] coordinates {
            (1e-10, 1e+32)
            (0.055276382009, 592081.06367)
            (0.110552763918, 74010.1331596)
            (0.165829145827, 21928.9283634)
            (0.221105527736, 9251.2666575)
            (0.276381909645, 4736.64852992)
            (0.331658291554, 2741.11604791)
            (0.386934673463, 1726.18386713)
            (0.442211055372, 1156.40833297)
            (0.497487437281, 812.182532958)
            (0.55276381919, 592.081066562)
            (0.608040201099, 444.839268663)
            (0.663316583009, 342.639506143)
            (0.718592964918, 269.495251086)
            (0.773869346827, 215.772983474)
            (0.829145728736, 175.431427161)
            (0.884422110645, 144.55104167)
            (0.939698492554, 120.51314201)
            (0.994974874463, 101.52281665)
            (1.05025125637, 86.3217767479)
            (1.10552763828, 74.0101333403)
            (1.16080402019, 63.9327358525)
            (1.2160804021, 55.6049085966)
            (1.27135678401, 48.6628640374)
            (1.32663316592, 42.8299382776)
            (1.38190954783, 37.8931882723)
            (1.43718592974, 33.6869063928)
            (1.49246231164, 30.0808345661)
            (1.54773869355, 26.9716229395)
            (1.60301507546, 24.2765618423)
            (1.65829145737, 21.9289283991)
            (1.71356783928, 19.8744945379)
            (1.76884422119, 18.0688802119)
            (1.8241206031, 16.4755284744)
            (1.87939698501, 15.0641427536)
            (1.93467336692, 13.8094709456)
            (1.98994974883, 12.6903520832)
            (2.04522613073, 11.688963473)
            (2.10050251264, 10.790222095)
            (2.15577889455, 9.98130559856)
            (2.21105527646, 9.25126666879)
            (2.26633165837, 8.59072077893)
            (2.32160804028, 7.9915919826)
            (2.37688442219, 7.44690488647)
            (2.4321608041, 6.95061357544)
            (2.48743718601, 6.4974602668)
            (2.54271356792, 6.0828580054)
            (2.59798994983, 5.70279289574)
            (2.65326633173, 5.3537422853)
            (2.70854271364, 5.03260602995)
            (2.76381909555, 4.73664853455)
            (2.81909547746, 4.46344970502)
            (2.87437185937, 4.21086329954)
            (2.92964824128, 3.97698144658)
            (2.98492462319, 3.76010432114)
            (3.0402010051, 3.55871415071)
            (3.09547738701, 3.37145286777)
            (3.15075376892, 3.19710284312)
            (3.20603015082, 3.03457023057)
            (3.26130653273, 2.88287053121)
            (3.31658291464, 2.74111605014)
            (3.37185929655, 2.60850497103)
            (3.42713567846, 2.48431181746)
            (3.48241206037, 2.36787910606)
            (3.53768844228, 2.25861002668)
            (3.59296482419, 2.15596200941)
            (3.6482412061, 2.05944105947)
            (3.70351758801, 1.96859675836)
            (3.75879396992, 1.88301784435)
            (3.81407035182, 1.80232829797)
            (3.86934673373, 1.72618386833)
            (3.92462311564, 1.65426898541)
            (3.97989949755, 1.58629401052)
            (4.03517587946, 1.52199278396)
            (4.09045226137, 1.46112043423)
            (4.14572864328, 1.4034514177)
            (4.20100502519, 1.34877776198)
            (4.2562814071, 1.29690748936)
            (4.31155778901, 1.24766319991)
            (4.36683417091, 1.20088079613)
            (4.42211055282, 1.15640833368)
            (4.47738693473, 1.11410498408)
            (4.53266331664, 1.07384009744)
            (4.58793969855, 1.0354923544)
            (4.64321608046, 0.998948997889)
            (4.69849246237, 0.964105136325)
            (4.75376884428, 0.930863110867)
            (4.80904522619, 0.899131920198)
            (4.8643216081, 0.868826696983)
            (4.91959799001, 0.839868230868)
            (4.97487437191, 0.812182533398)
            (5.03015075382, 0.785700440765)
            (5.08542713573, 0.760357250719)
            (5.14070351764, 0.73609239038)
            (5.19597989955, 0.712849112009)
            (5.25125628146, 0.690574214141)
            (5.30653266337, 0.669217785701)
            (5.36180904528, 0.648732971009)
            (5.41708542719, 0.629075753779)
            (5.4723618091, 0.610204758379)
            (5.527638191, 0.592081066851)
            (5.58291457291, 0.57466805026)
            (5.63819095482, 0.557931213157)
            (5.69346733673, 0.54183804999)
            (5.74874371864, 0.52635791247)
            (5.80402010055, 0.511461886926)
            (5.85929648246, 0.497122680848)
            (5.91457286437, 0.483314517819)
            (5.96984924628, 0.470013040166)
            (6.02512562819, 0.457195218681)
            (6.08040201009, 0.444839268861)
            (6.135678392, 0.432924573115)
            (6.19095477391, 0.421431608491)
            (6.24623115582, 0.410341879465)
            (6.30150753773, 0.39963785541)
            (6.35678391964, 0.389302912373)
            (6.41206030155, 0.37932127884)
            (6.46733668346, 0.369677985166)
            (6.52261306537, 0.360358816417)
            (6.57788944728, 0.351350268346)
            (6.63316582919, 0.342639506283)
            (6.68844221109, 0.334214326719)
            (6.743718593, 0.326063121394)
            (6.79899497491, 0.318174843692)
            (6.85427135682, 0.310538977196)
            (6.90954773873, 0.303145506231)
            (6.96482412064, 0.29598488827)
            (7.02010050255, 0.289048028058)
            (7.07537688446, 0.282326253346)
            (7.13065326637, 0.275811292115)
            (7.18592964828, 0.269495251187)
            (7.24120603018, 0.263370596145)
            (7.29648241209, 0.257430132445)
            (7.351758794, 0.251666987665)
            (7.40703517591, 0.246074594805)
            (7.46231155782, 0.240646676567)
            (7.51758793973, 0.235377230554)
            (7.57286432164, 0.230260515324)
            (7.62814070355, 0.225291037255)
            (7.68341708546, 0.220463538149)
            (7.73869346737, 0.21577298355)
            (7.79396984928, 0.21121455171)
            (7.84924623118, 0.206783623184)
            (7.90452261309, 0.20247577099)
            (7.959798995, 0.198286751322)
            (8.01507537691, 0.194212494768)
            (8.07035175882, 0.190249098003)
            (8.12562814073, 0.186392815938)
            (8.18090452264, 0.182640054285)
            (8.23618090455, 0.17898736252)
            (8.29145728646, 0.175431427218)
            (8.34673366837, 0.171969065741)
            (8.40201005027, 0.168597220253)
            (8.45728643218, 0.165312952049)
            (8.51256281409, 0.162113436176)
            (8.567839196, 0.158995956326)
            (8.62311557791, 0.155957899994)
            (8.67839195982, 0.15299675388)
            (8.73366834173, 0.150110099521)
            (8.78894472364, 0.147295609143)
            (8.84422110555, 0.144551041715)
            (8.89949748746, 0.141874239205)
            (8.95477386937, 0.139263123014)
            (9.01005025127, 0.136715690587)
            (9.06532663318, 0.134230012184)
            (9.12060301509, 0.131804227813)
            (9.175879397, 0.129436544304)
            (9.23115577891, 0.127125232528)
            (9.28643216082, 0.12486862474)
            (9.34170854273, 0.122665112057)
            (9.39698492464, 0.120513142044)
            (9.45226130655, 0.11841121642)
            (9.50753768846, 0.116357888862)
            (9.56281407036, 0.114351762923)
            (9.61809045227, 0.112391490028)
            (9.67336683418, 0.110475767578)
            (9.72864321609, 0.108603337126)
            (9.783919598, 0.106772982644)
            (9.83919597991, 0.104983528862)
            (9.89447236182, 0.103233839685)
            (9.94974874373, 0.101522816678)
            (10.0050251256, 0.0998493976154)
            (10.0603015075, 0.0982125550986)
            (10.1155778895, 0.0966112952292)
            (10.1708542714, 0.0950446563427)
            (10.2261306533, 0.0935117077947)
            (10.2814070352, 0.0920115488001)
            (10.3366834171, 0.0905433073214)
            (10.391959799, 0.0891061390037)
            (10.4472361809, 0.0876992261553)
            (10.5025125628, 0.0863217767701)
            (10.5577889447, 0.0849730235916)
            (10.6130653266, 0.083652223215)
            (10.6683417085, 0.0823586552265)
            (10.7236180905, 0.0810916213784)
            (10.7788944724, 0.0798504447973)
            (10.8341708543, 0.0786344692245)
            (10.8894472362, 0.0774430582873)
            (10.9447236181, 0.0762755947995)
            (11.0, 0.0751314800902)
    };
    


                
                    
                    \nextgroupplot[
                        % Default: empty ticks all round the border of the
                        % multiplot
                                xtick={  },
                                % 'left' means 'bottom'
                                xtick pos=left,
                                xticklabel=\empty,
                                ytick={  },
                                ytick pos=both,
                                yticklabel=\empty,
                            xtick={ 0, 1, 2, 3, 4, 5, 6, 7, 8, 9, 10, 11, 12, 13, 14, 15, 16, 17, 18, 19 },
                                xticklabel={},
                        axis equal=false,
                        %
                        title={  },
                        xlabel={  },
                        ylabel={  },
                    ]

                    

                    


    
    % Draw series plot
    \addplot[no markers,lightgray] coordinates {
            (1e-10, 153.922273845)
            (0.055276382009, 238.701111536)
            (0.110552763918, 361.056313561)
            (0.165829145827, 532.675812853)
            (0.221105527736, 766.511265817)
            (0.276381909645, 1075.82501607)
            (0.331658291554, 1472.76097042)
            (0.386934673463, 1966.48403786)
            (0.442211055372, 2561.03855182)
            (0.497487437281, 3253.18932019)
            (0.55276381919, 4030.60362478)
            (0.608040201099, 4870.77844795)
            (0.663316583009, 5741.0878394)
            (0.718592964918, 6600.20673603)
            (0.773869346827, 7400.96614734)
            (0.829145728736, 8094.44036483)
            (0.884422110645, 8634.8096701)
            (0.939698492554, 8984.34130705)
            (0.994974874463, 9117.7408907)
            (1.05025125637, 9025.17811761)
            (1.10552763828, 8713.48421234)
            (1.16080402019, 8205.31856637)
            (1.2160804021, 7536.4457376)
            (1.27135678401, 6751.57719513)
            (1.32663316592, 5899.44871358)
            (1.38190954783, 5027.88289267)
            (1.43718592974, 4179.52001033)
            (1.49246231164, 3388.7161727)
            (1.54773869355, 2679.8563395)
            (1.60301507546, 2067.0708493)
            (1.65829145737, 1555.1301561)
            (1.71356783928, 1141.15767361)
            (1.76884422119, 816.755557602)
            (1.8241206031, 570.172245296)
            (1.87939698501, 388.228647617)
            (1.93467336692, 257.831880711)
            (1.98994974883, 167.014118359)
            (2.04522613073, 105.520608141)
            (2.10050251264, 65.0262810893)
            (2.15577889455, 39.0848167358)
            (2.21105527646, 22.9136763353)
            (2.26633165837, 13.102344069)
            (2.32160804028, 7.3075325225)
            (2.37688442219, 3.97520971642)
            (2.4321608041, 2.1091956405)
            (2.48743718601, 1.09154392305)
            (2.54271356792, 0.550976528811)
            (2.59798994983, 0.27126421924)
            (2.65326633173, 0.130262514638)
            (2.70854271364, 0.0610118135359)
            (2.76381909555, 0.0278725011636)
            (2.81909547746, 0.0124195391716)
            (2.87437185937, 0.0053976230566)
            (2.92964824128, 0.00228805865943)
            (2.98492462319, 0.000946017755903)
            (3.0402010051, 0.000381503874194)
            (3.09547738701, 0.000150060416932)
            (3.15075376892, 5.75706159899e-05)
            (3.20603015082, 2.15428488938e-05)
            (3.26130653273, 7.86272208667e-06)
            (3.31658291464, 2.79904736217e-06)
            (3.37185929655, 9.71885494232e-07)
            (3.42713567846, 3.29145168536e-07)
            (3.48241206037, 1.08724495288e-07)
            (3.53768844228, 3.50295733213e-08)
            (3.59296482419, 1.10080356127e-08)
            (3.6482412061, 3.37405638346e-09)
            (3.70351758801, 1.00870082721e-09)
            (3.75879396992, 2.94130424394e-10)
            (3.81407035182, 8.36536805427e-11)
            (3.86934673373, 2.32058611239e-11)
            (3.92462311564, 6.27881743523e-12)
            (3.97989949755, 1.6570116187e-12)
            (4.03517587946, 4.26521357786e-13)
            (4.09045226137, 1.07083743193e-13)
            (4.14572864328, 2.62224844197e-14)
            (4.20100502519, 6.26313336597e-15)
            (4.2562814071, 1.45907297828e-15)
            (4.31155778901, 3.31535394929e-16)
            (4.36683417091, 7.34768142792e-17)
            (4.42211055282, 1.58832096147e-17)
            (4.47738693473, 3.34883498311e-18)
            (4.53266331664, 6.88678872168e-19)
            (4.58793969855, 1.38136135715e-19)
            (4.64321608046, 2.70249797477e-20)
            (4.69849246237, 5.15692699197e-21)
            (4.75376884428, 9.59807458082e-22)
            (4.80904522619, 1.7423876688e-22)
            (4.8643216081, 3.0851265896e-23)
            (4.91959799001, 5.32805525577e-24)
            (4.97487437191, 8.97494886451e-25)
            (5.03015075382, 1.4745611142e-25)
            (5.08542713573, 2.3629857484e-26)
            (5.14070351764, 3.69340522407e-27)
            (5.19597989955, 5.63067352872e-28)
            (5.25125628146, 8.37261780401e-29)
            (5.30653266337, 1.21431040892e-29)
            (5.36180904528, 1.71777271451e-30)
            (5.41708542719, 2.37011388022e-31)
            (5.4723618091, 3.18962964208e-32)
            (5.527638191, 4.18676741125e-33)
            (5.58291457291, 5.36024873276e-34)
            (5.63819095482, 6.69358190504e-35)
            (5.69346733673, 8.15266791471e-36)
            (5.74874371864, 9.68519689846e-37)
            (5.80402010055, 1.12223726944e-37)
            (5.85929648246, 1.26831886468e-38)
            (5.91457286437, 1.39810486861e-39)
            (5.96984924628, 1.50320625896e-40)
            (6.02512562819, 1.57639458395e-41)
            (6.08040201009, 1.61242240499e-42)
            (6.135678392, 1.6086451191e-43)
            (6.19095477391, 1.56534185707e-44)
            (6.24623115582, 1.48568138741e-45)
            (6.30150753773, 1.37533880934e-46)
            (6.35678391964, 1.24182742002e-47)
            (6.41206030155, 1.0936549508e-48)
            (6.46733668346, 9.39435425075e-50)
            (6.52261306537, 7.87083981648e-51)
            (6.57788944728, 6.43195128307e-52)
            (6.63316582919, 5.12662971042e-53)
            (6.68844221109, 3.98555388485e-54)
            (6.743718593, 3.02212867538e-55)
            (6.79899497491, 2.23514009666e-56)
            (6.85427135682, 1.61236766009e-57)
            (6.90954773873, 1.1344645757e-58)
            (6.96482412064, 7.78547888697e-60)
            (7.02010050255, 5.21131415483e-61)
            (7.07537688446, 3.40233215252e-62)
            (7.13065326637, 2.16657483963e-63)
            (7.18592964828, 1.34566889181e-64)
            (7.24120603018, 8.15211432597e-66)
            (7.29648241209, 4.81692452791e-67)
            (7.351758794, 2.77611180783e-68)
            (7.40703517591, 1.56052811059e-69)
            (7.46231155782, 8.55605899691e-71)
            (7.51758793973, 4.57555179769e-72)
            (7.57286432164, 2.38660591787e-73)
            (7.62814070355, 1.21418672846e-74)
            (7.68341708546, 6.02501012812e-76)
            (7.73869346737, 2.91606776708e-77)
            (7.79396984928, 1.37659115547e-78)
            (7.84924623118, 6.33840329609e-80)
            (7.90452261309, 2.846572792e-81)
            (7.959798995, 1.24690179626e-82)
            (8.01507537691, 5.32733170091e-84)
            (8.07035175882, 2.22000911634e-85)
            (8.12562814073, 9.02333972238e-87)
            (8.18090452264, 3.57723433563e-88)
            (8.23618090455, 1.38323192549e-89)
            (8.29145728646, 5.21687161265e-91)
            (8.34673366837, 1.919078876e-92)
            (8.40201005027, 6.88561970114e-94)
            (8.45728643218, 2.40968775892e-95)
            (8.51256281409, 8.22519194892e-97)
            (8.567839196, 2.73841234132e-98)
            (8.62311557791, 8.8924038166e-100)
            (8.67839195982, 2.816482394e-101)
            (8.73366834173, 8.70086512426e-103)
            (8.78894472364, 2.62171430444e-104)
            (8.84422110555, 7.70505640152e-106)
            (8.89949748746, 2.20868515386e-107)
            (8.95477386937, 6.17531809333e-109)
            (9.01005025127, 1.68403993397e-110)
            (9.06532663318, 4.47932908107e-112)
            (9.12060301509, 1.16209353492e-113)
            (9.175879397, 2.9406052738e-115)
            (9.23115577891, 7.257714996e-117)
            (9.28643216082, 1.74715171671e-118)
            (9.34170854273, 4.10231363095e-120)
            (9.39698492464, 9.3949534422e-122)
            (9.45226130655, 2.09859162704e-123)
            (9.50753768846, 4.57223738856e-125)
            (9.56281407036, 9.7162154625e-127)
            (9.61809045227, 2.0138774296e-128)
            (9.67336683418, 4.0713314835e-130)
            (9.72864321609, 8.02800111244e-132)
            (9.783919598, 1.54399509293e-133)
            (9.83919597991, 2.89635598391e-135)
            (9.89447236182, 5.29938525888e-137)
            (9.94974874373, 9.45728722734e-139)
            (10.0050251256, 1.64617194689e-140)
            (10.0603015075, 2.79480392826e-142)
            (10.1155778895, 4.62801784611e-144)
            (10.1708542714, 7.47491461775e-146)
            (10.2261306533, 1.17756524418e-147)
            (10.2814070352, 1.80938611408e-149)
            (10.3366834171, 2.71172133788e-151)
            (10.391959799, 3.96393382173e-153)
            (10.4472361809, 5.6516500623e-155)
            (10.5025125628, 7.8594410886e-157)
            (10.5577889447, 1.06604513828e-158)
            (10.6130653266, 1.41035049782e-160)
            (10.6683417085, 1.81989381829e-162)
            (10.7236180905, 2.29051201811e-164)
            (10.7788944724, 2.81181427356e-166)
            (10.8341708543, 3.36672945786e-168)
            (10.8894472362, 3.93185359343e-170)
            (10.9447236181, 4.47872078806e-172)
            (11.0, 4.97597477851e-174)
    };
    

    
    % Draw series plot
    \addplot[no markers,lightgray] coordinates {
            (1e-10, 0.229649734272)
            (0.055276382009, 0.358366194811)
            (0.110552763918, 0.552295988826)
            (0.165829145827, 0.840621529144)
            (0.221105527736, 1.26360953293)
            (0.276381909645, 1.87589724018)
            (0.331658291554, 2.75035631272)
            (0.386934673463, 3.98247119403)
            (0.442211055372, 5.69508343484)
            (0.497487437281, 8.04324512793)
            (0.55276381919, 11.2187968479)
            (0.608040201099, 15.4541466365)
            (0.663316583009, 21.024590191)
            (0.718592964918, 28.2483974875)
            (0.773869346827, 37.4838211993)
            (0.829145728736, 49.1221837412)
            (0.884422110645, 63.5762984274)
            (0.939698492554, 81.26369727)
            (0.994974874463, 102.584484953)
            (1.05025125637, 127.894112551)
            (1.10552763828, 157.47194396)
            (1.16080402019, 191.487130243)
            (1.2160804021, 229.963949083)
            (1.27135678401, 272.74932919)
            (1.32663316592, 319.485674844)
            (1.38190954783, 369.592248772)
            (1.43718592974, 422.258192783)
            (1.49246231164, 476.449724882)
            (1.54773869355, 530.933149662)
            (1.60301507546, 584.314103984)
            (1.65829145737, 635.092029359)
            (1.71356783928, 681.727354948)
            (1.76884422119, 722.717457043)
            (1.8241206031, 756.676306133)
            (1.87939698501, 782.411978476)
            (1.93467336692, 798.996012809)
            (1.98994974883, 805.818992434)
            (2.04522613073, 802.62771468)
            (2.10050251264, 789.540787009)
            (2.15577889455, 767.041310386)
            (2.21105527646, 735.947279202)
            (2.26633165837, 697.362226379)
            (2.32160804028, 652.610265062)
            (2.37688442219, 603.160853996)
            (2.4321608041, 550.549230472)
            (2.48743718601, 496.298472031)
            (2.54271356792, 441.848598702)
            (2.59798994983, 388.497109175)
            (2.65326633173, 337.354002203)
            (2.70854271364, 289.31283946)
            (2.76381909555, 245.037929586)
            (2.81909547746, 204.966406275)
            (2.87437185937, 169.322949328)
            (2.92964824128, 138.144223058)
            (2.98492462319, 111.309798756)
            (3.0402010051, 88.5763606895)
            (3.09547738701, 69.6123080384)
            (3.15075376892, 54.0303772272)
            (3.20603015082, 41.4165321142)
            (3.26130653273, 31.3540198915)
            (3.31658291464, 23.4420988523)
            (3.37185929655, 17.3094601046)
            (3.42713567846, 12.6227597671)
            (3.48241206037, 9.09094179978)
            (3.53768844228, 6.46617126974)
            (3.59296482419, 4.54223182119)
            (3.6482412061, 3.15119408804)
            (3.70351758801, 2.15906032199)
            (3.75879396992, 1.46095946089)
            (3.81407035182, 0.976327045037)
            (3.86934673373, 0.644371375917)
            (3.92462311564, 0.420011252816)
            (3.97989949755, 0.270376771349)
            (4.03517587946, 0.171894347211)
            (4.09045226137, 0.107928857072)
            (4.14572864328, 0.0669263787382)
            (4.20100502519, 0.0409865006981)
            (4.2562814071, 0.0247895202082)
            (4.31155778901, 0.0148074122279)
            (4.36683417091, 0.00873522263719)
            (4.42211055282, 0.00508923558952)
            (4.47738693473, 0.00292829548851)
            (4.53266331664, 0.00166402944512)
            (4.58793969855, 0.000933879583776)
            (4.64321608046, 0.000517612220192)
            (4.69849246237, 0.000283336109631)
            (4.75376884428, 0.000153173310134)
            (4.80904522619, 8.17801748288e-05)
            (4.8643216081, 4.31217857295e-05)
            (4.91959799001, 2.24558348445e-05)
            (4.97487437191, 1.15490281394e-05)
            (5.03015075382, 5.86604566081e-06)
            (5.08542713573, 2.94258627184e-06)
            (5.14070351764, 1.45779593719e-06)
            (5.19597989955, 7.13260267711e-07)
            (5.25125628146, 3.44653818603e-07)
            (5.30653266337, 1.64475767907e-07)
            (5.36180904528, 7.75183493601e-08)
            (5.41708542719, 3.60820227432e-08)
            (5.4723618091, 1.65867386243e-08)
            (5.527638191, 7.53034531498e-09)
            (5.58291457291, 3.37638955711e-09)
            (5.63819095482, 1.49511283769e-09)
            (5.69346733673, 6.53851551223e-10)
            (5.74874371864, 2.82402223593e-10)
            (5.80402010055, 1.20459437597e-10)
            (5.85929648246, 5.07454743067e-11)
            (5.91457286437, 2.11123981136e-11)
            (5.96984924628, 8.67484219044e-12)
            (6.02512562819, 3.52021642498e-12)
            (6.08040201009, 1.41078525727e-12)
            (6.135678392, 5.58388192137e-13)
            (6.19095477391, 2.18270631527e-13)
            (6.24623115582, 8.42632391405e-14)
            (6.30150753773, 3.21265995613e-14)
            (6.35678391964, 1.20969291256e-14)
            (6.41206030155, 4.49851608757e-15)
            (6.46733668346, 1.65214129093e-15)
            (6.52261306537, 5.99251125674e-16)
            (6.57788944728, 2.1466156799e-16)
            (6.63316582919, 7.59422572835e-17)
            (6.68844221109, 2.65336179771e-17)
            (6.743718593, 9.15573486802e-18)
            (6.79899497491, 3.12013732936e-18)
            (6.85427135682, 1.0501176951e-18)
            (6.90954773873, 3.49048687722e-19)
            (6.96482412064, 1.14582371929e-19)
            (7.02010050255, 3.71478234185e-20)
            (7.07537688446, 1.18941305893e-20)
            (7.13065326637, 3.76110794591e-21)
            (7.18592964828, 1.17458015621e-21)
            (7.24120603018, 3.62270763562e-22)
            (7.29648241209, 1.1034881637e-22)
            (7.351758794, 3.31960062408e-23)
            (7.40703517591, 9.86251692796e-24)
            (7.46231155782, 2.89383291046e-24)
            (7.51758793973, 8.38576923128e-25)
            (7.57286432164, 2.39991647985e-25)
            (7.62814070355, 6.78317645503e-26)
            (7.68341708546, 1.89345011342e-26)
            (7.73869346737, 5.21985435538e-27)
            (7.79396984928, 1.4211720881e-27)
            (7.84924623118, 3.82136673615e-28)
            (7.90452261309, 1.01478615669e-28)
            (7.959798995, 2.66142428266e-29)
            (8.01507537691, 6.89346320224e-30)
            (8.07035175882, 1.76337457194e-30)
            (8.12562814073, 4.45487427464e-31)
            (8.18090452264, 1.11150160858e-31)
            (8.23618090455, 2.73885162211e-32)
            (8.29145728646, 6.66516154418e-33)
            (8.34673366837, 1.60190449719e-33)
            (8.40201005027, 3.80229954008e-34)
            (8.45728643218, 8.91332616823e-35)
            (8.51256281409, 2.06355969555e-35)
            (8.567839196, 4.71821840407e-36)
            (8.62311557791, 1.065424819e-36)
            (8.67839195982, 2.37602685116e-37)
            (8.73366834173, 5.2331555604e-38)
            (8.78894472364, 1.13830779778e-38)
            (8.84422110555, 2.44534163758e-39)
            (8.89949748746, 5.18803790328e-40)
            (8.95477386937, 1.08705246097e-40)
            (9.01005025127, 2.2494773908e-41)
            (9.06532663318, 4.5972331226e-42)
            (9.12060301509, 9.27887207712e-43)
            (9.175879397, 1.84959913378e-43)
            (9.23115577891, 3.64119388747e-44)
            (9.28643216082, 7.07935616418e-45)
            (9.34170854273, 1.35933808667e-45)
            (9.39698492464, 2.57777480182e-46)
            (9.45226130655, 4.82776618476e-47)
            (9.50753768846, 8.9295844428e-48)
            (9.56281407036, 1.63117303806e-48)
            (9.61809045227, 2.94274469697e-49)
            (9.67336683418, 5.24310919843e-50)
            (9.72864321609, 9.22590515643e-51)
            (9.783919598, 1.60329277018e-51)
            (9.83919597991, 2.75169597367e-52)
            (9.89447236182, 4.66414268912e-53)
            (9.94974874373, 7.80776932241e-54)
            (10.0050251256, 1.29082059225e-54)
            (10.0603015075, 2.10760180545e-55)
            (10.1155778895, 3.39856049256e-56)
            (10.1708542714, 5.41234176863e-57)
            (10.2261306533, 8.51254041436e-58)
            (10.2814070352, 1.32226020173e-58)
            (10.3366834171, 2.02842255853e-59)
            (10.391959799, 3.07314960506e-60)
            (10.4472361809, 4.59825169405e-61)
            (10.5025125628, 6.79493868086e-62)
            (10.5577889447, 9.91658434235e-63)
            (10.6130653266, 1.42929687359e-63)
            (10.6683417085, 2.03454145453e-64)
            (10.7236180905, 2.86018686789e-65)
            (10.7788944724, 3.97105625223e-66)
            (10.8341708543, 5.44504452784e-67)
            (10.8894472362, 7.37361737442e-68)
            (10.9447236181, 9.86151241203e-69)
            (11.0, 1.30253745127e-69)
    };
    

    
    % Draw series plot
    \addplot[no markers,lightgray] coordinates {
            (1e-10, 0.00093781246772)
            (0.055276382009, 0.00146649233023)
            (0.110552763918, 0.0022742215311)
            (0.165829145827, 0.00349763845303)
            (0.221105527736, 0.00533465379491)
            (0.276381909645, 0.0080691305582)
            (0.331658291554, 0.0121042092197)
            (0.386934673463, 0.0180067476507)
            (0.442211055372, 0.026565825279)
            (0.497487437281, 0.0388687454025)
            (0.55276381919, 0.056398416531)
            (0.608040201099, 0.0811563442595)
            (0.663316583009, 0.115815641717)
            (0.718592964918, 0.163908372157)
            (0.773869346827, 0.230051057275)
            (0.829145728736, 0.320211191844)
            (0.884422110645, 0.442015967003)
            (0.939698492554, 0.60510199596)
            (0.994974874463, 0.821501556)
            (1.05025125637, 1.10605665292)
            (1.10552763828, 1.47684708971)
            (1.16080402019, 1.95561278467)
            (1.2160804021, 2.56814405602)
            (1.27135678401, 3.34460682501)
            (1.32663316592, 4.31976318892)
            (1.38190954783, 5.53304221879)
            (1.43718592974, 7.02841190816)
            (1.49246231164, 8.85400177873)
            (1.54773869355, 11.0614275974)
            (1.60301507546, 13.70477577)
            (1.65829145737, 16.8392158779)
            (1.71356783928, 20.5192258699)
            (1.76884422119, 24.7964355821)
            (1.8241206031, 29.7171200306)
            (1.87939698501, 35.3194032562)
            (1.93467336692, 41.6302647928)
            (1.98994974883, 48.6624719468)
            (2.04522613073, 56.4115894535)
            (2.10050251264, 64.8532409062)
            (2.15577889455, 73.9408107769)
            (2.21105527646, 83.6037792744)
            (2.26633165837, 93.7468726306)
            (2.32160804028, 104.250187448)
            (2.37688442219, 114.970409306)
            (2.4321608041, 125.743193976)
            (2.48743718601, 136.386716776)
            (2.54271356792, 146.706325329)
            (2.59798994983, 156.500158078)
            (2.65326633173, 165.565520633)
            (2.70854271364, 173.705750057)
            (2.76381909555, 180.737249017)
            (2.81909547746, 186.496341892)
            (2.87437185937, 190.845596938)
            (2.92964824128, 193.67927415)
            (2.98492462319, 194.927597517)
            (3.0402010051, 194.559610689)
            (3.09547738701, 192.584452888)
            (3.15075376892, 189.05098166)
            (3.20603015082, 184.045764187)
            (3.26130653273, 177.689552465)
            (3.31658291464, 170.132442734)
            (3.37185929655, 161.547989965)
            (3.42713567846, 152.126599321)
            (3.48241206037, 142.068544939)
            (3.53768844228, 131.576971181)
            (3.59296482419, 120.851213227)
            (3.6482412061, 110.080735199)
            (3.70351758801, 99.4399289527)
            (3.75879396992, 89.08395018)
            (3.81407035182, 79.1456963513)
            (3.86934673373, 69.7339586352)
            (3.92462311564, 60.9327125626)
            (3.97989949755, 52.801453975)
            (4.03517587946, 45.3764408318)
            (4.09045226137, 38.6726695445)
            (4.14572864328, 32.6863971343)
            (4.20100502519, 27.3980170133)
            (4.2562814071, 22.7751049096)
            (4.31155778901, 18.7754700454)
            (4.36683417091, 15.3500723852)
            (4.42211055282, 12.4456967348)
            (4.47738693473, 10.007305959)
            (4.53266331664, 7.9800262264)
            (4.58793969855, 6.31074506774)
            (4.64321608046, 4.94932680753)
            (4.69849246237, 3.8494687967)
            (4.75376884428, 2.96923555861)
            (4.80904522619, 2.27131662234)
            (4.8643216081, 1.7230579453)
            (4.91959799001, 1.29631713998)
            (4.97487437191, 0.96719005225)
            (5.03015075382, 0.715651450608)
            (5.08542713573, 0.525146490997)
            (5.14070351764, 0.382162929137)
            (5.19597989955, 0.275807342184)
            (5.25125628146, 0.197402326403)
            (5.30653266337, 0.140116050442)
            (5.36180904528, 0.098630829272)
            (5.41708542719, 0.0688536011291)
            (5.4723618091, 0.0476683161627)
            (5.527638191, 0.0327282014829)
            (5.58291457291, 0.0222845385861)
            (5.63819095482, 0.0150478457028)
            (5.69346733673, 0.0100770680127)
            (5.74874371864, 0.00669242059747)
            (5.80402010055, 0.00440779554637)
            (5.85929648246, 0.00287904752188)
            (5.91457286437, 0.00186494231151)
            (5.96984924628, 0.00119803953034)
            (6.02512562819, 0.000763248696503)
            (6.08040201009, 0.000482225499326)
            (6.135678392, 0.000302150591494)
            (6.19095477391, 0.000187752575516)
            (6.24623115582, 0.000115701112477)
            (6.30150753773, 7.07095939689e-05)
            (6.35678391964, 4.28556739506e-05)
            (6.41206030155, 2.57589106195e-05)
            (6.46733668346, 1.5354505425e-05)
            (6.52261306537, 9.07681218783e-06)
            (6.57788944728, 5.32132817816e-06)
            (6.63316582919, 3.09382621512e-06)
            (6.68844221109, 1.78386080635e-06)
            (6.743718593, 1.02003527128e-06)
            (6.79899497491, 5.78440365751e-07)
            (6.85427135682, 3.25305324661e-07)
            (6.90954773873, 1.81431599358e-07)
            (6.96482412064, 1.00351498767e-07)
            (7.02010050255, 5.50457729457e-08)
            (7.07537688446, 2.9944237784e-08)
            (7.13065326637, 1.61544333305e-08)
            (7.18592964828, 8.64289766776e-09)
            (7.24120603018, 4.585811412e-09)
            (7.29648241209, 2.41302703471e-09)
            (7.351758794, 1.25920767366e-09)
            (7.40703517591, 6.51660973353e-10)
            (7.46231155782, 3.34453111779e-10)
            (7.51758793973, 1.70230724067e-10)
            (7.57286432164, 8.59270355189e-11)
            (7.62814070355, 4.30141047637e-11)
            (7.68341708546, 2.13540927772e-11)
            (7.73869346737, 1.0513335872e-11)
            (7.79396984928, 5.13321116413e-12)
            (7.84924623118, 2.48557514955e-12)
            (7.90452261309, 1.19358634993e-12)
            (7.959798995, 5.68420805674e-13)
            (8.01507537691, 2.68457329276e-13)
            (8.07035175882, 1.2573892468e-13)
            (8.12562814073, 5.84054528584e-14)
            (8.18090452264, 2.69045806033e-14)
            (8.23618090455, 1.2291028749e-14)
            (8.29145728646, 5.56851532992e-15)
            (8.34673366837, 2.50195665674e-15)
            (8.40201005027, 1.11483173048e-15)
            (8.45728643218, 4.92638146559e-16)
            (8.51256281409, 2.15891692593e-16)
            (8.567839196, 9.38281172595e-17)
            (8.62311557791, 4.04407545158e-17)
            (8.67839195982, 1.72860066094e-17)
            (8.73366834173, 7.32755820406e-18)
            (8.78894472364, 3.08044214112e-18)
            (8.84422110555, 1.28426903772e-18)
            (8.89949748746, 5.30992205148e-19)
            (8.95477386937, 2.17725580103e-19)
            (9.01005025127, 8.8536009603e-20)
            (9.06532663318, 3.57042248515e-20)
            (9.12060301509, 1.42793503984e-20)
            (9.175879397, 5.66352040894e-21)
            (9.23115577891, 2.22768441261e-21)
            (9.28643216082, 8.68980516821e-22)
            (9.34170854273, 3.36167395794e-22)
            (9.39698492464, 1.28970480343e-22)
            (9.45226130655, 4.90697925591e-23)
            (9.50753768846, 1.85151529414e-23)
            (9.56281407036, 6.92834591424e-24)
            (9.61809045227, 2.57111202051e-24)
            (9.67336683418, 9.46240673485e-25)
            (9.72864321609, 3.45359496305e-25)
            (9.783919598, 1.25005859252e-25)
            (9.83919597991, 4.48723189749e-26)
            (9.89447236182, 1.59740792922e-26)
            (9.94974874373, 5.63952251215e-27)
            (10.0050251256, 1.97450396664e-27)
            (10.0603015075, 6.85587457174e-28)
            (10.1155778895, 2.36078748229e-28)
            (10.1708542714, 8.06195001205e-29)
            (10.2261306533, 2.73031330468e-29)
            (10.2814070352, 9.1700994769e-30)
            (10.3366834171, 3.05439208434e-30)
            (10.391959799, 1.00893851569e-30)
            (10.4472361809, 3.30516993952e-31)
            (10.5025125628, 1.07377197336e-31)
            (10.5577889447, 3.45954920029e-32)
            (10.6130653266, 1.10539153788e-32)
            (10.6683417085, 3.50269189129e-33)
            (10.7236180905, 1.10072014657e-33)
            (10.7788944724, 3.43037083325e-34)
            (10.8341708543, 1.06021612171e-34)
            (10.8894472362, 3.24965243507e-35)
            (10.9447236181, 9.87799042173e-36)
            (11.0, 2.97775844387e-36)
    };
    

    
    % Draw series plot
    \addplot[no markers,lightgray] coordinates {
            (1e-10, 5.78362179496e-06)
            (0.055276382009, 9.05346928034e-06)
            (0.110552763918, 1.4083871755e-05)
            (0.165829145827, 2.17731354423e-05)
            (0.221105527736, 3.34512051956e-05)
            (0.276381909645, 5.10733620392e-05)
            (0.331658291554, 7.74941659188e-05)
            (0.386934673463, 0.000116851812178)
            (0.442211055372, 0.000175103077295)
            (0.497487437281, 0.000260761810613)
            (0.55276381919, 0.000385910042361)
            (0.608040201099, 0.00056757081461)
            (0.663316583009, 0.000829556390325)
            (0.718592964918, 0.00120493509527)
            (0.773869346827, 0.00173929511643)
            (0.829145728736, 0.00249502432449)
            (0.884422110645, 0.00355687148113)
            (0.939698492554, 0.00503910543599)
            (0.994974874463, 0.00709464387591)
            (1.05025125637, 0.00992657979986)
            (1.10552763828, 0.0138025890863)
            (1.16080402019, 0.0190727520285)
            (1.2160804021, 0.0261913599257)
            (1.27135678401, 0.035743297669)
            (1.32663316592, 0.0484755862442)
            (1.38190954783, 0.065334625317)
            (1.43718592974, 0.087509584653)
            (1.49246231164, 0.116482242522)
            (1.54773869355, 0.154083348021)
            (1.60301507546, 0.202555282028)
            (1.65829145737, 0.264620400153)
            (1.71356783928, 0.343553955986)
            (1.76884422119, 0.443259924887)
            (1.8241206031, 0.56834738502)
            (1.87939698501, 0.724204379131)
            (1.93467336692, 0.917065403468)
            (1.98994974883, 1.15406788558)
            (2.04522613073, 1.44329226789)
            (2.10050251264, 1.79377966683)
            (2.15577889455, 2.21552059492)
            (2.21105527646, 2.71940798843)
            (2.26633165837, 3.31714785195)
            (2.32160804028, 4.02112128531)
            (2.37688442219, 4.84419256254)
            (2.4321608041, 5.79945933303)
            (2.48743718601, 6.89994293745)
            (2.54271356792, 8.15821926886)
            (2.59798994983, 9.58599352027)
            (2.65326633173, 11.1936254616)
            (2.70854271364, 12.9896154556)
            (2.76381909555, 14.9800650871)
            (2.81909547746, 17.1681298388)
            (2.87437185937, 19.5534844675)
            (2.92964824128, 22.1318243707)
            (2.98492462319, 24.8944280306)
            (3.0402010051, 27.8278063564)
            (3.09547738701, 30.9134642159)
            (3.15075376892, 34.127797516)
            (3.20603015082, 37.4421457975)
            (3.26130653273, 40.8230154706)
            (3.31658291464, 44.2324826578)
            (3.37185929655, 47.6287773454)
            (3.42713567846, 50.9670424901)
            (3.48241206037, 54.2002532777)
            (3.53768844228, 57.2802733358)
            (3.59296482419, 60.1590168634)
            (3.6482412061, 62.7896788341)
            (3.70351758801, 65.127990129)
            (3.75879396992, 67.1334510457)
            (3.81407035182, 68.7704954082)
            (3.86934673373, 70.0095386413)
            (3.92462311564, 70.8278667055)
            (3.97989949755, 71.2103285872)
            (4.03517587946, 71.1498028519)
            (4.09045226137, 70.6474181891)
            (4.14572864328, 69.7125184043)
            (4.20100502519, 68.3623733778)
            (4.2562814071, 66.6216484808)
            (4.31155778901, 64.5216552222)
            (4.36683417091, 62.0994149298)
            (4.42211055282, 59.3965745632)
            (4.47738693473, 56.4582189601)
            (4.53266331664, 53.3316267019)
            (4.58793969855, 50.0650172677)
            (4.64321608046, 46.70633531)
            (4.69849246237, 43.3021139214)
            (4.75376884428, 39.8964530091)
            (4.80904522619, 36.5301417616)
            (4.8643216081, 33.2399461467)
            (4.91959799001, 30.0580739278)
            (4.97487437191, 27.0118212964)
            (5.03015075382, 24.1233973456)
            (5.08542713573, 21.4099156173)
            (5.14070351764, 18.8835361392)
            (5.19597989955, 16.551736915)
            (5.25125628146, 14.4176908361)
            (5.30653266337, 12.4807224467)
            (5.36180904528, 10.7368188227)
            (5.41708542719, 9.17916987132)
            (5.4723618091, 7.79871540736)
            (5.527638191, 6.58467918372)
            (5.58291457291, 5.52507338984)
            (5.63819095482, 4.60716074187)
            (5.69346733673, 3.81786494532)
            (5.74874371864, 3.14412381994)
            (5.80402010055, 2.5731825825)
            (5.85929648246, 2.09282757123)
            (5.91457286437, 1.69156299582)
            (5.96984924628, 1.35873507567)
            (6.02512562819, 1.08460919128)
            (6.08040201009, 0.860406449008)
            (6.135678392, 0.678306402222)
            (6.19095477391, 0.531422647062)
            (6.24623115582, 0.413757694527)
            (6.30150753773, 0.320142985851)
            (6.35678391964, 0.246169238259)
            (6.41206030155, 0.18811154712)
            (6.46733668346, 0.14285288368)
            (6.52261306537, 0.107808859281)
            (6.57788944728, 0.0808559106643)
            (6.63316582919, 0.0602644195235)
            (6.68844221109, 0.0446377263501)
            (6.743718593, 0.0328575392982)
            (6.79899497491, 0.0240358723778)
            (6.85427135682, 0.017473368085)
            (6.90954773873, 0.0126236587148)
            (6.96482412064, 0.00906328729148)
            (7.02010050255, 0.00646663176741)
            (7.07537688446, 0.00458524344857)
            (7.13065326637, 0.00323101183985)
            (7.18592964828, 0.00226259374806)
            (7.24120603018, 0.00157458641036)
            (7.29648241209, 0.00108897597355)
            (7.351758794, 0.000748448598849)
            (7.40703517591, 0.000511207885569)
            (7.46231155782, 0.000346996418168)
            (7.51758793973, 0.0002340692277)
            (7.57286432164, 0.000156911775065)
            (7.62814070355, 0.000104534255719)
            (7.68341708546, 6.92075715347e-05)
            (7.73869346737, 4.55344895069e-05)
            (7.79396984928, 2.97727682266e-05)
            (7.84924623118, 1.93459403321e-05)
            (7.90452261309, 1.24925861829e-05)
            (7.959798995, 8.01690518512e-06)
            (8.01507537691, 5.11273193882e-06)
            (8.07035175882, 3.24034447705e-06)
            (8.12562814073, 2.04089764048e-06)
            (8.18090452264, 1.27744775683e-06)
            (8.23618090455, 7.94615359211e-07)
            (8.29145728646, 4.91204849122e-07)
            (8.34673366837, 3.01758990683e-07)
            (8.40201005027, 1.84225470459e-07)
            (8.45728643218, 1.11771482915e-07)
            (8.51256281409, 6.73913621149e-08)
            (8.567839196, 4.03802811827e-08)
            (8.62311557791, 2.40450855151e-08)
            (8.67839195982, 1.42290267552e-08)
            (8.73366834173, 8.36788973626e-09)
            (8.78894472364, 4.89044731477e-09)
            (8.84422110555, 2.84035813709e-09)
            (8.89949748746, 1.63941726545e-09)
            (8.95477386937, 9.40367884744e-10)
            (9.01005025127, 5.36040934757e-10)
            (9.06532663318, 3.03661690735e-10)
            (9.12060301509, 1.7095190257e-10)
            (9.175879397, 9.5642243335e-11)
            (9.23115577891, 5.31762187845e-11)
            (9.28643216082, 2.93817076994e-11)
            (9.34170854273, 1.61334964565e-11)
            (9.39698492464, 8.80383399701e-12)
            (9.45226130655, 4.77427108545e-12)
            (9.50753768846, 2.57296683028e-12)
            (9.56281407036, 1.37801258218e-12)
            (9.61809045227, 7.33439104458e-13)
            (9.67336683418, 3.87942019732e-13)
            (9.72864321609, 2.03920771973e-13)
            (9.783919598, 1.06524129384e-13)
            (9.83919597991, 5.53001619464e-14)
            (9.89447236182, 2.85296665479e-14)
            (9.94974874373, 1.46271215749e-14)
            (10.0050251256, 7.45268781233e-15)
            (10.0603015075, 3.77362612955e-15)
            (10.1155778895, 1.89887638862e-15)
            (10.1708542714, 9.4956872248e-16)
            (10.2261306533, 4.71897841353e-16)
            (10.2814070352, 2.33056630483e-16)
            (10.3366834171, 1.14384405271e-16)
            (10.391959799, 5.57909885411e-17)
            (10.4472361809, 2.70428947293e-17)
            (10.5025125628, 1.30266930941e-17)
            (10.5577889447, 6.23601380656e-18)
            (10.6130653266, 2.96668774466e-18)
            (10.6683417085, 1.40258273894e-18)
            (10.7236180905, 6.5898726985e-19)
            (10.7788944724, 3.07692877382e-19)
            (10.8341708543, 1.42774202828e-19)
            (10.8894472362, 6.5837590402e-20)
            (10.9447236181, 3.01710201054e-20)
            (11.0, 1.37403546905e-20)
    };
    

    
    % Draw series plot
    \addplot[no markers,lightgray] coordinates {
            (1e-10, 4.47079031791e-08)
            (0.055276382009, 7.00277641055e-08)
            (0.110552763918, 1.09141451372e-07)
            (0.165829145827, 1.69255463211e-07)
            (0.221105527736, 2.61173538367e-07)
            (0.276381909645, 4.01004417037e-07)
            (0.331658291554, 6.12636187172e-07)
            (0.386934673463, 9.31300084028e-07)
            (0.442211055372, 1.40867285036e-06)
            (0.497487437281, 2.12013811171e-06)
            (0.55276381919, 3.17505803452e-06)
            (0.608040201099, 4.73121505312e-06)
            (0.663316583009, 7.01499258574e-06)
            (0.718592964918, 1.03494018455e-05)
            (0.773869346827, 1.51927639393e-05)
            (0.829145728736, 2.21917645594e-05)
            (0.884422110645, 3.2253762867e-05)
            (0.939698492554, 4.66447153794e-05)
            (0.994974874463, 6.71209372901e-05)
            (1.05025125637, 9.61052435731e-05)
            (1.10552763828, 0.00013692087357)
            (1.16080402019, 0.00019410009376)
            (1.2160804021, 0.000273788583776)
            (1.27135678401, 0.000384271726569)
            (1.32663316592, 0.000536654820021)
            (1.38190954783, 0.000745736059062)
            (1.43718592974, 0.00103111892652)
            (1.49246231164, 0.00141861935228)
            (1.54773869355, 0.00194203256366)
            (1.60301507546, 0.0026453347795)
            (1.65829145737, 0.00358540551496)
            (1.71356783928, 0.00483536684997)
            (1.76884422119, 0.00648864600816)
            (1.8241206031, 0.00866387625629)
            (1.87939698501, 0.0115107575409)
            (1.93467336692, 0.0152170013046)
            (1.98994974883, 0.0200164822514)
            (2.04522613073, 0.0261987119574)
            (2.10050251264, 0.0341197335152)
            (2.15577889455, 0.0442145111141)
            (2.21105527646, 0.0570108518688)
            (2.26633165837, 0.0731448476547)
            (2.32160804028, 0.0933777607939)
            (2.37688442219, 0.118614198116)
            (2.4321608041, 0.14992132275)
            (2.48743718601, 0.188548742266)
            (2.54271356792, 0.235948586764)
            (2.59798994983, 0.293795153567)
            (2.65326633173, 0.364003350067)
            (2.70854271364, 0.448745018127)
            (2.76381909555, 0.550462078876)
            (2.81909547746, 0.671875303877)
            (2.87437185937, 0.815987406899)
            (2.92964824128, 0.986079070399)
            (2.98492462319, 1.1856964835)
            (3.0402010051, 1.4186289851)
            (3.09547738701, 1.68887548785)
            (3.15075376892, 2.0005985152)
            (3.20603015082, 2.35806492336)
            (3.26130653273, 2.76557270576)
            (3.31658291464, 3.22736369233)
            (3.37185929655, 3.74752245414)
            (3.42713567846, 4.3298622981)
            (3.48241206037, 4.97779987112)
            (3.53768844228, 5.69422056841)
            (3.59296482419, 6.48133763076)
            (3.6482412061, 7.3405484891)
            (3.70351758801, 8.27229253829)
            (3.75879396992, 9.27591505653)
            (3.81407035182, 10.3495423966)
            (3.86934673373, 11.4899738219)
            (3.92462311564, 12.6925954124)
            (3.97989949755, 13.9513212968)
            (4.03517587946, 15.2585670578)
            (4.09045226137, 16.6052595019)
            (4.14572864328, 17.9808860849)
            (4.20100502519, 19.3735861612)
            (4.2562814071, 20.7702849026)
            (4.31155778901, 22.1568692647)
            (4.36683417091, 23.5184038094)
            (4.42211055282, 24.8393825972)
            (4.47738693473, 26.1040118047)
            (4.53266331664, 27.2965162807)
            (4.58793969855, 28.4014620053)
            (4.64321608046, 29.4040854143)
            (4.69849246237, 30.2906198798)
            (4.75376884428, 31.0486093149)
            (4.80904522619, 31.6671989516)
            (4.8643216081, 32.1373938187)
            (4.91959799001, 32.4522763219)
            (4.97487437191, 32.6071755743)
            (5.03015075382, 32.5997826817)
            (5.08542713573, 32.4302080073)
            (5.14070351764, 32.1009784306)
            (5.19597989955, 31.6169746947)
            (5.25125628146, 30.9853110152)
            (5.30653266337, 30.2151610982)
            (5.36180904528, 29.3175365172)
            (5.41708542719, 28.3050249278)
            (5.4723618091, 27.191496813)
            (5.527638191, 25.9917902927)
            (5.58291457291, 24.7213839723)
            (5.63819095482, 23.3960678481)
            (5.69346733673, 22.0316219348)
            (5.74874371864, 20.6435115692)
            (5.80402010055, 19.246607327)
            (5.85929648246, 17.8549362109)
            (5.91457286437, 16.4814693195)
            (5.96984924628, 15.1379496362)
            (6.02512562819, 13.834761987)
            (6.08040201009, 12.5808456494)
            (6.135678392, 11.3836486399)
            (6.19095477391, 10.2491214013)
            (6.24623115582, 9.18174650688)
            (6.30150753773, 8.18460012111)
            (6.35678391964, 7.25944032101)
            (6.41206030155, 6.40681699839)
            (6.46733668346, 5.62619791298)
            (6.52261306537, 4.9161055383)
            (6.57788944728, 4.27425960562)
            (6.63316582919, 3.69772067294)
            (6.68844221109, 3.18303059128)
            (6.743718593, 2.72634636918)
            (6.79899497491, 2.32356461342)
            (6.85427135682, 1.97043441357)
            (6.90954773873, 1.66265721071)
            (6.96482412064, 1.39597282068)
            (7.02010050255, 1.16623134971)
            (7.07537688446, 0.969451231665)
            (7.13065326637, 0.801864022263)
            (7.18592964828, 0.659946904046)
            (7.24120603018, 0.540444087253)
            (7.29648241209, 0.440378441363)
            (7.351758794, 0.357054767705)
            (7.40703517591, 0.288056135046)
            (7.46231155782, 0.231234658341)
            (7.51758793973, 0.184698017424)
            (7.57286432164, 0.146792898282)
            (7.62814070355, 0.116086405299)
            (7.68341708546, 0.0913463476282)
            (7.73869346737, 0.0715211547411)
            (7.79396984928, 0.0557200316472)
            (7.84924623118, 0.0431938282802)
            (7.90452261309, 0.0333169737478)
            (7.959798995, 0.0255707167867)
            (8.01507537691, 0.0195278200299)
            (8.07035175882, 0.0148387778124)
            (8.12562814073, 0.0112195646734)
            (8.18090452264, 0.00844087340709)
            (8.23618090455, 0.00631876601735)
            (8.29145728646, 0.00470663659254)
            (8.34673366837, 0.00348837018886)
            (8.40201005027, 0.00257257456888)
            (8.45728643218, 0.00188776045453)
            (8.51256281409, 0.00137834933911)
            (8.567839196, 0.00100139455787)
            (8.62311557791, 0.000723910136206)
            (8.67839195982, 0.000520712010165)
            (8.73366834173, 0.00037268682327)
            (8.78894472364, 0.000265414098093)
            (8.84422110555, 0.000188077759953)
            (8.89949748746, 0.000132612482704)
            (8.95477386937, 9.30389657973e-05)
            (9.01005025127, 6.49499541653e-05)
            (9.06532663318, 4.51155576598e-05)
            (9.12060301509, 3.11822405265e-05)
            (9.175879397, 2.14447906414e-05)
            (9.23115577891, 1.46747195626e-05)
            (9.28643216082, 9.99197449045e-06)
            (9.34170854273, 6.76965176627e-06)
            (9.39698492464, 4.56367645889e-06)
            (9.45226130655, 3.06123642668e-06)
            (9.50753768846, 2.04320716877e-06)
            (9.56281407036, 1.35694248804e-06)
            (9.61809045227, 8.96693424075e-07)
            (9.67336683418, 5.89603475489e-07)
            (9.72864321609, 3.85753247707e-07)
            (9.783919598, 2.51126563512e-07)
            (9.83919597991, 1.62670657668e-07)
            (9.89447236182, 1.04847794063e-07)
            (9.94974874373, 6.72423493981e-08)
            (10.0050251256, 4.29101434705e-08)
            (10.0603015075, 2.72464898635e-08)
            (10.1155778895, 1.72145098178e-08)
            (10.1708542714, 1.08221180542e-08)
            (10.2261306533, 6.7696057973e-09)
            (10.2814070352, 4.21354851637e-09)
            (10.3366834171, 2.60955301442e-09)
            (10.391959799, 1.60811741734e-09)
            (10.4472361809, 9.86058982352e-10)
            (10.5025125628, 6.01618992022e-10)
            (10.5577889447, 3.65236092719e-10)
            (10.6130653266, 2.20627348019e-10)
            (10.6683417085, 1.32610681713e-10)
            (10.7236180905, 7.93105893046e-11)
            (10.7788944724, 4.71973214647e-11)
            (10.8341708543, 2.79471183493e-11)
            (10.8894472362, 1.64660800378e-11)
            (10.9447236181, 9.65332474399e-12)
            (11.0, 5.63115027116e-12)
    };
    

    
    % Draw series plot
    \addplot[no markers,lightgray] coordinates {
            (1e-10, 3.98675366542e-10)
            (0.055276382009, 6.24720707575e-10)
            (0.110552763918, 9.7487063726e-10)
            (0.165829145827, 1.51496523347e-09)
            (0.221105527736, 2.34451454649e-09)
            (0.276381909645, 3.61324812033e-09)
            (0.331658291554, 5.54545554017e-09)
            (0.386934673463, 8.47561573652e-09)
            (0.442211055372, 1.29003020213e-08)
            (0.497487437281, 1.95534360851e-08)
            (0.55276381919, 2.95148705441e-08)
            (0.608040201099, 4.43663057514e-08)
            (0.663316583009, 6.641409232e-08)
            (0.718592964918, 9.90060659086e-08)
            (0.773869346827, 1.46979902384e-07)
            (0.829145728736, 2.17294480026e-07)
            (0.884422110645, 3.19914571586e-07)
            (0.939698492554, 4.69044381888e-07)
            (0.994974874463, 6.84838935459e-07)
            (1.05025125637, 9.9576655887e-07)
            (1.10552763828, 1.44185377669e-06)
            (1.16080402019, 2.0791196839e-06)
            (1.2160804021, 2.98560500011e-06)
            (1.27135678401, 4.26952732384e-06)
            (1.32663316592, 6.08025556229e-06)
            (1.38190954783, 8.62300144156e-06)
            (1.43718592974, 1.21783842375e-05)
            (1.49246231164, 1.71283478565e-05)
            (1.54773869355, 2.39903102849e-05)
            (1.60301507546, 3.34619190548e-05)
            (1.65829145737, 4.64793891882e-05)
            (1.71356783928, 6.42931299e-05)
            (1.76884422119, 8.85652419489e-05)
            (1.8241206031, 0.000121494508064)
            (1.87939698501, 0.000165975722927)
            (1.93467336692, 0.000225801633588)
            (1.98994974883, 0.000305917399433)
            (2.04522613073, 0.000412739341041)
            (2.10050251264, 0.000554551829962)
            (2.15577889455, 0.000741998466324)
            (2.21105527646, 0.00098868617423)
            (2.26633165837, 0.00131192347439)
            (2.32160804028, 0.00173361690615)
            (2.37688442219, 0.00228135227826)
            (2.4321608041, 0.00298969001044)
            (2.48743718601, 0.00390170613531)
            (2.54271356792, 0.00507081237273)
            (2.59798994983, 0.00656288984181)
            (2.65326633173, 0.00845877117341)
            (2.70854271364, 0.0108571047278)
            (2.76381909555, 0.013877631974)
            (2.81909547746, 0.0176649044915)
            (2.87437185937, 0.0223924601413)
            (2.92964824128, 0.0282674683486)
            (2.98492462319, 0.0355358418031)
            (3.0402010051, 0.0444877959048)
            (3.09547738701, 0.0554638177425)
            (3.15075376892, 0.0688609831603)
            (3.20603015082, 0.0851395335622)
            (3.26130653273, 0.104829593722)
            (3.31658291464, 0.12853787839)
            (3.37185929655, 0.156954199584)
            (3.42713567846, 0.190857548986)
            (3.48241206037, 0.231121492093)
            (3.53768844228, 0.278718574081)
            (3.59296482419, 0.334723403594)
            (3.6482412061, 0.400314051807)
            (3.70351758801, 0.476771382398)
            (3.75879396992, 0.565475915885)
            (3.81407035182, 0.667901831494)
            (3.86934673373, 0.785607723821)
            (3.92462311564, 0.920223762077)
            (3.97989949755, 1.07343494864)
            (4.03517587946, 1.24696024228)
            (4.09045226137, 1.44252740066)
            (4.14572864328, 1.66184350608)
            (4.20100502519, 1.90656126777)
            (4.2562814071, 2.17824134041)
            (4.31155778901, 2.47831105946)
            (4.36683417091, 2.80802016504)
            (4.42211055282, 3.16839426204)
            (4.47738693473, 3.56018693863)
            (4.53266331664, 3.98383163125)
            (4.58793969855, 4.43939447391)
            (4.64321608046, 4.92652949477)
            (4.69849246237, 5.44443761611)
            (4.75376884428, 5.99183096671)
            (4.80904522619, 6.56690402213)
            (4.8643216081, 7.16731304264)
            (4.91959799001, 7.79016517714)
            (4.97487437191, 8.43201844197)
            (5.03015075382, 9.0888935673)
            (5.08542713573, 9.75629843345)
            (5.14070351764, 10.4292655009)
            (5.19597989955, 11.1024022791)
            (5.25125628146, 11.76995449)
            (5.30653266337, 12.4258811802)
            (5.36180904528, 13.0639406245)
            (5.41708542719, 13.6777854736)
            (5.4723618091, 14.2610652314)
            (5.527638191, 14.8075338288)
            (5.58291457291, 15.3111597993)
            (5.63819095482, 15.7662363732)
            (5.69346733673, 16.1674887033)
            (5.74874371864, 16.5101754148)
            (5.80402010055, 16.7901817545)
            (5.85929648246, 17.0041017828)
            (5.91457286437, 17.1493073136)
            (5.96984924628, 17.2240016539)
            (6.02512562819, 17.2272566036)
            (6.08040201009, 17.1590316535)
            (6.135678392, 17.0201748243)
            (6.19095477391, 16.8124051228)
            (6.24623115582, 16.5382771235)
            (6.30150753773, 16.2011286944)
            (6.35678391964, 15.8050133639)
            (6.41206030155, 15.3546192431)
            (6.46733668346, 14.8551767714)
            (6.52261306537, 14.3123578179)
            (6.57788944728, 13.7321688579)
            (6.63316582919, 13.1208410224)
            (6.68844221109, 12.4847198168)
            (6.743718593, 11.8301572046)
            (6.79899497491, 11.1634085699)
            (6.85427135682, 10.4905368189)
            (6.90954773873, 9.81732556652)
            (6.96482412064, 9.14920298927)
            (7.02010050255, 8.49117753851)
            (7.07537688446, 7.8477862973)
            (7.13065326637, 7.22305636134)
            (7.18592964828, 6.6204792329)
            (7.24120603018, 6.04299785528)
            (7.29648241209, 5.49300559247)
            (7.351758794, 4.97235618373)
            (7.40703517591, 4.4823834815)
            (7.46231155782, 4.02392961633)
            (7.51758793973, 3.59738012622)
            (7.57286432164, 3.20270453642)
            (7.62814070355, 2.83950087842)
            (7.68341708546, 2.50704268519)
            (7.73869346737, 2.20432708998)
            (7.79396984928, 1.93012277816)
            (7.84924623118, 1.68301668982)
            (7.90452261309, 1.46145853517)
            (7.959798995, 1.26380235928)
            (8.01507537691, 1.08834456832)
            (8.07035175882, 0.933358001743)
            (8.12562814073, 0.797121796337)
            (8.18090452264, 0.677946936521)
            (8.23618090455, 0.574197515765)
            (8.29145728646, 0.484307845464)
            (8.34673366837, 0.406795638663)
            (8.40201005027, 0.340271566476)
            (8.45728643218, 0.28344553575)
            (8.51256281409, 0.235130068667)
            (8.567839196, 0.194241180429)
            (8.62311557791, 0.159797152153)
            (8.67839195982, 0.130915584874)
            (8.73366834173, 0.106809099598)
            (8.78894472364, 0.0867800200625)
            (8.84422110555, 0.0702143414644)
            (8.89949748746, 0.0565752519018)
            (8.95477386937, 0.0453964355604)
            (9.01005025127, 0.0362753491081)
            (9.06532663318, 0.0288666266825)
            (9.12060301509, 0.0228757351286)
            (9.175879397, 0.0180529704588)
            (9.23115577891, 0.0141878592756)
            (9.28643216082, 0.0111040053276)
            (9.34170854273, 0.00865440149902)
            (9.39698492464, 0.00671721123536)
            (9.45226130655, 0.00519201048326)
            (9.50753768846, 0.00399647135248)
            (9.56281407036, 0.00306346156755)
            (9.61809045227, 0.0023385289838)
            (9.67336683418, 0.00177773763313)
            (9.72864321609, 0.00134582057615)
            (9.783919598, 0.00101461493373)
            (9.83919597991, 0.000761745548794)
            (9.89447236182, 0.000569525519126)
            (9.94974874373, 0.000424044115264)
            (10.0050251256, 0.000314415162274)
            (10.0603015075, 0.000232161665303)
            (10.1155778895, 0.000170715175164)
            (10.1708542714, 0.000125011030233)
            (10.2261306533, 9.11631092684e-05)
            (10.2814070352, 6.62040430596e-05)
            (10.3366834171, 4.78789353142e-05)
            (10.391959799, 3.44825238231e-05)
            (10.4472361809, 2.47313710887e-05)
            (10.5025125628, 1.7664116958e-05)
            (10.5577889447, 1.25640674087e-05)
            (10.6130653266, 8.89945010586e-06)
            (10.6683417085, 6.27755716029e-06)
            (10.7236180905, 4.40973780368e-06)
            (10.7788944724, 3.08481730269e-06)
            (10.8341708543, 2.14902128594e-06)
            (10.8894472362, 1.49089334011e-06)
            (10.9447236181, 1.03002323342e-06)
            (11.0, 7.08666748363e-07)
    };
    

    
    % Draw series plot
    \addplot[no markers,lightgray] coordinates {
            (1e-10, 3.92351303779e-12)
            (0.055276382009, 6.1499354523e-12)
            (0.110552763918, 9.60546777961e-12)
            (0.165829145827, 1.49492355853e-11)
            (0.221105527736, 2.31831239353e-11)
            (0.276381909645, 3.58242768385e-11)
            (0.331658291554, 5.5161416786e-11)
            (0.386934673463, 8.46342006198e-11)
            (0.442211055372, 1.29392434893e-10)
            (0.497487437281, 1.97117135231e-10)
            (0.55276381919, 2.99221211226e-10)
            (0.608040201099, 4.52598258858e-10)
            (0.663316583009, 6.82159427777e-10)
            (0.718592964918, 1.02449866526e-09)
            (0.773869346827, 1.53316682269e-09)
            (0.829145728736, 2.28623007164e-09)
            (0.884422110645, 3.39705796629e-09)
            (0.939698492554, 5.02965836244e-09)
            (0.994974874463, 7.4203863942e-09)
            (1.05025125637, 1.09085508037e-08)
            (1.10552763828, 1.59793864954e-08)
            (1.16080402019, 2.3324140471e-08)
            (1.2160804021, 3.39237379614e-08)
            (1.27135678401, 4.91647976417e-08)
            (1.32663316592, 7.09998312547e-08)
            (1.38190954783, 1.02167527072e-07)
            (1.43718592974, 1.4649437562e-07)
            (1.49246231164, 2.09305926263e-07)
            (1.54773869355, 2.97985137561e-07)
            (1.60301507546, 4.22727193668e-07)
            (1.65829145737, 5.97555535605e-07)
            (1.71356783928, 8.41683601419e-07)
            (1.76884422119, 1.18133198354e-06)
            (1.8241206031, 1.65214272803e-06)
            (1.87939698501, 2.30237291983e-06)
            (1.93467336692, 3.19710042597e-06)
            (1.98994974883, 4.42373795033e-06)
            (2.04522613073, 6.09923001015e-06)
            (2.10050251264, 8.37940409687e-06)
            (2.15577889455, 1.1471065578e-05)
            (2.21105527646, 1.56475697024e-05)
            (2.26633165837, 2.12687776745e-05)
            (2.32160804028, 2.88065118197e-05)
            (2.37688442219, 3.88768723354e-05)
            (2.4321608041, 5.22810701467e-05)
            (2.48743718601, 7.00567721263e-05)
            (2.54271356792, 9.35423513404e-05)
            (2.59798994983, 0.000124456890472)
            (2.65326633173, 0.000164999304705)
            (2.70854271364, 0.000217970533273)
            (2.76381909555, 0.000286923396867)
            (2.81909547746, 0.000376345428787)
            (2.87437185937, 0.000491880755423)
            (2.92964824128, 0.000640597916553)
            (2.98492462319, 0.000831311363056)
            (3.0402010051, 0.00107496522799)
            (3.09547738701, 0.00138508880872)
            (3.15075376892, 0.00177833398707)
            (3.20603015082, 0.00227510550664)
            (3.26130653273, 0.0029002955684)
            (3.31658291464, 0.00368413453292)
            (3.37185929655, 0.00466316955864)
            (3.42713567846, 0.00588138267723)
            (3.48241206037, 0.00739145901949)
            (3.53768844228, 0.00925621456284)
            (3.59296482419, 0.0115501907728)
            (3.6482412061, 0.0143614207569)
            (3.70351758801, 0.0177933679429)
            (3.75879396992, 0.0219670337459)
            (3.81407035182, 0.0270232251332)
            (3.86934673373, 0.0331249663744)
            (3.92462311564, 0.0404600315654)
            (3.97989949755, 0.0492435657541)
            (4.03517587946, 0.0597207527341)
            (4.09045226137, 0.0721694769329)
            (4.14572864328, 0.086902915486)
            (4.20100502519, 0.104271984793)
            (4.2562814071, 0.124667553926)
            (4.31155778901, 0.148522325602)
            (4.36683417091, 0.17631227445)
            (4.42211055282, 0.208557522648)
            (4.47738693473, 0.245822525071)
            (4.53266331664, 0.288715430724)
            (4.58793969855, 0.3378864849)
            (4.64321608046, 0.394025337979)
            (4.69849246237, 0.457857132624)
            (4.75376884428, 0.530137251948)
            (4.80904522619, 0.611644627515)
            (4.8643216081, 0.703173528025)
            (4.91959799001, 0.805523777666)
            (4.97487437191, 0.919489387041)
            (5.03015075382, 1.04584561939)
            (5.08542713573, 1.18533455979)
            (5.14070351764, 1.33864930451)
            (5.19597989955, 1.50641694054)
            (5.25125628146, 1.6891805404)
            (5.30653266337, 1.88738045283)
            (5.36180904528, 2.10133522404)
            (5.41708542719, 2.33122253547)
            (5.4723618091, 2.57706058903)
            (5.527638191, 2.83869040914)
            (5.58291457291, 3.11575955874)
            (5.63819095482, 3.40770778278)
            (5.69346733673, 3.71375509563)
            (5.74874371864, 4.03289281629)
            (5.80402010055, 4.36387802686)
            (5.85929648246, 4.70523188433)
            (5.91457286437, 5.05524215369)
            (5.96984924628, 5.41197025215)
            (6.02512562819, 5.77326300066)
            (6.08040201009, 6.13676917305)
            (6.135678392, 6.49996081579)
            (6.19095477391, 6.86015918731)
            (6.24623115582, 7.21456503771)
            (6.30150753773, 7.56029282133)
            (6.35678391964, 7.89440831125)
            (6.41206030155, 8.21396896955)
            (6.46733668346, 8.51606632501)
            (6.52261306537, 8.79786952444)
            (6.57788944728, 9.05666915896)
            (6.63316582919, 9.289920425)
            (6.68844221109, 9.49528466359)
            (6.743718593, 9.67066833284)
            (6.79899497491, 9.81425850697)
            (6.85427135682, 9.92455406087)
            (6.90954773873, 10.0003917905)
            (6.96482412064, 10.0409668333)
            (7.02010050255, 10.0458468882)
            (7.07537688446, 10.0149798827)
            (7.13065326637, 9.94869489815)
            (7.18592964828, 9.84769632888)
            (7.24120603018, 9.71305142167)
            (7.29648241209, 9.54617150252)
            (7.351758794, 9.34878735301)
            (7.40703517591, 9.12291933704)
            (7.46231155782, 8.87084299838)
            (7.51758793973, 8.59505094792)
            (7.57286432164, 8.29821193177)
            (7.62814070355, 7.98312801766)
            (7.68341708546, 7.65269085575)
            (7.73869346737, 7.30983796092)
            (7.79396984928, 6.95750992921)
            (7.84924623118, 6.59860944195)
            (7.90452261309, 6.23596283101)
            (7.959798995, 5.87228488071)
            (8.01507537691, 5.51014742985)
            (8.07035175882, 5.15195221548)
            (8.12562814073, 4.79990827295)
            (8.18090452264, 4.45601407789)
            (8.23618090455, 4.12204449044)
            (8.29145728646, 3.79954244243)
            (8.34673366837, 3.48981519908)
            (8.40201005027, 3.19393492935)
            (8.45728643218, 2.91274323698)
            (8.51256281409, 2.64685923714)
            (8.567839196, 2.39669071426)
            (8.62311557791, 2.16244786287)
            (8.67839195982, 1.9441590973)
            (8.73366834173, 1.74168841421)
            (8.78894472364, 1.55475380524)
            (8.84422110555, 1.3829462417)
            (8.89949748746, 1.22574878911)
            (8.95477386937, 1.0825554527)
            (9.01005025127, 0.952689404765)
            (9.06532663318, 0.835420298386)
            (9.12060301509, 0.729980427145)
            (9.175879397, 0.635579546059)
            (9.23115577891, 0.551418222353)
            (9.28643216082, 0.476699635327)
            (9.34170854273, 0.410639790816)
            (9.39698492464, 0.352476157039)
            (9.45226130655, 0.301474764177)
            (9.50753768846, 0.256935839776)
            (9.56281407036, 0.218198075665)
            (9.61809045227, 0.184641639974)
            (9.67336683418, 0.155690060024)
            (9.72864321609, 0.130811109061)
            (9.783919598, 0.109516832379)
            (9.83919597991, 0.0913628470502)
            (9.89447236182, 0.0759470448325)
            (9.94974874373, 0.0629078205652)
            (10.0050251256, 0.0519219391009)
            (10.0603015075, 0.0427021431447)
            (10.1155778895, 0.0349945928573)
            (10.1708542714, 0.0285762161783)
            (10.2261306533, 0.0232520369631)
            (10.2814070352, 0.0188525365385)
            (10.3366834171, 0.0152310934299)
            (10.391959799, 0.0122615360018)
            (10.4472361809, 0.00983583370714)
            (10.5025125628, 0.00786194465102)
            (10.5577889447, 0.00626183025919)
            (10.6130653266, 0.0049696419988)
            (10.6683417085, 0.0039300802833)
            (10.7236180905, 0.00309692183829)
            (10.7788944724, 0.0024317088277)
            (10.8341708543, 0.00190259084522)
            (10.8894472362, 0.00148330936436)
            (10.9447236181, 0.00115231331125)
            (11.0, 0.000891993980364)
    };
    

    
    % Draw series plot
    \addplot[no markers,lightgray] coordinates {
            (1e-10, 4.15023556101e-14)
            (0.055276382009, 6.50676202344e-14)
            (0.110552763918, 1.01695805513e-13)
            (0.165829145827, 1.58448123214e-13)
            (0.221105527736, 2.46103108717e-13)
            (0.276381909645, 3.8105971761e-13)
            (0.331658291554, 5.88186326565e-13)
            (0.386934673463, 9.05071206801e-13)
            (0.442211055372, 1.38834216824e-12)
            (0.497487437281, 2.12303044452e-12)
            (0.55276381919, 3.2363976026e-12)
            (0.608040201099, 4.91828239247e-12)
            (0.663316583009, 7.45093873666e-12)
            (0.718592964918, 1.12526410525e-11)
            (0.773869346827, 1.69411887035e-11)
            (0.829145728736, 2.5426069855e-11)
            (0.884422110645, 3.80417540222e-11)
            (0.939698492554, 5.67397962313e-11)
            (0.994974874463, 8.43647360141e-11)
            (1.05025125637, 1.2504895691e-10)
            (1.10552763828, 1.84775819199e-10)
            (1.16080402019, 2.72179956771e-10)
            (1.2160804021, 3.99680616476e-10)
            (1.27135678401, 5.85080971406e-10)
            (1.32663316592, 8.53817009441e-10)
            (1.38190954783, 1.24210861073e-09)
            (1.43718592974, 1.80135904246e-09)
            (1.49246231164, 2.60427556938e-09)
            (1.54773869355, 3.75335403519e-09)
            (1.60301507546, 5.39259825602e-09)
            (1.65829145737, 7.72364926635e-09)
            (1.71356783928, 1.10279036405e-08)
            (1.76884422119, 1.5696734867e-08)
            (1.8241206031, 2.22726362373e-08)
            (1.87939698501, 3.15050277845e-08)
            (1.93467336692, 4.44256766436e-08)
            (1.98994974883, 6.24502493737e-08)
            (2.04522613073, 8.75145459789e-08)
            (2.10050251264, 1.2225658278e-07)
            (2.15577889455, 1.70259048708e-07)
            (2.21105527646, 2.36370946308e-07)
            (2.26633165837, 3.27132676322e-07)
            (2.32160804028, 4.51335714513e-07)
            (2.37688442219, 6.20756701192e-07)
            (2.4321608041, 8.51116624732e-07)
            (2.48743718601, 1.16332931492e-06)
            (2.54271356792, 1.58512024213e-06)
            (2.59798994983, 2.15311731413e-06)
            (2.65326633173, 2.91554075133e-06)
            (2.70854271364, 3.93565009826e-06)
            (2.76381909555, 5.29614401126e-06)
            (2.81909547746, 7.10475379725e-06)
            (2.87437185937, 9.50132603971e-06)
            (2.92964824128, 1.26667544333e-05)
            (2.98492462319, 1.68341976673e-05)
            (3.0402010051, 2.23031104467e-05)
            (3.09547738701, 2.94567201742e-05)
            (3.15075376892, 3.87837040982e-05)
            (3.20603015082, 5.09049624763e-05)
            (3.26130653273, 6.66065439895e-05)
            (3.31658291464, 8.68799615059e-05)
            (3.37185929655, 0.000112971340235)
            (3.42713567846, 0.000146441066695)
            (3.48241206037, 0.000189235855432)
            (3.53768844228, 0.000243775419922)
            (3.59296482419, 0.000313056222215)
            (3.6482412061, 0.000400775079043)
            (3.70351758801, 0.000511475715073)
            (3.75879396992, 0.000650721669584)
            (3.81407035182, 0.000825299271845)
            (3.86934673373, 0.00104345469127)
            (3.92462311564, 0.00131516932663)
            (3.97989949755, 0.00165247800735)
            (4.03517587946, 0.00206983461914)
            (4.09045226137, 0.00258452981291)
            (4.14572864328, 0.00321716538511)
            (4.20100502519, 0.00399218970008)
            (4.2562814071, 0.0049384981314)
            (4.31155778901, 0.00609010189741)
            (4.36683417091, 0.0074868678232)
            (4.42211055282, 0.00917533044627)
            (4.47738693473, 0.0112095764647)
            (4.53266331664, 0.0136521997783)
            (4.58793969855, 0.0165753232739)
            (4.64321608046, 0.0200616810377)
            (4.69849246237, 0.0242057518416)
            (4.75376884428, 0.0291149315437)
            (4.80904522619, 0.0349107284939)
            (4.8643216081, 0.0417299621756)
            (4.91959799001, 0.0497259411977)
            (4.97487437191, 0.0590695924554)
            (5.03015075382, 0.0699505088924)
            (5.08542713573, 0.0825778789469)
            (5.14070351764, 0.0971812565797)
            (5.19597989955, 0.114011126928)
            (5.25125628146, 0.133339219283)
            (5.30653266337, 0.155458516456)
            (5.36180904528, 0.180682907828)
            (5.41708542719, 0.209346432816)
            (5.4723618091, 0.241802062139)
            (5.527638191, 0.278419966535)
            (5.58291457291, 0.319585226547)
            (5.63819095482, 0.365694942764)
            (5.69346733673, 0.417154713751)
            (5.74874371864, 0.474374458724)
            (5.80402010055, 0.537763573957)
            (5.85929648246, 0.607725425827)
            (5.91457286437, 0.684651199214)
            (5.96984924628, 0.768913137427)
            (6.02512562819, 0.860857228689)
            (6.08040201009, 0.960795414037)
            (6.135678392, 1.06899741186)
            (6.19095477391, 1.18568227465)
            (6.24623115582, 1.31100981336)
            (6.30150753773, 1.44507204305)
            (6.35678391964, 1.5878848204)
            (6.41206030155, 1.73937985706)
            (6.46733668346, 1.89939730388)
            (6.52261306537, 2.06767910703)
            (6.57788944728, 2.24386333944)
            (6.63316582919, 2.42747970753)
            (6.68844221109, 2.61794642451)
            (6.743718593, 2.81456862745)
            (6.79899497491, 3.01653849476)
            (6.85427135682, 3.22293719512)
            (6.90954773873, 3.43273876745)
            (6.96482412064, 3.64481599537)
            (7.02010050255, 3.85794829924)
            (7.07537688446, 4.07083162496)
            (7.13065326637, 4.28209026268)
            (7.18592964828, 4.49029048151)
            (7.24120603018, 4.69395581894)
            (7.29648241209, 4.89158381852)
            (7.351758794, 5.08166396634)
            (7.40703517591, 5.26269653884)
            (7.46231155782, 5.43321204144)
            (7.51758793973, 5.5917908915)
            (7.57286432164, 5.73708298094)
            (7.62814070355, 5.86782674388)
            (7.68341708546, 5.98286735436)
            (7.73869346737, 6.08117368789)
            (7.79396984928, 6.16185369876)
            (7.84924623118, 6.22416789283)
            (7.90452261309, 6.2675406111)
            (7.959798995, 6.29156888347)
            (8.01507537691, 6.29602866208)
            (8.07035175882, 6.28087829955)
            (8.12562814073, 6.24625919653)
            (8.18090452264, 6.19249360445)
            (8.23618090455, 6.12007963124)
            (8.29145728646, 6.02968355831)
            (8.34673366837, 5.92212963448)
            (8.40201005027, 5.7983875655)
            (8.45728643218, 5.65955796465)
            (8.51256281409, 5.50685606961)
            (8.567839196, 5.34159406214)
            (8.62311557791, 5.16516234966)
            (8.67839195982, 4.97901018092)
            (8.73366834173, 4.78462597176)
            (8.78894472364, 4.58351771117)
            (8.84422110555, 4.37719380349)
            (8.89949748746, 4.16714467987)
            (8.95477386937, 3.95482548241)
            (9.01005025127, 3.74164008812)
            (9.06532663318, 3.52892669936)
            (9.12060301509, 3.31794518271)
            (9.175879397, 3.10986629173)
            (9.23115577891, 2.9057628621)
            (9.28643216082, 2.70660302033)
            (9.34170854273, 2.51324540271)
            (9.39698492464, 2.32643633897)
            (9.45226130655, 2.14680891672)
            (9.50753768846, 1.97488380956)
            (9.56281407036, 1.81107172287)
            (9.61809045227, 1.65567728895)
            (9.67336683418, 1.50890422572)
            (9.72864321609, 1.37086156245)
            (9.783919598, 1.24157072982)
            (9.83919597991, 1.1209733117)
            (9.89447236182, 1.0089392604)
            (9.94974874373, 0.905275385829)
            (10.0050251256, 0.809733941535)
            (10.0603015075, 0.722021146101)
            (10.1155778895, 0.641805495848)
            (10.1708542714, 0.568725744228)
            (10.2261306533, 0.502398443375)
            (10.2814070352, 0.44242496374)
            (10.3366834171, 0.388397927857)
            (10.391959799, 0.339907013656)
            (10.4472361809, 0.296544100851)
            (10.5025125628, 0.257907750537)
            (10.5577889447, 0.223607022934)
            (10.6130653266, 0.193264651036)
            (10.6683417085, 0.166519598737)
            (10.7236180905, 0.143029040729)
            (10.7788944724, 0.122469808159)
            (10.8341708543, 0.104539348839)
            (10.8894472362, 0.0889562537668)
            (10.9447236181, 0.07546040313)
            (11.0, 0.0638127848877)
    };
    

    
    % Draw series plot
    \addplot[no markers,lightgray] coordinates {
            (1e-10, 4.63882917976e-16)
            (0.055276382009, 7.27404162251e-16)
            (0.110552763918, 1.13746901608e-15)
            (0.165829145827, 1.7737800327e-15)
            (0.221105527736, 2.75839425944e-15)
            (0.276381909645, 4.27769002731e-15)
            (0.331658291554, 6.61543908117e-15)
            (0.386934673463, 1.02024489467e-14)
            (0.442211055372, 1.56908538872e-14)
            (0.497487437281, 2.40649585906e-14)
            (0.55276381919, 3.68061211678e-14)
            (0.608040201099, 5.61372813693e-14)
            (0.663316583009, 8.53845107417e-14)
            (0.718592964918, 1.29509970411e-13)
            (0.773869346827, 1.95895160215e-13)
            (0.829145728736, 2.9548851205e-13)
            (0.884422110645, 4.4448170314e-13)
            (0.939698492554, 6.66750833984e-13)
            (0.994974874463, 9.97400632707e-13)
            (1.05025125637, 1.48789438058e-12)
            (1.10552763828, 2.21345634033e-12)
            (1.16080402019, 3.2837206873e-12)
            (1.2160804021, 4.85800380131e-12)
            (1.27135678401, 7.16713993528e-12)
            (1.32663316592, 1.05446048605e-11)
            (1.38190954783, 1.54707414586e-11)
            (1.43718592974, 2.26354086362e-11)
            (1.49246231164, 3.30264536736e-11)
            (1.54773869355, 4.80542648771e-11)
            (1.60301507546, 6.97265738472e-11)
            (1.65829145737, 1.00893016344e-10)
            (1.71356783928, 1.45586222882e-10)
            (1.76884422119, 2.0949605238e-10)
            (1.8241206031, 3.00626874164e-10)
            (1.87939698501, 4.30205690804e-10)
            (1.93467336692, 6.13932879872e-10)
            (1.98994974883, 8.73699380287e-10)
            (2.04522613073, 1.23993681423e-09)
            (2.10050251264, 1.75482362993e-09)
            (2.15577889455, 2.47664514728e-09)
            (2.21105527646, 3.48570387748e-09)
            (2.26633165837, 4.89230569382e-09)
            (2.32160804028, 6.84751629965e-09)
            (2.37688442219, 9.55760231463e-09)
            (2.4321608041, 1.3303356501e-08)
            (2.48743718601, 1.84658751656e-08)
            (2.54271356792, 2.55608300956e-08)
            (2.59798994983, 3.52838855105e-08)
            (2.65326633173, 4.85706871012e-08)
            (2.70854271364, 6.66758379841e-08)
            (2.76381909555, 9.12765276793e-08)
            (2.81909547746, 1.24608058829e-07)
            (2.87437185937, 1.69640499697e-07)
            (2.92964824128, 2.30308171835e-07)
            (2.98492462319, 3.11806773438e-07)
            (3.0402010051, 4.2097677315e-07)
            (3.09547738701, 5.6679644399e-07)
            (3.15075376892, 7.61013728252e-07)
            (3.20603015082, 1.01895324807e-06)
            (3.26130653273, 1.36054345408e-06)
            (3.31658291464, 1.81161942477e-06)
            (3.37185929655, 2.40556952117e-06)
            (3.42713567846, 3.1854093378e-06)
            (3.48241206037, 4.20638458738e-06)
            (3.53768844228, 5.53922617583e-06)
            (3.59296482419, 7.27420626928e-06)
            (3.6482412061, 9.52617417065e-06)
            (3.70351758801, 1.24407858887e-05)
            (3.75879396992, 1.62021819978e-05)
            (3.81407035182, 2.10424153684e-05)
            (3.86934673373, 2.72529841976e-05)
            (3.92462311564, 3.51988870697e-05)
            (3.97989949755, 4.53356860373e-05)
            (4.03517587946, 5.82301413602e-05)
            (4.09045226137, 7.45850678532e-05)
            (4.14572864328, 9.52691578681e-05)
            (4.20100502519, 0.000121352619612)
            (4.2562814071, 0.000154149591319)
            (4.31155778901, 0.000195268410873)
            (4.36683417091, 0.000246670945528)
            (4.42211055282, 0.00031074231548)
            (4.47738693473, 0.000390372475681)
            (4.53266331664, 0.000489051249289)
            (4.58793969855, 0.000610978529334)
            (4.64321608046, 0.000761191477684)
            (4.69849246237, 0.000945710646355)
            (4.75376884428, 0.00117170701861)
            (4.80904522619, 0.00144769200841)
            (4.8643216081, 0.00178373245756)
            (4.91959799001, 0.00219169262056)
            (4.97487437191, 0.00268550501732)
            (5.03015075382, 0.00328147185144)
            (5.08542713573, 0.0039985984261)
            (5.14070351764, 0.00485895962743)
            (5.19597989955, 0.00588810007617)
            (5.25125628146, 0.00711546795999)
            (5.30653266337, 0.00857488184269)
            (5.36180904528, 0.0103050288927)
            (5.41708542719, 0.0123499919782)
            (5.4723618091, 0.0147598019343)
            (5.527638191, 0.0175910100246)
            (5.58291457291, 0.0209072741944)
            (5.63819095482, 0.0247799511668)
            (5.69346733673, 0.0292886847692)
            (5.74874371864, 0.0345219791383)
            (5.80402010055, 0.0405777436471)
            (5.85929648246, 0.0475637945866)
            (5.91457286437, 0.0555982968471)
            (5.96984924628, 0.0648101271426)
            (6.02512562819, 0.0753391387665)
            (6.08040201009, 0.0873363065191)
            (6.135678392, 0.100963729387)
            (6.19095477391, 0.116394467853)
            (6.24623115582, 0.133812192442)
            (6.30150753773, 0.153410620372)
            (6.35678391964, 0.175392718004)
            (6.41206030155, 0.199969648278)
            (6.46733668346, 0.227359444549)
            (6.52261306537, 0.257785395193)
            (6.57788944728, 0.291474127125)
            (6.63316582919, 0.328653380962)
            (6.68844221109, 0.369549475914)
            (6.743718593, 0.414384468623)
            (6.79899497491, 0.463373017037)
            (6.85427135682, 0.516718967796)
            (6.90954773873, 0.574611693593)
            (6.96482412064, 0.637222215239)
            (7.02010050255, 0.704699151575)
            (7.07537688446, 0.77716454884)
            (7.13065326637, 0.854709649192)
            (7.18592964828, 0.937390665761)
            (7.24120603018, 1.02522463849)
            (7.29648241209, 1.11818545085)
            (7.351758794, 1.21620009225)
            (7.40703517591, 1.31914525377)
            (7.46231155782, 1.42684434653)
            (7.51758793973, 1.53906503119)
            (7.57286432164, 1.65551734441)
            (7.62814070355, 1.77585250327)
            (7.68341708546, 1.8996624615)
            (7.73869346737, 2.026480282)
            (7.79396984928, 2.15578137836)
            (7.84924623118, 2.28698566472)
            (7.90452261309, 2.4194606377)
            (7.959798995, 2.55252539697)
            (8.01507537691, 2.68545559298)
            (8.07035175882, 2.81748927098)
            (8.12562814073, 2.94783356115)
            (8.18090452264, 3.07567214498)
            (8.23618090455, 3.20017340925)
            (8.29145728646, 3.32049918108)
            (8.34673366837, 3.43581392131)
            (8.40201005027, 3.54529423923)
            (8.45728643218, 3.64813858006)
            (8.51256281409, 3.74357692783)
            (8.567839196, 3.83088036061)
            (8.62311557791, 3.90937029333)
            (8.67839195982, 3.97842724465)
            (8.73366834173, 4.0374989697)
            (8.78894472364, 4.08610780982)
            (8.84422110555, 4.12385712213)
            (8.89949748746, 4.15043666835)
            (8.95477386937, 4.1656268601)
            (9.01005025127, 4.16930177968)
            (9.06532663318, 4.16143091814)
            (9.12060301509, 4.14207959698)
            (9.175879397, 4.11140806548)
            (9.23115577891, 4.06966929105)
            (9.28643216082, 4.01720548512)
            (9.34170854273, 3.95444343149)
            (9.39698492464, 3.88188870618)
            (9.45226130655, 3.8001188983)
            (9.50753768846, 3.70977595914)
            (9.56281407036, 3.61155782092)
            (9.61809045227, 3.50620943824)
            (9.67336683418, 3.39451341261)
            (9.72864321609, 3.27728036445)
            (9.783919598, 3.15533921719)
            (9.83919597991, 3.02952755475)
            (9.89447236182, 2.90068220711)
            (9.94974874373, 2.76963020835)
            (10.0050251256, 2.63718025959)
            (10.0603015075, 2.50411481356)
            (10.1155778895, 2.37118288127)
            (10.1708542714, 2.23909364248)
            (10.2261306533, 2.10851092278)
            (10.2814070352, 1.98004858053)
            (10.3366834171, 1.8542668272)
            (10.391959799, 1.73166948612)
            (10.4472361809, 1.6127021765)
            (10.5025125628, 1.4977513932)
            (10.5577889447, 1.38714443782)
            (10.6130653266, 1.28115014388)
            (10.6683417085, 1.17998032785)
            (10.7236180905, 1.08379188946)
            (10.7788944724, 0.992689478197)
            (10.8341708543, 0.906728638954)
            (10.8894472362, 0.825919347753)
            (10.9447236181, 0.750229848739)
            (11.0, 0.679590705546)
    };
    

    
    % Draw series plot
    \addplot[no markers,lightgray] coordinates {
            (1e-10, 5.41535117479e-18)
            (0.055276382009, 8.49286472812e-18)
            (0.110552763918, 1.32861313532e-17)
            (0.165829145827, 2.07328798005e-17)
            (0.221105527736, 3.2272861269e-17)
            (0.276381909645, 5.01108841497e-17)
            (0.331658291554, 7.7614598092e-17)
            (0.386934673463, 1.19914448266e-16)
            (0.442211055372, 1.84806127753e-16)
            (0.497487437281, 2.84104409449e-16)
            (0.55276381919, 4.35668674431e-16)
            (0.608040201099, 6.66425255992e-16)
            (0.663316583009, 1.01686503374e-15)
            (0.718592964918, 1.54771835556e-15)
            (0.773869346827, 2.34983466269e-15)
            (0.829145728736, 3.55876599057e-15)
            (0.884422110645, 5.37623582419e-15)
            (0.939698492554, 8.10165867772e-15)
            (0.994974874463, 1.2178290447e-14)
            (1.05025125637, 1.82606177661e-14)
            (1.10552763828, 2.73124938967e-14)
            (1.16080402019, 4.07496611556e-14)
            (1.2160804021, 6.06461728657e-14)
            (1.27135678401, 9.00325498849e-14)
            (1.32663316592, 1.33325264606e-13)
            (1.38190954783, 1.9694370843e-13)
            (1.43718592974, 2.9019406202e-13)
            (1.49246231164, 4.26532064713e-13)
            (1.54773869355, 6.25362163586e-13)
            (1.60301507546, 9.14593838225e-13)
            (1.65829145737, 1.33426377114e-12)
            (1.71356783928, 1.94165412534e-12)
            (1.76884422119, 2.81850493057e-12)
            (1.8241206031, 4.08114928401e-12)
            (1.87939698501, 5.89471641044e-12)
            (1.93467336692, 8.49298006856e-12)
            (1.98994974883, 1.22060191358e-11)
            (2.04522613073, 1.74986579191e-11)
            (2.10050251264, 2.50237382754e-11)
            (2.15577889455, 3.5695740038e-11)
            (2.21105527646, 5.07922374372e-11)
            (2.26633165837, 7.20933242629e-11)
            (2.32160804028, 1.02072678659e-10)
            (2.37688442219, 1.44158656274e-10)
            (2.4321608041, 2.03090071374e-10)
            (2.48743718601, 2.85399634146e-10)
            (2.54271356792, 4.00068985595e-10)
            (2.59798994983, 5.59413716912e-10)
            (2.65326633173, 7.80275715121e-10)
            (2.70854271364, 1.08562496743e-09)
            (2.76381909555, 1.50670527338e-09)
            (2.81909547746, 2.08590030176e-09)
            (2.87437185937, 2.88055080058e-09)
            (2.92964824128, 3.96802393968e-09)
            (2.98492462319, 5.45242602067e-09)
            (3.0402010051, 7.47346548118e-09)
            (3.09547738701, 1.02181209081e-08)
            (3.15075376892, 1.39359569071e-08)
            (3.20603015082, 1.89591693276e-08)
            (3.26130653273, 2.57287430119e-08)
            (3.31658291464, 3.48284851954e-08)
            (3.37185929655, 4.70291745364e-08)
            (3.42713567846, 6.33456620037e-08)
            (3.48241206037, 8.51105026621e-08)
            (3.53768844228, 1.14068619318e-07)
            (3.59296482419, 1.52498638895e-07)
            (3.6482412061, 2.03367946404e-07)
            (3.70351758801, 2.7053022394e-07)
            (3.75879396992, 3.58976347166e-07)
            (3.81407035182, 4.75152073839e-07)
            (3.86934673373, 6.27359064534e-07)
            (3.92462311564, 8.26259524539e-07)
            (3.97989949755, 1.08550926178e-06)
            (4.03517587946, 1.42254934809e-06)
            (4.09045226137, 1.85959299591e-06)
            (4.14572864328, 2.42485188255e-06)
            (4.20100502519, 3.15405514921e-06)
            (4.2562814071, 4.09232487026e-06)
            (4.31155778901, 5.29648414225e-06)
            (4.36683417091, 6.83788831104e-06)
            (4.42211055282, 8.80588647878e-06)
            (4.47738693473, 1.13120395588e-05)
            (4.53266331664, 1.44952430262e-05)
            (4.58793969855, 1.85279273927e-05)
            (4.64321608046, 2.36235375464e-05)
            (4.69849246237, 3.00455236551e-05)
            (4.75376884428, 3.81181114977e-05)
            (4.80904522619, 4.82391589908e-05)
            (4.8643216081, 6.08954483489e-05)
            (4.91959799001, 7.66808097196e-05)
            (4.97487437191, 9.63175220922e-05)
            (5.03015075382, 0.000120681490488)
            (5.08542713573, 0.000150831754408)
            (5.14070351764, 0.000188044940554)
            (5.19597989955, 0.000233855332045)
            (5.25125628146, 0.00029010128547)
            (5.30653266337, 0.000358978784761)
            (5.36180904528, 0.000443102975075)
            (5.41708542719, 0.000545578568624)
            (5.4723618091, 0.00067008005496)
            (5.527638191, 0.000820942677825)
            (5.58291457291, 0.00100326515586)
            (5.63819095482, 0.0012230251216)
            (5.69346733673, 0.00148720822812)
            (5.74874371864, 0.001803951821)
            (5.80402010055, 0.00218270399079)
            (5.85929648246, 0.00263439870173)
            (5.91457286437, 0.00317164753344)
            (5.96984924628, 0.00380894836628)
            (6.02512562819, 0.00456291108551)
            (6.08040201009, 0.00545250006927)
            (6.135678392, 0.00649929285688)
            (6.19095477391, 0.00772775396512)
            (6.24623115582, 0.00916552232894)
            (6.30150753773, 0.0108437102894)
            (6.35678391964, 0.0127972114367)
            (6.41206030155, 0.0150650139448)
            (6.46733668346, 0.0176905153086)
            (6.52261306537, 0.0207218336287)
            (6.57788944728, 0.0242121097858)
            (6.63316582919, 0.0282197940278)
            (6.68844221109, 0.0328089096686)
            (6.743718593, 0.0380492857891)
            (6.79899497491, 0.0440167500659)
            (6.85427135682, 0.0507932721437)
            (6.90954773873, 0.0584670473594)
            (6.96482412064, 0.0671325101265)
            (7.02010050255, 0.0768902659553)
            (7.07537688446, 0.0878469309174)
            (7.13065326637, 0.100114867428)
            (7.18592964828, 0.113811805517)
            (7.24120603018, 0.129060339341)
            (7.29648241209, 0.145987289559)
            (7.351758794, 0.164722923407)
            (7.40703517591, 0.185400025823)
            (7.46231155782, 0.208152816874)
            (7.51758793973, 0.233115712954)
            (7.57286432164, 0.260421931799)
            (7.62814070355, 0.290201944241)
            (7.68341708546, 0.322581778813)
            (7.73869346737, 0.357681188759)
            (7.79396984928, 0.395611694607)
            (7.84924623118, 0.436474519275)
            (7.90452261309, 0.480358436479)
            (7.959798995, 0.527337557057)
            (8.01507537691, 0.577469081516)
            (8.07035175882, 0.630791050613)
            (8.12562814073, 0.687320128966)
            (8.18090452264, 0.747049459453)
            (8.23618090455, 0.809946628415)
            (8.29145728646, 0.875951783276)
            (8.34673366837, 0.944975945119)
            (8.40201005027, 1.01689955883)
            (8.45728643218, 1.09157132267)
            (8.51256281409, 1.16880733736)
            (8.567839196, 1.24839061227)
            (8.62311557791, 1.33007096234)
            (8.67839195982, 1.41356532513)
            (8.73366834173, 1.49855852159)
            (8.78894472364, 1.58470447778)
            (8.84422110555, 1.67162791773)
            (8.89949748746, 1.75892652965)
            (8.95477386937, 1.84617359945)
            (9.01005025127, 1.9329210968)
            (9.06532663318, 2.01870318998)
            (9.12060301509, 2.10304015724)
            (9.175879397, 2.18544265327)
            (9.23115577891, 2.26541628162)
            (9.28643216082, 2.34246641614)
            (9.34170854273, 2.41610320778)
            (9.39698492464, 2.4858467075)
            (9.45226130655, 2.55123203142)
            (9.50753768846, 2.61181449127)
            (9.56281407036, 2.66717461155)
            (9.61809045227, 2.71692295433)
            (9.67336683418, 2.76070467459)
            (9.72864321609, 2.79820373139)
            (9.783919598, 2.82914668521)
            (9.83919597991, 2.85330601757)
            (9.89447236182, 2.87050291668)
            (9.94974874373, 2.88060948124)
            (10.0050251256, 2.88355030457)
            (10.0603015075, 2.87930341154)
            (10.1155778895, 2.86790053203)
            (10.1708542714, 2.84942670611)
            (10.2261306533, 2.82401922796)
            (10.2814070352, 2.79186594666)
            (10.3366834171, 2.75320295351)
            (10.391959799, 2.70831169543)
            (10.4472361809, 2.65751556413)
            (10.5025125628, 2.60117601853)
            (10.5577889447, 2.53968830564)
            (10.6130653266, 2.47347685069)
            (10.6683417085, 2.40299039139)
            (10.7236180905, 2.32869693443)
            (10.7788944724, 2.25107861279)
            (10.8341708543, 2.1706265227)
            (10.8894472362, 2.08783561641)
            (10.9447236181, 2.00319972411)
            (11.0, 1.91720677323)
    };
    

    
    % Draw series plot
    \addplot[no markers,lightgray] coordinates {
            (1e-10, 6.54819627712e-20)
            (0.055276382009, 1.02706636967e-19)
            (0.110552763918, 1.60727636819e-19)
            (0.165829145827, 2.50956148141e-19)
            (0.221105527736, 3.90949219607e-19)
            (0.276381909645, 6.07656420967e-19)
            (0.331658291554, 9.42347455628e-19)
            (0.386934673463, 1.45807298451e-18)
            (0.442211055372, 2.25093365453e-18)
            (0.497487437281, 3.46705998652e-18)
            (0.55276381919, 5.32813533292e-18)
            (0.608040201099, 8.16966726289e-18)
            (0.663316583009, 1.24982359834e-17)
            (0.718592964918, 1.90769219238e-17)
            (0.773869346827, 2.90524735315e-17)
            (0.829145728736, 4.41441497344e-17)
            (0.884422110645, 6.69234664946e-17)
            (0.939698492554, 1.01227597763e-16)
            (0.994974874463, 1.5276879855e-16)
            (1.05025125637, 2.30030607126e-16)
            (1.10552763828, 3.455825607e-16)
            (1.16080402019, 5.18004152415e-16)
            (1.2160804021, 7.74693467874e-16)
            (1.27135678401, 1.15595726051e-15)
            (1.32663316592, 1.7209525267e-15)
            (1.38190954783, 2.55629654804e-15)
            (1.43718592974, 3.78851321935e-15)
            (1.49246231164, 5.60198072926e-15)
            (1.54773869355, 8.26474853531e-15)
            (1.60301507546, 1.21655824935e-14)
            (1.65829145737, 1.78669903109e-14)
            (1.71356783928, 2.61809334011e-14)
            (1.76884422119, 3.82766640217e-14)
            (1.8241206031, 5.58339385669e-14)
            (1.87939698501, 8.12601610634e-14)
            (1.93467336692, 1.17997368844e-13)
            (1.98994974883, 1.70955153779e-13)
            (2.04522613073, 2.47119662849e-13)
            (2.10050251264, 3.56408159173e-13)
            (2.15577889455, 5.12865166119e-13)
            (2.21105527646, 7.36332547867e-13)
            (2.26633165837, 1.05477547469e-12)
            (2.32160804028, 1.50751373801e-12)
            (2.37688442219, 2.14969954284e-12)
            (2.4321608041, 3.0585070067e-12)
            (2.48743718601, 4.34166614965e-12)
            (2.54271356792, 6.14919975831e-12)
            (2.59798994983, 8.68952440846e-12)
            (2.65326633173, 1.22514824889e-11)
            (2.70854271364, 1.72344135505e-11)
            (2.76381909555, 2.41890942279e-11)
            (2.81909547746, 3.38733330974e-11)
            (2.87437185937, 4.73272739114e-11)
            (2.92964824128, 6.59751332832e-11)
            (2.98492462319, 9.17623005577e-11)
            (3.0402010051, 1.27339617258e-10)
            (3.09547738701, 1.76310435156e-10)
            (3.15075376892, 2.43560986603e-10)
            (3.20603015082, 3.35701020913e-10)
            (3.26130653273, 4.61649988972e-10)
            (3.31658291464, 6.33414829961e-10)
            (3.37185929655, 8.67119304497e-10)
            (3.42713567846, 1.1843626767e-09)
            (3.48241206037, 1.61400843792e-09)
            (3.53768844228, 2.19453301577e-09)
            (3.59296482419, 2.97710167631e-09)
            (3.6482412061, 4.02958615473e-09)
            (3.70351758801, 5.44179846963e-09)
            (3.75879396992, 7.33229100236e-09)
            (3.81407035182, 9.85716807026e-09)
            (3.86934673373, 1.32214735486e-08)
            (3.92462311564, 1.76938682636e-08)
            (3.97989949755, 2.36254967428e-08)
            (4.03517587946, 3.14741737443e-08)
            (4.09045226137, 4.18353067073e-08)
            (4.14572864328, 5.54813227623e-08)
            (4.20100502519, 7.34118023096e-08)
            (4.2562814071, 9.6917052176e-08)
            (4.31155778901, 1.27658499692e-07)
            (4.36683417091, 1.67770077861e-07)
            (4.42211055282, 2.1998572809e-07)
            (4.47738693473, 2.87799302112e-07)
            (4.53266331664, 3.75664534898e-07)
            (4.58793969855, 4.89244427081e-07)
            (4.64321608046, 6.35721366135e-07)
            (4.69849246237, 8.24181683804e-07)
            (4.75376884428, 1.06609115369e-06)
            (4.80904522619, 1.37588124478e-06)
            (4.8643216081, 1.77166983895e-06)
            (4.91959799001, 2.27614467525e-06)
            (4.97487437191, 2.91764309036e-06)
            (5.03015075382, 3.73146777934e-06)
            (5.08542713573, 4.76148540579e-06)
            (5.14070351764, 6.06206305336e-06)
            (5.19597989955, 7.70040684093e-06)
            (5.25125628146, 9.75937763481e-06)
            (5.30653266337, 1.23408707914e-05)
            (5.36180904528, 1.55698603588e-05)
            (5.41708542719, 1.95992232512e-05)
            (5.4723618091, 2.46154756643e-05)
            (5.527638191, 3.08455724844e-05)
            (5.58291457291, 3.85649406889e-05)
            (5.63819095482, 4.81069397332e-05)
            (5.69346733673, 5.98739656208e-05)
            (5.74874371864, 7.43504406462e-05)
            (5.80402010055, 9.21179575146e-05)
            (5.85929648246, 0.000113872874412)
            (5.91457286437, 0.000140446686285)
            (5.96984924628, 0.000172829526647)
            (6.02512562819, 0.000212197183025)
            (6.08040201009, 0.000259942037113)
            (6.135678392, 0.00031770836682)
            (6.19095477391, 0.000387432470777)
            (6.24623115582, 0.00047138809533)
            (6.30150753773, 0.000572237658264)
            (6.35678391964, 0.000693089770923)
            (6.41206030155, 0.000837563559486)
            (6.46733668346, 0.001009860275)
            (6.52261306537, 0.00121484265846)
            (6.57788944728, 0.00145812248979)
            (6.63316582919, 0.00174615669565)
            (6.68844221109, 0.00208635231866)
            (6.743718593, 0.00248718055784)
            (6.79899497491, 0.00295829997379)
            (6.85427135682, 0.00351068881233)
            (6.90954773873, 0.00415678623274)
            (6.96482412064, 0.00491064203289)
            (7.02010050255, 0.00578807423928)
            (7.07537688446, 0.00680683367829)
            (7.13065326637, 0.0079867743626)
            (7.18592964828, 0.00935002821765)
            (7.24120603018, 0.0109211823369)
            (7.29648241209, 0.0127274565951)
            (7.351758794, 0.0147988790707)
            (7.40703517591, 0.0171684563328)
            (7.46231155782, 0.0198723352496)
            (7.51758793973, 0.0229499525673)
            (7.57286432164, 0.0264441681175)
            (7.62814070355, 0.0304013771305)
            (7.68341708546, 0.0348715967826)
            (7.73869346737, 0.0399085217986)
            (7.79396984928, 0.0455695436738)
            (7.84924623118, 0.0519157278937)
            (7.90452261309, 0.0590117434215)
            (7.959798995, 0.0669257387163)
            (8.01507537691, 0.0757291586391)
            (8.07035175882, 0.0854964968305)
            (8.12562814073, 0.0963049784944)
            (8.18090452264, 0.10823416903)
            (8.23618090455, 0.121365504604)
            (8.29145728646, 0.135781741573)
            (8.34673366837, 0.151566322639)
            (8.40201005027, 0.168802658769)
            (8.45728643218, 0.187573327187)
            (8.51256281409, 0.20795918721)
            (8.567839196, 0.23003841725)
            (8.62311557791, 0.253885478029)
            (8.67839195982, 0.279570008804)
            (8.73366834173, 0.307155665257)
            (8.78894472364, 0.336698909601)
            (8.84422110555, 0.368247765295)
            (8.89949748746, 0.401840550605)
            (8.95477386937, 0.437504606981)
            (9.01005025127, 0.475255039822)
            (9.06532663318, 0.515093490606)
            (9.12060301509, 0.557006960564)
            (9.175879397, 0.600966706983)
            (9.23115577891, 0.646927233817)
            (9.28643216082, 0.694825398515)
            (9.34170854273, 0.744579656831)
            (9.39698492464, 0.796089466779)
            (9.45226130655, 0.849234871895)
            (9.50753768846, 0.903876282488)
            (9.56281407036, 0.959854471627)
            (9.61809045227, 1.01699080023)
            (9.67336683418, 1.07508768287)
            (9.72864321609, 1.13392930257)
            (9.783919598, 1.19328257957)
            (9.83919597991, 1.25289839484)
            (9.89447236182, 1.31251306543)
            (9.94974874373, 1.37185006401)
            (10.0050251256, 1.43062197119)
            (10.0603015075, 1.48853264427)
            (10.1155778895, 1.54527958232)
            (10.1708542714, 1.600556463)
            (10.2261306533, 1.65405582323)
            (10.2814070352, 1.70547185177)
            (10.3366834171, 1.75450325944)
            (10.391959799, 1.8008561898)
            (10.4472361809, 1.84424713146)
            (10.5025125628, 1.88440579205)
            (10.5577889447, 1.9210778932)
            (10.6130653266, 1.9540278464)
            (10.6683417085, 1.9830412702)
            (10.7236180905, 2.00792731149)
            (10.7788944724, 2.02852073546)
            (10.8341708543, 2.04468375234)
            (10.8894472362, 2.05630755269)
            (10.9447236181, 2.06331352732)
            (11.0, 2.06565415259)
    };
    

    
    % Draw series plot
    \addplot[no markers,lightgray] coordinates {
            (1e-10, 8.1513808932e-22)
            (0.055276382009, 1.27864221703e-21)
            (0.110552763918, 2.00153959627e-21)
            (0.165829145827, 3.12663098239e-21)
            (0.221105527736, 4.87400939864e-21)
            (0.276381909645, 7.58216780527e-21)
            (0.331658291554, 1.17705759942e-20)
            (0.386934673463, 1.82347298545e-20)
            (0.442211055372, 2.81902057749e-20)
            (0.497487437281, 4.34905052052e-20)
            (0.55276381919, 6.69557614768e-20)
            (0.608040201099, 1.02867633471e-19)
            (0.663316583009, 1.57712762251e-19)
            (0.718592964918, 2.41297166428e-19)
            (0.773869346827, 3.68412960551e-19)
            (0.829145728736, 5.61325629836e-19)
            (0.884422110645, 8.53477622856e-19)
            (0.939698492554, 1.29499084989e-18)
            (0.994974874463, 1.96082373374e-18)
            (1.05025125637, 2.96283662411e-18)
            (1.10552763828, 4.46759862e-18)
            (1.16080402019, 6.72260951914e-18)
            (1.2160804021, 1.00948278779e-17)
            (1.27135678401, 1.51271545837e-17)
            (1.32663316592, 2.26210555273e-17)
            (1.38190954783, 3.37571505757e-17)
            (1.43718592974, 5.02708224281e-17)
            (1.49246231164, 7.47073779541e-17)
            (1.54773869355, 1.10791971864e-16)
            (1.60301507546, 1.63964709615e-16)
            (1.65829145737, 2.4215295367e-16)
            (1.71356783928, 3.568834853e-16)
            (1.76884422119, 5.24880485707e-16)
            (1.8241206031, 7.70356401827e-16)
            (1.87939698501, 1.1282887546e-15)
            (1.93467336692, 1.64909668084e-15)
            (1.98994974883, 2.40530010249e-15)
            (2.04522613073, 3.50098069723e-15)
            (2.10050251264, 5.08519319383e-15)
            (2.15577889455, 7.37093348391e-15)
            (2.21105527646, 1.06619053164e-14)
            (2.26633165837, 1.53902064282e-14)
            (2.32160804028, 2.21692684807e-14)
            (2.37688442219, 3.18680562535e-14)
            (2.4321608041, 4.57148271024e-14)
            (2.48743718601, 6.54418973504e-14)
            (2.54271356792, 9.34871622719e-14)
            (2.59798994983, 1.33273979766e-13)
            (2.65326633173, 1.89599003397e-13)
            (2.70854271364, 2.69168372161e-13)
            (2.76381909555, 3.81337311331e-13)
            (2.81909547746, 5.39127973685e-13)
            (2.87437185937, 7.60626982134e-13)
            (2.92964824128, 1.07089991056e-12)
            (2.98492462319, 1.50460787848e-12)
            (3.0402010051, 2.10957550998e-12)
            (3.09547738701, 2.95164493244e-12)
            (3.15075376892, 4.12126409079e-12)
            (3.20603015082, 5.74240833568e-12)
            (3.26130653273, 7.98463360206e-12)
            (3.31658291464, 1.10793225741e-11)
            (3.37185929655, 1.53415315047e-11)
            (3.42713567846, 2.11992999501e-11)
            (3.48241206037, 2.92328809246e-11)
            (3.53768844228, 4.02271263532e-11)
            (3.59296482419, 5.52412752773e-11)
            (3.6482412061, 7.57017077088e-11)
            (3.70351758801, 1.03524931427e-10)
            (3.75879396992, 1.4128027146e-10)
            (3.81407035182, 1.9240457043e-10)
            (3.86934673373, 2.614848562e-10)
            (3.92462311564, 3.54629609605e-10)
            (3.97989949755, 4.79955243036e-10)
            (4.03517587946, 6.48222021503e-10)
            (4.09045226137, 8.73663346412e-10)
            (4.14572864328, 1.1750646054e-09)
            (4.20100502519, 1.57716332067e-09)
            (4.2562814071, 2.11246190596e-09)
            (4.31155778901, 2.82356895404e-09)
            (4.36683417091, 3.76621582904e-09)
            (4.42211055282, 5.01313390222e-09)
            (4.47738693473, 6.65902583363e-09)
            (4.53266331664, 8.8269240313e-09)
            (4.58793969855, 1.16763034264e-08)
            (4.64321608046, 1.54134071272e-08)
            (4.69849246237, 2.03043561181e-08)
            (4.75376884428, 2.66917524363e-08)
            (4.80904522619, 3.50156545195e-08)
            (4.8643216081, 4.58400099771e-08)
            (4.91959799001, 5.98858823223e-08)
            (4.97487437191, 7.8073112938e-08)
            (5.03015075382, 1.01572427902e-07)
            (5.08542713573, 1.31870443113e-07)
            (5.14070351764, 1.70850554135e-07)
            (5.19597989955, 2.20893335042e-07)
            (5.25125628146, 2.85000831438e-07)
            (5.30653266337, 3.66950037178e-07)
            (5.36180904528, 4.71481915595e-07)
            (5.41708542719, 6.0453359027e-07)
            (5.4723618091, 7.73522817077e-07)
            (5.527638191, 9.87695590851e-07)
            (5.58291457291, 1.25854977255e-06)
            (5.63819095482, 1.60034998546e-06)
            (5.69346733673, 2.03075176448e-06)
            (5.74874371864, 2.57155609647e-06)
            (5.80402010055, 3.24961911084e-06)
            (5.85929648246, 4.09794581817e-06)
            (5.91457286437, 5.15700150475e-06)
            (5.96984924628, 6.47627972369e-06)
            (6.02512562819, 8.11617183386e-06)
            (6.08040201009, 1.01501897758e-05)
            (6.135678392, 1.26676012871e-05)
            (6.19095477391, 1.57765450907e-05)
            (6.24623115582, 1.96077027684e-05)
            (6.30150753773, 2.43186140877e-05)
            (6.35678391964, 3.00987334786e-05)
            (6.41206030155, 3.71753371589e-05)
            (6.46733668346, 4.58204030399e-05)
            (6.52261306537, 5.63585989479e-05)
            (6.57788944728, 6.917652878e-05)
            (6.63316582919, 8.47334008366e-05)
            (6.68844221109, 0.00010357329757)
            (6.743718593, 0.000126339241121)
            (6.79899497491, 0.000153789264009)
            (6.85427135682, 0.000186814708867)
            (6.90954773873, 0.000226460994736)
            (6.96482412064, 0.000273951099686)
            (7.02010050255, 0.0003307120199)
            (7.07537688446, 0.000398404473153)
            (7.13065326637, 0.000478956119177)
            (7.18592964828, 0.000574598569962)
            (7.24120603018, 0.000687908458698)
            (7.29648241209, 0.00082185282598)
            (7.351758794, 0.000979839065024)
            (7.40703517591, 0.00116576964306)
            (7.46231155782, 0.00138410178263)
            (7.51758793973, 0.00163991224338)
            (7.57286432164, 0.00193896729068)
            (7.62814070355, 0.00228779787185)
            (7.68341708546, 0.00269377994184)
            (7.73869346737, 0.00316521978838)
            (7.79396984928, 0.00371144410064)
            (7.84924623118, 0.00434289440472)
            (7.90452261309, 0.00507122535429)
            (7.959798995, 0.00590940621532)
            (8.01507537691, 0.00687182472037)
            (8.07035175882, 0.00797439229165)
            (8.12562814073, 0.00923464944417)
            (8.18090452264, 0.0106718699827)
            (8.23618090455, 0.0123071624014)
            (8.29145728646, 0.0141635666854)
            (8.34673366837, 0.0162661445043)
            (8.40201005027, 0.0186420605788)
            (8.45728643218, 0.0213206528046)
            (8.51256281409, 0.02433348853)
            (8.567839196, 0.0277144042178)
            (8.62311557791, 0.0314995255766)
            (8.67839195982, 0.0357272651393)
            (8.73366834173, 0.0404382941889)
            (8.78894472364, 0.0456754859059)
            (8.84422110555, 0.0514838266322)
            (8.89949748746, 0.0579102922278)
            (8.95477386937, 0.0650036866403)
            (9.01005025127, 0.0728144400203)
            (9.06532663318, 0.0813943640063)
            (9.12060301509, 0.0907963621635)
            (9.175879397, 0.101074094011)
            (9.23115577891, 0.112281591599)
            (9.28643216082, 0.124472828192)
            (9.34170854273, 0.137701239323)
            (9.39698492464, 0.152019197219)
            (9.45226130655, 0.16747744042)
            (9.50753768846, 0.184124461323)
            (9.56281407036, 0.202005855273)
            (9.61809045227, 0.221163635808)
            (9.67336683418, 0.241635521636)
            (9.72864321609, 0.263454201888)
            (9.783919598, 0.286646587171)
            (9.83919597991, 0.31123305484)
            (9.89447236182, 0.337226697786)
            (9.94974874373, 0.364632586799)
            (10.0050251256, 0.393447057227)
            (10.0603015075, 0.423657031213)
            (10.1155778895, 0.455239387166)
            (10.1708542714, 0.488160388337)
            (10.2261306533, 0.522375182427)
            (10.2814070352, 0.557827383956)
            (10.3366834171, 0.594448750776)
            (10.391959799, 0.632158965456)
            (10.4472361809, 0.670865531488)
            (10.5025125628, 0.710463793157)
            (10.5577889447, 0.750837086665)
            (10.6130653266, 0.791857028601)
            (10.6683417085, 0.833383946154)
            (10.7236180905, 0.87526745162)
            (10.7788944724, 0.917347161733)
            (10.8341708543, 0.959453560213)
            (10.8894472362, 1.00140899974)
            (10.9447236181, 1.04302883726)
            (11.0, 1.08412269434)
    };
    

    
    % Draw series plot
    \addplot[no markers,lightgray] coordinates {
            (1e-10, 1.03976040908e-23)
            (0.055276382009, 1.63111975876e-23)
            (0.110552763918, 2.55390744209e-23)
            (0.165829145827, 3.99108697209e-23)
            (0.221105527736, 6.22506604186e-23)
            (0.276381909645, 9.69088562236e-23)
            (0.331658291554, 1.50573902211e-22)
            (0.386934673463, 2.33508490549e-22)
            (0.442211055372, 3.61428487726e-22)
            (0.497487437281, 5.5835295366e-22)
            (0.55276381919, 8.60918396959e-22)
            (0.608040201099, 1.32489631451e-21)
            (0.663316583009, 2.03501928335e-21)
            (0.718592964918, 3.11976512477e-21)
            (0.773869346827, 4.77355578356e-21)
            (0.829145728736, 7.29002199193e-21)
            (0.884422110645, 1.11117486249e-20)
            (0.939698492554, 1.69045148308e-20)
            (0.994974874463, 2.56678650154e-20)
            (1.05025125637, 3.88994549226e-20)
            (1.10552763828, 5.88388286259e-20)
            (1.16080402019, 8.88282801533e-20)
            (1.2160804021, 1.33845946215e-19)
            (1.27135678401, 2.01291726151e-19)
            (1.32663316592, 3.02143571946e-19)
            (1.38190954783, 4.52655214109e-19)
            (1.43718592974, 6.7684375898e-19)
            (1.49246231164, 1.01012719457e-18)
            (1.54773869355, 1.50463247484e-18)
            (1.60301507546, 2.2369255602e-18)
            (1.65829145737, 3.31924543107e-18)
            (1.71356783928, 4.91579692327e-18)
            (1.76884422119, 7.26633194873e-18)
            (1.8241206031, 1.07202093746e-17)
            (1.87939698501, 1.57854887997e-17)
            (1.93467336692, 2.31995485821e-17)
            (1.98994974883, 3.40304559284e-17)
            (2.04522613073, 4.98221808743e-17)
            (2.10050251264, 7.28021883053e-17)
            (2.15577889455, 1.06177590688e-16)
            (2.21105527646, 1.54556769736e-16)
            (2.26633165837, 2.24548382718e-16)
            (2.32160804028, 3.25610621891e-16)
            (2.37688442219, 4.71252784398e-16)
            (2.4321608041, 6.80731782793e-16)
            (2.48743718601, 9.81442498201e-16)
            (2.54271356792, 1.41227875098e-15)
            (2.59798994983, 2.02834919027e-15)
            (2.65326633173, 2.90758054599e-15)
            (2.70854271364, 4.15994433226e-15)
            (2.76381909555, 5.94032240843e-15)
            (2.81909547746, 8.46640905706e-15)
            (2.87437185937, 1.20435691858e-14)
            (2.92964824128, 1.70992833555e-14)
            (2.98492462319, 2.4230776939e-14)
            (3.0402010051, 3.42707416701e-14)
            (3.09547738701, 4.83778317932e-14)
            (3.15075376892, 6.81610120542e-14)
            (3.20603015082, 9.58500609612e-14)
            (3.26130653273, 1.34528868624e-13)
            (3.31658291464, 1.88453982419e-13)
            (3.37185929655, 2.63488631701e-13)
            (3.42713567846, 3.67692844317e-13)
            (3.48241206037, 5.12124092364e-13)
            (3.53768844228, 7.11921271759e-13)
            (3.59296482419, 9.87769165733e-13)
            (3.6482412061, 1.36787273704e-12)
            (3.70351758801, 1.89061306878e-12)
            (3.75879396992, 2.60811268047e-12)
            (3.81407035182, 3.59101135899e-12)
            (3.86934673373, 4.93484981727e-12)
            (3.92462311564, 6.76858418507e-12)
            (3.97989949755, 9.26591821564e-12)
            (4.03517587946, 1.2660353251e-11)
            (4.09045226137, 1.72651325824e-11)
            (4.14572864328, 2.34996148849e-11)
            (4.20100502519, 3.19240737586e-11)
            (4.2562814071, 4.32855160188e-11)
            (4.31155778901, 5.8577876773e-11)
            (4.36683417091, 7.9120930502e-11)
            (4.42211055282, 1.06663512054e-10)
            (4.47738693473, 1.43518241953e-10)
            (4.53266331664, 1.92736986942e-10)
            (4.58793969855, 2.58338871517e-10)
            (4.64321608046, 3.456059288e-10)
            (4.69849246237, 4.61465611903e-10)
            (4.75376884428, 6.14984592839e-10)
            (4.80904522619, 8.18004816912e-10)
            (4.8643216081, 1.08596097748e-09)
            (4.91959799001, 1.43892882184e-09)
            (4.97487437191, 1.90296647231e-09)
            (5.03015075382, 2.51182682929e-09)
            (5.08542713573, 3.30913881933e-09)
            (5.14070351764, 4.35117961248e-09)
            (5.19597989955, 5.71038998184e-09)
            (5.25125628146, 7.47982194372e-09)
            (5.30653266337, 9.77875316463e-09)
            (5.36180904528, 1.27597581011e-08)
            (5.41708542719, 1.66175935228e-08)
            (5.4723618091, 2.16003384139e-08)
            (5.527638191, 2.8023328139e-08)
            (5.58291457291, 3.62865435923e-08)
            (5.63819095482, 4.68962617917e-08)
            (5.69346733673, 6.04919496364e-08)
            (5.74874371864, 7.78795926922e-08)
            (5.80402010055, 1.00072902064e-07)
            (5.85929648246, 1.28344141781e-07)
            (5.91457286437, 1.6428667477e-07)
            (5.96984924628, 2.09891746705e-07)
            (6.02512562819, 2.67642524252e-07)
            (6.08040201009, 3.40628989314e-07)
            (6.135678392, 4.32687977088e-07)
            (6.19095477391, 5.48573447717e-07)
            (6.24623115582, 6.94163015589e-07)
            (6.30150753773, 8.76707844575e-07)
            (6.35678391964, 1.10513427161e-06)
            (6.41206030155, 1.39040696595e-06)
            (6.46733668346, 1.74596509007e-06)
            (6.52261306537, 2.18824482487e-06)
            (6.57788944728, 2.73730378154e-06)
            (6.63316582919, 3.41756527199e-06)
            (6.68844221109, 4.25870317617e-06)
            (6.743718593, 5.29669125377e-06)
            (6.79899497491, 6.57504422833e-06)
            (6.85427135682, 8.14628184605e-06)
            (6.90954773873, 1.00736514057e-05)
            (6.96482412064, 1.24331489861e-05)
            (7.02010050255, 1.53158847828e-05)
            (7.07537688446, 1.88308436104e-05)
            (7.13065326637, 2.31080977366e-05)
            (7.18592964828, 2.83025357784e-05)
            (7.24120603018, 3.45981783905e-05)
            (7.29648241209, 4.2213158876e-05)
            (7.351758794, 5.14054546038e-05)
            (7.40703517591, 6.24794631508e-05)
            (7.46231155782, 7.57935253184e-05)
            (7.51758793973, 9.17685054763e-05)
            (7.57286432164, 0.000110897547937)
            (7.62814070355, 0.000133757136071)
            (7.68341708546, 0.000161019588457)
            (7.73869346737, 0.000193467133268)
            (7.79396984928, 0.000232007708055)
            (7.84924623118, 0.000277692636822)
            (7.90452261309, 0.000331736339393)
            (7.959798995, 0.000395538229233)
            (8.01507537691, 0.000470706954639)
            (8.07035175882, 0.000559087134115)
            (8.12562814073, 0.000662788729366)
            (8.18090452264, 0.000784219188087)
            (8.23618090455, 0.000926118473171)
            (8.29145728646, 0.00109159707451)
            (8.34673366837, 0.00128417707377)
            (8.40201005027, 0.00150783630066)
            (8.45728643218, 0.00176705558139)
            (8.51256281409, 0.00206686903478)
            (8.567839196, 0.00241291731952)
            (8.62311557791, 0.00281150367679)
            (8.67839195982, 0.00326965254472)
            (8.73366834173, 0.0037951704469)
            (8.78894472364, 0.00439670877428)
            (8.84422110555, 0.00508382799037)
            (8.89949748746, 0.00586706269266)
            (8.95477386937, 0.00675798686097)
            (9.01005025127, 0.00776927851503)
            (9.06532663318, 0.00891478289191)
            (9.12060301509, 0.0102095731391)
            (9.175879397, 0.0116700074036)
            (9.23115577891, 0.0133137810838)
            (9.28643216082, 0.0151599728984)
            (9.34170854273, 0.0172290833236)
            (9.39698492464, 0.0195430638519)
            (9.45226130655, 0.0221253354433)
            (9.50753768846, 0.0250007944677)
            (9.56281407036, 0.0281958043886)
            (9.61809045227, 0.0317381714076)
            (9.67336683418, 0.0356571022844)
            (9.72864321609, 0.0399831425718)
            (9.783919598, 0.044748093561)
            (9.83919597991, 0.0499849063231)
            (9.89447236182, 0.0557275513597)
            (9.94974874373, 0.0620108625451)
            (10.0050251256, 0.0688703542486)
            (10.0603015075, 0.0763420107788)
            (10.1155778895, 0.0844620475861)
            (10.1708542714, 0.0932666439921)
            (10.2261306533, 0.102791647597)
            (10.2814070352, 0.113072250928)
            (10.3366834171, 0.124142641338)
            (10.391959799, 0.136035625659)
            (10.4472361809, 0.148782231605)
            (10.5025125628, 0.162411288449)
            (10.5577889447, 0.176948990038)
            (10.6130653266, 0.192418443748)
            (10.6683417085, 0.208839209493)
            (10.7236180905, 0.22622683345)
            (10.7788944724, 0.244592381606)
            (10.8341708543, 0.263941978707)
            (10.8894472362, 0.284276358564)
            (10.9447236181, 0.305590432023)
            (11.0, 0.327872879135)
    };
    

    
    % Draw series plot
    \addplot[no markers,lightgray] coordinates {
            (1e-10, 1.3541174871e-25)
            (0.055276382009, 2.12441166766e-25)
            (0.110552763918, 3.32695756548e-25)
            (0.165829145827, 5.20094303956e-25)
            (0.221105527736, 8.11602143786e-25)
            (0.276381909645, 1.26424290739e-24)
            (0.331658291554, 1.96582167119e-24)
            (0.386934673463, 3.05129335487e-24)
            (0.442211055372, 4.72770163323e-24)
            (0.497487437281, 7.31210483891e-24)
            (0.55276381919, 1.12891439565e-23)
            (0.608040201099, 1.7398261931e-23)
            (0.663316583009, 2.67655992322e-23)
            (0.718592964918, 4.11030764796e-23)
            (0.773869346827, 6.30083257322e-23)
            (0.829145728736, 9.64157096971e-23)
            (0.884422110645, 1.47273265759e-22)
            (0.939698492554, 2.24556838106e-22)
            (0.994974874463, 3.41786499362e-22)
            (1.05025125637, 5.19289780297e-22)
            (1.10552763828, 7.87573146215e-22)
            (1.16080402019, 1.1923349542e-21)
            (1.2160804021, 1.80190505745e-21)
            (1.27135678401, 2.71826494768e-21)
            (1.32663316592, 4.09334092726e-21)
            (1.38190954783, 6.15304799841e-21)
            (1.43718592974, 9.23270459482e-21)
            (1.49246231164, 1.38290972249e-20)
            (1.54773869355, 2.06768783095e-20)
            (1.60301507546, 3.08604579112e-20)
            (1.65829145737, 4.59775697843e-20)
            (1.71356783928, 6.83779239415e-20)
            (1.76884422119, 1.01510755871e-19)
            (1.8241206031, 1.50430003186e-19)
            (1.87939698501, 2.22527208534e-19)
            (1.93467336692, 3.28592788232e-19)
            (1.98994974883, 4.84349869813e-19)
            (2.04522613073, 7.12666918087e-19)
            (2.10050251264, 1.04674347811e-18)
            (2.15577889455, 1.53468829273e-18)
            (2.21105527646, 2.24608583461e-18)
            (2.26633165837, 3.28139694842e-18)
            (2.32160804028, 4.78539116135e-18)
            (2.37688442219, 6.96630313424e-18)
            (2.4321608041, 1.01231004496e-17)
            (2.48743718601, 1.46842231693e-17)
            (2.54271356792, 2.12625162452e-17)
            (2.59798994983, 3.0732974306e-17)
            (2.65326633173, 4.43425617562e-17)
            (2.70854271364, 6.38650444023e-17)
            (2.76381909555, 9.18188627989e-17)
            (2.81909547746, 1.31773127322e-16)
            (2.87437185937, 1.8877655165e-16)
            (2.92964824128, 2.69957561481e-16)
            (2.98492462319, 3.8536227451e-16)
            (3.0402010051, 5.49122386121e-16)
            (3.09547738701, 7.81079705047e-16)
            (3.15075376892, 1.1090415327e-15)
            (3.20603015082, 1.57190581123e-15)
            (3.26130653273, 2.22398309027e-15)
            (3.31658291464, 3.14096208782e-15)
            (3.37185929655, 4.42812793917e-15)
            (3.42713567846, 6.2316618377e-15)
            (3.48241206037, 8.75414727183e-15)
            (3.53768844228, 1.22758075425e-14)
            (3.59296482419, 1.71835363442e-14)
            (3.6482412061, 2.40105038042e-14)
            (3.70351758801, 3.34900851728e-14)
            (3.75879396992, 4.66291483613e-14)
            (3.81407035182, 6.48074550437e-14)
            (3.86934673373, 8.9912216351e-14)
            (3.92462311564, 1.24519879414e-13)
            (3.97989949755, 1.72141241923e-13)
            (4.03517587946, 2.37551307183e-13)
            (4.09045226137, 3.27232303494e-13)
            (4.14572864328, 4.49967520452e-13)
            (4.20100502519, 6.17635736044e-13)
            (4.2562814071, 8.46271894407e-13)
            (4.31155778901, 1.15748046909e-12)
            (4.36683417091, 1.58031505605e-12)
            (4.42211055282, 2.1537730389e-12)
            (4.47738693473, 2.93010002902e-12)
            (4.53266331664, 3.97915823475e-12)
            (4.58793969855, 5.39418981508e-12)
            (4.64321608046, 7.29940552642e-12)
            (4.69849246237, 9.85995678758e-12)
            (4.75376884428, 1.32950135232e-11)
            (4.80904522619, 1.78948807021e-11)
            (4.8643216081, 2.40433558268e-11)
            (4.91959799001, 3.22468733794e-11)
            (4.97487437191, 4.31724199674e-11)
            (5.03015075382, 5.76967600479e-11)
            (5.08542713573, 7.69702170725e-11)
            (5.14070351764, 1.02499146449e-10)
            (5.19597989955, 1.36252361651e-10)
            (5.25125628146, 1.80798190924e-10)
            (5.30653266337, 2.39480620537e-10)
            (5.36180904528, 3.16645208965e-10)
            (5.41708542719, 4.17928241877e-10)
            (5.4723618091, 5.50626071551e-10)
            (5.527638191, 7.24165905813e-10)
            (5.58291457291, 9.50704675027e-10)
            (5.63819095482, 1.245889246e-09)
            (5.69346733673, 1.62981945301e-09)
            (5.74874371864, 2.12826552077e-09)
            (5.80402010055, 2.77420387257e-09)
            (5.85929648246, 3.60975054189e-09)
            (5.91457286437, 4.68859002368e-09)
            (5.96984924628, 6.07902011233e-09)
            (6.02512562819, 7.86776090516e-09)
            (6.08040201009, 1.01647096838e-08)
            (6.135678392, 1.31088639754e-08)
            (6.19095477391, 1.68756840986e-08)
            (6.24623115582, 2.16862255005e-08)
            (6.30150753773, 2.78184420459e-08)
            (6.35678391964, 3.56211462718e-08)
            (6.41206030155, 4.55312139697e-08)
            (6.46733668346, 5.80947411598e-08)
            (6.52261306537, 7.39930048884e-08)
            (6.57788944728, 9.40742490686e-08)
            (6.63316582919, 1.19392517097e-07)
            (6.68844221109, 1.51254989074e-07)
            (6.743718593, 1.9127955865e-07)
            (6.79899497491, 2.41464708914e-07)
            (6.85427135682, 3.04274125361e-07)
            (6.90954773873, 3.82738924279e-07)
            (6.96482412064, 4.80580885565e-07)
            (7.02010050255, 6.02360669102e-07)
            (7.07537688446, 7.53655673677e-07)
            (7.13065326637, 9.41272977885e-07)
            (7.18592964828, 1.17350369534e-06)
            (7.24120603018, 1.46042609436e-06)
            (7.29648241209, 1.81426598846e-06)
            (7.351758794, 2.24982421172e-06)
            (7.40703517591, 2.78498246791e-06)
            (7.46231155782, 3.44130049651e-06)
            (7.51758793973, 4.24471934903e-06)
            (7.57286432164, 5.22638762754e-06)
            (7.62814070355, 6.42362981807e-06)
            (7.68341708546, 7.88107836615e-06)
            (7.73869346737, 9.6519939003e-06)
            (7.79396984928, 1.17998010191e-05)
            (7.84924623118, 1.43998703234e-05)
            (7.90452261309, 1.75415808972e-05)
            (7.959798995, 2.13307012126e-05)
            (8.01507537691, 2.58921304507e-05)
            (8.07035175882, 3.13730464632e-05)
            (8.12562814073, 3.79465110355e-05)
            (8.18090452264, 4.58155877057e-05)
            (8.23618090455, 5.52180321004e-05)
            (8.29145728646, 6.64316195153e-05)
            (8.34673366837, 7.97801792142e-05)
            (8.40201005027, 9.56404095656e-05)
            (8.45728643218, 0.000114449552571)
            (8.51256281409, 0.000136714010456)
            (8.567839196, 0.000163018990632)
            (8.62311557791, 0.000194039268369)
            (8.67839195982, 0.000230551158756)
            (8.73366834173, 0.000273445790728)
            (8.78894472364, 0.000323743776024)
            (8.84422110555, 0.00038261136448)
            (8.89949748746, 0.000451378174021)
            (8.95477386937, 0.000531556578655)
            (9.01005025127, 0.000624862830542)
            (9.06532663318, 0.000733239982479)
            (9.12060301509, 0.000858882664624)
            (9.175879397, 0.00100426375373)
            (9.23115577891, 0.00117216295425)
            (9.28643216082, 0.00136569728836)
            (9.34170854273, 0.0015883534654)
            (9.39698492464, 0.0018440220715)
            (9.45226130655, 0.00213703348519)
            (9.50753768846, 0.00247219538684)
            (9.56281407036, 0.00285483168638)
            (9.61809045227, 0.00329082264719)
            (9.67336683418, 0.00378664593265)
            (9.72864321609, 0.00434941824702)
            (9.783919598, 0.00498693718404)
            (9.83919597991, 0.00570772283526)
            (9.89447236182, 0.00652105864605)
            (9.94974874373, 0.00743703094243)
            (10.0050251256, 0.00846656648493)
            (10.0603015075, 0.00962146734027)
            (10.1155778895, 0.0109144422969)
            (10.1708542714, 0.0123591339882)
            (10.2261306533, 0.0139701408306)
            (10.2814070352, 0.0157630328296)
            (10.3366834171, 0.0177543602649)
            (10.391959799, 0.0199616542272)
            (10.4472361809, 0.0224034179571)
            (10.5025125628, 0.0250991079223)
            (10.5577889447, 0.0280691035736)
            (10.6130653266, 0.0313346647378)
            (10.6683417085, 0.0349178756422)
            (10.7236180905, 0.0388415746231)
            (10.7788944724, 0.0431292686466)
            (10.8341708543, 0.0478050318695)
            (10.8894472362, 0.0528933875908)
            (10.9447236181, 0.0584191730897)
            (11.0, 0.0644073870157)
    };
    

    
    % Draw series plot
    \addplot[no markers,lightgray] coordinates {
            (1e-10, 1e+32)
            (0.055276382009, 592081.06367)
            (0.110552763918, 74010.1331596)
            (0.165829145827, 21928.9283634)
            (0.221105527736, 9251.2666575)
            (0.276381909645, 4736.64852992)
            (0.331658291554, 2741.11604791)
            (0.386934673463, 1726.18386713)
            (0.442211055372, 1156.40833297)
            (0.497487437281, 812.182532958)
            (0.55276381919, 592.081066562)
            (0.608040201099, 444.839268663)
            (0.663316583009, 342.639506143)
            (0.718592964918, 269.495251086)
            (0.773869346827, 215.772983474)
            (0.829145728736, 175.431427161)
            (0.884422110645, 144.55104167)
            (0.939698492554, 120.51314201)
            (0.994974874463, 101.52281665)
            (1.05025125637, 86.3217767479)
            (1.10552763828, 74.0101333403)
            (1.16080402019, 63.9327358525)
            (1.2160804021, 55.6049085966)
            (1.27135678401, 48.6628640374)
            (1.32663316592, 42.8299382776)
            (1.38190954783, 37.8931882723)
            (1.43718592974, 33.6869063928)
            (1.49246231164, 30.0808345661)
            (1.54773869355, 26.9716229395)
            (1.60301507546, 24.2765618423)
            (1.65829145737, 21.9289283991)
            (1.71356783928, 19.8744945379)
            (1.76884422119, 18.0688802119)
            (1.8241206031, 16.4755284744)
            (1.87939698501, 15.0641427536)
            (1.93467336692, 13.8094709456)
            (1.98994974883, 12.6903520832)
            (2.04522613073, 11.688963473)
            (2.10050251264, 10.790222095)
            (2.15577889455, 9.98130559856)
            (2.21105527646, 9.25126666879)
            (2.26633165837, 8.59072077893)
            (2.32160804028, 7.9915919826)
            (2.37688442219, 7.44690488647)
            (2.4321608041, 6.95061357544)
            (2.48743718601, 6.4974602668)
            (2.54271356792, 6.0828580054)
            (2.59798994983, 5.70279289574)
            (2.65326633173, 5.3537422853)
            (2.70854271364, 5.03260602995)
            (2.76381909555, 4.73664853455)
            (2.81909547746, 4.46344970502)
            (2.87437185937, 4.21086329954)
            (2.92964824128, 3.97698144658)
            (2.98492462319, 3.76010432114)
            (3.0402010051, 3.55871415071)
            (3.09547738701, 3.37145286777)
            (3.15075376892, 3.19710284312)
            (3.20603015082, 3.03457023057)
            (3.26130653273, 2.88287053121)
            (3.31658291464, 2.74111605014)
            (3.37185929655, 2.60850497103)
            (3.42713567846, 2.48431181746)
            (3.48241206037, 2.36787910606)
            (3.53768844228, 2.25861002668)
            (3.59296482419, 2.15596200941)
            (3.6482412061, 2.05944105947)
            (3.70351758801, 1.96859675836)
            (3.75879396992, 1.88301784435)
            (3.81407035182, 1.80232829797)
            (3.86934673373, 1.72618386833)
            (3.92462311564, 1.65426898541)
            (3.97989949755, 1.58629401052)
            (4.03517587946, 1.52199278396)
            (4.09045226137, 1.46112043423)
            (4.14572864328, 1.4034514177)
            (4.20100502519, 1.34877776198)
            (4.2562814071, 1.29690748936)
            (4.31155778901, 1.24766319991)
            (4.36683417091, 1.20088079613)
            (4.42211055282, 1.15640833368)
            (4.47738693473, 1.11410498408)
            (4.53266331664, 1.07384009744)
            (4.58793969855, 1.0354923544)
            (4.64321608046, 0.998948997889)
            (4.69849246237, 0.964105136325)
            (4.75376884428, 0.930863110867)
            (4.80904522619, 0.899131920198)
            (4.8643216081, 0.868826696983)
            (4.91959799001, 0.839868230868)
            (4.97487437191, 0.812182533398)
            (5.03015075382, 0.785700440765)
            (5.08542713573, 0.760357250719)
            (5.14070351764, 0.73609239038)
            (5.19597989955, 0.712849112009)
            (5.25125628146, 0.690574214141)
            (5.30653266337, 0.669217785701)
            (5.36180904528, 0.648732971009)
            (5.41708542719, 0.629075753779)
            (5.4723618091, 0.610204758379)
            (5.527638191, 0.592081066851)
            (5.58291457291, 0.57466805026)
            (5.63819095482, 0.557931213157)
            (5.69346733673, 0.54183804999)
            (5.74874371864, 0.52635791247)
            (5.80402010055, 0.511461886926)
            (5.85929648246, 0.497122680848)
            (5.91457286437, 0.483314517819)
            (5.96984924628, 0.470013040166)
            (6.02512562819, 0.457195218681)
            (6.08040201009, 0.444839268861)
            (6.135678392, 0.432924573115)
            (6.19095477391, 0.421431608491)
            (6.24623115582, 0.410341879465)
            (6.30150753773, 0.39963785541)
            (6.35678391964, 0.389302912373)
            (6.41206030155, 0.37932127884)
            (6.46733668346, 0.369677985166)
            (6.52261306537, 0.360358816417)
            (6.57788944728, 0.351350268346)
            (6.63316582919, 0.342639506283)
            (6.68844221109, 0.334214326719)
            (6.743718593, 0.326063121394)
            (6.79899497491, 0.318174843692)
            (6.85427135682, 0.310538977196)
            (6.90954773873, 0.303145506231)
            (6.96482412064, 0.29598488827)
            (7.02010050255, 0.289048028058)
            (7.07537688446, 0.282326253346)
            (7.13065326637, 0.275811292115)
            (7.18592964828, 0.269495251187)
            (7.24120603018, 0.263370596145)
            (7.29648241209, 0.257430132445)
            (7.351758794, 0.251666987665)
            (7.40703517591, 0.246074594805)
            (7.46231155782, 0.240646676567)
            (7.51758793973, 0.235377230554)
            (7.57286432164, 0.230260515324)
            (7.62814070355, 0.225291037255)
            (7.68341708546, 0.220463538149)
            (7.73869346737, 0.21577298355)
            (7.79396984928, 0.21121455171)
            (7.84924623118, 0.206783623184)
            (7.90452261309, 0.20247577099)
            (7.959798995, 0.198286751322)
            (8.01507537691, 0.194212494768)
            (8.07035175882, 0.190249098003)
            (8.12562814073, 0.186392815938)
            (8.18090452264, 0.182640054285)
            (8.23618090455, 0.17898736252)
            (8.29145728646, 0.175431427218)
            (8.34673366837, 0.171969065741)
            (8.40201005027, 0.168597220253)
            (8.45728643218, 0.165312952049)
            (8.51256281409, 0.162113436176)
            (8.567839196, 0.158995956326)
            (8.62311557791, 0.155957899994)
            (8.67839195982, 0.15299675388)
            (8.73366834173, 0.150110099521)
            (8.78894472364, 0.147295609143)
            (8.84422110555, 0.144551041715)
            (8.89949748746, 0.141874239205)
            (8.95477386937, 0.139263123014)
            (9.01005025127, 0.136715690587)
            (9.06532663318, 0.134230012184)
            (9.12060301509, 0.131804227813)
            (9.175879397, 0.129436544304)
            (9.23115577891, 0.127125232528)
            (9.28643216082, 0.12486862474)
            (9.34170854273, 0.122665112057)
            (9.39698492464, 0.120513142044)
            (9.45226130655, 0.11841121642)
            (9.50753768846, 0.116357888862)
            (9.56281407036, 0.114351762923)
            (9.61809045227, 0.112391490028)
            (9.67336683418, 0.110475767578)
            (9.72864321609, 0.108603337126)
            (9.783919598, 0.106772982644)
            (9.83919597991, 0.104983528862)
            (9.89447236182, 0.103233839685)
            (9.94974874373, 0.101522816678)
            (10.0050251256, 0.0998493976154)
            (10.0603015075, 0.0982125550986)
            (10.1155778895, 0.0966112952292)
            (10.1708542714, 0.0950446563427)
            (10.2261306533, 0.0935117077947)
            (10.2814070352, 0.0920115488001)
            (10.3366834171, 0.0905433073214)
            (10.391959799, 0.0891061390037)
            (10.4472361809, 0.0876992261553)
            (10.5025125628, 0.0863217767701)
            (10.5577889447, 0.0849730235916)
            (10.6130653266, 0.083652223215)
            (10.6683417085, 0.0823586552265)
            (10.7236180905, 0.0810916213784)
            (10.7788944724, 0.0798504447973)
            (10.8341708543, 0.0786344692245)
            (10.8894472362, 0.0774430582873)
            (10.9447236181, 0.0762755947995)
            (11.0, 0.0751314800902)
    };
    

    
    % Draw series plot
    \addplot[no markers,solid] coordinates {
            (1e-10, 1e+32)
            (0.055276382009, 592320.124623)
            (0.110552763918, 74371.7440575)
            (0.165829145827, 22462.4483174)
            (0.221105527736, 10019.0469012)
            (0.276381909645, 5814.35756384)
            (0.331658291554, 4216.63955696)
            (0.386934673463, 3696.66850072)
            (0.442211055372, 3723.16871058)
            (0.497487437281, 4073.45422993)
            (0.55276381919, 4633.96027572)
            (0.608040201099, 5331.15359194)
            (0.663316583009, 6104.86858802)
            (0.718592964918, 6898.11550836)
            (0.773869346827, 7654.4547577)
            (0.829145728736, 8319.31670436)
            (0.884422110645, 8843.38261561)
            (0.939698492554, 9186.72833455)
            (0.994974874463, 9322.67685632)
            (1.05025125637, 9240.51008725)
            (1.10552763828, 8946.4570777)
            (1.16080402019, 8462.7133142)
            (1.2160804021, 7824.6092075)
            (1.27135678401, 7076.37012707)
            (1.32663316592, 6266.13310828)
            (1.38190954783, 5440.96746102)
            (1.43718592974, 4642.58207444)
            (1.49246231164, 3904.21865213)
            (1.54773869355, 3248.97858937)
            (1.60301507546, 2689.57152541)
            (1.65829145737, 2229.25858262)
            (1.71356783928, 1863.62720344)
            (1.76884422119, 1582.78816877)
            (1.8241206031, 1373.61833436)
            (1.87939698501, 1221.76005555)
            (1.93467336692, 1113.20014071)
            (1.98994974883, 1035.3603296)
            (2.04522613073, 977.718785655)
            (2.10050251264, 932.038993556)
            (2.15577889455, 892.308732246)
            (2.21105527646, 854.493424895)
            (2.26633165837, 816.193790082)
            (2.32160804028, 776.275838944)
            (2.37688442219, 734.518505525)
            (2.4321608041, 691.304657156)
            (2.48743718601, 647.366657622)
            (2.54271356792, 603.588092387)
            (2.59798994983, 560.857802577)
            (2.65326633173, 519.969783183)
            (2.70854271364, 481.561646913)
            (2.76381909555, 446.084396749)
            (2.81909547746, 413.796671035)
            (2.87437185937, 384.777173077)
            (2.92964824128, 358.949591121)
            (2.98492462319, 336.114955431)
            (3.0402010051, 315.987087867)
            (3.09547738701, 298.227582499)
            (3.15075376892, 282.477594209)
            (3.20603015082, 268.384565377)
            (3.26130653273, 255.622836809)
            (3.31658291464, 243.907817525)
            (3.37185929655, 233.003988606)
            (3.42713567846, 222.727464646)
            (3.48241206037, 212.944125582)
            (3.53768844228, 203.564470637)
            (3.59296482419, 194.536355643)
            (3.6482412061, 185.836683655)
            (3.70351758801, 177.462957644)
            (3.75879396992, 169.425403828)
            (3.81407035182, 161.740161383)
            (3.86934673373, 154.423830381)
            (3.92462311564, 147.489489925)
            (3.97989949755, 140.944152079)
            (4.03517587946, 134.787508387)
            (4.09045226137, 129.011754422)
            (4.14572864328, 123.602240758)
            (4.20100502519, 118.538690839)
            (4.2562814071, 113.796741036)
            (4.31155778901, 109.349589327)
            (4.36683417091, 105.169580131)
            (4.42211055282, 101.229598353)
            (4.47738693473, 97.5041910226)
            (4.53266331664, 93.9703765288)
            (4.58793969855, 90.6081368644)
            (4.64321608046, 87.4006161223)
            (4.69849246237, 84.334068172)
            (4.75376884428, 81.3976082355)
            (4.80904522619, 78.582827757)
            (4.8643216081, 75.8833307088)
            (4.91959799001, 73.2942436833)
            (4.97487437191, 70.8117432471)
            (5.03015075382, 68.4326334693)
            (5.08542713573, 66.153995508)
            (5.14070351764, 63.9729206511)
            (5.19597989955, 61.8863290063)
            (5.25125628146, 59.8908686075)
            (5.30653266337, 57.9828842734)
            (5.36180904528, 56.1584421604)
            (5.41708542719, 54.4133944233)
            (5.4723618091, 52.7434684872)
            (5.527638191, 51.1443667715)
            (5.58291457291, 49.611864939)
            (5.63819095482, 48.1418994809)
            (5.69346733673, 46.7306383713)
            (5.74874371864, 45.3745313453)
            (5.80402010055, 44.0703388655)
            (5.85929648246, 42.8151408996)
            (5.91457286437, 41.606328159)
            (5.96984924628, 40.4415794324)
            (6.02512562819, 39.3188291173)
            (6.08040201009, 38.236229083)
            (6.135678392, 37.1921086618)
            (6.19095477391, 36.1849359905)
            (6.24623115582, 35.213283183)
            (6.30150753773, 34.2757970222)
            (6.35678391964, 33.3711760815)
            (6.41206030155, 32.4981544901)
            (6.46733668346, 31.6554919812)
            (6.52261306537, 30.8419694325)
            (6.57788944728, 30.0563888296)
            (6.63316582919, 29.2975764495)
            (6.68844221109, 28.5643880542)
            (6.743718593, 27.855714977)
            (6.79899497491, 27.1704901524)
            (6.85427135682, 26.5076933531)
            (6.90954773873, 25.8663551261)
            (6.96482412064, 25.2455591467)
            (7.02010050255, 24.6444429125)
            (7.07537688446, 24.0621968658)
            (7.13065326637, 23.498062161)
            (7.18592964828, 22.951327375)
            (7.24120603018, 22.4213244975)
            (7.29648241209, 21.907424543)
            (7.351758794, 21.4090330974)
            (7.40703517591, 20.9255860658)
            (7.46231155782, 20.4565458231)
            (7.51758793973, 20.0013979052)
            (7.57286432164, 19.5596483102)
            (7.62814070355, 19.1308214201)
            (7.68341708546, 18.7144585064)
            (7.73869346737, 18.310116743)
            (7.79396984928, 17.9173686305)
            (7.84924623118, 17.5358017208)
            (7.90452261309, 17.1650185355)
            (7.959798995, 16.8046365764)
            (8.01507537691, 16.4542883447)
            (8.07035175882, 16.1136213038)
            (8.12562814073, 15.7822977406)
            (8.18090452264, 15.4599945022)
            (8.23618090455, 15.1464026018)
            (8.29145728646, 14.8412267025)
            (8.34673366837, 14.5441845007)
            (8.40201005027, 14.2550060345)
            (8.45728643218, 13.9734329495)
            (8.51256281409, 13.6992177523)
            (8.567839196, 13.4321230787)
            (8.62311557791, 13.1719210011)
            (8.67839195982, 12.9183923906)
            (8.73366834173, 12.6713263473)
            (8.78894472364, 12.4305197003)
            (8.84422110555, 12.1957765801)
            (8.89949748746, 11.9669080559)
            (8.95477386937, 11.7437318302)
            (9.01005025127, 11.5260719807)
            (9.06532663318, 11.3137587381)
            (9.12060301509, 11.106628288)
            (9.175879397, 10.9045225883)
            (9.23115577891, 10.7072891932)
            (9.28643216082, 10.5147810759)
            (9.34170854273, 10.326856449)
            (9.39698492464, 10.1433785766)
            (9.45226130655, 9.96421558)
            (9.50753768846, 9.78924023684)
            (9.56281407036, 9.61832977541)
            (9.61809045227, 9.45136566729)
            (9.67336683418, 9.28823342048)
            (9.72864321609, 9.12882237609)
            (9.783919598, 8.97302551078)
            (9.83919597991, 8.820739247)
            (9.89447236182, 8.67186327229)
            (9.94974874373, 8.52630036831)
            (10.0050251256, 8.38395624963)
            (10.0603015075, 8.24473941196)
            (10.1155778895, 8.10856098899)
            (10.1708542714, 7.97533461651)
            (10.2261306533, 7.84497630283)
            (10.2814070352, 7.71740430401)
            (10.3366834171, 7.59253900267)
            (10.391959799, 7.47030278948)
            (10.4472361809, 7.35061994621)
            (10.5025125628, 7.23341652997)
            (10.5577889447, 7.11862025816)
            (10.6130653266, 7.00616039397)
            (10.6683417085, 6.89596763267)
            (10.7236180905, 6.78797398883)
            (10.7788944724, 6.68211268508)
            (10.8341708543, 6.57831804274)
            (10.8894472362, 6.47652537508)
            (10.9447236181, 6.37667088382)
            (11.0, 6.27869155948)
    };
    


                
            \end{groupplot}
        \end{tikzpicture}
    };
    \node[below] at (plot.south) { Aantal deeltjes };
    \node[above, rotate=90] at (plot.west) { Aantal events };
\end{tikzpicture}
\end{sansmath}
\caption{Histogram van de componenten van het signaal uit een \hisparc
detector.  Bovenste plot: er kunnen 1, 2, 3, of meerdere deeltjes door een
detector gaan.  Middelste plot: doordat het energieverlies een kansproces
is, evenals het aantal fotonen dat uiteindelijk bij de \pmt uitkomt, is
het signaal van bv. 2 deeltjes soms iets kleiner, en soms iets groter.  De
componenten zijn dus verbreedt.  Verder is een afvallende distributie
toegevoegd die de bijdrage van fotonen laat zien.  Onderste plot: het
totale signaal (zwart) is een optelsom van de verschillende
deeltjescomponenten (grijs).}
\label{fig:spectrum_componenten}
\end{figure}


\subsection{Drempels en triggervoorwaarden}

Het is in een scintillatorplaat nooit volmaakt donker.  Er zal altijd een
klein aantal fotonen op de \pmt vallen.  Verder staat er zó veel spanning
op de \pmt dat er soms spontaan een elektron van een dynode wordt
afgetrokken wat vervolgens leidt tot een klein, maar versterkt, signaal.
Dit misleidende `signaal' noemen we \emph{ruis}.  Door een
\emph{drempelwaarde} in te stellen is het mogelijk om de ruis te negeren.
Alle signalen met een piekwaarde \emph{lager} dan de drempelwaarde worden
genegeerd.

Laagenergetische air showers komen veel vaker voor dan hoogenergetische
showers.  Bij laagenergetische showers bereiken maar zeer weinig deeltjes
de grond.  Deze showers willen we het liefste niet detecteren, aangezien
we te weinig informatie kunnen krijgen om de energie of richting van de
shower te reconstrueren.  Helaas zijn deze showers zó talrijk dat een
typische \hisparc detector honderden deeltjes per seconde meet.  Dit
betekent dat het moeilijk wordt om hoogenergetische showers te herkennen.

Door meer dan één detector te gebruiken in een station is het mogelijk om
deze achtergrond te negeren.  De kans dat een laagenergetische shower meer
dan één detector raakt is zeer klein.  Voor een hoogenergetische air
shower is dat echter geen probleem.  De gemiddelde deeltjesdichtheid op de
grond is dan zó hoog dat de kans dat er deeltjes door meer dan één
detector gaan zeer groot is.  We kunnen dus stellen dat als twee of meer
detectoren tegelijkertijd deeltjes meten er zeer waarschijnlijk sprake is
van een hoogenergetische air shower.  De \hisparc elektronica reageert
alleen indien dit het geval is.  Dit heet een \emph{triggervoorwaarde}.

De standaard triggervoorwaarde voor een twee-plaats opstelling is
\emph{minimaal twee detectoren met een piekwaarde boven de
\SI{30}{\milli\volt}} en voor een vier-plaats opstelling: \emph{minimaal
twee detectoren met een piek\-waarde boven de \SI{70}{\milli\volt} óf
minimaal drie detectoren met een piekwaarde boven de
\SI{30}{\milli\volt}}.


\section{Inregelen \pmts}

De correcte afstelling van de hoogspanning van de \pmt kan worden
gecontroleerd door een pulshoogtehistogram te bekijken
(\figref{fig:afstelling_pmt}).  Hoe hoger de spanning, hoe verder het
histogram naar rechts is verschoven, omdat alle signalen meer worden
versterkt.  We proberen om de bult van \SI{1}{\mip} bij een spanning van
ongeveer \SI{200}{\milli\volt} te leggen.  In de linker plot is de
spanning te laag, in de rechter plot te hoog.  De middelste plot laat het
goede beeld zien.

\begin{figure}
\centering
\pgfkeys{/artist/width/.initial=.37\linewidth}
% \usepackage{tikz}
% \usetikzlibrary{arrows,pgfplots.groupplots}
% \usepackage{pgfplots}
% \pgfplotsset{compat=1.3}
% \usepackage[detect-family]{siunitx}
% \usepackage[eulergreek]{sansmath}
% \sisetup{text-sf=\sansmath}
% \usepackage{relsize}
%
\pgfkeysifdefined{/artist/width}
    {\pgfkeysgetvalue{/artist/width}{\defaultwidth}}
    {\def\defaultwidth{ .4\linewidth }}
%
%
\begin{sansmath}
\begin{tikzpicture}[font=\sffamily]
\node[inner sep=0pt] (plot) {
    \begin{tikzpicture}[
            inner sep=.3333em,
            font=\sffamily,
            every pin/.style={inner sep=2pt, font={\sffamily\smaller}},
            every label/.style={inner sep=2pt, font={\sffamily\smaller}},
            every pin edge/.style={<-, >=stealth', shorten <=2pt},
            pin distance=2.5ex,
        ]
        \begin{groupplot}[
                xmode=normal,
                ymode=normal,
                width=\defaultwidth,
                %
                xmin={ 0 },
                xmax={ 1000 },
                ymin={ 1 },
                ymax={  },
                %
                group style={rows=1,columns=3,
                             horizontal sep=4pt, vertical sep=4pt},
                %
                tick align=outside,
                max space between ticks=40,
                every tick/.style={},
                axis on top,
                %
                xtick=\empty, ytick=\empty,
                scaled ticks=false,
            ]
            
                
                \nextgroupplot[
                    % Default: empty ticks all round the border of the
                    % multiplot
                            xtick={  },
                            xtick pos=both,
                            xticklabel=\empty,
                        xticklabel={},
                    axis equal=false,
                    %
                    title={  },
                    xlabel={  },
                    ylabel={  },
                ]

                

                




    
    % Draw series plot
    \addplot[no markers,solid] coordinates {
        (0.0, 1.0)
        (1.93467336683, 1.0)
        (3.86934673367, 1.0)
        (5.8040201005, 1.0)
        (7.73869346734, 1.0)
        (9.67336683417, 1.0)
        (11.608040201, 1.0)
        (13.5427135678, 1.0)
        (15.4773869347, 1.0)
        (17.4120603015, 1.0)
        (19.3467336683, 1.0)
        (21.2814070352, 1.0)
        (23.216080402, 1.0)
        (25.1507537688, 1.0)
        (27.0854271357, 1.0)
        (29.0201005025, 1.0)
        (30.9547738693, 8843.38261488)
        (32.8894472362, 9186.72833414)
        (34.824120603, 9322.67685628)
        (36.7587939698, 9240.51008757)
        (38.6934673367, 8946.45707834)
        (40.6281407035, 8462.71331512)
        (42.5628140704, 7824.60920863)
        (44.4974874372, 7076.37012833)
        (46.432160804, 6266.1331096)
        (48.3668341709, 5440.96746231)
        (50.3015075377, 4642.58207566)
        (52.2361809045, 3904.21865322)
        (54.1708542714, 3248.97859032)
        (56.1055276382, 2689.5715262)
        (58.040201005, 2229.25858325)
        (59.9748743719, 1863.62720393)
        (61.9095477387, 1582.78816914)
        (63.8442211055, 1373.61833463)
        (65.7788944724, 1221.76005574)
        (67.7135678392, 1113.20014084)
        (69.648241206, 1035.36032969)
        (71.5829145729, 977.718785729)
        (73.5175879397, 932.038993617)
        (75.4522613065, 892.308732302)
        (77.3869346734, 854.493424949)
        (79.3216080402, 816.193790138)
        (81.256281407, 776.275839002)
        (83.1909547739, 734.518505585)
        (85.1256281407, 691.304657217)
        (87.0603015075, 647.366657683)
        (88.9949748744, 603.588092447)
        (90.9296482412, 560.857802635)
        (92.864321608, 519.969783238)
        (94.7989949749, 481.561646964)
        (96.7336683417, 446.084396795)
        (98.6683417085, 413.796671076)
        (100.603015075, 384.777173114)
        (102.537688442, 358.949591154)
        (104.472361809, 336.114955459)
        (106.407035176, 315.987087892)
        (108.341708543, 298.227582521)
        (110.27638191, 282.477594228)
        (112.211055276, 268.384565394)
        (114.145728643, 255.622836825)
        (116.08040201, 243.90781754)
        (118.015075377, 233.003988619)
        (119.949748744, 222.727464658)
        (121.884422111, 212.944125594)
        (123.819095477, 203.564470648)
        (125.753768844, 194.536355653)
        (127.688442211, 185.836683665)
        (129.623115578, 177.462957654)
        (131.557788945, 169.425403837)
        (133.492462312, 161.740161392)
        (135.427135678, 154.42383039)
        (137.361809045, 147.489489933)
        (139.296482412, 140.944152086)
        (141.231155779, 134.787508394)
        (143.165829146, 129.011754428)
        (145.100502513, 123.602240763)
        (147.035175879, 118.538690844)
        (148.969849246, 113.796741041)
        (150.904522613, 109.349589332)
        (152.83919598, 105.169580135)
        (154.773869347, 101.229598357)
        (156.708542714, 97.5041910265)
        (158.64321608, 93.9703765325)
        (160.577889447, 90.6081368679)
        (162.512562814, 87.4006161256)
        (164.447236181, 84.3340681752)
        (166.381909548, 81.3976082385)
        (168.316582915, 78.5828277598)
        (170.251256281, 75.8833307115)
        (172.185929648, 73.2942436859)
        (174.120603015, 70.8117432495)
        (176.055276382, 68.4326334716)
        (177.989949749, 66.1539955102)
        (179.924623116, 63.9729206531)
        (181.859296482, 61.8863290083)
        (183.793969849, 59.8908686093)
        (185.728643216, 57.9828842752)
        (187.663316583, 56.158442162)
        (189.59798995, 54.4133944249)
        (191.532663317, 52.7434684887)
        (193.467336683, 51.1443667729)
        (195.40201005, 49.6118649403)
        (197.336683417, 48.1418994821)
        (199.271356784, 46.7306383725)
        (201.206030151, 45.3745313465)
        (203.140703518, 44.0703388666)
        (205.075376884, 42.8151409006)
        (207.010050251, 41.60632816)
        (208.944723618, 40.4415794333)
        (210.879396985, 39.3188291182)
        (212.814070352, 38.2362290839)
        (214.748743719, 37.1921086626)
        (216.683417085, 36.1849359912)
        (218.618090452, 35.2132831837)
        (220.552763819, 34.2757970229)
        (222.487437186, 33.3711760822)
        (224.422110553, 32.4981544907)
        (226.35678392, 31.6554919818)
        (228.291457286, 30.8419694331)
        (230.226130653, 30.0563888301)
        (232.16080402, 29.29757645)
        (234.095477387, 28.5643880547)
        (236.030150754, 27.8557149775)
        (237.964824121, 27.1704901529)
        (239.899497487, 26.5076933536)
        (241.834170854, 25.8663551265)
        (243.768844221, 25.2455591471)
        (245.703517588, 24.6444429129)
        (247.638190955, 24.0621968662)
        (249.572864322, 23.4980621614)
        (251.507537688, 22.9513273753)
        (253.442211055, 22.4213244978)
        (255.376884422, 21.9074245433)
        (257.311557789, 21.4090330977)
        (259.246231156, 20.9255860661)
        (261.180904523, 20.4565458233)
        (263.115577889, 20.0013979055)
        (265.050251256, 19.5596483104)
        (266.984924623, 19.1308214203)
        (268.91959799, 18.7144585066)
        (270.854271357, 18.3101167433)
        (272.788944724, 17.9173686307)
        (274.72361809, 17.535801721)
        (276.658291457, 17.1650185357)
        (278.592964824, 16.8046365765)
        (280.527638191, 16.4542883449)
        (282.462311558, 16.113621304)
        (284.396984925, 15.7822977408)
        (286.331658291, 15.4599945024)
        (288.266331658, 15.1464026019)
        (290.201005025, 14.8412267027)
        (292.135678392, 14.5441845009)
        (294.070351759, 14.2550060346)
        (296.005025126, 13.9734329496)
        (297.939698492, 13.6992177524)
        (299.874371859, 13.4321230789)
        (301.809045226, 13.1719210012)
        (303.743718593, 12.9183923907)
        (305.67839196, 12.6713263474)
        (307.613065327, 12.4305197004)
        (309.547738693, 12.1957765802)
        (311.48241206, 11.966908056)
        (313.417085427, 11.7437318303)
        (315.351758794, 11.5260719808)
        (317.286432161, 11.3137587382)
        (319.221105528, 11.106628288)
        (321.155778894, 10.9045225884)
        (323.090452261, 10.7072891932)
        (325.025125628, 10.5147810759)
        (326.959798995, 10.326856449)
        (328.894472362, 10.1433785766)
        (330.829145729, 9.96421558005)
        (332.763819095, 9.78924023688)
        (334.698492462, 9.61832977545)
        (336.633165829, 9.45136566733)
        (338.567839196, 9.28823342052)
        (340.502512563, 9.12882237613)
        (342.43718593, 8.97302551081)
        (344.371859296, 8.82073924702)
        (346.306532663, 8.67186327232)
        (348.24120603, 8.52630036834)
        (350.175879397, 8.38395624965)
        (352.110552764, 8.24473941199)
        (354.045226131, 8.10856098901)
        (355.979899497, 7.97533461653)
        (357.914572864, 7.84497630285)
        (359.849246231, 7.71740430402)
        (361.783919598, 7.59253900269)
        (363.718592965, 7.47030278949)
        (365.653266332, 7.35061994622)
        (367.587939698, 7.23341652998)
        (369.522613065, 7.11862025817)
        (371.457286432, 7.00616039398)
        (373.391959799, 6.89596763267)
        (375.326633166, 6.78797398884)
        (377.261306533, 6.68211268509)
        (379.195979899, 6.57831804274)
        (381.130653266, 6.47652537508)
        (383.065326633, 6.37667088382)
        (385.0, 6.27869155948)
    };





    \draw[gray]
        ({rel axis cs:0, 0} -| {axis cs:200, 0 }) --
        ({rel axis cs:1, 1} -| {axis cs:200, 0 });





    \node[,
          below left=2pt
        ]
        at (rel axis cs:1,
            1)
        { $V_\mathrm{PMT}$ te laag };


            
                
                \nextgroupplot[
                    % Default: empty ticks all round the border of the
                    % multiplot
                            xtick={  },
                            xtick pos=both,
                            xticklabel=\empty,
                        xticklabel={},
                        xticklabel pos=right,
                    axis equal=false,
                    %
                    title={  },
                    xlabel={  },
                    ylabel={  },
                ]

                

                




    
    % Draw series plot
    \addplot[no markers,solid] coordinates {
        (0.0, 1.0)
        (11.0552763819, 1.0)
        (22.1105527638, 1.0)
        (33.1658291457, 22462.4483561)
        (44.2211055276, 10019.046913)
        (55.2763819095, 5814.35756823)
        (66.3316582915, 4216.63955858)
        (77.3869346734, 3696.66850105)
        (88.4422110553, 3723.16871021)
        (99.4974874372, 4073.45422912)
        (110.552763819, 4633.96027462)
        (121.608040201, 5331.15359067)
        (132.663316583, 6104.86858667)
        (143.718592965, 6898.11550703)
        (154.773869347, 7654.45475649)
        (165.829145729, 8319.31670336)
        (176.884422111, 8843.38261488)
        (187.939698492, 9186.72833414)
        (198.994974874, 9322.67685628)
        (210.050251256, 9240.51008757)
        (221.105527638, 8946.45707834)
        (232.16080402, 8462.71331512)
        (243.216080402, 7824.60920863)
        (254.271356784, 7076.37012833)
        (265.326633166, 6266.1331096)
        (276.381909548, 5440.96746231)
        (287.43718593, 4642.58207566)
        (298.492462312, 3904.21865322)
        (309.547738693, 3248.97859032)
        (320.603015075, 2689.5715262)
        (331.658291457, 2229.25858325)
        (342.713567839, 1863.62720393)
        (353.768844221, 1582.78816914)
        (364.824120603, 1373.61833463)
        (375.879396985, 1221.76005574)
        (386.934673367, 1113.20014084)
        (397.989949749, 1035.36032969)
        (409.045226131, 977.718785729)
        (420.100502513, 932.038993617)
        (431.155778894, 892.308732302)
        (442.211055276, 854.493424949)
        (453.266331658, 816.193790138)
        (464.32160804, 776.275839002)
        (475.376884422, 734.518505585)
        (486.432160804, 691.304657217)
        (497.487437186, 647.366657683)
        (508.542713568, 603.588092447)
        (519.59798995, 560.857802635)
        (530.653266332, 519.969783238)
        (541.708542714, 481.561646964)
        (552.763819095, 446.084396795)
        (563.819095477, 413.796671076)
        (574.874371859, 384.777173114)
        (585.929648241, 358.949591154)
        (596.984924623, 336.114955459)
        (608.040201005, 315.987087892)
        (619.095477387, 298.227582521)
        (630.150753769, 282.477594228)
        (641.206030151, 268.384565394)
        (652.261306533, 255.622836825)
        (663.316582915, 243.90781754)
        (674.371859296, 233.003988619)
        (685.427135678, 222.727464658)
        (696.48241206, 212.944125594)
        (707.537688442, 203.564470648)
        (718.592964824, 194.536355653)
        (729.648241206, 185.836683665)
        (740.703517588, 177.462957654)
        (751.75879397, 169.425403837)
        (762.814070352, 161.740161392)
        (773.869346734, 154.42383039)
        (784.924623116, 147.489489933)
        (795.979899497, 140.944152086)
        (807.035175879, 134.787508394)
        (818.090452261, 129.011754428)
        (829.145728643, 123.602240763)
        (840.201005025, 118.538690844)
        (851.256281407, 113.796741041)
        (862.311557789, 109.349589332)
        (873.366834171, 105.169580135)
        (884.422110553, 101.229598357)
        (895.477386935, 97.5041910265)
        (906.532663317, 93.9703765325)
        (917.587939698, 90.6081368679)
        (928.64321608, 87.4006161256)
        (939.698492462, 84.3340681752)
        (950.753768844, 81.3976082385)
        (961.809045226, 78.5828277598)
        (972.864321608, 75.8833307115)
        (983.91959799, 73.2942436859)
        (994.974874372, 70.8117432495)
        (1006.03015075, 68.4326334716)
        (1017.08542714, 66.1539955102)
        (1028.14070352, 63.9729206531)
        (1039.1959799, 61.8863290083)
        (1050.25125628, 59.8908686093)
        (1061.30653266, 57.9828842752)
        (1072.36180905, 56.158442162)
        (1083.41708543, 54.4133944249)
        (1094.47236181, 52.7434684887)
        (1105.52763819, 51.1443667729)
        (1116.58291457, 49.6118649403)
        (1127.63819095, 48.1418994821)
        (1138.69346734, 46.7306383725)
        (1149.74874372, 45.3745313465)
        (1160.8040201, 44.0703388666)
        (1171.85929648, 42.8151409006)
        (1182.91457286, 41.60632816)
        (1193.96984925, 40.4415794333)
        (1205.02512563, 39.3188291182)
        (1216.08040201, 38.2362290839)
        (1227.13567839, 37.1921086626)
        (1238.19095477, 36.1849359912)
        (1249.24623116, 35.2132831837)
        (1260.30150754, 34.2757970229)
        (1271.35678392, 33.3711760822)
        (1282.4120603, 32.4981544907)
        (1293.46733668, 31.6554919818)
        (1304.52261307, 30.8419694331)
        (1315.57788945, 30.0563888301)
        (1326.63316583, 29.29757645)
        (1337.68844221, 28.5643880547)
        (1348.74371859, 27.8557149775)
        (1359.79899497, 27.1704901529)
        (1370.85427136, 26.5076933536)
        (1381.90954774, 25.8663551265)
        (1392.96482412, 25.2455591471)
        (1404.0201005, 24.6444429129)
        (1415.07537688, 24.0621968662)
        (1426.13065327, 23.4980621614)
        (1437.18592965, 22.9513273753)
        (1448.24120603, 22.4213244978)
        (1459.29648241, 21.9074245433)
        (1470.35175879, 21.4090330977)
        (1481.40703518, 20.9255860661)
        (1492.46231156, 20.4565458233)
        (1503.51758794, 20.0013979055)
        (1514.57286432, 19.5596483104)
        (1525.6281407, 19.1308214203)
        (1536.68341709, 18.7144585066)
        (1547.73869347, 18.3101167433)
        (1558.79396985, 17.9173686307)
        (1569.84924623, 17.535801721)
        (1580.90452261, 17.1650185357)
        (1591.95979899, 16.8046365765)
        (1603.01507538, 16.4542883449)
        (1614.07035176, 16.113621304)
        (1625.12562814, 15.7822977408)
        (1636.18090452, 15.4599945024)
        (1647.2361809, 15.1464026019)
        (1658.29145729, 14.8412267027)
        (1669.34673367, 14.5441845009)
        (1680.40201005, 14.2550060346)
        (1691.45728643, 13.9734329496)
        (1702.51256281, 13.6992177524)
        (1713.5678392, 13.4321230789)
        (1724.62311558, 13.1719210012)
        (1735.67839196, 12.9183923907)
        (1746.73366834, 12.6713263474)
        (1757.78894472, 12.4305197004)
        (1768.84422111, 12.1957765802)
        (1779.89949749, 11.966908056)
        (1790.95477387, 11.7437318303)
        (1802.01005025, 11.5260719808)
        (1813.06532663, 11.3137587382)
        (1824.12060302, 11.106628288)
        (1835.1758794, 10.9045225884)
        (1846.23115578, 10.7072891932)
        (1857.28643216, 10.5147810759)
        (1868.34170854, 10.326856449)
        (1879.39698492, 10.1433785766)
        (1890.45226131, 9.96421558005)
        (1901.50753769, 9.78924023688)
        (1912.56281407, 9.61832977545)
        (1923.61809045, 9.45136566733)
        (1934.67336683, 9.28823342052)
        (1945.72864322, 9.12882237613)
        (1956.7839196, 8.97302551081)
        (1967.83919598, 8.82073924702)
        (1978.89447236, 8.67186327232)
        (1989.94974874, 8.52630036834)
        (2001.00502513, 8.38395624965)
        (2012.06030151, 8.24473941199)
        (2023.11557789, 8.10856098901)
        (2034.17085427, 7.97533461653)
        (2045.22613065, 7.84497630285)
        (2056.28140704, 7.71740430402)
        (2067.33668342, 7.59253900269)
        (2078.3919598, 7.47030278949)
        (2089.44723618, 7.35061994622)
        (2100.50251256, 7.23341652998)
        (2111.55778894, 7.11862025817)
        (2122.61306533, 7.00616039398)
        (2133.66834171, 6.89596763267)
        (2144.72361809, 6.78797398884)
        (2155.77889447, 6.68211268509)
        (2166.83417085, 6.57831804274)
        (2177.88944724, 6.47652537508)
        (2188.94472362, 6.37667088382)
        (2200.0, 6.27869155948)
    };





    \draw[gray]
        ({rel axis cs:0, 0} -| {axis cs:200, 0 }) --
        ({rel axis cs:1, 1} -| {axis cs:200, 0 });





    \node[,
          below left=2pt
        ]
        at (rel axis cs:1,
            1)
        { $V_\mathrm{PMT}$ correct };


            
                
                \nextgroupplot[
                    % Default: empty ticks all round the border of the
                    % multiplot
                            xtick={  },
                            xtick pos=both,
                            xticklabel=\empty,
                        xticklabel={},
                    axis equal=false,
                    %
                    title={  },
                    xlabel={  },
                    ylabel={  },
                ]

                

                




    
    % Draw series plot
    \addplot[no markers,solid] coordinates {
        (0.0, 1.0)
        (22.1105527638, 1.0)
        (44.2211055276, 74371.7442561)
        (66.3316582915, 22462.4483561)
        (88.4422110553, 10019.046913)
        (110.552763819, 5814.35756823)
        (132.663316583, 4216.63955858)
        (154.773869347, 3696.66850105)
        (176.884422111, 3723.16871021)
        (198.994974874, 4073.45422912)
        (221.105527638, 4633.96027462)
        (243.216080402, 5331.15359067)
        (265.326633166, 6104.86858667)
        (287.43718593, 6898.11550703)
        (309.547738693, 7654.45475649)
        (331.658291457, 8319.31670336)
        (353.768844221, 8843.38261488)
        (375.879396985, 9186.72833414)
        (397.989949749, 9322.67685628)
        (420.100502513, 9240.51008757)
        (442.211055276, 8946.45707834)
        (464.32160804, 8462.71331512)
        (486.432160804, 7824.60920863)
        (508.542713568, 7076.37012833)
        (530.653266332, 6266.1331096)
        (552.763819095, 5440.96746231)
        (574.874371859, 4642.58207566)
        (596.984924623, 3904.21865322)
        (619.095477387, 3248.97859032)
        (641.206030151, 2689.5715262)
        (663.316582915, 2229.25858325)
        (685.427135678, 1863.62720393)
        (707.537688442, 1582.78816914)
        (729.648241206, 1373.61833463)
        (751.75879397, 1221.76005574)
        (773.869346734, 1113.20014084)
        (795.979899497, 1035.36032969)
        (818.090452261, 977.718785729)
        (840.201005025, 932.038993617)
        (862.311557789, 892.308732302)
        (884.422110553, 854.493424949)
        (906.532663317, 816.193790138)
        (928.64321608, 776.275839002)
        (950.753768844, 734.518505585)
        (972.864321608, 691.304657217)
        (994.974874372, 647.366657683)
        (1017.08542714, 603.588092447)
        (1039.1959799, 560.857802635)
        (1061.30653266, 519.969783238)
        (1083.41708543, 481.561646964)
        (1105.52763819, 446.084396795)
        (1127.63819095, 413.796671076)
        (1149.74874372, 384.777173114)
        (1171.85929648, 358.949591154)
        (1193.96984925, 336.114955459)
        (1216.08040201, 315.987087892)
        (1238.19095477, 298.227582521)
        (1260.30150754, 282.477594228)
        (1282.4120603, 268.384565394)
        (1304.52261307, 255.622836825)
        (1326.63316583, 243.90781754)
        (1348.74371859, 233.003988619)
        (1370.85427136, 222.727464658)
        (1392.96482412, 212.944125594)
        (1415.07537688, 203.564470648)
        (1437.18592965, 194.536355653)
        (1459.29648241, 185.836683665)
        (1481.40703518, 177.462957654)
        (1503.51758794, 169.425403837)
        (1525.6281407, 161.740161392)
        (1547.73869347, 154.42383039)
        (1569.84924623, 147.489489933)
        (1591.95979899, 140.944152086)
        (1614.07035176, 134.787508394)
        (1636.18090452, 129.011754428)
        (1658.29145729, 123.602240763)
        (1680.40201005, 118.538690844)
        (1702.51256281, 113.796741041)
        (1724.62311558, 109.349589332)
        (1746.73366834, 105.169580135)
        (1768.84422111, 101.229598357)
        (1790.95477387, 97.5041910265)
        (1813.06532663, 93.9703765325)
        (1835.1758794, 90.6081368679)
        (1857.28643216, 87.4006161256)
        (1879.39698492, 84.3340681752)
        (1901.50753769, 81.3976082385)
        (1923.61809045, 78.5828277598)
        (1945.72864322, 75.8833307115)
        (1967.83919598, 73.2942436859)
        (1989.94974874, 70.8117432495)
        (2012.06030151, 68.4326334716)
        (2034.17085427, 66.1539955102)
        (2056.28140704, 63.9729206531)
        (2078.3919598, 61.8863290083)
        (2100.50251256, 59.8908686093)
        (2122.61306533, 57.9828842752)
        (2144.72361809, 56.158442162)
        (2166.83417085, 54.4133944249)
        (2188.94472362, 52.7434684887)
        (2211.05527638, 51.1443667729)
        (2233.16582915, 49.6118649403)
        (2255.27638191, 48.1418994821)
        (2277.38693467, 46.7306383725)
        (2299.49748744, 45.3745313465)
        (2321.6080402, 44.0703388666)
        (2343.71859296, 42.8151409006)
        (2365.82914573, 41.60632816)
        (2387.93969849, 40.4415794333)
        (2410.05025126, 39.3188291182)
        (2432.16080402, 38.2362290839)
        (2454.27135678, 37.1921086626)
        (2476.38190955, 36.1849359912)
        (2498.49246231, 35.2132831837)
        (2520.60301508, 34.2757970229)
        (2542.71356784, 33.3711760822)
        (2564.8241206, 32.4981544907)
        (2586.93467337, 31.6554919818)
        (2609.04522613, 30.8419694331)
        (2631.15577889, 30.0563888301)
        (2653.26633166, 29.29757645)
        (2675.37688442, 28.5643880547)
        (2697.48743719, 27.8557149775)
        (2719.59798995, 27.1704901529)
        (2741.70854271, 26.5076933536)
        (2763.81909548, 25.8663551265)
        (2785.92964824, 25.2455591471)
        (2808.04020101, 24.6444429129)
        (2830.15075377, 24.0621968662)
        (2852.26130653, 23.4980621614)
        (2874.3718593, 22.9513273753)
        (2896.48241206, 22.4213244978)
        (2918.59296482, 21.9074245433)
        (2940.70351759, 21.4090330977)
        (2962.81407035, 20.9255860661)
        (2984.92462312, 20.4565458233)
        (3007.03517588, 20.0013979055)
        (3029.14572864, 19.5596483104)
        (3051.25628141, 19.1308214203)
        (3073.36683417, 18.7144585066)
        (3095.47738693, 18.3101167433)
        (3117.5879397, 17.9173686307)
        (3139.69849246, 17.535801721)
        (3161.80904523, 17.1650185357)
        (3183.91959799, 16.8046365765)
        (3206.03015075, 16.4542883449)
        (3228.14070352, 16.113621304)
        (3250.25125628, 15.7822977408)
        (3272.36180905, 15.4599945024)
        (3294.47236181, 15.1464026019)
        (3316.58291457, 14.8412267027)
        (3338.69346734, 14.5441845009)
        (3360.8040201, 14.2550060346)
        (3382.91457286, 13.9734329496)
        (3405.02512563, 13.6992177524)
        (3427.13567839, 13.4321230789)
        (3449.24623116, 13.1719210012)
        (3471.35678392, 12.9183923907)
        (3493.46733668, 12.6713263474)
        (3515.57788945, 12.4305197004)
        (3537.68844221, 12.1957765802)
        (3559.79899497, 11.966908056)
        (3581.90954774, 11.7437318303)
        (3604.0201005, 11.5260719808)
        (3626.13065327, 11.3137587382)
        (3648.24120603, 11.106628288)
        (3670.35175879, 10.9045225884)
        (3692.46231156, 10.7072891932)
        (3714.57286432, 10.5147810759)
        (3736.68341709, 10.326856449)
        (3758.79396985, 10.1433785766)
        (3780.90452261, 9.96421558005)
        (3803.01507538, 9.78924023688)
        (3825.12562814, 9.61832977545)
        (3847.2361809, 9.45136566733)
        (3869.34673367, 9.28823342052)
        (3891.45728643, 9.12882237613)
        (3913.5678392, 8.97302551081)
        (3935.67839196, 8.82073924702)
        (3957.78894472, 8.67186327232)
        (3979.89949749, 8.52630036834)
        (4002.01005025, 8.38395624965)
        (4024.12060302, 8.24473941199)
        (4046.23115578, 8.10856098901)
        (4068.34170854, 7.97533461653)
        (4090.45226131, 7.84497630285)
        (4112.56281407, 7.71740430402)
        (4134.67336683, 7.59253900269)
        (4156.7839196, 7.47030278949)
        (4178.89447236, 7.35061994622)
        (4201.00502513, 7.23341652998)
        (4223.11557789, 7.11862025817)
        (4245.22613065, 7.00616039398)
        (4267.33668342, 6.89596763267)
        (4289.44723618, 6.78797398884)
        (4311.55778894, 6.68211268509)
        (4333.66834171, 6.57831804274)
        (4355.77889447, 6.47652537508)
        (4377.88944724, 6.37667088382)
        (4400.0, 6.27869155948)
    };





    \draw[gray]
        ({rel axis cs:0, 0} -| {axis cs:200, 0 }) --
        ({rel axis cs:1, 1} -| {axis cs:200, 0 });





    \node[,
          below left=2pt
        ]
        at (rel axis cs:1,
            1)
        { $V_\mathrm{PMT}$ te hoog };


            
        \end{groupplot}
        \end{tikzpicture}
    };
    \node[below] at (plot.south) { Pulseheight [\si{\milli\volt}] };
    \node[above, rotate=90] at (plot.west) { Counts };
    \end{tikzpicture}
\end{sansmath}
\caption{De correcte afstelling van de hoogspanning van de \pmt kan worden
gecontroleerd door een pulshoogtehistogram te bekijken.  Hoe hoger de
spanning, hoe verder het histogram naar rechts is verschoven.  We proberen
om de bult van \SI{1}{\mip} bij een spanning van ongeveer
\SI{200}{\milli\volt} te leggen.  In de linker plot is de spanning te
laag, in de rechter plot te hoog.  De middelste plot laat het goede beeld
zien.}
\label{fig:afstelling_pmt}
\end{figure}

We kunnen de spanning op een snelle manier goed instellen.  We maken dan
gebruik van een aantal \emph{tellers} in de \hisparc software.  Deze
tellers laten zien hoeveel pulsen boven de drempelwaardes uitkomen.  Dit
doen ze per detector, onafhankelijk van of er daadwerkelijk getriggerd
wordt.  Dit betekent dat als slechts één detector een signaal hoger dan
\SI{30}{\milli\volt} meet, dit meegeteld wordt.

Aangezien de vorm van het signaal van \hisparc vrijwel gelijk is voor alle
detectoren kunnen we de tellers gebruiken om te controleren of een
detector goed is ingesteld.  We hebben daartoe één \hisparc detector met
de hand goed afgesteld en de tellerwaardes genoteerd.  Als een andere
detector op diezelfde waardes wordt afgesteld, blijkt de vorm van het
signaal inderdaad goed overeen te komen en is deze detector óók goed
afgesteld.  De procedure is dan als volgt:

\begin{enumerate}
\item Allereerst: stop de \daq modus (\figref{fig:stopdaq}).
\item Onthoudt goed: klik na iedere aanpassing van een instelling in de
software op \emph{apply settings}! (\figref{fig:applysettings})
\item Controleer dat de drempelwaardes zijn afgesteld op
\SI{30}{\milli\volt} voor de \emph{low threshold} en \SI{70}{\milli\volt}
voor de \emph{high threshold} (\figref{fig:drempels}).
\item Stel de hoogspanning in op \SI{300}{\volt} (\figref{fig:spanningen}).
\item Het aantal singles kun je meten door over te schakelen naar het
tabblad \emph{statistics} (\figref{fig:tabstatistics}).
\item Verhoog de spanning nu zodanig dat het aantal singles voor de
\emph{high threshold} ongeveer 120 is.  Meet minstens \SI{30}{\second}.
\item Als de spanning goed is ingesteld, controleer dan nogmaals het
aantal singles door minstens \SI{120}{\second} te meten
(\figref{fig:counters}).
\item Pas de spanning eventueel nog aan in heel kleine stapjes.  De
software kan geen kleiner stapjes maken dan ongeveer \SI{4}{\volt}.  Meet
daarna minstens \emph{twee keer} en gebruik alleen de laatste meting.  De
\pmts hebben namelijk even tijd nodig om `af te koelen' of `op te warmen'.
\item Controleer dat het aantal singles voor de \emph{low threshold}
grofweg uitkomt op 250.  Het mag minder dan een factor 2 afwijken.  Wijkt
het méér af, neem dan contact op met de \hisparc clustercoördinator
(\figref{fig:counters}).
\item Schakel over naar \daq mode (\figref{fig:startdaq}).
\item Er wordt gevraagd om de nieuwe instellingen te bewaren.  Klik op
\emph{Yes} (\figref{fig:savesettings}).
\item Laat de detector een tijdje meten (ongeveer \SI{30}{\minute}) en
controleer dat de \SI{1}{\mip} bult uitkomt bij ongeveer
\SIrange{150}{200}{\milli\volt} (\figref{fig:histogram}).
\end{enumerate}

\begin{figure}
\centering
\subfloat[]{\includegraphics[width=.3\linewidth]{screenshot-stopdaq}
  \label{fig:stopdaq}}
\hfill
\subfloat[]{\includegraphics[width=.3\linewidth]{screenshot-applysettings}
  \label{fig:applysettings}}
\hfill
\subfloat[]{\includegraphics[width=.3\linewidth]{screenshot-drempels}
  \label{fig:drempels}}

\vspace{1em}

\subfloat[]{\includegraphics[width=.3\linewidth]{screenshot-spanningen}
  \label{fig:spanningen}}
\hfill
\subfloat[]{\includegraphics[width=.3\linewidth]{screenshot-tabstatistics}
  \label{fig:tabstatistics}}
\hfill
\subfloat[]{\includegraphics[width=.3\linewidth]{screenshot-counters}
  \label{fig:counters}}

\vspace{1em}

\subfloat[]{\includegraphics[width=.3\linewidth]{start-daq}
  \label{fig:startdaq}}
\hfill
\subfloat[]{\includegraphics[width=.3\linewidth]{screenshot-savesettings}
  \label{fig:savesettings}}
\hfill
\subfloat[]{\includegraphics[width=.3\linewidth]{screenshot-histogram}
  \label{fig:histogram}}

\caption{Screenshots van de \daq software.  Deze verduidelijken de
verschillende stappen voor het inregelen van de \pmts.  Zie de
beschrijving in de lopende tekst.}
\end{figure}

De detector is nu goed afgesteld.


\begin{thebibliography}{9}
\bibitem{hamamatsu} Hamamatsu, \emph{Photomultiplier Tubes, Construction
and Operating Characteristics Connections to External Circuits} (1997).
\bibitem{9107B} ET Enterprises, Ltd., \emph{9107B series data sheet}
(2010), \url{http://my.et-enterprises.com/pdf/9107B.pdf}.
\end{thebibliography}

\end{document}
