\documentclass[oneside, 11pt]{article}

\usepackage[T1]{fontenc}
\usepackage[utf8]{inputenc}
\usepackage[dutch]{babel}

\usepackage{fouriernc}
\usepackage[detect-all, load-configurations=binary,
            separate-uncertainty=true, per-mode=symbol,
            retain-explicit-plus, range-phrase={ tot }]{siunitx}

\usepackage{setspace}
\setstretch{1.2}

\setlength{\parskip}{\smallskipamount}
\setlength{\parindent}{0pt}

\usepackage{geometry}
\geometry{marginparwidth=0.5cm, verbose, a4paper, tmargin=3cm, bmargin=3cm, lmargin=2cm, rmargin=2cm}

\usepackage{float}

\usepackage[fleqn]{amsmath}
\numberwithin{equation}{section}
\numberwithin{figure}{section}

\usepackage{graphicx}
\graphicspath{{Figures/}}
\usepackage{subfig}

\usepackage{tikz}
\usetikzlibrary{plotmarks}

\usepackage{fancyhdr}
\pagestyle{fancy}
\fancyhf{}
\rhead{\thepage}
\renewcommand{\footrulewidth}{0pt}
\renewcommand{\headrulewidth}{0pt}

\usepackage{relsize}
\usepackage{xspace}
\usepackage{url}

\newcommand{\figref}[1]{Figuur~\ref{#1}}

\newcommand{\hisparc}{\textsmaller{HiSPARC}\xspace}
\newcommand{\kascade}{\textsmaller{KASCADE}\xspace}
\newcommand{\sapphire}{\textsmaller{SAPPHiRE}\xspace}
\newcommand{\jsparc}{\textsmaller{jSparc}\xspace}
\newcommand{\hdf}{\textsmaller{HDF5}\xspace}
\newcommand{\aires}{\textsmaller{AIRES}\xspace}
\newcommand{\csv}{\textsmaller{CSV}\xspace}
\newcommand{\python}{\textsmaller{PYTHON}\xspace}
\newcommand{\corsika}{\textsmaller{CORSIKA}\xspace}
\newcommand{\labview}{\textsmaller{LabVIEW}\xspace}
\newcommand{\daq}{\textsmaller{DAQ}\xspace}
\newcommand{\adc}{\textsmaller{ADC}\xspace}
\newcommand{\adcs}{\textsmaller{ADC}s\xspace}
\newcommand{\Adcs}{A\textsmaller{DC}s\xspace}
\newcommand{\hi}{\textsc{h i}\xspace}
\newcommand{\hii}{\textsc{h ii}\xspace}
\newcommand{\mip}{\textsmaller{MIP}\xspace}
\newcommand{\hisparcii}{\textsmaller{HiSPARC II}\xspace}
\newcommand{\hisparciii}{\textsmaller{HiSPARC III}\xspace}
\newcommand{\pmt}{\textsmaller{PMT}\xspace}
\newcommand{\pmts}{\textsmaller{PMT}s\xspace}

\DeclareSIUnit{\electronvolt}{\ensuremath{\mathrm{e\!\!\:V}}}

\DeclareSIUnit{\unitsigma}{\ensuremath{\sigma}}
\DeclareSIUnit{\mip}{\textsmaller{MIP}}
\DeclareSIUnit{\adc}{\textsmaller{ADC}}

\DeclareSIUnit{\gauss}{G}
\DeclareSIUnit{\parsec}{pc}
\DeclareSIUnit{\year}{yr}



\title{Data retrieval}
\author{D.B.R.A. Fokkema}
\docrecept{3}{DR}
\version{1.1}

\begin{document}

\maketitle

\begin{tabular}{|>{\raggedright}p{3cm}|>{\raggedright}p{12cm}|}
\hline
leerjaar & niveau \tabularnewline
\hline
5/6 VWO & beginnend / enigszins gevorderd \tabularnewline
\hline
\end{tabular}


\section{lesmateriaal}

Deze lessenserie van drie lessen geeft een inleiding op \hisparc en de
data die opgeslagen wordt. Aan de hand van een werkblad wordt de leerling
bekend gemaakt met de interface naar de \hisparc data, en de verschillende
stappen die nodig zijn om tot een klein eigen onderzoek te komen naar de
correlaties tussen kosmische straling en bijvoorbeeld het weer,
aardmagnetisch veld of activiteit van de zon.

\textbf{Alles wat nodig is voor deze lessenserie vindt u hier:}
\url{https://docs.hisparc.nl/zips/recept_data_retrieval.zip}

\begin{tabular}{ |>{\raggedright}p{2.5cm}|>{\raggedright}p{8cm}|>{\raggedright}p{4cm}|}
\hline
lesnummer & lesbeschrijving & bron \tabularnewline
\hline
1 & Introductie (klassikaal) \\
\emph{Cosmic air showers} \\
\emph{Werkblad: Cosmic air showers}
& \url{https://docs.hisparc.nl/infopakket/} \tabularnewline
\hline
2 & Computerpracticum \\
\emph{Werkblad: Data retrieval}
& \url{https://docs.hisparc.nl/infopakket/} \tabularnewline
\hline
3 & Computerpracticum / eigen onderzoek \\
Afmaken werkblad
& \tabularnewline
\hline
\end{tabular}


\section{Les 1}

Introductie van kosmische straling. In deze les starten we met
achtergrond materiaal en een werkblad. Tijdens deze les kunnen
leerlingen ook een station bezoeken als dat op school staat. Belangrijk
is dat er een introductie op kosmische straling gegeven wordt en dat de
terminologie duidelijk wordt gemaakt.
Behandel in ieder geval:

\begin{itemize}
  \item Wat is kosmische straling?
  \item Hoe ontstaat een deeltjeslawine?
  \item Hoe worden deze deeltjes in de lawine op aarde gedetecteerd?
\end{itemize}

\textbf{Opmerking:}
Van de \hisparc site (\url{www.hisparc.nl}) kunnen diverse bestaande
powerpoint-presentaties gedownload worden. Deze presentaties mogen naar
believen aangepast worden, om door leerlingen en docenten in de klas te
gebruiken.


\section{Les 2}

Het werkblad neemt de leerling mee op ontdekkingstocht door de
meetgegevens van \hisparc. Aan de hand van in eerste instantie
stap-voor-stap opdrachten voert de leerling een aantal beperkte analyses
uit. Het werkblad is aardig wat leeswerk, maar data analyseren gaat dan
ook niet vanzelf. Het is belangrijk om deze analyses zelf ook een keer uit
te voeren en na te denken over wat de gegevens betekenen. Dat maakt het
een stuk makkelijker om de leerling te helpen bij vragen.


\section{Les 3}

Afmaken werkblad. Aan het eind van het werkblad staat een aantal
suggesties voor een kort eigen onderzoek. De leerlingen moeten dan meer
data downloaden (bijvoorbeeld een week, of een maand), of data van
verschillende stations op dezelfde dag met elkaar vergelijken. Ook kan op
internet gezocht worden naar momenten van verhoogde zonneactiviteit. Je
kunt de leerlingen vragen een kort documentje te maken met uitleg wat ze
hebben onderzocht, wat ze hebben gevonden (grafieken) en wat de resultaten
betekenen.


% \section{Verdieping}
%
% \begin{tabular}{ |>{\raggedright}p{2.5cm}|>{\raggedright}p{8cm}|>{\raggedright}p{4cm}|}
% \hline
% leerdoelen & beschrijving bron & bron \tabularnewline
% \hline
% HiSPARC en Excell & werkblad met databron & \url{www.hisparc.nl/excell.doc} \tabularnewline
% \hline
% HiSPARC elektronica & interactief lesmateriaal & \url{www.hisparc.nl/elektro.htm} \tabularnewline
% \hline
% \end{tabular}

\end{document}
