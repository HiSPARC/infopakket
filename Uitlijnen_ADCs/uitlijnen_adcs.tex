\documentclass[oneside, 11pt]{article}

\usepackage[T1]{fontenc}
\usepackage[utf8]{inputenc}
\usepackage[dutch]{babel}

\usepackage{fouriernc}
\usepackage[detect-all, load-configurations=binary,
            separate-uncertainty=true, per-mode=symbol,
            retain-explicit-plus, range-phrase={ tot }]{siunitx}

\usepackage{setspace}
\setstretch{1.2}

\setlength{\parskip}{\smallskipamount}
\setlength{\parindent}{0pt}

\usepackage{geometry}
\geometry{marginparwidth=0.5cm, verbose, a4paper, tmargin=3cm, bmargin=3cm, lmargin=2cm, rmargin=2cm}

\usepackage{float}

\usepackage[fleqn]{amsmath}
\numberwithin{equation}{section}
\numberwithin{figure}{section}

\usepackage{graphicx}
\graphicspath{{Figures/}}
\usepackage{subfig}

\usepackage{tikz}
\usetikzlibrary{plotmarks}

\usepackage{fancyhdr}
\pagestyle{fancy}
\fancyhf{}
\rhead{\thepage}
\renewcommand{\footrulewidth}{0pt}
\renewcommand{\headrulewidth}{0pt}

\usepackage{relsize}
\usepackage{xspace}
\usepackage{url}

\newcommand{\figref}[1]{Figuur~\ref{#1}}

\newcommand{\hisparc}{\textsmaller{HiSPARC}\xspace}
\newcommand{\kascade}{\textsmaller{KASCADE}\xspace}
\newcommand{\sapphire}{\textsmaller{SAPPHiRE}\xspace}
\newcommand{\jsparc}{\textsmaller{jSparc}\xspace}
\newcommand{\hdf}{\textsmaller{HDF5}\xspace}
\newcommand{\aires}{\textsmaller{AIRES}\xspace}
\newcommand{\csv}{\textsmaller{CSV}\xspace}
\newcommand{\python}{\textsmaller{PYTHON}\xspace}
\newcommand{\corsika}{\textsmaller{CORSIKA}\xspace}
\newcommand{\labview}{\textsmaller{LabVIEW}\xspace}
\newcommand{\daq}{\textsmaller{DAQ}\xspace}
\newcommand{\adc}{\textsmaller{ADC}\xspace}
\newcommand{\adcs}{\textsmaller{ADC}s\xspace}
\newcommand{\Adcs}{A\textsmaller{DC}s\xspace}
\newcommand{\hi}{\textsc{h i}\xspace}
\newcommand{\hii}{\textsc{h ii}\xspace}
\newcommand{\mip}{\textsmaller{MIP}\xspace}
\newcommand{\hisparcii}{\textsmaller{HiSPARC II}\xspace}
\newcommand{\hisparciii}{\textsmaller{HiSPARC III}\xspace}
\newcommand{\pmt}{\textsmaller{PMT}\xspace}
\newcommand{\pmts}{\textsmaller{PMT}s\xspace}

\DeclareSIUnit{\electronvolt}{\ensuremath{\mathrm{e\!\!\:V}}}

\DeclareSIUnit{\unitsigma}{\ensuremath{\sigma}}
\DeclareSIUnit{\mip}{\textsmaller{MIP}}
\DeclareSIUnit{\adc}{\textsmaller{ADC}}

\DeclareSIUnit{\gauss}{G}
\DeclareSIUnit{\parsec}{pc}
\DeclareSIUnit{\year}{yr}



\begin{document}

\title{Uitlijnen van de \adcs}
\author{D. B. R. A. Fokkema}
\date{}

\maketitle

\section{De \hisparc uitleeselektronica}

De \hisparc detectoren bestaan uit een rechthoekige \emph{scintillator}
die via een driehoekige \emph{lichtgeleider} verbonden zijn met een
\emph{fotobuis} (\pmt \footnote{\pmt staat voor PhotoMultiplier Tube
(fotoversterkerbuis).}).  De fotobuis is verantwoordelijk voor de detectie
van de kleine lichtflitsjes in de scintillator die worden veroorzaakt door
geladen deeltjes uit de kosmische straling.  Deze lichtflitsjes worden
omgezet in een kleine elektrische stroom.  Deze stroom wordt door een
bekende weerstand geleid en de spanning over deze weerstand kan worden
gemeten.  Hoe feller het lichtflitsje, hoe hoger de spanning
(\figref{fig:schema-pmt}).

\begin{figure}
\centering
\framebox{Schema PMT (stroom door weerstand, uitlezen spanning)}
\caption{Elektrisch schema van het uitlezen van een PMT.}
\label{fig:schema-pmt}
\end{figure}

De \pmts zijn via lange kabels verbonden met de uitleeselektronica van
\hisparc (het `rode kastje').  De uitleeselektronica is verantwoordelijk
voor het omzetten van de elektrische spanningen van de \pmts in een
signaal dat de computer kan begrijpen.

\section{Uitlijnen van de \adcs}

\subsection{Stappenplan}


\begin{thebibliography}{9}
\bibitem{software-handleiding} Het \hisparc team, \emph{\hisparc software
documentatie} (2009--2012),
\url{http://docs.hisparc.nl/station-software/doc/}.
\end{thebibliography}

\end{document}
