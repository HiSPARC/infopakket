\documentclass[twoside, 11pt]{exam}

\input{../common_style}

\newcommand{\defaultstyle}{headandfoot}

% style definitions
\pagestyle{\defaultstyle}
\chead{\oddeven{\rightthumb}{\leftthumb}}
\cfoot{\theshorttitle\ -- \thepage}
\lfoot{\oddeven{}{\textcolor{gray}{\smaller Versie \theversion}}}
\rfoot{\oddeven{\textcolor{gray}{\smaller Versie \theversion}}{}}

\renewcommand{\thequestion}{\textbf{Opdracht \arabic{question}:}}
\renewcommand{\solutiontitle}{\noindent\textbf{Antwoord:}\enspace}
\newcommand{\makelines}[1]{\ifprintanswers\else\fillwithlines{#1\linefillheight}\fi}

\ifdefined\showanswers
  \printanswers
\else
  \noprintanswers
\fi


\title{Uitwerkingen: Cosmic air showers}
\author{J.M.C. Montanus}
\docwerkblad{4}{CS}
\version{1.1}

\begin{document}

\maketitle

\begin{questions}

\uplevel{\section{Kosmische deeltjes}}

\question Er is een naam voor het fysisch verschijnsel dat \emph{poollicht} of
\emph{noorderlicht} veroorzaakt. Zoek (in de literatuur of op internet)
deze naam op.
\makelines{2}
\begin{solution}
Fluorescentie
\end{solution}


\uplevel{\section{De energie van een deeltje}}

\question Als een muon een snelheid heeft dat \SI{99.9}{\percent} is van de lichtsnelheid, dan
is zijn energie \SI{2.36}{\giga\electronvolt}. Druk de energie van
\SI{2.36}{\giga\electronvolt} uit in joule (\si{\joule}).
\makelines{2}
\begin{solution}
\begin{equation}
    \SI{2.36}{\giga\electronvolt} \times
    \SI{1.60e-10}{\joule\per\giga\electronvolt} = \SI{3.78e-10}{\joule}
    \nonumber
\end{equation}
\end{solution}

\question Een proton heeft in rust een energie van \SI{938}{\mega\electronvolt}.
Bereken de energie van een kosmisch proton dat beweegt met een snelheid
dat \SI{99.99}{\percent} is van de lichtsnelheid.
\makelines{4}
\begin{solution}
\begin{equation}
    E(\num{0.9999}c) = \frac{938}{\sqrt{1-\num{0.9999}^2}}\si{\mega\electronvolt}
    = \SI{66.3}{\giga\electronvolt}. \nonumber
\end{equation}
\end{solution}

\question In 2007 kwam voor het eerst een harde schijf op de markt met een
geheugen van 1 terabyte (\si{\tera\byte}). Het geheugen van hard disks
(harde schijven) groeit ruwweg met een factor 1,6 per jaar. Stel dat
deze groei doorzet (wat niet de verwachting is), in welk jaar zal dan
een hard disk met een geheugen van 1 exabyte (\si{\exa\byte}) op de
markt komen?
\makelines{4}
\begin{solution}
\begin{equation}
    \num{1.6}^t = \frac{\num{e18}}{\num{e12}} \, \, \,
    \Rightarrow \, \, \, t = \num{29.4} \,\,\,
    \Rightarrow \, \, \, \textrm{in } 2036. \nonumber
\end{equation}
\end{solution}


\uplevel{\section{Showers}}

\uplevel{\subsection{Electromagnetische showers}}

\question Het omgekeerde proces van paar-creatie komt ook voor in de natuur:
paar-annihilatie. Zoek (in de literatuur of op internet) op wat dat is
en geef de bijbehorende reactievergelijking.
\makelines{2}
\begin{solution}
$e^- + e^+ \rightarrow \gamma + \gamma$ ,  maar bijvoorbeeld
$e^- + e^+ \rightarrow \mu^- + \mu^+$ en
$e^- + e^+ \rightarrow q + \bar{q}$ ($q=$quark en $\bar{q}=$antiquark)
zijn ook mogelijk.
\end{solution}


\uplevel{\subsection{Eenvoudig model voor de elektromagnetische shower}}

\question Ga uit van het Heitler model. Als een kosmisch gamma-deeltje van
\SI{1}{\exa\electronvolt} de dampkring binnenkomt, na hoeveel stappen
stopt, volgens het Heitler model, dan de shower?
\makelines{4}
\begin{solution}
\begin{equation}
\frac{10^{18}}{2^n}=84 \cdot 10^6
\;\Rightarrow\; n=33,5
\;\Rightarrow\; \textrm{na } 34 \textrm{ stappen}. \nonumber
\end{equation}
\end{solution}


\uplevel{\subsection{Hadronische showers}}

\question Elektronen en muonen zijn elementaire geladen deeltjes die behoren tot
de zogenaamde \emph{leptonen}. Er is nog een soort geladen elementair
deeltje dat tot de leptonen behoort. Zoek (in de literatuur of op
internet) op hoe dat deeltje wordt genoemd.
\makelines{2}
\begin{solution}
Het tau-deeltje ($\tau$)
\end{solution}

\question Het proton is geen elementair deeltje omdat het is samengesteld uit 3
quarks (quarks zijn wel elementaire deeltjes). Zoek (in de literatuur of
op internet) op uit hoeveel quarks een geladen pion bestaat.
\makelines{2}
\begin{solution}
Twee (een quark en een antiquark)
\end{solution}


\uplevel{\section{Levensduur en relativiteit}}

\question In een shower ontstaat een geladen pion met een energie van
\SI{1}{\tera\electronvolt}. Zoek de rustmass van een geladen pion op
(bijv. in BINAS) en bereken de verwachte levensduur van dit pion.
\makelines{3}
\begin{solution}
De rustmassa van een geladen pion is \SI{140}{\mega\electronvolt\per c\squared}
(opzoeken, bijv. in BINAS), dus de energie van een geladen pion in rust
is \SI{140}{\mega\electronvolt}
\begin{equation}
\tau (v) = \tau_0 \cdot \frac{E(v)}{E_0}
= \num{2.6e-8} \cdot \frac{\num{e12}}{\num{140e6}}
\;\Rightarrow\; \tau (v) = \SI{186}{\micro\second} \nonumber
\end{equation}
\end{solution}

\question In een shower ontstaat een geladen pion met een energie van
\SI{100}{\giga\electronvolt}. Ga er vanuit dat door toeval het pion geen
sterke wisselwerking meer ondergaat. In plaats daarvan vervalt het na
een tijdje. Bereken de verwachte afstand die dit pion dan zal hebben
afgelegd.
\makelines{4}
\begin{solution}
\begin{eqnarray}
    \tau (v) = \tau_0 \cdot \frac{E(v)}{E_0}
    = \num{2.6} \cdot \num{e-8} \cdot \frac{\num{e11}}{\num{140e6}}
    \;\Rightarrow\; \tau (v) = \SI{18.6}{\micro\second} \nonumber \\
    \sqrt{1-\frac{v^2}{c^2}} = \frac{E_0}{E(v)}
    = \frac{\num{140e6}}{\num{e11}} = \num{1.40e-5}
    \;\Rightarrow\; v \approx c = \SI{3e8}{\meter\per\second} \nonumber \\
    x = v \cdot \tau(v) = \num{3e8} \cdot \num{1.86e-5}
    = \SI{5.6e3}{\meter}\, (=\SI{5.6}{\kilo\meter}). \nonumber
\end{eqnarray}
\end{solution}

\end{questions}
\end{document}
