\documentclass[oneside, 11pt]{article}

\usepackage[T1]{fontenc}
\usepackage[utf8]{inputenc}
\usepackage[dutch]{babel}

\usepackage{fouriernc}
\usepackage[detect-all, load-configurations=binary,
            separate-uncertainty=true, per-mode=symbol,
            retain-explicit-plus, range-phrase={ tot }]{siunitx}

\usepackage{setspace}
\setstretch{1.2}

\setlength{\parskip}{\smallskipamount}
\setlength{\parindent}{0pt}

\usepackage{geometry}
\geometry{marginparwidth=0.5cm, verbose, a4paper, tmargin=3cm, bmargin=3cm, lmargin=2cm, rmargin=2cm}

\usepackage{float}

\usepackage[fleqn]{amsmath}
\numberwithin{equation}{section}
\numberwithin{figure}{section}

\usepackage{graphicx}
\graphicspath{{Figures/}}
\usepackage{subfig}

\usepackage{tikz}
\usetikzlibrary{plotmarks}

\usepackage{fancyhdr}
\pagestyle{fancy}
\fancyhf{}
\rhead{\thepage}
\renewcommand{\footrulewidth}{0pt}
\renewcommand{\headrulewidth}{0pt}

\usepackage{relsize}
\usepackage{xspace}
\usepackage{url}

\newcommand{\figref}[1]{Figuur~\ref{#1}}

\newcommand{\hisparc}{\textsmaller{HiSPARC}\xspace}
\newcommand{\kascade}{\textsmaller{KASCADE}\xspace}
\newcommand{\sapphire}{\textsmaller{SAPPHiRE}\xspace}
\newcommand{\jsparc}{\textsmaller{jSparc}\xspace}
\newcommand{\hdf}{\textsmaller{HDF5}\xspace}
\newcommand{\aires}{\textsmaller{AIRES}\xspace}
\newcommand{\csv}{\textsmaller{CSV}\xspace}
\newcommand{\python}{\textsmaller{PYTHON}\xspace}
\newcommand{\corsika}{\textsmaller{CORSIKA}\xspace}
\newcommand{\labview}{\textsmaller{LabVIEW}\xspace}
\newcommand{\daq}{\textsmaller{DAQ}\xspace}
\newcommand{\adc}{\textsmaller{ADC}\xspace}
\newcommand{\adcs}{\textsmaller{ADC}s\xspace}
\newcommand{\Adcs}{A\textsmaller{DC}s\xspace}
\newcommand{\hi}{\textsc{h i}\xspace}
\newcommand{\hii}{\textsc{h ii}\xspace}
\newcommand{\mip}{\textsmaller{MIP}\xspace}
\newcommand{\hisparcii}{\textsmaller{HiSPARC II}\xspace}
\newcommand{\hisparciii}{\textsmaller{HiSPARC III}\xspace}
\newcommand{\pmt}{\textsmaller{PMT}\xspace}
\newcommand{\pmts}{\textsmaller{PMT}s\xspace}

\DeclareSIUnit{\electronvolt}{\ensuremath{\mathrm{e\!\!\:V}}}

\DeclareSIUnit{\unitsigma}{\ensuremath{\sigma}}
\DeclareSIUnit{\mip}{\textsmaller{MIP}}
\DeclareSIUnit{\adc}{\textsmaller{ADC}}

\DeclareSIUnit{\gauss}{G}
\DeclareSIUnit{\parsec}{pc}
\DeclareSIUnit{\year}{yr}



\title{Opdrachten: Cosmic air showers}
\author{J.M.C. Montanus}
\docwerkblad{4}{CS}
\version{1.0}

\begin{document}

\maketitle

\section{Kosmische deeltjes}

\begin{minipage}[t]{1\columnwidth}
\paragraph{Opdracht 1:}
Er is een naam voor het fysisch verschijnsel dat \emph{poollicht} of
\emph{noorderlicht} veroorzaakt. Zoek (in de literatuur of op internet)
deze naam op.
\begin{center}
    \rule{\textwidth}{0.3mm}\\
    \rule{\textwidth}{0.3mm}\\
\end{center}
\end{minipage}
\bigskip{}


\section{De energie van een deeltje}

\begin{minipage}[t]{1\columnwidth}
\paragraph{Opdracht 2:}
Als een muon een snelheid heeft dat 99,9\% is van de lichtsnelheid, dan
is zijn energie \SI{2,36}{\giga\electronvolt}. Druk de energie van
\SI{2,36}{\giga\electronvolt} uit in joule (\SI{}{\joule}).
\begin{center}
    \rule{\textwidth}{0.3mm}\\
    \rule{\textwidth}{0.3mm}\\
\end{center}
\end{minipage}
\bigskip{}

\begin{minipage}[t]{1\columnwidth}
\paragraph{Opdracht 3:}
Een proton heeft in rust een energie van \SI{938}{\mega\electronvolt}.
Bereken de energie van een kosmisch proton dat beweegt met een snelheid
dat 99,99\% is van de lichtsnelheid.
\begin{center}
    \rule{\textwidth}{0.3mm}\\
    \rule{\textwidth}{0.3mm}\\
    \rule{\textwidth}{0.3mm}\\
    \rule{\textwidth}{0.3mm}\\
\end{center}
\end{minipage}
\bigskip{}

\begin{minipage}[t]{1\columnwidth}
\paragraph{Opdracht 4:}
In 2007 kwam voor het eerst een harde schijf op de markt met een
geheugen van 1 terabyte (\SI{}{\tera\byte}). Het geheugen van hard disks
(harde schijven) groeit ruwweg met een factor 1,6 per jaar. Stel dat
deze groei doorzet (wat niet de verwachting is), in welk jaar zal dan
een hard disk met een geheugen van 1 exabyte (\SI{}{\exa\byte}) op de
markt komen? 
\begin{center}
    \rule{\textwidth}{0.3mm}\\
    \rule{\textwidth}{0.3mm}\\
    \rule{\textwidth}{0.3mm}\\
    \rule{\textwidth}{0.3mm}\\
\end{center}
\end{minipage}


\section{Showers}

\subsection{Electromagnetische showers}

\begin{minipage}[t]{1\columnwidth}
\paragraph{Opdracht 5:}
Het omgekeerde proces van paar-creatie komt ook voor in de natuur:
paar-annihilatie. Zoek (in de literatuur of op internet) op wat dat is
en geef de bijbehorende reactievergelijking. 
\begin{center}
    \rule{\textwidth}{0.3mm}\\
    \rule{\textwidth}{0.3mm}\\
\end{center}
\end{minipage}


\subsection{Eenvoudig model voor de elektromagnetische shower}

\begin{minipage}[t]{1\columnwidth}
\paragraph{Opdracht 6:}
Ga uit van het Heitler model. Als een kosmisch gamma-deeltje van
\SI{1}{\exa\electronvolt} de dampkring binnenkomt, na hoeveel stappen
stopt, volgens het Heitler model, dan de shower? 
\begin{center}
    \rule{\textwidth}{0.3mm}\\
    \rule{\textwidth}{0.3mm}\\
    \rule{\textwidth}{0.3mm}\\
    \rule{\textwidth}{0.3mm}\\
\end{center}
\end{minipage}


\subsection{Hadronische showers}

\begin{minipage}[t]{1\columnwidth}
\paragraph{Opdracht 7:}
Elektronen en muonen zijn elementaire geladen deeltjes die behoren tot
de zogenaamde \emph{leptonen}. Er is nog een soort geladen elementair
deeltje dat tot de leptonen behoort. Zoek (in de literatuur of op
internet) op hoe dat deeltje wordt genoemd.  
\begin{center}
    \rule{\textwidth}{0.3mm}\\
    \rule{\textwidth}{0.3mm}\\
\end{center}
\end{minipage}
\bigskip{}

\begin{minipage}[t]{1\columnwidth}
\paragraph{Opdracht 8:}
Het proton is geen elementair deeltje omdat het is samengesteld uit 3
quarks (quarks zijn wel elementaire deeltjes). Zoek (in de literatuur of
op internet) op uit hoeveel quarks een geladen pion bestaat.  
\begin{center}
    \rule{\textwidth}{0.3mm}\\
    \rule{\textwidth}{0.3mm}\\
\end{center}
\end{minipage}
\bigskip{}


\section{Levensduur en relativiteit}

\begin{minipage}[t]{1\columnwidth}
\paragraph{Opdracht 9:}
In een shower ontstaat een geladen pion met een energie van
\SI{1}{\tera\electronvolt}. Zoek de rustmass van een geladen pion op
(bijv. in BINAS) en bereken de verwachte levensduur van dit pion.
\begin{center}
    \rule{\textwidth}{0.3mm}\\
    \rule{\textwidth}{0.3mm}\\
    \rule{\textwidth}{0.3mm}\\
\end{center}
\end{minipage}
\bigskip{}

\begin{minipage}[t]{1\columnwidth}
\paragraph{Opdracht 10:}
In een shower ontstaat een geladen pion met een energie van
\SI{100}{\giga\electronvolt}. Ga er vanuit dat door toeval het pion geen
sterke wisselwerking meer ondergaat. In plaats daarvan vervalt het na
een tijdje. Bereken de verwachte afstand die dit pion dan zal hebben
afgelegd.  
\begin{center}
    \rule{\textwidth}{0.3mm}\\
    \rule{\textwidth}{0.3mm}\\
    \rule{\textwidth}{0.3mm}\\
    \rule{\textwidth}{0.3mm}\\
\end{center}
\end{minipage}
\bigskip{}

\end{document}
