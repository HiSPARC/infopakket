\documentclass[oneside, 11pt]{article}

\usepackage[T1]{fontenc}
\usepackage[utf8]{inputenc}
\usepackage[dutch]{babel}

\usepackage{fouriernc}
\usepackage[detect-all, load-configurations=binary,
            separate-uncertainty=true, per-mode=symbol,
            retain-explicit-plus, range-phrase={ tot }]{siunitx}

\usepackage{setspace}
\setstretch{1.2}

\setlength{\parskip}{\smallskipamount}
\setlength{\parindent}{0pt}

\usepackage{geometry}
\geometry{marginparwidth=0.5cm, verbose, a4paper, tmargin=3cm, bmargin=3cm, lmargin=2cm, rmargin=2cm}

\usepackage{float}

\usepackage[fleqn]{amsmath}
\numberwithin{equation}{section}
\numberwithin{figure}{section}

\usepackage{graphicx}
\graphicspath{{Figures/}}
\usepackage{subfig}

\usepackage{tikz}
\usetikzlibrary{plotmarks}

\usepackage{fancyhdr}
\pagestyle{fancy}
\fancyhf{}
\rhead{\thepage}
\renewcommand{\footrulewidth}{0pt}
\renewcommand{\headrulewidth}{0pt}

\usepackage{relsize}
\usepackage{xspace}
\usepackage{url}

\newcommand{\figref}[1]{Figuur~\ref{#1}}

\newcommand{\hisparc}{\textsmaller{HiSPARC}\xspace}
\newcommand{\kascade}{\textsmaller{KASCADE}\xspace}
\newcommand{\sapphire}{\textsmaller{SAPPHiRE}\xspace}
\newcommand{\jsparc}{\textsmaller{jSparc}\xspace}
\newcommand{\hdf}{\textsmaller{HDF5}\xspace}
\newcommand{\aires}{\textsmaller{AIRES}\xspace}
\newcommand{\csv}{\textsmaller{CSV}\xspace}
\newcommand{\python}{\textsmaller{PYTHON}\xspace}
\newcommand{\corsika}{\textsmaller{CORSIKA}\xspace}
\newcommand{\labview}{\textsmaller{LabVIEW}\xspace}
\newcommand{\daq}{\textsmaller{DAQ}\xspace}
\newcommand{\adc}{\textsmaller{ADC}\xspace}
\newcommand{\adcs}{\textsmaller{ADC}s\xspace}
\newcommand{\Adcs}{A\textsmaller{DC}s\xspace}
\newcommand{\hi}{\textsc{h i}\xspace}
\newcommand{\hii}{\textsc{h ii}\xspace}
\newcommand{\mip}{\textsmaller{MIP}\xspace}
\newcommand{\hisparcii}{\textsmaller{HiSPARC II}\xspace}
\newcommand{\hisparciii}{\textsmaller{HiSPARC III}\xspace}
\newcommand{\pmt}{\textsmaller{PMT}\xspace}
\newcommand{\pmts}{\textsmaller{PMT}s\xspace}

\DeclareSIUnit{\electronvolt}{\ensuremath{\mathrm{e\!\!\:V}}}

\DeclareSIUnit{\unitsigma}{\ensuremath{\sigma}}
\DeclareSIUnit{\mip}{\textsmaller{MIP}}
\DeclareSIUnit{\adc}{\textsmaller{ADC}}

\DeclareSIUnit{\gauss}{G}
\DeclareSIUnit{\parsec}{pc}
\DeclareSIUnit{\year}{yr}



%\signature{Het HiSPARC team}
%\address{Science Park 105\\
%1098 XG \\ Amsterdam}

\begin{document}

\title{Brief rector}
\author{H. Montanus}
\date{}

\maketitle
 
%\begin{letter}{Hier kan de naam van een docent \\ of middelbare school staan}
Aan de rector van ` naam middelbare school ' 

%\opening{Geachte heer, mevr.  \\of   \\Geachte HiSPARC gebruiker}
Geachte heer, mevr. 

\hisparc is een grootschalig experiment waarmee middelbare scholieren
onderzoek verrichten aan kosmische straling. Het omvat momenteel een
netwerk van zo'n 100 stations waarvan de meeste zich op de daken van
middelbare scholen bevinden. Het \hisparc team waardeert het dat uw
middelbare school daar deel van uitmaakt.  

Voor de aanschaf van de detectoren heeft uw school indertijd een financi"ele investering gepleegd. Daarnaast hebben uw docenten tijd en energie ge\"investeerd in het operationeel maken en houden van de detectoren. Om deze investeringen optimaal te laten renderen voor uw onderwijs hebben wij tegelijk met deze brief de betreffende docent van uw school een \hisparc map  toegezonden.

De bedoeling van de \hisparc map is om leerlingen een ingang te geven in het onderhouden en optimaal laten functioneren van hun \hisparc station en het doen van onderzoek met de meetgegevens. De \hisparc map bevat daarom lesmateriaal in de vorm van een aantal praktische opdrachten. Uw docenten kunnen dit gebruiken voor de invulling van een aantal van hun lessen. Met de uitgewerkte praktische opdrachten kunnen leerlingen zelfstandig aan de slag. Op deze wijze ervaren leerlingen verschillende facetten van wetenschapelijk onderzoek. 

We wensen U (uw school?) veel succes met de \hisparc map.
 
%\closing{Hoogachtend,}
Hoogachtend,

%\encl{HiSPARC editorial}
\hisparc map
 
%\end{letter}
 
\end{document}
