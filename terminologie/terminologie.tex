\documentclass[oneside, 11pt]{article}

\usepackage[T1]{fontenc}
\usepackage[utf8]{inputenc}
\usepackage[dutch]{babel}

\usepackage{fouriernc}
\usepackage[detect-all, load-configurations=binary,
            separate-uncertainty=true, per-mode=symbol,
            retain-explicit-plus, range-phrase={ tot }]{siunitx}

\usepackage{setspace}
\setstretch{1.2}

\setlength{\parskip}{\smallskipamount}
\setlength{\parindent}{0pt}

\usepackage{geometry}
\geometry{marginparwidth=0.5cm, verbose, a4paper, tmargin=3cm, bmargin=3cm, lmargin=2cm, rmargin=2cm}

\usepackage{float}

\usepackage[fleqn]{amsmath}
\numberwithin{equation}{section}
\numberwithin{figure}{section}

\usepackage{graphicx}
\graphicspath{{Figures/}}
\usepackage{subfig}

\usepackage{tikz}
\usetikzlibrary{plotmarks}

\usepackage{fancyhdr}
\pagestyle{fancy}
\fancyhf{}
\rhead{\thepage}
\renewcommand{\footrulewidth}{0pt}
\renewcommand{\headrulewidth}{0pt}

\usepackage{relsize}
\usepackage{xspace}
\usepackage{url}

\newcommand{\figref}[1]{Figuur~\ref{#1}}

\newcommand{\hisparc}{\textsmaller{HiSPARC}\xspace}
\newcommand{\kascade}{\textsmaller{KASCADE}\xspace}
\newcommand{\sapphire}{\textsmaller{SAPPHiRE}\xspace}
\newcommand{\jsparc}{\textsmaller{jSparc}\xspace}
\newcommand{\hdf}{\textsmaller{HDF5}\xspace}
\newcommand{\aires}{\textsmaller{AIRES}\xspace}
\newcommand{\csv}{\textsmaller{CSV}\xspace}
\newcommand{\python}{\textsmaller{PYTHON}\xspace}
\newcommand{\corsika}{\textsmaller{CORSIKA}\xspace}
\newcommand{\labview}{\textsmaller{LabVIEW}\xspace}
\newcommand{\daq}{\textsmaller{DAQ}\xspace}
\newcommand{\adc}{\textsmaller{ADC}\xspace}
\newcommand{\adcs}{\textsmaller{ADC}s\xspace}
\newcommand{\Adcs}{A\textsmaller{DC}s\xspace}
\newcommand{\hi}{\textsc{h i}\xspace}
\newcommand{\hii}{\textsc{h ii}\xspace}
\newcommand{\mip}{\textsmaller{MIP}\xspace}
\newcommand{\hisparcii}{\textsmaller{HiSPARC II}\xspace}
\newcommand{\hisparciii}{\textsmaller{HiSPARC III}\xspace}
\newcommand{\pmt}{\textsmaller{PMT}\xspace}
\newcommand{\pmts}{\textsmaller{PMT}s\xspace}

\DeclareSIUnit{\electronvolt}{\ensuremath{\mathrm{e\!\!\:V}}}

\DeclareSIUnit{\unitsigma}{\ensuremath{\sigma}}
\DeclareSIUnit{\mip}{\textsmaller{MIP}}
\DeclareSIUnit{\adc}{\textsmaller{ADC}}

\DeclareSIUnit{\gauss}{G}
\DeclareSIUnit{\parsec}{pc}
\DeclareSIUnit{\year}{yr}



\title{Terminologie}
\author{A.P.L.S. de Laat}
\docalgemeen{1}{TL}
\version{1.1}

\begin{document}

\maketitle

\section{Begrippenlijst}

Hieronder staat een overzicht van veel gebruikte termen bij \hisparc. Bij elke term volgt een
korte omschrijving of een voorbeeld. Indien verschillend staan de termen er
zowel in het Nederlands als in het Engels, omdat veel van de documentatie en
delen van de website in het Engels zijn.


\section{Fysica}

\textbf{Kosmische straling -- Cosmic rays} \\
Geladen deeltjes, atoomkernen en fotonen die met hoge snelheid door
het heelal bewegen.

\textbf{Deeltjeslawine -- Air shower} \\
Een kaskade van deeltjes die met vrijwel de lichtsnelheid richting de
Aarde bewegen. Deze kaskade wordt geïnitieerd door de botsing tussen een
hoog energetische kosmische straling en atomen in de atmosfeer.


\section{Organisatie}

\textbf{\hisparc -- High School Project on Astrophysics Research with Cosmics} \\
\hisparc is een outreach- en onderzoeksproject waarbij metingen worden
uitgevoerd aan de deeltjeslawine geïnitieerd door kosmische straling uit
het heelal.

\textbf{Nikhef} \\
Het onderzoeksinstituut van waaruit \hisparc gecoördineerd wordt. Hier wordt tevens alle data opgeslagen.

\textbf{Cluster} \\
Voor een goed overzicht zijn stations die rond een centraal punt liggen gegroepeerd in clusters. Meestal vormen alle stations
rond grote steden een cluster. Dit maakt het eenvoudig om stations te
vinden die dicht bij een ander station liggen. Voorbeelden van clusters
in Nederlands zijn Utrecht, Nijmegen en Leiden.

\textbf{Subcluster} \\
De clusters zijn weer opgedeeld in subclusters, dit zijn vaak alle
stations in een stad of dorp. Voorbeelden hiervan zijn Science Park in
cluster Amsterdam en Hengelo in cluster Enschede.

\textbf{Station} \\
Een station is een meetstation. Het bestaat uit 2 of 4 detectoren die met elkaar
verbonden zijn aan een pc. Elk station heeft een uniek nummer waarmee
die geïdentificeerd wordt. Een voorbeeld hiervan is station 101 op het
St. Michael College in Zaanstad.

\textbf{Clustercoördinator} \\
Voor elke cluster is er een locale coördinator die contact met de
scholen onderhoudt en dient als eerste aanspreekpunt voor eventuele
vragen of problemen.


\section{Hardware}

\textbf{Detector} \\
Een detector bestaat uit scintillatiemateriaal waar een fotobuis aan
bevestigd is die de signalen doorgeeft aan de \hisparc elektronica. De
detectoren zijn meestal geplaatst in ski-boxen op de daken van
deelnemende scholen. Soms wordt er verwezen naar een detector als
\emph{de skibox}.

\textbf{Scintillatorplaat -- Scintillator} \\
Het scintillatorplaat is datgene wat energie absorbeert van geladen
deeltjes die erdoorheen bewegen. Deze energie wordt omgezet in licht dat
uitgezonden wordt. Bij \hisparc gebruiken we een plastic scintillator,
dit is een plastic met daarin een fluor die zorgt voor de emissie van licht.

\textbf{Foto versterker buis -- Photo Multiplier Tube (PMT)} \\
Een foto versterker buis kan de fotonen uit de scintillator omzetten in
een meetbaar elektrisch signaal voor de \hisparc elektronica.

\textbf{\hisparc elektronica -- \hisparc electronics} \\
Dit zijn de elektronica kastjes waar de foto buizen en \gps antenne aan
verbonden zijn. De elektronica leest de signalen uit, zet de signalen van
de foto buizen om naar digitale waarden en controleert of er een air
shower heeft plaatsgevonden. Ook voorziet het metingen van tijdstempels
met behulp van de interne \gps module.

\textbf{Primary \& Secondary} \\
Elk station heef een Primary, dat is de \hisparc elektronica die twee
detectoren uitleest en verbonden is met de \gps antenne. Sommige
stations hebben vier detectoren, in dat geval is er een tweede \hisparc
elektronica kastje, de Secondary, die de andere twee detectoren uitleest. De
Secondary is verbonden met de Primary om samen te kijken naar air showers.

\textbf{\gps module} \\
Dit zit in elke Primary en verwerkt de signalen van de \gps antenne. Dit
wordt gebruikt voor de nauwkeurige positie bepaling en voor de
tijdstempels van events.

\textbf{Analoog digitaal converter -- \adc} \\
Een elektronica component die een spanning uitleest en dit omzet tot een
digitale waarde. Dit wordt gebruikt om de \pmts uit te lezen.


\section{Data}

\textbf{Aankomsttijden -- Arrival time} \\
Dit is de tijd waarop het eerste signaal van deeltjes in een detector
gezien wordt bij een detectie. Zo is de volgorde te bepalen waarop de
verschillende detectoren geraakt zijn. Daarmee kan gereconstrueerd worden
uit welke richting de deeltjeslawine kwam.

\textbf{Detectie -- Event} \\
Als een station getriggerd wordt door een deeltjeslawine spreek je van
een detectie.

\textbf{Single} \\
Een single is wanneer \'e\'en detector een signaal over de drempelwaarde meet,
maar de andere detector(en) niet.

\textbf{Detectie frequentie -- Event rate} \\
Dit geeft aan hoe vaak (gemiddeld) per seconde een detectie plaats vindt.

\textbf{Coïncidentie -- Coincidence} \\
Als meerdere stations een detectie kort na elkaar doen is het
waarschijnlijk dat ze dezelfde deeltjeslawine gedetecteerd hebben. In
zo'n geval wordt er gesproken over een coïncidentie.

\textbf{Histogram -- Histogram} \\
Weergave van gegevens door te tonen hoeveel van de waarden tussen steeds
twee andere waarden vallen. Zoals het aantal metingen tussen 9 en 10
uur, tussen 10 en 11 uur, en tussen 11 en 12 uur enz.

\textbf{Minimaal ioniserend deeltje -- Minimum Ionizing Particle (\mip)} \\
Een deeltje met zo'n energie dat het weinig energie verliest door
ionisatie.

\textbf{Drempel -- Threshold} \\
De drempelwaarde is de sterkte van een signaal dat nodig is om de
\hisparc elektronica het te laten herkennen als een significant signaal. Pas als
meerdere detectoren een sterk signaal geven zal er een detectie zijn.
Dit wordt gebruikt om zwakke achtergrond straling en ruis te filteren.

\textbf{Nullijn -- Baseline} \\
Dit is de sterkte van het signaal als er geen
deeltjes door de detector gaan.

\textbf{Pulshoogte -- Pulseheight} \\
De pulshoogte is het verschil tussen de nullijn en de maximale signaalsterkte.

\textbf{Pulsintegraal -- Pulseintegral} \\
De puls integraal is de oppervlakte van het signaal onder de nullijn.
Dit is een maat voor het aantal deeltjes dat bij een detectie door de
detector gingen.

\textbf{Nanoseconde -- Nanosecond (ns)} \\
Om de aankomsttijden heel nauwkeurig te bepalen werken we op het niveau
van nanoseconden, dit is een zeer precieze tijd meting, een nanoseconde
is gelijk aan \SI{1e-9}{\second} or $\frac{1}{1000000000} s$.

\textbf{\adc counts} \\
De digitale waarden die uit de \adc komen. Deze hebben geen eenheid,
maar kunnen doormiddel van kalibratie vertaald worden naar millivolt.
Dit kan waarden tussen 0 en 4096 aannemen, hetgeen ongeveer overeenkomt met
\SIrange{+0.113}{-2}{\volt}.

\textbf{Zenit -- Zenith} \\
De zenit is het punt recht boven de waarnemer. Als we spreken over een
zenit hoek is het de hoek tussen een punt aan de hemel en het punt recht
omhoog.

\textbf{Azimut -- Azimuth} \\
Dit is de kompasrichting van een punt, dus de hoek tussen het noorden en
het punt, met de klok mee draaiend.


\section{Software}

\textbf{Data acquisitie - \hisparc\daq} \\
De software op de \hisparc PC die de \hisparc elektronica uitleest.

\textbf{\sapphire} \\
Een \python framework om analyses met \hisparc data uit te voeren:
\url{https://pypi.python.org/pypi/hisparc-sapphire/}.

\textbf{\jsparc} \\
Een JavaScript bibliotheek om websites te maken die met \hisparc data
werken.

\textbf{GitHub} \\
Een website voor software ontwikkeling. Hier is de broncode voor de
\hisparc software te vinden: \url{http://www.github.com/HiSPARC/}.

\textbf{Publieke database -- Public Database} \\
Via deze site is toegang tot de \hisparc data mogelijk. De website toont
voor elke dag een samenvatting van de data van een station. Ook zijn
hier de metingen en andere gegevens op te vragen. De site is te bereiken
via: \url{https://data.hisparc.nl/}.

\textbf{Nagios} \\
De status monitor die in de gaten houdt hoe het gaat met de stations.
Als er problemen optreden verstuurd deze automatisch mailtjes om men er
van op de hoogte te brengen.

\textbf{\hdf} \\
\hdf is een bestandsformaat dat we gebruiken om data in op te slaan.

\end{document}
