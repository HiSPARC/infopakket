\documentclass[12pt]{letter}

\usepackage[T1]{fontenc}
\usepackage[utf8]{inputenc}
\usepackage[dutch]{babel}

\usepackage{fouriernc}

\usepackage{setspace}
\setstretch{1.2}

\setlength{\parskip}{\smallskipamount}
\setlength{\parindent}{0pt}

\usepackage{graphicx}
\graphicspath{{Figures/}}

\usepackage{fancyhdr}
\fancypagestyle{empty}{
    \fancyhf{}\fancyhead[R]{
      \includegraphics[height=0.5in, keepaspectratio=true]{HiSPARC_header.pdf}}
      \renewcommand{\headrulewidth}{0pt}}

\usepackage{geometry}
\geometry{a4paper, vmargin=3cm, inner=3cm, outer=2cm, head=14pt}

\usepackage{relsize}
\usepackage{xspace}

\newcommand{\hisparc}{\textsmaller{HiSPARC}\xspace}


\signature{Het \hisparc team}
\address{Science Park 105\\
         1098 XG \\
         Amsterdam}

\begin{document}

\begin{letter}{[naam van een docent] \\ [middelbare school staan]}

\opening{Geachte heer, mevr.} % of Geachte HiSPARC gebruiker

\hisparc is een grootschalig experiment waarmee middelbare scholieren
onderzoek verrichten aan kosmische straling. Het omvat momenteel een
netwerk van zo'n 100 stations waarvan de meeste zich op de daken van
middelbare scholen bevinden. Het \hisparc team waardeert het dat uw
middelbare school daar deel van uitmaakt.  

Om de waarde van \hisparc voor uw leerlingen te benadrukken vindt U in
deze brief een \hisparc map met lesmateriaal en praktische opdrachten.

Het doel van het materiaal is om docenten en leerlingen een ingang te bieden voor onderhoud en onderzoek. Hiervoor zijn een aantal opdrachten samengesteld die de docent voor zijn lessen kan gebruiken.
Door deze opdrachten uit te voeren ervaren leerlingen de verschillende
facetten van wetenschappelijk onderzoek:
\begin{itemize}
	\item  het onderhouden van een experimentele opstelling
	\item  het opvragen en verwerken van de digitaal opgeslagen meetgegevens
	\item op basis van de meetgegevens de onderzoeksvraag beantwoorden. 
\end{itemize}
We wensen U veel succes met de \hisparc map.
\\
 
\closing{Hoogachtend,}

\encl{\hisparc map}
 
\end{letter}

\end{document}
