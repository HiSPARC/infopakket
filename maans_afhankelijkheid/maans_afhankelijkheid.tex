\documentclass[twoside, 11pt]{exam}

\input{../common_style}

\newcommand{\defaultstyle}{headandfoot}

% style definitions
\pagestyle{\defaultstyle}
\chead{\oddeven{\rightthumb}{\leftthumb}}
\cfoot{\theshorttitle\ -- \thepage}
\lfoot{\oddeven{}{\textcolor{gray}{\smaller Versie \theversion}}}
\rfoot{\oddeven{\textcolor{gray}{\smaller Versie \theversion}}{}}

\renewcommand{\thequestion}{\textbf{Opdracht \arabic{question}:}}
\renewcommand{\solutiontitle}{\noindent\textbf{Antwoord:}\enspace}
\newcommand{\makelines}[1]{\ifprintanswers\else\fillwithlines{#1\linefillheight}\fi}

\ifdefined\showanswers
  \printanswers
\else
  \noprintanswers
\fi



\title{Afhankelijkheid van de maanstand}
\author{C.G. van Veen}
\docopdrachten{1}{AFZ}
\version{1.1}


\begin{document}

\maketitle

\begin{questions}

\begin{EnvUplevel}
\section{Inleiding}

De maan zien we in zijn rondje rond de aarde steeds in een andere fase.

De schijnbare diameter van de maan aan de hemel is ongeveer een halve
graad.

Met de \hisparc stations kun je kosmische straling meten. We weten dat de
intensiteit/aantal per tijdseenheid van deze kosmische showers varieert.

Probeer met behulp van de dataretrieval tool op tijdstippen dat de maan in
bepaalde fasen staat (bijvoorbeeld nieuwe maan of volle maan) uit te
zoeken of de stand van de maan invloed heeft op het aantal events dat een
station heeft gemeten.
\end{EnvUplevel}

\question Is er een invloed van de maanstand op het aantal kosmische
showers te meten?

\begin{EnvUplevel}
\section{Plan van aanpak.}

\begin{itemize}
\item Zorg dat je bekend bent met de dataretrieval tool (zie
  infopakket \url{https://docs.hisparc.nl/infopakket/})
\item Zorg dat je voor een tijdsperiode het aantal events kunt
  selecteren met de gedownloade data.
\item Zorg ervoor dat je weet wanneer de maan in een bepaalde fase aan
  de hemel stond. Zie bijvoorbeeld:
  \url{https://www.kalender-365.nl/maan/maankalender.html})
\item Probeer voor deze tijdsperioden een correlatie te zoeken tussen
  maanstand en aantal showers of intensiteit van showers.
\end{itemize}
\end{EnvUplevel}


\uplevel{\section{Vervolgopdrachten}}

\question Is er een invloed van de maanstand op de intensiteit van de
kosmische showers te meten?

\question Kun je tijdens bijvoorbeeld volle maan een richtingsvoorkeur van
kosmische showers vinden?


\end{questions}
\end{document}
