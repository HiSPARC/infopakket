\documentclass[twoside, 11pt]{exam}

\input{../common_style}

\newcommand{\defaultstyle}{headandfoot}

% style definitions
\pagestyle{\defaultstyle}
\chead{\oddeven{\rightthumb}{\leftthumb}}
\cfoot{\theshorttitle\ -- \thepage}
\lfoot{\oddeven{}{\textcolor{gray}{\smaller Versie \theversion}}}
\rfoot{\oddeven{\textcolor{gray}{\smaller Versie \theversion}}{}}

\renewcommand{\thequestion}{\textbf{Opdracht \arabic{question}:}}
\renewcommand{\solutiontitle}{\noindent\textbf{Antwoord:}\enspace}
\newcommand{\makelines}[1]{\ifprintanswers\else\fillwithlines{#1\linefillheight}\fi}

\ifdefined\showanswers
  \printanswers
\else
  \noprintanswers
\fi


\usepackage{lipsum}

\title{Afhankelijkheid van de zonneactiviteit}
\author{N.G. Schultheiss}
\docwerkblad{5}{AFZ}
\version{1.0}

\begin{document}

\maketitle

\begin{questions}

\uplevel{\section{Inleiding}}

\uplevel{Naast de planeten, manen en kometen komen in het zonnestelsel ook atomaire deeltjes voor. De zon straalt een stroom deeltjes uit, deze staat bekend als de zonnewind. Deze deeltjes hebben een energie rond de \SI{10}{\kilo\eV} met uitschieters tot \SI{10}{\MeV} De hoeveelheid zonnewind hangt af van de zonne-activiteit.}
\uplevel{Daarnaast komen er deeltjes van buiten het zonnestelsel (galactie). Dit noemen we de kosmische straling. Hiervan komt grootste deel van binnen ons eigen melkwegstelsel. Daarnaast komt een klein deel van de kosmische straling van buiten het zonnestelsel, dit wordt extra-galactische kosmische straling genoemd.}
\uplevel{Binnen het zonnestelsel kunnen deeltjes uit beide stromen interacties met elkaar aangaan. Als deze interacties elastisch gebeuren wordt de kinetische energie verdeeld. Er zijn interacties waarbij een deel van de energie wordt omgezet in nieuwe deeltjes.}

\question{Zoek de energie van kosmische straling uit ons melkwegstelsel en de energie van extra-galactische kosmische straling op.}
\makelines{2}
\begin{solution}
  42
\end{solution}

\question Foobar.
\begin{parts}
  \part deel 1
  \makelines{2}
  \begin{solution}
    Eh...
  \end{solution}
  \part deel 2
  \makelines{2}
  \begin{solution}
    Die weet ik!
  \end{solution}
\end{parts}

\question And so on.
\makelines{4}
\begin{solution}
  \ldots{}and so forth!
\end{solution}


\uplevel{\section{Middenstuk}}

\question This is numbered correctly.
\makelines{2}
\begin{solution}
  4?
\end{solution}

\uplevel{\section{Conclusie}}

\question And this one too.
\makelines{1}
\begin{solution}
  Did I lose count?
\end{solution}


\end{questions}
\end{document}
