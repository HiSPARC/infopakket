\documentclass[twoside, 11pt]{exam}

\input{../common_style}

\newcommand{\defaultstyle}{headandfoot}

% style definitions
\pagestyle{\defaultstyle}
\chead{\oddeven{\rightthumb}{\leftthumb}}
\cfoot{\theshorttitle\ -- \thepage}
\lfoot{\oddeven{}{\textcolor{gray}{\smaller Versie \theversion}}}
\rfoot{\oddeven{\textcolor{gray}{\smaller Versie \theversion}}{}}

\renewcommand{\thequestion}{\textbf{Opdracht \arabic{question}:}}
\renewcommand{\solutiontitle}{\noindent\textbf{Antwoord:}\enspace}
\newcommand{\makelines}[1]{\ifprintanswers\else\fillwithlines{#1\linefillheight}\fi}

\ifdefined\showanswers
  \printanswers
\else
  \noprintanswers
\fi



\title{Afhankelijkheid van de zonneactiviteit}
\author{N.G. Schultheiss}
\docopdrachten{3}{AFZ}
\version{1.1}


\begin{document}

\maketitle

\begin{questions}

\begin{EnvUplevel}
\section{Inleiding}

Naast planeten, manen en kometen komen in het zonnestelsel ook atomaire
deeltjes voor. De zon straalt een stroom deeltjes uit, deze staat bekend
als de zonnewind. Deze deeltjes hebben een energie rond de
\SI{10}{\kilo\eV} met uitschieters tot \SI{10}{\MeV} De hoeveelheid
zonnewind hangt af van de zonne-activiteit.

Daarnaast komen er deeltjes van buiten het zonnestelsel (galactie). Dit
noemen we de kosmische straling. Hiervan komt grootste deel van binnen ons
eigen melkwegstelsel. Daarnaast komt een klein deel van de kosmische
straling van buiten het zonnestelsel, dit wordt extra-galactische
kosmische straling genoemd.

Binnen het zonnestelsel kunnen deeltjes uit beide stromen interacties met
elkaar aangaan. Als deze interacties elastisch gebeuren wordt de
kinetische energie verdeeld. Er zijn interacties waarbij een deel van de
energie wordt omgezet in nieuwe deeltjes.
\end{EnvUplevel}

\question Zoek de energie van kosmische straling uit ons melkwegstelsel en
de energie van extra-galactische kosmische straling op.
\makelines{2}
\begin{solution}
  $10^{15}$ $10^{20}$
\end{solution}

\question Geef twee argumenten voor het feit dat kosmische straling door
interactie met zonnewind minder energie per kosmische straal / deeltje
krijgt. Wat gebeurd er met de hoeveelheid kosmische straling?
\begin{parts}
  \part Argument 1
  \makelines{2}
  \begin{solution}
    Als de energie verdeeld wordt, krijgt het kosmisch deeltje minder
    energie en het zonnewind deeltje meer energie.
  \end{solution}
  \part Argument 2
  \makelines{2}
  \begin{solution}
    Een deel van de energie is nodig voor de creatie van nieuwe deeltjes.
  \end{solution}
  \part De hoeveelheid straling.
  \makelines{2}
  \begin{solution}
    Omdat de zonnewind het aardoppervlak niet kan bereiken, lijkt het of
    de totale straling toeneemt.
  \end{solution}
\end{parts}

\question Formuleer een hypothese betreffende de afhankelijkheid van
kosmische straling en zonnewind.
\makelines{4}
\begin{solution}
  \ldots{}and so forth!
\end{solution}


\begin{EnvUplevel}
\section{Opzet van het onderzoek}

Om de afhankelijkheid van kosmische straling van zonnewind te onderzoeken
hebben we metingen van zonnewind en komische straling nodig. Deze zijn
bijvoorbeeld op te halen op: ESA / SPENVIS of
\url{https://www.swpc.noaa.gov/SWN/}.

Het ophalen van meetgegevens voor door HiSPARC gemeten kosmische straling
is beschreven in
\url{https://docs.hisparc.nl/infopakket//} Data
retrieval.

Laadt de meetgegevens in een spreadsheet programma en onderzoek of er een
correlatie te ontdekken is. Met welke nauwkeurigheid is deze correlatie te
leggen?
\end{EnvUplevel}


\uplevel{\section{Conclusie}}

\question De onderzochte correlatie tussen zonnewind en kosmische straling
is op de volgende wijze te beargumenteren:
\makelines{1}
\begin{solution}
  Hier komt een kort helder betoog.
\end{solution}


\end{questions}
\end{document}
