\documentclass[twoside, 11pt]{exam}

\input{../common_style}

\newcommand{\defaultstyle}{headandfoot}

% style definitions
\pagestyle{\defaultstyle}
\chead{\oddeven{\rightthumb}{\leftthumb}}
\cfoot{\theshorttitle\ -- \thepage}
\lfoot{\oddeven{}{\textcolor{gray}{\smaller Versie \theversion}}}
\rfoot{\oddeven{\textcolor{gray}{\smaller Versie \theversion}}{}}

\renewcommand{\thequestion}{\textbf{Opdracht \arabic{question}:}}
\renewcommand{\solutiontitle}{\noindent\textbf{Antwoord:}\enspace}
\newcommand{\makelines}[1]{\ifprintanswers\else\fillwithlines{#1\linefillheight}\fi}

\ifdefined\showanswers
  \printanswers
\else
  \noprintanswers
\fi


\usepackage{lipsum}

\title{Event Histogram Onderzoeken} \author{C.G.N. van Veen}
\docwerkblad{7}{EHO} \version{1.0}

\begin{document}

\maketitle

\begin{questions}

\begin{EnvUplevel}
\section{Inleiding}

Op de site \url{data.hisparc.nl} zie je een overzicht van de
stations van \hisparc en hun status, zie \figref{fig:data_hisparc}. Als
je op een station klikt dan kun je de allereerst de metingen van de
vorige dag zien. Als in een station twee detectoren op ongeveer
hetzelfde moment een puls meten spreken we van een event. Een event
geeft dus aan dat er een kosmische shower is gedetecteerd.

De metingen van het aantal showers wat per uur gemeten is staat uitgezet
in het event histogram, zie \figref{fig:Event_histogram}

Soms zie je dat het aantal events (dus  constant over de uren van de dag,
maar er zijn ook stations die een afwijkende hoeveelheid events laten
zien.

Die afwijkingen in het event histogram komen dan ook vaker
voor. Zoek zo'n station waarbij het event histogram afwijkingen laat
zien en probeer uit te zoeken wat die afwijking veroorzaakt.

Probeer met behulp van de dataretrieval tool op tijdstippen dat
je de afwijkingen in het event histogram ziet, gegevens te verzamelen
over het weer, aantal events , stations in de buurt van het afwijkende
station etc.
\end{EnvUplevel}

\question Waardoor kunnen we het afwijkende event
histogram verklaren?


\begin{EnvUplevel}
\section{Plan van aanpak}

\begin{itemize}
    \item Zorg dat je bekend bent met de dataretrieval tool
    (zie infopakket \url{http://docs.hisparc.nl/infopakket/})
    \item Zorg dat je voor een tijdsperiode het aantal events kunt selecteren met de gedownloade data.
    \item Zorg ervoor dat je ook van dichtbijgelegen stations gegevens (ook
    weer gegevens) verzameld.
    \item Probeer voor deze tijdsperioden een
    correlatie tussen de gegevens en zo een conclusie over de oorzaak van
    het afwijkende event histogram te vinden.
\end{itemize}
\end{EnvUplevel}


\uplevel{\section{Vervolgopdrachten}}

\question Hebben afwijkende event histogrammen altijd dezelfde oorzaak?

\begin{figure}
    \centering
    \includegraphics[width=0.90\textwidth]{data_hisparc}
    \caption{website met het overzicht van de stations}
    \label{fig:data_hisparc}
\end{figure}

\begin{figure}
    \centering
    \includegraphics[width=0.90\textwidth]{Event_histogram} \caption{Een
    event histogram met afwijkende waarde rond het middaguur.}
    \label{fig:Event_histogram}
\end{figure}

\end{questions}
\end{document}
